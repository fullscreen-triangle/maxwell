\section{Pure Fabrication: Dreams as Plato's Cave}

\subsection{The Unconstrained State}

Having established that consciousness involves continuous internal fabrication ($\Theta_0$) constrained by external input ($\Psi_0$), and that this process is inherently private, we now examine what occurs when the external constraint is removed. During REM sleep, thalamic gating reduces sensory input to $\Psi_0 \approx 0$, leaving the fabrication mechanism to operate without correction. This state of pure or nearly-pure fabrication is what we experience as dreaming.

The philosophical significance of this state was recognised over two millennia ago in Plato's allegory of the cave, though Plato could not have known the neuroscientific mechanisms underlying his metaphor. We argue that Plato's cave is not merely a useful analogy for dreaming but rather a formally equivalent description of the epistemic situation when external input is absent.

\subsection{Plato's Cave: The Original Formulation}

\subsubsection{The Allegory (Republic, Book VII, circa 380 BCE)}

Plato describes prisoners who have been confined in a cave since childhood, chained in fixed positions and facing a wall. Behind them burns a fire. Between the fire and the prisoners, people carry objects—statues of animals, humans, and tools—casting shadows on the wall. The prisoners perceive only these shadows and the echoes of sounds, never seeing the actual objects or the fire.

Having experienced nothing but shadows their entire lives, the prisoners naturally believe the shadows \textit{are} reality. They develop expertise in predicting shadow patterns, compete in naming shadows, honor those most skilled at shadow-interpretation. Their entire epistemic framework—what they know, how they reason, and what they value—is constructed from shadow-observations.

Plato then describes one prisoner being freed by an external agent who forcibly turns him toward the fire. Initially, the prisoner is blinded by the light, pained by the reorientation, and confused by seeing three-dimensional objects for the first time. The objects appear less real than the familiar shadows. Only gradually does the freed prisoner come to understand that objects are more real than shadows, that the fire is the source of illumination, and that the cave is a limited domain.

Eventually, the freed prisoner is led outside the cave entirely, experiencing sunlight for the first time—overwhelming, then illuminating. Upon returning to the cave to inform the other prisoners of the truth, the freed prisoner finds that his vision has become readjusted to darkness, he appears clumsy in the dim light, and he is mocked by prisoners who never left. They reject the testimony about an external world, seeing it as evidence that leaving the cave damages one's faculties.

\subsubsection{The Epistemic Structure}

The key features of Plato's cave, relevant to our framework:

\begin{enumerate}
\item \textbf{Complete sensory environment}: The prisoners' entire sensory experience comes from shadows. There is no alternative input source.

\item \textbf{Self-consistency}: The shadow-world is internally coherent. Shadows follow predictable patterns, have stable relationships, and exhibit a causal structure.

\item \textbf{Epistemic closure}: All knowledge, reasoning, and concepts are derived from shadow-observations. The prisoners cannot formulate concepts that transcend shadow-experience.

\item \textbf{Impossibility of self-liberation}: Plato specifies that prisoners cannot free themselves. The chains, the enforced orientation, and the epistemic framework all prevent self-escape.

\item \textbf{External intervention necessity}: Liberation requires an external agent to physically turn the prisoner around, providing novel sensory input that cannot be generated internally.

\item \textbf{Initial resistance}: Upon seeing objects and fire, the freed prisoner initially finds shadows more compelling—the familiar seems more real than the novel.

\item \textbf{Gradual recognition}: Only through sustained exposure to the new input does the prisoner recognise the greater reality of objects over shadows.
\end{enumerate}

\subsection{Dreams as the Cave State}

We propose that dreaming during REM sleep is formally equivalent to being imprisoned in Plato's cave.

\subsubsection{The Mapping}

\begin{table}[H]
\centering
\caption{Formal Correspondence Between Plato's Cave and REM Dreaming}
\begin{tabular}{@{}lll@{}}
\toprule
\textbf{Cave Element} & \textbf{Dream Equivalent} & \textbf{Formal Description} \\
\midrule
Shadows on wall & Dream imagery & $\Theta_0$ (internal fabrication) \\
Actual objects & External reality & $\Psi_0$ (sensory input) \\
Fire (light source) & Waking consciousness & Reality testing enabled \\
Chains & REM sleep paralysis & Motor atonia \\
Fixed orientation & Sensory gating & Thalamic blocking ($\Psi_0 \to 0$) \\
Shadow-world expertise & Dream logic & Internal consistency rules \\
Prisoners' consensus & Dream reality belief & Absence of doubt \\
External liberator & Alarm/stimulus & $\Psi_0 > 0$ restored \\
Painful reorientation & Sleep inertia & Gradual wake transition \\
Return to cave & Falling asleep again & $\Psi_0 \to 0$ again \\
\bottomrule
\end{tabular}
\end{table}

\subsubsection{Epistemic Equivalence}

The cave prisoners' situation is epistemically identical to the dreamer's:

\begin{itemize}
\item \textbf{Single information source}: Prisoners have only shadows ($\Theta_0$ unconstrained); dreamers have only internal fabrication ($\Theta_0$ with $\Psi_0 \approx 0$)

\item \textbf{No external reference}: Prisoners cannot see objects to compare against shadows; dreamers cannot access external reality to compare against dream content

\item \textbf{Internal coherence}: Shadow-world has consistent patterns; dream-world has internal logic (though different from waking logic)

\item \textbf{Belief in reality}: Prisoners believe shadows are real; dreamers believe dream is real (while dreaming)

\item \textbf{Inability to self-liberate}: Prisoners cannot turn themselves around; dreamers cannot wake themselves through pure internal decision
\end{itemize}

The last point is crucial and will be examined in depth. If the cave-dream equivalence holds, then just as prisoners cannot free themselves without external intervention, dreamers cannot wake themselves without external input.

\subsection{The Self-Liberation Impossibility}

\subsubsection{Why Prisoners Cannot Free Themselves}

Plato's prisoners face multiple barriers to self-liberation:

\textbf{Physical constraint}: The chains prevent turning around. Even if a prisoner wished to see the fire, the physical limitations prevent it.

\textbf{Epistemic constraint}: More fundamentally, why would a prisoner \textit{wish} to turn around? To question whether shadows are real requires conceiving of an alternative to shadows. But the prisoners' entire conceptual framework is constructed from shadow-observations. What would "more real than shadows" even mean to someone who has only experienced shadows?

\textbf{Verification constraint}: Suppose a prisoner somehow conceives the thought, "Perhaps shadows are not the fundamental reality." How would they test this hypothesis? Any test they devise must use shadow-observations (their only data source). Testing shadows using shadows is circular—like trying to calibrate a thermometer using only that thermometer's readings.

\textbf{Social constraint}: The other prisoners reinforce the shadow-reality. Consensus validates the belief system. To question shadows is to question the shared epistemic framework, marking oneself as confused or impaired.

\subsubsection{The Information-Theoretic Impossibility}

We can formalise this using information theory. For the prisoner to recognise shadows as mere representations requires:

\begin{equation}
I_{\text{liberation}} = I(\text{objects}|\text{shadows}) = H(\text{objects}) - H(\text{objects}|\text{shadows})
\end{equation}

where $H(\cdot)$ is entropy (uncertainty) and $I(\cdot|\cdot)$ is mutual information. This is the information about objects that cannot be inferred from shadows alone.

For Plato's setup:
\begin{itemize}
\item Shadows provide 2D projections of 3D objects
\item Multiple 3D configurations can produce identical 2D shadows
\item Depth, colour, texture information is lost in projection
\item Rotational symmetries create ambiguities
\end{itemize}

Thus $I_{\text{liberation}} > 0$—there exists information about objects that is fundamentally unavailable from shadow-observations. No amount of clever reasoning about shadows can recover this information because it is not present in the shadow-data.

The prisoner needs external input—seeing the actual objects—to acquire the missing information. Internal processing of existing data (shadows) cannot generate genuinely new information about external reality.

\begin{figure*}[htbp]
    \centering
    \includegraphics[width=\textwidth]{figures/cognitive_processing_analysis.png}
    \caption{
        \textbf{Cognitive state dynamics and network coupling during simulated dream-wake transitions.}
        \textbf{(Top row, left)} Temporal evolution of four cognitive state levels over 180 seconds: Consciousness (purple) oscillates at low amplitude ($\sim$3), Working Memory (green) shows periodic peaks to level 7, Executive function (red) remains suppressed near zero, and Attention (blue) exhibits minimal activation.
        \textbf{(Top row, center)} Neural oscillations across cognitive domains over 10 seconds, showing offset stacked traces: Working Memory (blue baseline), Attention (green, offset +50), Executive (red, offset +100), and Consciousness (purple, offset +150). High-frequency.
        \textbf{(Top row, right)} Cognitive performance metrics: Reaction Time (red, left axis) oscillates between 350-430 ms with period $\sim$25 seconds, while Accuracy (blue, right axis) varies between 0.6-1.3 in anti-phase relationship, suggesting periodic attentional lapses.
        \textbf{(Middle row, left)} Cognitive network coupling matrix showing weak coupling across all domain pairs: strongest coupling between Attention-Working Memory (0.4) and Working Memory-Executive (0.6), but near-zero coupling to Consciousness ($<0.2$ for all pairs), indicating functional isolation of conscious awareness.
        \textbf{(Middle row, center)} Key coupling relationships quantified: Attention-Executive coupling = 0.0037, Working Memory-Consciousness coupling = 0.0039, Attention-Consciousness coupling = 0.0030. All values fall orders of magnitude below waking thresholds ($>0.3$).
        \textbf{(Middle row, right)} Reaction Time vs. Attention neural activity scatter plot showing weak positive correlation ($r = 0.329$), contrary to expected strong negative correlation ($r = -0.8$ to $-0.2$) during conscious processing. The positive correlation suggests that increased neural activity paradoxically slows responses, consistent with dream-like dissociation.
        \textbf{(Bottom row, left)} Processing efficiency over time showing periodic peaks to 0.8 efficiency every $\sim$50 seconds, with baseline efficiency oscillating between 0.2-0.4, indicating intermittent rather than sustained cognitive integration.
        \textbf{(Bottom row, center)} Cognitive resource allocation showing periodic oscillations in all three resource pools (Attention, Working Memory, Executive) with period $\sim$50 seconds and amplitude $\sim$0.5-1.8 arbitrary units, but minimal coherence between pools.
        \textbf{(Bottom row, right)} Validation summary indicating \textsc{fail} status: Attention-Executive coupling = 0.004 (required $\geq 0.4$), WM-Consciousness coupling = 0.004 (required $\geq 0.35$), RT-Neural correlation = 0.329 (expected $-0.8$ to $-0.2$), Cognitive coherence = 0.003 (required $\geq 0.3$).
    }
    \label{fig:cognitive_processing}
\end{figure*}


\subsubsection{Dreams and the Same Impossibility}

The dreamer faces an equivalent information-theoretic constraint. To recognize one is dreaming requires:

\begin{equation}
I_{\text{waking}} = I(\text{reality}|\text{dream}) = H(\text{reality}) - H(\text{reality}|\text{dream})
\end{equation}

During dreaming with $\Psi_0 \approx 0$:
\begin{itemize}
\item Dream content is generated from $\Theta_0$ (learned from historical $\Psi_0$)
\item Current external reality state is not accessible
\item The correspondence between current $\Theta_0$ and current external reality is unknown
\item Multiple external states are consistent with the same $\Theta_0$ content
\end{itemize}

Thus $I_{\text{waking}} > 0$—information about current reality is unavailable from dream content alone. The dreamer would need external input (restoration of $\Psi_0 > 0$) to determine whether dream content matches external reality.

Crucially, even if a dreamer thinks "Am I dreaming?", answering this requires comparing dream content against an external reference. With $\Psi_0 \approx 0$, there is no external reference to access. The question becomes undefined, like asking "What is north of the North Pole?"

\subsection{Pure Fabrication Without Constraint}

\subsubsection{The Dream Logic Phenomenon}

A striking feature of dreams is "dream logic"—reasoning that seems valid during the dream but appears absurd upon waking. Examples:

\begin{itemize}
\item "I needed to reach the top floor, so I flapped my arms and flew up the stairwell. This seemed perfectly reasonable."
\item "My childhood dog was also my algebra teacher. I didn't find this contradictory."
\item "The building kept shifting between my house and my workplace, but I didn't notice the inconsistency."
\end{itemize}

These are not random nonsense but reflect the operation of $\Theta_0$ without $\Psi_0$ constraint. The internal model generates content based on learned patterns, but without external error signals, violations of waking logic go undetected.

In Plato's framework: the prisoners develop "shadow logic"—ways of reasoning about shadows that seem perfectly valid within the shadow-world but appear limited when compared to reasoning about actual objects. For instance, prisoners might believe shadows have no depth (true for shadows, false for objects) and build elaborate theories around this "fundamental truth."

\subsubsection{The Statistical Manifold of Dreams}

While dream content can violate physical laws, it does not violate \textit{learned statistical structure}. Recall from the fabrication section:

\begin{equation}
\Theta_0(t) \subseteq \text{span}\left\{\Psi_0(\tau) : \tau < t\right\}
\end{equation}

Dreams draw from the space of previously experienced patterns. This is why:

\begin{itemize}
\item Dream faces, while often unfamiliar, have eyes/nose/mouth in standard configurations
\item Dream spaces follow approximate architectural rules (rooms have walls/floors/ceilings)
\item Dream conversations use language with typical syntax
\item Dream movements feel kinesthetically familiar
\end{itemize}

Even violations of physical laws occur \textit{against a background of learned structure}. Flying dreams involve wing-flapping or running motions (learned motor patterns), not spontaneous levitation. Morphing objects transform through visually coherent intermediate states, not discontinuous jumps.

This reveals that $\Theta_0$ is not generating random noise but rather sampling from the learned statistical manifold of waking experience. Without $\Psi_0$ to constrain which samples are reality-consistent, the fabrication mechanism explores the manifold freely.

\subsection{Sleep Deprivation: The Necessity of Constraint}

The cave-dream framework explains a striking clinical phenomenon: sleep deprivation produces hallucinations during waking.

\subsubsection{The Progressive Degradation}

Extended wakefulness (24-72+ hours without sleep) produces predictable cognitive effects (Killgore, 2010; Banks \& Dinges, 2007):

\textbf{First 24 hours}:
\begin{itemize}
\item Attention lapses (microsleeps 1-3 seconds)
\item Slowed reaction time
\item Reduced working memory
\item Mood deterioration
\end{itemize}

\textbf{24-48 hours}:
\begin{itemize}
\item Visual distortions (objects appear to shimmer, walls breathe)
\item Auditory misperceptions (hearing name called when not spoken)
\item Illusions (shadows appearing as figures, patterns becoming faces)
\item Reduced reality monitoring
\end{itemize}

\textbf{48-72+ hours}:
\begin{itemize}
\item Frank hallucinations (seeing people/objects that aren't present)
\item Paranoid thoughts
\item Dissociation
\item Delirium
\end{itemize}

\subsubsection{The Mechanistic Explanation}

Sleep deprivation does not eliminate external input ($\Psi_0$ remains available), so why do hallucinations emerge during waking?

The BMD framework provides the answer: sustained wakefulness degrades the input filter $\Im_{\text{input}}$, weakening the constraint on the output filter $\Im_{\text{output}}$.

During normal waking:
\begin{equation}
\text{BMD}_{\text{normal}} = \Im_{\text{input}}(\Psi_0) \circ \Im_{\text{output}}(\Theta_0) \quad \text{with strong coupling}
\end{equation}

The input filter robustly processes $\Psi_0$, providing strong error signals that keep $\Theta_0$ aligned with reality.

During sleep deprivation:
\begin{equation}
\text{BMD}_{\text{deprived}} = \Im_{\text{input}}^{\text{weak}}(\Psi_0) \circ \Im_{\text{output}}(\Theta_0) \quad \text{with degraded coupling}
\end{equation}

The input filter becomes unreliable—sensory processing is impaired, attention falters, working memory cannot maintain context. External input $\Psi_0$ is still present but is processed poorly by the degraded $\Im_{\text{input}}$. As a result, the constraint on $\Theta_0$ weakens, and fabrication begins to dominate.

This produces a hybrid state:
\begin{equation}
\mathcal{R}_{\text{exp}}^{\text{deprived}} = \alpha_{\text{deprived}} \Theta_0 + (1-\alpha_{\text{deprived}}) \Psi_0 \quad \text{where } \alpha_{\text{deprived}} \approx 0.6\text{--}0.8
\end{equation}

This is intermediate between normal waking ($\alpha \approx 0.3$) and dreaming ($\alpha \approx 0.95$). The individual is technically awake (eyes open, motor control present) but internal fabrication increasingly dominates experience.

\subsubsection{Microsleeps and Dream Intrusions}

Additionally, sleep deprivation produces involuntary microsleeps—brief (1-10 second) periods where the brain enters sleep-like states despite behavioral wakefulness. EEG shows bursts of theta rhythm characteristic of drowsiness/sleep.

During these microsleeps, $\Psi_0$ processing is further suppressed, and dream content can intrude into waking experience. The subject may see vivid imagery overlaid on the actual environment or experience brief dreamlike scenarios while eyes remain open.

This is precisely the cave situation occurring during waking: external input is present but not processed, so fabrication takes over. The subject becomes a prisoner seeing shadows (internal fabrication) rather than objects (external reality), despite having their eyes open.

\subsubsection{Why Sleep is Mandatory}

The sleep deprivation phenomenon reveals why sleep is not merely beneficial but \textit{mandatory} for maintaining reality testing.

During waking, the continuous operation of $\Im_{\text{input}}$ processing $\Psi_0$ consumes resources:
\begin{itemize}
\item Neural metabolites accumulate (adenosine, oxidative waste)
\item Synaptic weights become saturated (need downscaling)
\item Attentional systems deplete glucose reserves
\item Error correction mechanisms fatigue
\end{itemize}

Sleep allows these systems to recover:
\begin{itemize}
\item Metabolite clearance (glymphatic system active during sleep)
\item Synaptic downscaling (pruning unnecessary connections)
\item Memory consolidation (integrating new $\Psi_0$ into $\Theta_0$)
\item System maintenance impossible during active $\Psi_0$ processing
\end{itemize}

Critically, these maintenance operations \textit{require} reducing $\Psi_0$ input. You cannot maintain a sensory processing system while simultaneously using it at full capacity, just as you cannot perform maintenance on a running engine.

Thus, the cave state (dreaming with $\Psi_0 \approx 0$) is not a dysfunction but a necessary maintenance mode. The danger arises not from entering this state nightly but from failing to alternate properly between constrained (waking) and unconstrained (dreaming) operation.

\subsection{Implications for Continuous Constraint}

\subsubsection{The Fragility of Reality Testing}

The sleep deprivation data demonstrate that reality testing is not a robust, fail-safe operation but a fragile achievement requiring continuous external input.

Removing or degrading $\Psi_0$ for even moderate durations (24-48 hours) produces profound alterations in experienced reality. This fragility reveals that waking consciousness is not a "default state" maintained automatically but an active process requiring ongoing sensory constraint.

This has evolutionary implications: why would natural selection produce a system so dependent on external input that brief deprivation causes hallucinations? The answer lies in the computational efficiency of fabrication-constrained-by-input versus attempting to maintain pure bottom-up perception:

\begin{itemize}
\item Pure bottom-up: Process all sensory data from scratch → computationally expensive
\item Top-down prediction + error correction: Generate predictions, correct only deviations → computationally efficient
\end{itemize}

The predictive coding architecture (Friston, 2010) is efficient but requires the error correction signal from $\Psi_0$. Without it, predictions ($\Theta_0$) run unchecked.

\subsubsection{Sensory Deprivation as Artificial Cave}

Sensory deprivation experiments (floatation tanks, Ganzfeld stimulation, prolonged darkness) demonstrate that even with adequate sleep, reducing $\Psi_0$ toward zero during waking produces cave-like states.

In floatation tanks (Suedfeld \& Borrie, 1999):
\begin{itemize}
\item After 30-60 minutes: Visual/auditory hallucinations begin
\item After 1-2 hours: Vivid imagery, dream-like experiences
\item Upon emerging: Brief disorientation, need to "remember" external reality
\end{itemize}

This is faster than sleep deprivation (hours vs. days) because external input is more completely eliminated. The subject is rested (input filter intact) but has nothing to filter ($\Psi_0 \approx 0$). Internal fabrication fills the void.

The rapid onset demonstrates that $\Theta_0$ is \textit{always generating} content; during normal waking, this generation is constrained and masked by $\Psi_0$. Remove the constraint, and the continuous fabrication becomes phenomenologically dominant within minutes.

\subsection{The Cave as Permanent Condition}

\subsubsection{Plato's Metaphysical Interpretation}

Plato intended the cave as a metaphor for the human epistemic condition more generally: even when "awake," we perceive only shadows (sensory appearances) rather than Forms (true reality). Liberation requires philosophical enlightenment, not merely physical reorientation.

Our framework suggests that Plato was more correct than he knew, but in a different sense. Even during waking, with eyes open and $\Psi_0$ flowing, we do not perceive raw reality. We perceive a constructed model ($\Theta_0$) constrained by sensory data ($\Psi_0$). The constraint makes this a \textit{useful} model—it tracks external reality sufficiently for adaptive behavior—but it remains a model, a fabrication.

From this perspective:
\begin{itemize}
\item \textbf{Dreaming} is the cave at night: fabrication without constraint, pure shadow-world
\item \textbf{Waking} is the cave by daylight: fabrication constrained by external input, shadows that correspond to objects
\item \textbf{True liberation} would require accessing reality without any fabrication—likely impossible for biological systems
\end{itemize}

\subsubsection{The Degrees of Constraint}

We can understand cognitive states as lying on a continuum of external constraint:

\begin{table}[H]
\centering
\caption{Cognitive States as Degrees of Constraint}
\begin{tabular}{@{}llll@{}}
\toprule
\textbf{State} & \textbf{$\alpha$ (fabrication)} & \textbf{$\Psi_0$ status} & \textbf{Constraint level} \\
\midrule
Normal waking & 0.3-0.4 & High, processed well & Strong \\
Drowsy & 0.5-0.6 & Moderate, processing degraded & Moderate \\
Sleep deprived & 0.6-0.8 & High, processing impaired & Weak \\
Sensory deprivation & 0.7-0.9 & Low/absent & Minimal \\
Hypnagogic & 0.5-0.7 & Decreasing & Moderate \\
REM dreaming & 0.95-1.0 & Minimal (gated) & None/negligible \\
\bottomrule
\end{tabular}
\end{table}

Health requires cycling between constrained (waking) and unconstrained (dreaming) states. The danger lies in:
\begin{itemize}
\item Insufficient unconstrained time (sleep deprivation → system maintenance failure)
\item Excessive unconstrained time (coma, vegetative state → loss of reality tracking)
\item Intermediate states during waking (sleep deprivation, sensory deprivation → hallucinations)
\end{itemize}

\subsection{Summary and Transition}

We have established:

\begin{enumerate}
\item Dreaming is formally equivalent to Plato's cave—a state of pure or near-pure internal fabrication without external constraint

\item The impossibility of prisoner self-liberation maps to the impossibility of dreamer self-waking—both require external input providing information unavailable from internal processing

\item Sleep deprivation produces hallucinations by degrading the input filter, allowing fabrication to dominate during waking—demonstrating continuous need for constraint

\item Reality testing is fragile, requiring active maintenance through ongoing $\Psi_0$ input—removal produces rapid deterioration toward cave-like states

\item All cognitive states involve fabrication; they differ in degree of external constraint, not in presence/absence of internal generation
\end{enumerate}

This framework explains \textit{why} dreamers cannot wake themselves through internal decision alone: like Plato's prisoners, they lack the external information necessary to recognize their epistemic situation. But now we must address a modern challenge to this conclusion: reports of lucid dreaming, where individuals claim to recognize they are dreaming while remaining asleep. If verified, this would seem to violate the cave-dream equivalence.

In the next section, we examine a complementary thought experiment—Mizraji's prisoner parable—which formalizes the information requirements for self-liberation and provides tools for analyzing the lucid dreaming question.
