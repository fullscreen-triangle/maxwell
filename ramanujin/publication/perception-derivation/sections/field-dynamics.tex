\section{Electric Field Dynamics and Emotional Substrate}
\label{sec:field}

\subsection{The Imperceptibility Problem}

The preceding sections established the geometric structures of thought and perception within configuration space. However, a fundamental question remains: how does the system distinguish between configurations that are geometrically identical but semantically distinct? Two thoughts may occupy the same position in configuration space and yet mean entirely different things. The resolution requires introducing an additional structure that provides contextual identity to geometric configurations.

\begin{definition}[Configuration Equivalence]
\label{def:config_equiv}
Two configurations $\mathbf{c}_1, \mathbf{c}_2 \in \mathcal{M}$ are geometrically equivalent if:
\begin{equation}
d(\mathbf{c}_1, \mathbf{c}_2) < \varepsilon
\label{eq:geom_equiv}
\end{equation}
for threshold $\varepsilon > 0$ in the Fisher metric. Geometrically equivalent configurations may nonetheless be semantically distinct.
\end{definition}

The existence of geometrically equivalent but semantically distinct configurations implies that geometric structure alone is insufficient to characterize the content of thought. An additional degree of freedom must exist that distinguishes configurations even when their positions in configuration space coincide.

\subsection{Electric Field as Reality Approximation}

We propose that this additional structure is provided by the electric field generated by ionic distributions within the neural substrate. The electric field represents the system's best approximation of aspects of external reality that are inaccessible to direct measurement.

\begin{definition}[Neural Electric Field]
\label{def:electric_field}
The neural electric field $\mathbf{E}(\mathbf{r}, t)$ is the vector field generated by ionic charge distributions:
\begin{equation}
\mathbf{E}(\mathbf{r}, t) = -\nabla \phi(\mathbf{r}, t)
\label{eq:electric_field}
\end{equation}
where the potential $\phi$ satisfies the Poisson equation:
\begin{equation}
\nabla^2 \phi = -\frac{\rho}{\varepsilon_0 \varepsilon_r}
\label{eq:poisson}
\end{equation}
with charge density $\rho(\mathbf{r}, t)$ and relative permittivity $\varepsilon_r \approx 80$ for the aqueous neural environment.
\end{definition}

The characteristic magnitude of neural electric fields is substantial. Transmembrane potentials of approximately 70 mV across membranes of 5 nm thickness yield field strengths of $|\mathbf{E}| \approx 1.4 \times 10^7$ V/m. While highly localized, these fields collectively generate a spatially varying field landscape across the neural tissue.

\subsection{Proton Flux as Field Generator}

The electric field is ultimately generated by the movement of charged particles, principally protons (\ce{H+}) arising from metabolic processes and ionic pumping \citep{mitchell1961coupling}.

\begin{theorem}[Proton Flux Dynamics]
\label{thm:proton_flux}
The proton flux $\mathbf{J}_{\ce{H+}}$ satisfies the drift-diffusion equation:
\begin{equation}
\mathbf{J}_{\ce{H+}} = -D_{\ce{H+}} \nabla [\ce{H+}] + \mu_{\ce{H+}} [\ce{H+}] \mathbf{E}
\label{eq:proton_flux}
\end{equation}
where $D_{\ce{H+}} = 9.3 \times 10^{-9}$ m$^2$/s is the proton diffusion coefficient, $\mu_{\ce{H+}}$ is the proton mobility, and $[\ce{H+}]$ is the proton concentration.
\end{theorem}

The proton diffusion coefficient in aqueous solution is anomalously high due to the Grotthuss mechanism, whereby protons ``hop'' along hydrogen-bonded water networks rather than diffusing as conventional ions \citep{agmon1995grotthuss}. This rapid transport enables proton flux to generate field fluctuations on femtosecond timescales, far faster than neural firing timescales.

\begin{theorem}[Field Fluctuation Timescale]
\label{thm:field_timescale}
The characteristic timescale for electric field fluctuations is:
\begin{equation}
\tau_{\mathbf{E}} = \frac{L^2}{D_{\ce{H+}}} \approx 10^{-15} \text{ to } 10^{-12} \text{ s}
\label{eq:field_timescale}
\end{equation}
for length scales $L$ ranging from molecular ($\sim$1 nm) to nanoscale domains ($\sim$100 nm).
\end{theorem}

These femtosecond to picosecond fluctuations are imperceptible to the neural processing mechanisms, which operate on millisecond timescales. The system cannot directly measure the field but can only respond to its time-averaged effects.

\subsection{Variance Minimization in the Field}

The system's response to the imperceptible field fluctuations is mediated by variance minimization. The field configuration at any instant defines a landscape of thermodynamic stability, and the system evolves toward configurations that minimize variance within this landscape.

\begin{definition}[Field-Dependent Variance]
\label{def:field_variance}
The variance of a configuration $\mathbf{c}$ in the presence of field $\mathbf{E}$ is:
\begin{equation}
P(\mathbf{c}, \mathbf{E}) = P_0(\mathbf{c}) + \int_V \mathbf{c} \cdot \mathbf{E} \, dV
\label{eq:field_variance}
\end{equation}
where $P_0(\mathbf{c})$ is the field-independent variance and the integral represents field-configuration coupling.
\end{definition}

\begin{theorem}[Emergent Emotional Field]
\label{thm:emotional_field}
The time-averaged effect of field fluctuations on configuration variance defines an effective emotional field $\mathcal{E}$:
\begin{equation}
\mathcal{E}(\mathbf{c}) = \langle P(\mathbf{c}, \mathbf{E}) \rangle_t - P_0(\mathbf{c})
\label{eq:emotional_field}
\end{equation}
where $\langle \cdot \rangle_t$ denotes time averaging over the field fluctuation timescale $\tau_{\mathbf{E}}$.
\end{theorem}

The emotional field $\mathcal{E}$ is not a physical electric field but rather an effective field capturing the system's sensitivity to external conditions. It represents the organism's approximation of aspects of reality that cannot be directly perceived but that nonetheless influence the thermodynamic stability of internal configurations.

\subsection{Emotional Field Properties}

The emotional field exhibits characteristic properties that explain phenomenological observations about emotional experience.

\begin{theorem}[Emotional Field Continuity]
\label{thm:emotional_continuity}
The emotional field $\mathcal{E}$ is continuous with respect to configuration space:
\begin{equation}
|\mathcal{E}(\mathbf{c}_1) - \mathcal{E}(\mathbf{c}_2)| \leq L \cdot d(\mathbf{c}_1, \mathbf{c}_2)
\label{eq:emotional_lipschitz}
\end{equation}
for Lipschitz constant $L > 0$. That is, similar configurations experience similar emotional fields.
\end{theorem}

\begin{proof}
The time-averaged field effect depends on the coupling integral (Equation \ref{eq:field_variance}), which varies continuously with configuration. The Lipschitz bound follows from the boundedness of the electric field magnitude and the smoothness of configuration space.
\end{proof}

\begin{theorem}[Emotional Valence]
\label{thm:valence}
The emotional field can be decomposed into positive and negative components:
\begin{equation}
\mathcal{E} = \mathcal{E}^+ - \mathcal{E}^-
\label{eq:valence_decomp}
\end{equation}
where $\mathcal{E}^+ \geq 0$ corresponds to variance-reducing (pleasurable) field configurations and $\mathcal{E}^- \geq 0$ corresponds to variance-increasing (aversive) field configurations.
\end{theorem}

The valence decomposition explains the hedonic dimension of emotional experience. Configurations that reduce variance below baseline are experienced as positive (pleasurable), while configurations that increase variance above baseline are experienced as negative (aversive). The emotional field directs the system toward variance-minimizing configurations, implementing a hedonic gradient descent.

\subsection{Field-Configuration Coupling}

The emotional field does not merely label configurations but actively influences configuration dynamics through field-configuration coupling.

\begin{theorem}[Modified Dynamics]
\label{thm:modified_dynamics}
In the presence of the emotional field, the configuration dynamics (Equation \ref{eq:hole_langevin}) become:
\begin{equation}
\frac{d\mathbf{c}}{dt} = -\nabla_{\mathbf{c}} (F + \mathcal{E}) + \sqrt{2D} \, \boldsymbol{\xi}(t)
\label{eq:modified_langevin}
\end{equation}
The effective potential is $F_{\text{eff}} = F + \mathcal{E}$, combining thermodynamic free energy and emotional contribution.
\end{theorem}

This coupling has profound consequences for thought dynamics. A configuration that is thermodynamically stable (low $F$) but emotionally aversive (high $\mathcal{E}^-$) will be destabilized, driving the system away despite intrinsic stability. Conversely, a thermodynamically unstable configuration (high $F$) can be stabilized if it is emotionally attractive (high $\mathcal{E}^+$). This explains the phenomenon of emotionally driven cognition, whereby emotional salience can override purely logical considerations.

\subsection{Semantic Identity from Field Context}

We now resolve the configuration equivalence problem introduced at the beginning of this section. Geometrically equivalent configurations are distinguished by their position in the emotional field landscape.

\begin{theorem}[Semantic Identity]
\label{thm:semantic_identity}
Two configurations $\mathbf{c}_1, \mathbf{c}_2$ with $d(\mathbf{c}_1, \mathbf{c}_2) < \varepsilon$ are semantically distinct if and only if:
\begin{equation}
|\mathcal{E}(\mathbf{c}_1) - \mathcal{E}(\mathbf{c}_2)| > \delta
\label{eq:semantic_distinction}
\end{equation}
for emotional discrimination threshold $\delta > 0$.
\end{theorem}

The content of a thought is thus determined not only by its geometric structure but also by its emotional context. The same pattern of oscillatory holes, occurring in different emotional field configurations, constitutes different thoughts. This explains why memories of objectively similar events can have vastly different significance depending on emotional context during encoding and retrieval.

\begin{corollary}[Emotional Memory Modulation]
\label{cor:emotional_memory}
Memory encoding and retrieval are modulated by the emotional field:
\begin{equation}
M(\mathbf{c}, \mathcal{E}) = M_0(\mathbf{c}) \cdot f(\mathcal{E})
\label{eq:emotional_memory}
\end{equation}
where $M_0$ is baseline memory strength and $f(\mathcal{E})$ is the emotional modulation function, with $f > 1$ for high-valence (positive or negative) emotional states.
\end{corollary}

This result explains the enhanced memory for emotionally significant events, well-documented in psychological research \citep{mcgaugh2000memory,labar2006cognitive}. The emotional field provides the contextual scaffold that gives geometric configurations their semantic meaning.

