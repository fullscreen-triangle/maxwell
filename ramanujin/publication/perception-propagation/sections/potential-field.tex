\section{Potential Fields Over Categorical Space}

Potential fields over categorical space provide the mathematical structure that determines output states in bounded systems. States are determined by position in the potential field, not by input geometry.

\begin{definition}[Potential Field]
\label{def:potential-field}
A potential field over categorical space is a function $\Pfield: \Cspace \to \mathbb{R}$ that assigns a potential value to each categorical state $C_i \in \Cspace$. The potential field determines gradients that guide state transitions.
\end{definition}

\begin{definition}[Potential Field Gradient]
\label{def:potential-gradient}
The gradient of potential field $\Pfield$ at categorical state $C_i$ is:
\begin{equation}
\nabla \Pfield(C_i) = \left(\frac{\partial \Pfield}{\partial C_1}, \frac{\partial \Pfield}{\partial C_2}, \ldots\right)
\end{equation}
where partial derivatives are taken with respect to categorical dimensions.
\end{definition}

\begin{theorem}[State Determination by Potential Field]
\label{thm:state-determination}
For a bounded system $\mathcal{S}$, output state $s = f(I)$ is determined by the potential field position $\Pfield(\text{Category}(s))$ and its gradient $\nabla \Pfield(\text{Category}(s))$, not by input $I$ directly.
\end{theorem}

\begin{proof}
By the Sufficiency Principle (Theorem \ref{thm:sufficiency-principle}), output states are determined by whether they produce appropriate gradients for subsequent transitions.

The gradient $\nabla \Pfield(\text{Category}(s))$ determines:
\begin{itemize}
\item Direction of subsequent state transitions
\item Stability of current state
\item Accessibility of neighboring states
\end{itemize}

Therefore, state $s$ is determined by $\Pfield(\text{Category}(s))$ and $\nabla \Pfield(\text{Category}(s))$, not by input $I$. Multiple inputs $I_1, I_2, \ldots$ can produce the same categorical state $\text{Category}(s)$, and thus the same potential field position. $\square$
\end{proof}

\begin{definition}[Potential Field Attractor]
\label{def:attractor}
A potential field attractor is a categorical state $C^*$ where $\Pfield(C^*)$ is a local minimum, and $\nabla \Pfield(C^*) = 0$. States near $C^*$ are drawn toward it by the gradient.
\end{definition}

\begin{proposition}[Convergence to Attractors]
\label{prop:convergence}
Multiple input paths converge to the same attractor $C^*$ if they activate gradients pointing toward $C^*$. The convergence is path-independent: different inputs produce the same final state if they reach the same attractor.
\end{proposition}

\begin{proof}
Consider inputs $I_1, I_2$ producing outputs $s_1 = f(I_1)$, $s_2 = f(I_2)$ with categorical states $C_1 = \text{Category}(s_1)$, $C_2 = \text{Category}(s_2)$.

If both $C_1$ and $C_2$ are in the basin of attraction for $C^*$, then:
\begin{align}
\nabla \Pfield(C_1) &\to C^* \\
\nabla \Pfield(C_2) &\to C^*
\end{align}

Both paths converge to $C^*$ regardless of starting points $C_1, C_2$. Therefore, convergence is path-independent. $\square$
\end{proof}

\begin{definition}[Field Dynamics]
\label{def:field-dynamics}
Potential field dynamics describe how $\Pfield$ evolves over time:
\begin{equation}
\frac{d\Pfield}{dt} = \mathcal{F}(\Pfield, \text{inputs}, \text{system state})
\end{equation}
where $\mathcal{F}$ captures how external inputs and internal state modify the potential field.
\end{definition}

\begin{theorem}[Field Shaping]
\label{thm:field-shaping}
Potential fields are shaped by:
\begin{enumerate}
\item Historical completions (past categorical states modify field structure)
\item External inputs (sensory data when available)
\item System constraints (bounded capacity, energy limits)
\item Multi-agent interactions (consensus calibration, to be discussed)
\end{enumerate}
\end{theorem}

\begin{proof}
The potential field $\Pfield$ is not static but evolves based on:

\textbf{Historical completions:} Each completed categorical state $C_i$ modifies the field structure, creating gradients that favor or disfavor future completions based on past success.

\textbf{External inputs:} When available, sensory inputs provide constraints that shape the field, creating attractors aligned with external reality.

\textbf{System constraints:} Bounded capacity (Axiom \ref{ax:bounded-space}) limits field complexity. Energy constraints favor low-potential states.

\textbf{Multi-agent interactions:} When multiple agents share categorical space, their individual fields interact, creating consensus attractors (to be formalized in Section \ref{sec:consensus}).

Therefore, $\Pfield$ is dynamically shaped by these factors. $\square$
\end{proof}

\begin{corollary}[Intuition as Potential Field]
\label{cor:intuition}
The phenomenological experience of "intuition" or "what feels right" corresponds to following gradients in the potential field $\Pfield$. States with low potential (attractors) feel "right," while states requiring climbing potential barriers feel "wrong."
\end{corollary}

\begin{proof}
By Theorem \ref{thm:state-determination}, state transitions follow potential field gradients. Low-potential states are energetically favorable and naturally reached. High-potential states require energy input and feel effortful.

The phenomenological "feeling" of rightness corresponds to low potential (easy transitions), while wrongness corresponds to high potential (difficult transitions). Therefore, intuition is the experience of potential field gradients. $\square$
\end{proof}

Potential fields over categorical space provide the mathematical mechanism by which bounded systems determine output states. The field structure enables path independence: multiple inputs converge to the same field positions, producing identical outputs despite different input geometries.

