\section{The Sufficiency Principle}

The Sufficiency Principle establishes that output states in bounded systems are determined by functional adequacy rather than representational accuracy. This principle resolves the bounded-unbounded mismatch by replacing correspondence requirements with sufficiency requirements.

\begin{definition}[Sufficient State]
\label{def:sufficient-state}
A state $s \in H$ is \textit{sufficient} for purpose $P$ if it produces appropriate gradients in the potential field $\Pfield$ over categorical space $\Cspace$ that enable subsequent state transitions consistent with $P$, regardless of whether $s$ accurately represents external reality.
\end{definition}

\begin{theorem}[Sufficiency Principle]
\label{thm:sufficiency-principle}
For a bounded system $\mathcal{S} = (H, \mathcal{U}, f, \Cspace, \prec)$ with transformation function $f: \mathcal{U} \to H$, output states are determined by position in potential field $\Pfield: \Cspace \to \mathbb{R}$, not by input geometry. A state $s = f(I)$ is sufficient if $\Pfield(\text{Category}(s))$ produces appropriate gradients for subsequent transitions.
\end{theorem}

\begin{proof}
Assume output state $s$ must accurately represent input $I \in \mathcal{U}$. Since $|\mathcal{U}| = \infty$ (unbounded) and $|H| < \infty$ (bounded), the function $f: \mathcal{U} \to H$ cannot be injective. Therefore, multiple inputs $I_1 \neq I_2$ map to the same output $s = f(I_1) = f(I_2)$.

If accuracy were required, then $s$ would need to represent both $I_1$ and $I_2$ simultaneously, which is impossible for distinct $I_1, I_2$.

However, if sufficiency is required, then $s$ need only produce appropriate gradients $\nabla \Pfield(\text{Category}(s))$ that enable subsequent state transitions. Both $I_1$ and $I_2$ can be sufficient for the same purpose $P$ if they activate the same potential field gradient.

Therefore, output states are determined by potential field position, not input geometry. $\square$
\end{proof}

\begin{corollary}[Path Irrelevance]
\label{cor:path-irrelevance}
The path from input $I$ to output $s = f(I)$ is irrelevant. Only the final categorical state $\text{Category}(s)$ and its position in the potential field matter.
\end{corollary}

\begin{proof}
By Theorem \ref{thm:sufficiency-principle}, output states are determined by $\Pfield(\text{Category}(s))$, not by how $s$ was reached. Therefore, different paths $I_1 \to s$ and $I_2 \to s$ are equivalent if they produce the same categorical state and potential field position. $\square$
\end{proof}

\begin{proposition}[Sufficiency Over Accuracy]
\label{prop:sufficiency-over-accuracy}
Bounded systems operate on sufficiency, not accuracy. A state $s$ is valid if it is sufficient for producing appropriate subsequent states, regardless of whether it accurately represents external reality.
\end{proposition}

\begin{proof}
By Axiom \ref{ax:bounded-space}, $|H| < \infty$ while $|\mathcal{U}| = \infty$. Accurate representation would require $|H| \geq |\mathcal{U}|$, which is impossible.

However, sufficiency requires only that $s$ produces appropriate gradients $\nabla \Pfield(\text{Category}(s))$. This is achievable with finite $H$ because:
\begin{itemize}
\item Multiple inputs can activate the same gradient (many-to-one mapping)
\item Gradients are determined by categorical position, not input details
\item Finite categorical space $\Cspace$ with $|\Cspace| = \Nmax$ (finite but large) is sufficient
\end{itemize}

Therefore, bounded systems must operate on sufficiency, not accuracy. $\square$
\end{proof}

\begin{definition}[Functional Adequacy]
\label{def:functional-adequacy}
A state $s$ is \textit{functionally adequate} if it enables the system to continue operating successfully (producing appropriate subsequent states) despite potential disconnection from objective external reality.
\end{definition}

\begin{theorem}[Sufficiency Enables Functionality]
\label{thm:sufficiency-enables}
Sufficient states enable bounded systems to function successfully in unbounded reality spaces, even when states do not accurately represent external reality.
\end{theorem}

\begin{proof}
Consider a bounded system $\mathcal{S}$ operating in unbounded reality $\mathcal{U}$. By Proposition \ref{prop:sufficiency-over-accuracy}, $\mathcal{S}$ cannot achieve accurate representation.

However, if $\mathcal{S}$ operates on sufficiency:
\begin{enumerate}
\item Multiple inputs $I_1, I_2, \ldots$ can produce the same sufficient state $s$
\item State $s$ produces gradients $\nabla \Pfield(\text{Category}(s))$ that guide subsequent transitions
\item These gradients enable appropriate responses to future inputs
\item System continues operating successfully despite incomplete information
\end{enumerate}

Therefore, sufficiency enables functionality where accuracy is impossible. $\square$
\end{proof}

The Sufficiency Principle is the mathematical foundation for understanding how finite systems operate successfully in infinite reality spaces. It replaces the impossible requirement of accurate representation with the achievable requirement of functional adequacy.

