\section{Categorical States and Completion}

Categorical completion provides the mathematical structure for understanding how bounded systems process information through discrete state assignments in an ordered sequence.

\begin{definition}[Categorical Sequence]
\label{def:categorical-sequence}
A categorical sequence is an ordered set $\mathcal{C} = \{C_1, C_2, C_3, \ldots\}$ where each $C_i$ is a categorical state. The ordering relation $\prec$ satisfies:
\begin{itemize}
\item Transitivity: If $C_i \prec C_j$ and $C_j \prec C_k$, then $C_i \prec C_k$
\item Irreflexivity: $C_i \nprec C_i$ for all $i$
\item Totality: For any distinct $C_i, C_j$, either $C_i \prec C_j$ or $C_j \prec C_i$
\end{itemize}
\end{definition}

\begin{definition}[Categorical Completion]
\label{def:categorical-completion}
A categorical state $C_i$ is \textit{completed} when a physical event occupies that categorical position in the sequence $\mathcal{C}$. Completion is irreversible: once $C_i$ is completed, it cannot be re-occupied.
\end{definition}

\begin{theorem}[Categorical Irreversibility]
\label{thm:categorical-irreversibility}
Once a categorical state $C_i$ is completed, it cannot be re-occupied. If a system attempts to "re-think" or "re-process" a completed state, it must occupy a new categorical position $C_j$ with $C_i \prec C_j$.
\end{theorem}

\begin{proof}
Assume $C_i$ is completed at time $t_1$. By Definition \ref{def:categorical-completion}, this means a physical event occupied position $C_i$ in the sequence $\mathcal{C}$.

At time $t_2 > t_1$, if the system attempts to process "the same" information, it must:
\begin{enumerate}
\item Access the completed state $C_i$ (which is now in the past of the sequence)
\item Create a new processing event (which must occupy a new position in $\mathcal{C}$)
\end{enumerate}

Since the new processing event occurs at $t_2 > t_1$, and categorical ordering reflects temporal ordering, the new event must occupy $C_j$ with $C_i \prec C_j$.

Therefore, "re-thinking" $C_i$ actually creates a new categorical state $C_j$ that refers to $C_i$ but is distinct from it. $\square$
\end{proof}

\begin{corollary}[Temporal Direction from Categorical Ordering]
\label{cor:temporal-direction}
Categorical irreversibility creates temporal direction. The ordering $C_i \prec C_j$ provides an arrow of time: systems move forward through categorical space, never backward.
\end{corollary}

\begin{proof}
By Theorem \ref{thm:categorical-irreversibility}, completed states cannot be re-occupied. Therefore, the system must always move to new categorical positions. The ordering $\prec$ ensures this movement is unidirectional: if $C_i \prec C_j$, then $C_j \nprec C_i$ (irreflexivity and transitivity).

Therefore, categorical completion creates an inherent temporal direction. $\square$
\end{proof}

\begin{definition}[Categorical Completion Rate]
\label{def:completion-rate}
The categorical completion rate $\dot{C} = dC/dt$ is the number of categorical states completed per unit physical time. This rate determines the perceived temporal flow for bounded observers.
\end{definition}

\begin{theorem}[Time as Completion Rate]
\label{thm:time-completion-rate}
For a bounded observer system, perceived temporal duration $\Delta t_{\text{perceived}}$ is proportional to the number of categorical completions:
\begin{equation}
\Delta t_{\text{perceived}} \propto \int_{t_1}^{t_2} \dot{C}(\tau) \, d\tau = C(t_2) - C(t_1)
\end{equation}
\end{theorem}

\begin{proof}
Physical time $t$ progresses continuously. However, bounded observers process reality through discrete categorical assignments (Axiom \ref{ax:categorical-completion}).

Each categorical completion requires computational resources. With finite capacity (Axiom \ref{ax:bounded-space}), the completion rate is bounded: $\dot{C} \leq \dot{C}_{\text{max}} < \infty$.

Perceived temporal duration corresponds to the number of categorical assignments completed, not to physical time directly. Therefore:
\begin{equation}
\Delta t_{\text{perceived}} = \alpha \cdot (C(t_2) - C(t_1))
\end{equation}
where $\alpha$ is a proportionality constant relating categorical completions to temporal experience. $\square$
\end{proof}

\begin{proposition}[Categorical State Space]
\label{prop:categorical-space}
The categorical state space $\Cspace$ has cardinality $|\Cspace| = \Nmax$, where $\Nmax$ is the maximum categorical complexity from Theorem \ref{thm:recursive-enumeration}.
\end{proposition}

\begin{proof}
By Theorem \ref{thm:recursive-enumeration}, the maximum number of categorical distinctions is $\Nmax \approx (10^{84}) \uparrow\uparrow (10^{80})$. Each distinction corresponds to a categorical state position in $\mathcal{C}$.

Therefore, $|\Cspace| = \Nmax$. $\square$
\end{proof}

\begin{definition}[Categorical Equivalence]
\label{def:categorical-equivalence}
Two physical configurations are \textit{categorically equivalent} if they occupy the same categorical state $C_i$ despite being physically distinct. Formally, configurations $\omega_1$ and $\omega_2$ are equivalent if:
\begin{equation}
\text{Category}(\omega_1) = \text{Category}(\omega_2) = C_i
\end{equation}
\end{definition}

Categorical equivalence is central to path independence: multiple physically distinct inputs can produce the same categorical output, enabling the many-to-one mapping structure that characterizes bounded information systems.

