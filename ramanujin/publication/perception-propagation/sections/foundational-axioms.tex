\section{Foundational Axioms}

We establish the mathematical foundations through four axioms that define bounded information systems operating within unbounded reality spaces.

\begin{axiom}[Bounded Space]
\label{ax:bounded-space}
For any observer system $\mathcal{S}$, there exists a finite bounded space $H$ such that all computational operations and state representations are constrained to $H$. Formally, for any state $s$ accessible to $\mathcal{S}$, we have $s \in H$ where $|H| < \infty$.
\end{axiom}

\begin{axiom}[Partition Depth]
\label{ax:partition-depth}
Any bounded system $\mathcal{S}$ must partition reality into observer-environment distinction. This partition creates a boundary $\partial H$ separating internal states $H$ from external reality $\mathcal{U}$, where $\mathcal{U}$ is unbounded ($|\mathcal{U}| = \infty$ or $|\mathcal{U}| \gg |H|$).
\end{axiom}

\begin{axiom}[Categorical Completion]
\label{ax:categorical-completion}
States in bounded systems are categorical completions—discrete assignments to ordered positions in a categorical sequence $\mathcal{C} = \{C_1, C_2, C_3, \ldots\}$ where $C_i \prec C_j$ indicates $C_i$ was completed before $C_j$. Once a categorical state $C_i$ is completed, it cannot be re-occupied (categorical irreversibility).
\end{axiom}

\begin{axiom}[Oscillatory Substrate]
\label{ax:oscillatory-substrate}
Information processing occurs through oscillatory dynamics—phase-locked configurations that can be completed by multiple categorically equivalent but physically distinct arrangements. Phase-lock degeneracy means that $\sim 10^6$ different weak force configurations can produce identical oscillatory signatures.
\end{axiom}

These axioms establish the mathematical constraints under which bounded systems operate. Axiom \ref{ax:bounded-space} ensures finite capacity, Axiom \ref{ax:partition-depth} establishes the observer-environment distinction, Axiom \ref{ax:categorical-completion} provides temporal structure through categorical ordering, and Axiom \ref{ax:oscillatory-substrate} enables the path independence that will be central to our framework.

\begin{definition}[Bounded Information System]
A bounded information system is a tuple $\mathcal{S} = (H, \mathcal{U}, f, \Cspace, \prec)$ where:
\begin{itemize}
\item $H$ is a finite bounded space (Axiom \ref{ax:bounded-space})
\item $\mathcal{U}$ is an unbounded input space with $\partial H$ as partition boundary (Axiom \ref{ax:partition-depth})
\item $f: \mathcal{U} \to H$ is a transformation function mapping inputs to outputs
\item $\Cspace = \{C_1, C_2, \ldots\}$ is a categorical sequence (Axiom \ref{ax:categorical-completion})
\item $\prec$ is an ordering relation on $\Cspace$ (Axiom \ref{ax:categorical-completion})
\end{itemize}
\end{definition}

\begin{proposition}[Necessity of Boundedness]
\label{prop:necessity-boundedness}
Any system that processes information must be bounded. If $|H| = \infty$, then the system cannot be physically realized or computationally implemented.
\end{proposition}

\begin{proof}
Assume $|H| = \infty$. Then either:
\begin{enumerate}
\item The system requires infinite storage, which violates physical finiteness
\item The system requires infinite computation time per operation, which violates temporal finiteness
\item The system cannot be specified finitely, which violates constructibility
\end{enumerate}
All three cases contradict the definition of a realizable system. Therefore, $|H| < \infty$ necessarily. $\square$
\end{proof}

\begin{proposition}[Partition Necessity]
\label{prop:partition-necessity}
The observer-environment partition is geometrically necessary. Without $\partial H$, there is no distinction between system and environment, hence no information processing can occur.
\end{proposition}

\begin{proof}
Information processing requires:
\begin{enumerate}
\item A system that processes (internal to $H$)
\item Information to process (external from $\mathcal{U}$)
\item A mechanism to transfer information across boundary $\partial H$
\end{enumerate}
If $\partial H$ does not exist, then $H = \mathcal{U}$, meaning the system and environment are identical. In this case, there is no information to process (no distinction between system state and environment state), and no processing can occur. Therefore, $\partial H$ is necessary. $\square$
\end{proof}

These foundational axioms and propositions establish the mathematical framework within which all subsequent results will be derived. The bounded-unbounded mismatch ($|H| < \infty$ but $|\mathcal{U}| = \infty$) creates the fundamental problem that sufficiency-based processing solves.

