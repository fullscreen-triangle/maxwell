\section{Path Independence and Infinite Substitutability}

The Path Independence Theorem establishes that for any output state, there exists an infinite set of input configurations that produce it. This many-to-one mapping structure is fundamental to bounded information systems.

\begin{theorem}[Path Independence]
\label{thm:path-independence}
For any output state $O \in H$ in a bounded system $\mathcal{S} = (H, \mathcal{U}, f, \Cspace, \prec)$ with transformation function $f: \mathcal{U} \to H$, there exists an infinite set of input configurations $\{I_1, I_2, I_3, \ldots\}$ such that:
\begin{equation}
f(I_i) = O \quad \text{for all } i \in \mathbb{N}
\end{equation}
\end{theorem}

\begin{proof}
By Axiom \ref{ax:bounded-space}, $|H| < \infty$. By Axiom \ref{ax:partition-depth}, $|\mathcal{U}| = \infty$ (unbounded input space).

Since $f: \mathcal{U} \to H$ maps infinite space to finite space, $f$ cannot be injective. Therefore, for any $O \in H$, the preimage $f^{-1}(O) = \{I \in \mathcal{U} : f(I) = O\}$ must be infinite.

Formally, if $|f^{-1}(O)| < \infty$ for all $O \in H$, then:
\begin{equation}
|\mathcal{U}| = \sum_{O \in H} |f^{-1}(O)| \leq |H| \cdot \max_{O \in H} |f^{-1}(O)| < \infty
\end{equation}
contradicting $|\mathcal{U}| = \infty$.

Therefore, there exists at least one $O \in H$ such that $|f^{-1}(O)| = \infty$. In fact, by the pigeonhole principle, most outputs have infinite preimages. $\square$
\end{proof}

\begin{corollary}[Infinite Substitutability]
\label{cor:infinite-substitutability}
For any output state $O$, inputs are infinitely substitutable: any $I_i \in f^{-1}(O)$ can be replaced by any $I_j \in f^{-1}(O)$ without changing the output.
\end{corollary}

\begin{proof}
By Theorem \ref{thm:path-independence}, $f^{-1}(O)$ is infinite. For any $I_i, I_j \in f^{-1}(O)$, we have $f(I_i) = O = f(I_j)$.

Therefore, $I_i$ and $I_j$ are substitutable: they produce identical outputs despite being distinct inputs. $\square$
\end{proof}

\begin{definition}[Output Indeterminacy]
\label{def:output-indeterminacy}
Output indeterminacy is the property that output states cannot be reverse-mapped to unique inputs. Given output $O$, the input $I$ such that $f(I) = O$ cannot be uniquely determined.
\end{definition}

\begin{theorem}[Reverse Mapping Impossibility]
\label{thm:reverse-mapping}
For bounded systems with $|\mathcal{U}| = \infty$ and $|H| < \infty$, reverse mapping from outputs to inputs is impossible. Given output $O$, there exist infinitely many inputs $I$ such that $f(I) = O$.
\end{theorem}

\begin{proof}
By Theorem \ref{thm:path-independence}, for any $O \in H$, the set $f^{-1}(O)$ is infinite. Therefore, given $O$, we cannot uniquely determine which $I \in f^{-1}(O)$ produced it.

The function $f^{-1}: H \to \mathcal{P}(\mathcal{U})$ (where $\mathcal{P}(\mathcal{U})$ is the power set) maps each output to an infinite set of possible inputs. This is not a function from $H$ to $\mathcal{U}$, so reverse mapping is impossible. $\square$
\end{proof}

\begin{example}[The Zoo Scenario]
\label{ex:zoo-scenario}
Consider three agents $\{A_1, A_2, A_3\}$ with transformation functions $\{f_1, f_2, f_3\}$ mapping inputs to outputs. Let $O$ be the output state "initiate flight response."

Agent $A_1$ receives input $I_1$ (visual pattern: "cat-like predator"). Agent $A_2$ receives input $I_2$ (learned category: "lion"). Agent $A_3$ receives input $I_3$ (social signal: "others running").

By the Path Independence Theorem:
\begin{align}
f_1(I_1) &= O \\
f_2(I_2) &= O \\
f_3(I_3) &= O
\end{align}
where $I_1 \neq I_2 \neq I_3$ are categorically distinct inputs, yet all produce identical output $O$.

This demonstrates infinite substitutability: inputs $I_1$, $I_2$, $I_3$, and infinitely many others (stock market crash, world cup win, Fields medal announcement, etc.) all produce the same output $O$.
\end{example}

\begin{proposition}[Output Geometry Independence]
\label{prop:output-geometry}
Output states are independent of input geometry. The same output can be produced by inputs with completely different geometric structures, physical configurations, or categorical origins.
\end{proposition}

\begin{proof}
By Theorem \ref{thm:path-independence}, for any output $O$, there exist infinitely many inputs $\{I_1, I_2, \ldots\}$ such that $f(I_i) = O$.

These inputs can have:
\begin{itemize}
\item Different geometric structures (spatial arrangements)
\item Different physical configurations (molecular compositions)
\item Different categorical origins (sensory, learned, social, abstract)
\item Different temporal sequences
\end{itemize}

Yet all produce the same output $O$. Therefore, output is independent of input geometry. $\square$
\end{proof}

\begin{theorem}[Finite Output Modes]
\label{thm:finite-output-modes}
Bounded systems have finite output modes. While inputs are infinitely substitutable, outputs are constrained to a finite set $H$ with $|H| < \infty$.
\end{theorem}

\begin{proof}
By Axiom \ref{ax:bounded-space}, $|H| < \infty$. Therefore, the set of possible outputs is finite.

While $|\mathcal{U}| = \infty$ (infinite input variety), the system can only produce $|H|$ distinct outputs. This creates the many-to-one mapping structure: infinite inputs map to finite outputs. $\square$
\end{proof}

Path independence and infinite substitutability are mathematical necessities for bounded systems operating in unbounded reality spaces. They enable the flexibility and adaptability that characterize successful information processing despite finite computational capacity.

