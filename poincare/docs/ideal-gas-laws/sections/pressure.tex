\section{Pressure: Categorical Density}
\label{sec:pressure}

\subsection{Classical Pressure and Its Problems}

Classical kinetic theory derives pressure from momentum transfer at container walls:
\begin{equation}
P_{\text{classical}} = \frac{1}{3}\rho\langle v^2 \rangle = \frac{Nk_B T}{V}
\end{equation}

This definition has fundamental problems:

\textbf{Problem 1: Boundary localization.} The derivation assumes pressure arises from wall collisions. Yet we measure pressure in the bulk of fluids---deep ocean pressure, atmospheric pressure at altitude, pressure inside stars. How can a boundary phenomenon determine a bulk property?

\textbf{Problem 2: Circular definition with temperature.} Since $T \propto \langle v^2 \rangle$, the relation $P \propto T$ becomes $P \propto \langle v^2 \rangle$---a tautology rather than physics.

\textbf{Problem 3: No explanation of extensivity.} Why does pressure scale as $N/V$? What makes it an intensive variable?

The triple equivalence resolves these by defining pressure as categorical density---an intrinsic bulk property.

\subsection{Categorical Pressure}

Pressure measures the density of categorical distinctions---how many categories are packed into a given volume:
\begin{equation}
\boxed{P_{\text{cat}} = k_B T \left(\frac{\partial M}{\partial V}\right)_S}
\label{eq:categorical-pressure}
\end{equation}

\textbf{Physical interpretation:} Compressing a gas (decreasing $V$) forces the same number of categories into smaller volume. The system resists this compression---this resistance is pressure.

\subsubsection{Bulk Property, Not Boundary}

The categorical density $\partial M/\partial V$ exists throughout the volume, not just at boundaries. Wall collisions are one \textit{manifestation} of categorical density, not its \textit{definition}.

Consider a point in the bulk of a gas. The categorical density at that point is:
\begin{equation}
\rho_M = \frac{dM}{dV}
\end{equation}

This is the local number of categorical distinctions per unit volume. Pressure is $k_B T$ times this density.

\subsubsection{Derivation from Entropy}

From the thermodynamic relation:
\begin{equation}
P = -\left(\frac{\partial F}{\partial V}\right)_T = -\frac{\partial}{\partial V}(U - TS)
\end{equation}

Using $S = k_B M \ln n$ and assuming $U$ is volume-independent (ideal gas):
\begin{equation}
P = T\left(\frac{\partial S}{\partial V}\right)_T = k_B T \ln n \left(\frac{\partial M}{\partial V}\right)_T
\end{equation}

For $\ln n = 1$ (one nat per category):
\begin{equation}
P = k_B T \left(\frac{\partial M}{\partial V}\right)_T
\end{equation}

\subsubsection{Scaling of Categories with Volume}

For an ideal gas, the number of accessible spatial categories scales as:
\begin{equation}
M = \alpha \frac{V}{V_0}
\end{equation}

where $V_0$ is the elementary volume (thermal de Broglie wavelength cubed) and $\alpha$ is a constant.

Thus:
\begin{equation}
\frac{\partial M}{\partial V} = \frac{\alpha}{V_0} = \frac{M}{V}
\end{equation}

But for $N$ particles each contributing categories:
\begin{equation}
M = N \cdot m
\end{equation}

where $m$ is categories per particle. The derivative becomes:
\begin{equation}
\frac{\partial M}{\partial V} = \frac{N}{V} \cdot \frac{\partial m}{\partial \ln V}
\end{equation}

For translational modes, $m \propto \ln V$, giving $\partial m/\partial\ln V = 1$:
\begin{equation}
\frac{\partial M}{\partial V} = \frac{N}{V}
\end{equation}

Therefore:
\begin{equation}
P = k_B T \cdot \frac{N}{V}
\end{equation}

This is the ideal gas law, derived from categorical density.

\subsection{Oscillatory Pressure}

In the oscillatory perspective, pressure arises from the spatial amplitude of oscillations. Particles execute oscillations with amplitudes $A_i$; these amplitudes push against boundaries.
\begin{equation}
\boxed{P_{\text{osc}} = \frac{1}{3V}\sum_i m_i \omega_i^2 A_i^2}
\label{eq:oscillatory-pressure}
\end{equation}

\textbf{Physical interpretation:} Each oscillator exerts a restoring force $F = m\omega^2 A$. The pressure is the average force per unit area from all oscillators.

\subsubsection{Derivation from Virial Theorem}

For a system of particles in a container, the virial theorem states:
\begin{equation}
PV = \frac{2}{3}\langle E_k \rangle_{\text{total}} = \frac{1}{3}\sum_i m_i \langle v_i^2 \rangle
\end{equation}

For harmonic oscillators, $\langle v^2 \rangle = \omega^2 A^2$:
\begin{equation}
PV = \frac{1}{3}\sum_i m_i \omega_i^2 A_i^2
\end{equation}

Dividing by $V$:
\begin{equation}
P = \frac{1}{3V}\sum_i m_i \omega_i^2 A_i^2
\end{equation}

\subsubsection{Connection to Thermal Energy}

For oscillators in thermal equilibrium:
\begin{equation}
\frac{1}{2}m\omega^2 A^2 = \frac{1}{2}k_B T
\end{equation}

Thus:
\begin{equation}
m\omega^2 A^2 = k_B T
\end{equation}

Summing over $N$ oscillators:
\begin{equation}
\sum_i m_i \omega_i^2 A_i^2 = N k_B T
\end{equation}

Substituting:
\begin{equation}
P = \frac{N k_B T}{3V} \times 3 = \frac{N k_B T}{V}
\end{equation}

(The factor of 3 accounts for three spatial dimensions.)

\subsection{Partition Pressure}

In the partition perspective, pressure arises from the rate of boundary-crossing partitions:
\begin{equation}
\boxed{P_{\text{part}} = \frac{k_B T}{V} \sum_a \frac{1}{\tau_{p,a}^{\text{boundary}}}}
\label{eq:partition-pressure}
\end{equation}

\textbf{Physical interpretation:} Faster boundary crossings (shorter partition lags at boundaries) mean higher pressure. The system transitions rapidly at the boundary, exerting force.

\subsubsection{Derivation from Transition Rates}

The rate at which particles cross a boundary element $dA$ is:
\begin{equation}
\Phi = n \langle v \rangle / 4
\end{equation}

where $n = N/V$ is number density and $\langle v \rangle$ is mean speed.

Each crossing is a partition with lag:
\begin{equation}
\tau_p^{\text{boundary}} = \frac{1}{\Phi \cdot dA}
\end{equation}

The pressure is momentum flux:
\begin{equation}
P = \frac{\text{momentum transferred}}{\text{area} \times \text{time}} = n m \langle v^2 \rangle / 3
\end{equation}

In partition terms:
\begin{equation}
P = \frac{k_B T}{V} \cdot N = k_B T \cdot n
\end{equation}

This is consistent with $\sum_a 1/\tau_p^{\text{boundary}} = N$ for an ideal gas.

\subsection{Equivalence of Three Definitions}

\textbf{Theorem.} The three pressure definitions are equivalent for ideal gases:
\begin{equation}
P_{\text{cat}} = P_{\text{osc}} = P_{\text{part}} = \frac{Nk_B T}{V}
\end{equation}

\textbf{Proof:}

\textit{Categorical:}
\begin{equation}
P_{\text{cat}} = k_B T \left(\frac{\partial M}{\partial V}\right)_S = k_B T \cdot \frac{N}{V}
\end{equation}

\textit{Oscillatory:}
\begin{equation}
P_{\text{osc}} = \frac{1}{3V}\sum_i m_i \omega_i^2 A_i^2 = \frac{N k_B T}{V}
\end{equation}

\textit{Partition:}
\begin{equation}
P_{\text{part}} = \frac{k_B T}{V} \cdot N = \frac{N k_B T}{V}
\end{equation}

All three give the ideal gas result. $\square$

\subsection{Pressure as Intensive Variable}

The categorical perspective explains why pressure is intensive. The categorical density:
\begin{equation}
\rho_M = \frac{M}{V} \propto \frac{N}{V}
\end{equation}

is intensive (doubling both $N$ and $V$ leaves $\rho_M$ unchanged). Since $P = k_B T \cdot \rho_M$, pressure inherits intensivity.

\subsection{Pressure Saturation}

At extremely high density, all available categories become occupied:
\begin{equation}
M \to M_{\max}
\end{equation}

The categorical density saturates:
\begin{equation}
\frac{\partial M}{\partial V} \to 0 \quad \text{as} \quad M \to M_{\max}
\end{equation}

This predicts pressure saturation at extreme densities (e.g., nuclear matter). The pressure cannot increase indefinitely because there are no more categories to compress.

\subsection{Summary}

Pressure admits three equivalent definitions:
\begin{align}
P_{\text{cat}} &= k_B T \left(\frac{\partial M}{\partial V}\right)_S \quad \text{(categorical density)} \\
P_{\text{osc}} &= \frac{1}{3V}\sum_i m_i \omega_i^2 A_i^2 \quad \text{(oscillation amplitude)} \\
P_{\text{part}} &= \frac{k_B T}{V} \sum_a \frac{1}{\tau_p^{\text{boundary}}} \quad \text{(boundary partition rate)}
\end{align}

All three:
\begin{enumerate}
\item Are bulk properties (not localized at boundaries)
\item Explain why $P \propto N/V$ (categorical density)
\item Give correct ideal gas law
\item Predict pressure saturation at extreme density
\end{enumerate}

