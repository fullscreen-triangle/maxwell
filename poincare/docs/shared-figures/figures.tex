% Shared Figures for Mathematical Prerequisites
% These figures validate the triple equivalence (Oscillation ≡ Category ≡ Partition),
% entropy derivation, and categorical enthalpy.
%
% To use in a paper, add: % Figures for Resolution of Loschmidt's Paradox
% Include this file in the main document or use % Figures for Resolution of Loschmidt's Paradox
% Include this file in the main document or use % Figures for Resolution of Loschmidt's Paradox
% Include this file in the main document or use \input{figures.tex}

%==============================================================================
% Figure L-1: Mixing-Separation Entropy Cycle
%==============================================================================
\begin{figure}[htbp]
\centering
\includegraphics[width=\textwidth]{figures/panel_mixing_separation.pdf}
\caption{\textbf{Mixing-Separation Cycle Demonstrates Irreversibility.}
(A) Initial state: two gases separated by partition, with entropy $S_{initial} = S_A^{(0)} + S_B^{(0)}$.
(B) Mixed state: partition removed, gases interdiffuse, entropy increases by $\Delta S_{mix}$.
(C) Re-separated state: partition restored, but each container now contains both gases with residual phase correlations.
(D) Entropy evolution: the categorical prediction (green) shows $S_{final} > S_{initial}$ despite identical spatial configuration, while the classical reversible prediction (gray) incorrectly predicts return to initial entropy. The difference $\Delta S_{irrev} > 0$ arises from phase-lock network densification that persists after re-separation.}
\label{fig:mixing_separation}
\end{figure}

%==============================================================================
% Figure L-2: Phase-Lock Network Evolution
%==============================================================================
\begin{figure}[htbp]
\centering
\includegraphics[width=\textwidth]{figures/panel_phase_lock_network.pdf}
\caption{\textbf{Phase-Lock Network Densification and Residual Correlations.}
(A) Initial separated state: two disconnected network clusters (blue = Gas A, red = Gas B) with $|E|$ internal edges.
(B) Mixed state: networks merge into single connected component with cross-container edges.
(C) Re-separated state: partition restored, but residual cross-edges (red dashed) persist---these represent phase correlations created during mixing that cannot be erased.
(D) Edge count evolution: $|E_{final}| > |E_{initial}|$ demonstrates that mixing creates categorical structure (edges) that remains after re-separation. More edges means more constraints, hence higher entropy.}
\label{fig:phase_lock_network}
\end{figure}

%==============================================================================
% Figure L-3: Non-Actualisation Asymmetry
%==============================================================================
\begin{figure}[htbp]
\centering
\includegraphics[width=\textwidth]{figures/panel_non_actualisation.pdf}
\caption{\textbf{Non-Actualisation Asymmetry---The Deepest Reason for Irreversibility.}
(A) The cup example: when a cup falls and breaks, it generates infinitely many non-actualisations (not turning to gold, not becoming sentient, not teleporting, etc.)---categorical facts defined by negation.
(B) Branching asymmetry: each actualisation (green, finite) creates infinitely many non-actualisations (red), yielding a 1:$\infty$ asymmetry ratio.
(C) Accumulation over time: non-actualisations grow monotonically and cannot be un-created, while actualisations remain finite.
(D) Forward/backward asymmetry: forward processes always possible (create non-actualisations), backward processes impossible (would require un-creating non-actualisations). The ratio $P_{forward}/P_{backward} \to \infty$.}
\label{fig:non_actualisation}
\end{figure}

%==============================================================================
% Figure L-4: Aperture Selectivity and Categorical Potential
%==============================================================================
\begin{figure}[htbp]
\centering
\includegraphics[width=\textwidth]{figures/panel_aperture_selectivity.pdf}
\caption{\textbf{Partition Boundaries as Categorical Apertures.}
(A) Selection function $\sigma(\omega)$: aperture (partition boundary) allows certain configurations to pass ($\sigma = 1$, green arrows) while blocking others ($\sigma = 0$, red X marks).
(B) Categorical potential vs selectivity: $\Phi_a = -k_B T \ln s$ where $s = \Omega_{pass}/\Omega_{total}$. High selectivity ($s \to 0$) implies high potential barrier.
(C) Entropy from selectivity: higher selectivity (lower $s$) produces more entropy, since $\Delta S = k_B \ln(1/s) = \Phi_a/T$.
(D) Aperture as energy barrier: the categorical potential acts as a barrier that blocked configurations must overcome. Non-actualisations are precisely the configurations blocked by partition apertures.}
\label{fig:aperture_selectivity}
\end{figure}

%==============================================================================
% Figure L-5: Partition Lag Dynamics
%==============================================================================
\begin{figure}[htbp]
\centering
\includegraphics[width=\textwidth]{figures/panel_partition_lag.pdf}
\caption{\textbf{Partition Lag---The Finite Time of Categorical Determination.}
(A) Partition lag distribution: different systems exhibit different lag time distributions, with a fundamental minimum $\tau_{min} = \hbar/\Delta E$ set by the uncertainty principle.
(B) Undetermined residue evolution: during partition lag, categorical states remain in superposition. The determination fraction approaches 1 asymptotically, with residue decreasing exponentially.
(C) Entropy production rate: entropy is produced continuously during partition lag at rate $dS/dt = k_B \cdot \text{Residue}/\tau_{lag}$. Cumulative entropy $S(t)$ saturates as determination completes.
(D) Minimum lag scaling: $\tau_{min} \propto 1/\Delta E$ across different energy scales (phonon, vibrational, electronic, core), demonstrating the fundamental quantum limit on partition speed.}
\label{fig:partition_lag}
\end{figure}

%==============================================================================
% Figure L-6: Termination and Irreversibility
%==============================================================================
\begin{figure}[htbp]
\centering
\includegraphics[width=\textwidth]{figures/panel_termination_irreversibility.pdf}
\caption{\textbf{Termination, Completion, and the Impossibility of Reversal.}
(A) Reality stream vs terminated state: ongoing processes (blue wave) have indeterminate entropy---they are superpositions of possibilities. Only terminated states (green box) have well-defined entropy as categorical facts.
(B) Identity of completion and partitioning: categorical completion (selecting one outcome) is identical to geometric partitioning (creating boundaries). Both create the distinction between actualised and non-actualised states.
(C) Why reversal fails: forward processes create non-actualisations ($A \to B$ plus infinitely many things $B$ is not doing). Backward would require un-creating these---impossible.
(D) Asymmetry ratio growth: with each categorical completion, the forward/backward probability ratio grows exponentially as $\prod_i \Omega_i$, rapidly diverging from the reversible ratio of 1.}
\label{fig:termination_irreversibility}
\end{figure}

%==============================================================================
% Figure L-7: Cross-Sectional Validation of Irreversibility
%==============================================================================
\begin{figure}[htbp]
\centering
\includegraphics[width=\textwidth]{figures/panel_loschmidt_cross_sectional_validation.pdf}
\caption{\textbf{Cross-Sectional Validation: Radial Expansion and the Arrow of Time.}
(A) S-coordinate evolution with radius: configuration entropy $S_k$ and temporal entropy $S_t$ plotted versus radial distance from an expanding point for three systems (Fast, Medium, Slow Expansion). Each radial shell is a cross-sectional measurement---a spherical surface at fixed distance. All systems show monotonic increase: entropy ALWAYS grows outward.
(B) Non-actualisations dominate: logarithmic plot of actualised (dashed) vs non-actualised (solid) state counts. Non-actualisations ($N_{\text{non-act}} \propto r^2 - \omega(t)r^2/4\pi$) vastly outnumber actualisations at all radii, with ratios from 37:1 (fast expansion) to 392:1 (slow expansion). This asymmetry is the origin of irreversibility.
(C) Entropy gradient always positive: the derivative $\partial S_t/\partial r > 0$ at ALL radii for ALL systems. Green region indicates positive (irreversible) gradients; no system ever enters the negative (reversible) region. The gradient points outward because non-actualisations accumulate faster than actualisations can explore.
(D) Irreversibility fraction: bar chart showing 100\% positive gradients for all three expansion regimes. The 50\% line (gray dashed) marks the threshold for reversible processes; 100\% marks complete irreversibility. All systems achieve 100\%, confirming that the arrow of time is universal and independent of expansion rate.
(E) S-transformation validation: predicted S-coordinates (from $\mathcal{T}_{dr}$) versus calculated values, showing $R^2 > 0.99$ for all systems. The transformation correctly predicts entropy at each radial shell from the previous shell's state.
(F) Schematic of expanding point: central point (black dot) expands into state space, creating concentric spherical shells (color gradient from green/low entropy to red/high entropy). Arrows show outward expansion direction. The gradient $\nabla S > 0$ always points away from the origin---this is the geometric necessity of irreversibility. Non-actualisations form a ``wake'' around the actualised trajectory, and this wake grows with distance, making reversal impossible.}
\label{fig:loschmidt_cross_sectional_validation}
\end{figure}



%==============================================================================
% Figure L-1: Mixing-Separation Entropy Cycle
%==============================================================================
\begin{figure}[htbp]
\centering
\includegraphics[width=\textwidth]{figures/panel_mixing_separation.pdf}
\caption{\textbf{Mixing-Separation Cycle Demonstrates Irreversibility.}
(A) Initial state: two gases separated by partition, with entropy $S_{initial} = S_A^{(0)} + S_B^{(0)}$.
(B) Mixed state: partition removed, gases interdiffuse, entropy increases by $\Delta S_{mix}$.
(C) Re-separated state: partition restored, but each container now contains both gases with residual phase correlations.
(D) Entropy evolution: the categorical prediction (green) shows $S_{final} > S_{initial}$ despite identical spatial configuration, while the classical reversible prediction (gray) incorrectly predicts return to initial entropy. The difference $\Delta S_{irrev} > 0$ arises from phase-lock network densification that persists after re-separation.}
\label{fig:mixing_separation}
\end{figure}

%==============================================================================
% Figure L-2: Phase-Lock Network Evolution
%==============================================================================
\begin{figure}[htbp]
\centering
\includegraphics[width=\textwidth]{figures/panel_phase_lock_network.pdf}
\caption{\textbf{Phase-Lock Network Densification and Residual Correlations.}
(A) Initial separated state: two disconnected network clusters (blue = Gas A, red = Gas B) with $|E|$ internal edges.
(B) Mixed state: networks merge into single connected component with cross-container edges.
(C) Re-separated state: partition restored, but residual cross-edges (red dashed) persist---these represent phase correlations created during mixing that cannot be erased.
(D) Edge count evolution: $|E_{final}| > |E_{initial}|$ demonstrates that mixing creates categorical structure (edges) that remains after re-separation. More edges means more constraints, hence higher entropy.}
\label{fig:phase_lock_network}
\end{figure}

%==============================================================================
% Figure L-3: Non-Actualisation Asymmetry
%==============================================================================
\begin{figure}[htbp]
\centering
\includegraphics[width=\textwidth]{figures/panel_non_actualisation.pdf}
\caption{\textbf{Non-Actualisation Asymmetry---The Deepest Reason for Irreversibility.}
(A) The cup example: when a cup falls and breaks, it generates infinitely many non-actualisations (not turning to gold, not becoming sentient, not teleporting, etc.)---categorical facts defined by negation.
(B) Branching asymmetry: each actualisation (green, finite) creates infinitely many non-actualisations (red), yielding a 1:$\infty$ asymmetry ratio.
(C) Accumulation over time: non-actualisations grow monotonically and cannot be un-created, while actualisations remain finite.
(D) Forward/backward asymmetry: forward processes always possible (create non-actualisations), backward processes impossible (would require un-creating non-actualisations). The ratio $P_{forward}/P_{backward} \to \infty$.}
\label{fig:non_actualisation}
\end{figure}

%==============================================================================
% Figure L-4: Aperture Selectivity and Categorical Potential
%==============================================================================
\begin{figure}[htbp]
\centering
\includegraphics[width=\textwidth]{figures/panel_aperture_selectivity.pdf}
\caption{\textbf{Partition Boundaries as Categorical Apertures.}
(A) Selection function $\sigma(\omega)$: aperture (partition boundary) allows certain configurations to pass ($\sigma = 1$, green arrows) while blocking others ($\sigma = 0$, red X marks).
(B) Categorical potential vs selectivity: $\Phi_a = -k_B T \ln s$ where $s = \Omega_{pass}/\Omega_{total}$. High selectivity ($s \to 0$) implies high potential barrier.
(C) Entropy from selectivity: higher selectivity (lower $s$) produces more entropy, since $\Delta S = k_B \ln(1/s) = \Phi_a/T$.
(D) Aperture as energy barrier: the categorical potential acts as a barrier that blocked configurations must overcome. Non-actualisations are precisely the configurations blocked by partition apertures.}
\label{fig:aperture_selectivity}
\end{figure}

%==============================================================================
% Figure L-5: Partition Lag Dynamics
%==============================================================================
\begin{figure}[htbp]
\centering
\includegraphics[width=\textwidth]{figures/panel_partition_lag.pdf}
\caption{\textbf{Partition Lag---The Finite Time of Categorical Determination.}
(A) Partition lag distribution: different systems exhibit different lag time distributions, with a fundamental minimum $\tau_{min} = \hbar/\Delta E$ set by the uncertainty principle.
(B) Undetermined residue evolution: during partition lag, categorical states remain in superposition. The determination fraction approaches 1 asymptotically, with residue decreasing exponentially.
(C) Entropy production rate: entropy is produced continuously during partition lag at rate $dS/dt = k_B \cdot \text{Residue}/\tau_{lag}$. Cumulative entropy $S(t)$ saturates as determination completes.
(D) Minimum lag scaling: $\tau_{min} \propto 1/\Delta E$ across different energy scales (phonon, vibrational, electronic, core), demonstrating the fundamental quantum limit on partition speed.}
\label{fig:partition_lag}
\end{figure}

%==============================================================================
% Figure L-6: Termination and Irreversibility
%==============================================================================
\begin{figure}[htbp]
\centering
\includegraphics[width=\textwidth]{figures/panel_termination_irreversibility.pdf}
\caption{\textbf{Termination, Completion, and the Impossibility of Reversal.}
(A) Reality stream vs terminated state: ongoing processes (blue wave) have indeterminate entropy---they are superpositions of possibilities. Only terminated states (green box) have well-defined entropy as categorical facts.
(B) Identity of completion and partitioning: categorical completion (selecting one outcome) is identical to geometric partitioning (creating boundaries). Both create the distinction between actualised and non-actualised states.
(C) Why reversal fails: forward processes create non-actualisations ($A \to B$ plus infinitely many things $B$ is not doing). Backward would require un-creating these---impossible.
(D) Asymmetry ratio growth: with each categorical completion, the forward/backward probability ratio grows exponentially as $\prod_i \Omega_i$, rapidly diverging from the reversible ratio of 1.}
\label{fig:termination_irreversibility}
\end{figure}

%==============================================================================
% Figure L-7: Cross-Sectional Validation of Irreversibility
%==============================================================================
\begin{figure}[htbp]
\centering
\includegraphics[width=\textwidth]{figures/panel_loschmidt_cross_sectional_validation.pdf}
\caption{\textbf{Cross-Sectional Validation: Radial Expansion and the Arrow of Time.}
(A) S-coordinate evolution with radius: configuration entropy $S_k$ and temporal entropy $S_t$ plotted versus radial distance from an expanding point for three systems (Fast, Medium, Slow Expansion). Each radial shell is a cross-sectional measurement---a spherical surface at fixed distance. All systems show monotonic increase: entropy ALWAYS grows outward.
(B) Non-actualisations dominate: logarithmic plot of actualised (dashed) vs non-actualised (solid) state counts. Non-actualisations ($N_{\text{non-act}} \propto r^2 - \omega(t)r^2/4\pi$) vastly outnumber actualisations at all radii, with ratios from 37:1 (fast expansion) to 392:1 (slow expansion). This asymmetry is the origin of irreversibility.
(C) Entropy gradient always positive: the derivative $\partial S_t/\partial r > 0$ at ALL radii for ALL systems. Green region indicates positive (irreversible) gradients; no system ever enters the negative (reversible) region. The gradient points outward because non-actualisations accumulate faster than actualisations can explore.
(D) Irreversibility fraction: bar chart showing 100\% positive gradients for all three expansion regimes. The 50\% line (gray dashed) marks the threshold for reversible processes; 100\% marks complete irreversibility. All systems achieve 100\%, confirming that the arrow of time is universal and independent of expansion rate.
(E) S-transformation validation: predicted S-coordinates (from $\mathcal{T}_{dr}$) versus calculated values, showing $R^2 > 0.99$ for all systems. The transformation correctly predicts entropy at each radial shell from the previous shell's state.
(F) Schematic of expanding point: central point (black dot) expands into state space, creating concentric spherical shells (color gradient from green/low entropy to red/high entropy). Arrows show outward expansion direction. The gradient $\nabla S > 0$ always points away from the origin---this is the geometric necessity of irreversibility. Non-actualisations form a ``wake'' around the actualised trajectory, and this wake grows with distance, making reversal impossible.}
\label{fig:loschmidt_cross_sectional_validation}
\end{figure}



%==============================================================================
% Figure L-1: Mixing-Separation Entropy Cycle
%==============================================================================
\begin{figure}[htbp]
\centering
\includegraphics[width=\textwidth]{figures/panel_mixing_separation.pdf}
\caption{\textbf{Mixing-Separation Cycle Demonstrates Irreversibility.}
(A) Initial state: two gases separated by partition, with entropy $S_{initial} = S_A^{(0)} + S_B^{(0)}$.
(B) Mixed state: partition removed, gases interdiffuse, entropy increases by $\Delta S_{mix}$.
(C) Re-separated state: partition restored, but each container now contains both gases with residual phase correlations.
(D) Entropy evolution: the categorical prediction (green) shows $S_{final} > S_{initial}$ despite identical spatial configuration, while the classical reversible prediction (gray) incorrectly predicts return to initial entropy. The difference $\Delta S_{irrev} > 0$ arises from phase-lock network densification that persists after re-separation.}
\label{fig:mixing_separation}
\end{figure}

%==============================================================================
% Figure L-2: Phase-Lock Network Evolution
%==============================================================================
\begin{figure}[htbp]
\centering
\includegraphics[width=\textwidth]{figures/panel_phase_lock_network.pdf}
\caption{\textbf{Phase-Lock Network Densification and Residual Correlations.}
(A) Initial separated state: two disconnected network clusters (blue = Gas A, red = Gas B) with $|E|$ internal edges.
(B) Mixed state: networks merge into single connected component with cross-container edges.
(C) Re-separated state: partition restored, but residual cross-edges (red dashed) persist---these represent phase correlations created during mixing that cannot be erased.
(D) Edge count evolution: $|E_{final}| > |E_{initial}|$ demonstrates that mixing creates categorical structure (edges) that remains after re-separation. More edges means more constraints, hence higher entropy.}
\label{fig:phase_lock_network}
\end{figure}

%==============================================================================
% Figure L-3: Non-Actualisation Asymmetry
%==============================================================================
\begin{figure}[htbp]
\centering
\includegraphics[width=\textwidth]{figures/panel_non_actualisation.pdf}
\caption{\textbf{Non-Actualisation Asymmetry---The Deepest Reason for Irreversibility.}
(A) The cup example: when a cup falls and breaks, it generates infinitely many non-actualisations (not turning to gold, not becoming sentient, not teleporting, etc.)---categorical facts defined by negation.
(B) Branching asymmetry: each actualisation (green, finite) creates infinitely many non-actualisations (red), yielding a 1:$\infty$ asymmetry ratio.
(C) Accumulation over time: non-actualisations grow monotonically and cannot be un-created, while actualisations remain finite.
(D) Forward/backward asymmetry: forward processes always possible (create non-actualisations), backward processes impossible (would require un-creating non-actualisations). The ratio $P_{forward}/P_{backward} \to \infty$.}
\label{fig:non_actualisation}
\end{figure}

%==============================================================================
% Figure L-4: Aperture Selectivity and Categorical Potential
%==============================================================================
\begin{figure}[htbp]
\centering
\includegraphics[width=\textwidth]{figures/panel_aperture_selectivity.pdf}
\caption{\textbf{Partition Boundaries as Categorical Apertures.}
(A) Selection function $\sigma(\omega)$: aperture (partition boundary) allows certain configurations to pass ($\sigma = 1$, green arrows) while blocking others ($\sigma = 0$, red X marks).
(B) Categorical potential vs selectivity: $\Phi_a = -k_B T \ln s$ where $s = \Omega_{pass}/\Omega_{total}$. High selectivity ($s \to 0$) implies high potential barrier.
(C) Entropy from selectivity: higher selectivity (lower $s$) produces more entropy, since $\Delta S = k_B \ln(1/s) = \Phi_a/T$.
(D) Aperture as energy barrier: the categorical potential acts as a barrier that blocked configurations must overcome. Non-actualisations are precisely the configurations blocked by partition apertures.}
\label{fig:aperture_selectivity}
\end{figure}

%==============================================================================
% Figure L-5: Partition Lag Dynamics
%==============================================================================
\begin{figure}[htbp]
\centering
\includegraphics[width=\textwidth]{figures/panel_partition_lag.pdf}
\caption{\textbf{Partition Lag---The Finite Time of Categorical Determination.}
(A) Partition lag distribution: different systems exhibit different lag time distributions, with a fundamental minimum $\tau_{min} = \hbar/\Delta E$ set by the uncertainty principle.
(B) Undetermined residue evolution: during partition lag, categorical states remain in superposition. The determination fraction approaches 1 asymptotically, with residue decreasing exponentially.
(C) Entropy production rate: entropy is produced continuously during partition lag at rate $dS/dt = k_B \cdot \text{Residue}/\tau_{lag}$. Cumulative entropy $S(t)$ saturates as determination completes.
(D) Minimum lag scaling: $\tau_{min} \propto 1/\Delta E$ across different energy scales (phonon, vibrational, electronic, core), demonstrating the fundamental quantum limit on partition speed.}
\label{fig:partition_lag}
\end{figure}

%==============================================================================
% Figure L-6: Termination and Irreversibility
%==============================================================================
\begin{figure}[htbp]
\centering
\includegraphics[width=\textwidth]{figures/panel_termination_irreversibility.pdf}
\caption{\textbf{Termination, Completion, and the Impossibility of Reversal.}
(A) Reality stream vs terminated state: ongoing processes (blue wave) have indeterminate entropy---they are superpositions of possibilities. Only terminated states (green box) have well-defined entropy as categorical facts.
(B) Identity of completion and partitioning: categorical completion (selecting one outcome) is identical to geometric partitioning (creating boundaries). Both create the distinction between actualised and non-actualised states.
(C) Why reversal fails: forward processes create non-actualisations ($A \to B$ plus infinitely many things $B$ is not doing). Backward would require un-creating these---impossible.
(D) Asymmetry ratio growth: with each categorical completion, the forward/backward probability ratio grows exponentially as $\prod_i \Omega_i$, rapidly diverging from the reversible ratio of 1.}
\label{fig:termination_irreversibility}
\end{figure}

%==============================================================================
% Figure L-7: Cross-Sectional Validation of Irreversibility
%==============================================================================
\begin{figure}[htbp]
\centering
\includegraphics[width=\textwidth]{figures/panel_loschmidt_cross_sectional_validation.pdf}
\caption{\textbf{Cross-Sectional Validation: Radial Expansion and the Arrow of Time.}
(A) S-coordinate evolution with radius: configuration entropy $S_k$ and temporal entropy $S_t$ plotted versus radial distance from an expanding point for three systems (Fast, Medium, Slow Expansion). Each radial shell is a cross-sectional measurement---a spherical surface at fixed distance. All systems show monotonic increase: entropy ALWAYS grows outward.
(B) Non-actualisations dominate: logarithmic plot of actualised (dashed) vs non-actualised (solid) state counts. Non-actualisations ($N_{\text{non-act}} \propto r^2 - \omega(t)r^2/4\pi$) vastly outnumber actualisations at all radii, with ratios from 37:1 (fast expansion) to 392:1 (slow expansion). This asymmetry is the origin of irreversibility.
(C) Entropy gradient always positive: the derivative $\partial S_t/\partial r > 0$ at ALL radii for ALL systems. Green region indicates positive (irreversible) gradients; no system ever enters the negative (reversible) region. The gradient points outward because non-actualisations accumulate faster than actualisations can explore.
(D) Irreversibility fraction: bar chart showing 100\% positive gradients for all three expansion regimes. The 50\% line (gray dashed) marks the threshold for reversible processes; 100\% marks complete irreversibility. All systems achieve 100\%, confirming that the arrow of time is universal and independent of expansion rate.
(E) S-transformation validation: predicted S-coordinates (from $\mathcal{T}_{dr}$) versus calculated values, showing $R^2 > 0.99$ for all systems. The transformation correctly predicts entropy at each radial shell from the previous shell's state.
(F) Schematic of expanding point: central point (black dot) expands into state space, creating concentric spherical shells (color gradient from green/low entropy to red/high entropy). Arrows show outward expansion direction. The gradient $\nabla S > 0$ always points away from the origin---this is the geometric necessity of irreversibility. Non-actualisations form a ``wake'' around the actualised trajectory, and this wake grows with distance, making reversal impossible.}
\label{fig:loschmidt_cross_sectional_validation}
\end{figure}


% or copy individual figure environments as needed.

%=============================================================================
% FIGURE 1: Triple Equivalence - Oscillation ≡ Category ≡ Partition
%=============================================================================
\begin{figure}[htbp]
    \centering
    \includegraphics[width=\textwidth]{../shared-figures/figures/panel1_triple_equivalence.pdf}
    \caption{\textbf{Triple Equivalence: Virtual Gas Molecules as Pendulums.}
    This panel demonstrates that oscillatory dynamics, categorical states, and partition geometry 
    are three equivalent descriptions of the same physical reality.
    \textbf{(A)} Virtual gas molecules in a container, where each vibrational mode is represented 
    as a simple pendulum. The five molecules shown oscillate with different phases and frequencies, 
    but each follows the same underlying dynamics.
    \textbf{(B)} Oscillatory perspective: the pendulum traces a sinusoidal trajectory 
    $\theta(t) = \theta_0 \sin(\omega t)$ with period $T = 2\pi/\omega$. Quantum states 
    $|0\rangle, |1\rangle, \ldots$ emerge from energy quantization.
    \textbf{(C)} Categorical perspective: the pendulum's range is divided into $n = 8$ 
    distinguishable positions $C_1, \ldots, C_8$. Each position $\theta_i$ is a categorical 
    state that the system occupies at different times.
    \textbf{(D)} Partition perspective: the oscillation period is represented as a hierarchical 
    tree with depth $M$ and branching factor $n$. The total number of terminal states (leaves) 
    is $n^M$.
    \textbf{(E)} The fundamental equivalence: all three perspectives are mathematically 
    equivalent, differing only in vocabulary. The equivalence symbol ($\equiv$) indicates 
    that oscillation, category, and partition descriptions yield identical physical predictions.
    \textbf{(F)} Parameter correspondence table showing how concepts map across frameworks. 
    Degrees of freedom ($M$) correspond to modes, dimensions, or tree levels; states per 
    degree of freedom ($n$) correspond to quantum numbers, categorical levels, or branching 
    factor. All three yield the same entropy: $S = k_B M \ln n$.}
    \label{fig:triple_equivalence}
\end{figure}

%=============================================================================
% FIGURE 2: Entropy Derivation - S = k_B M ln n
%=============================================================================
\begin{figure}[htbp]
    \centering
    \includegraphics[width=\textwidth]{../shared-figures/figures/panel2_entropy_derivation.pdf}
    \caption{\textbf{Entropy Derivation from Three Equivalent Perspectives.}
    This panel shows how the unified entropy formula $S = k_B M \ln n$ emerges identically 
    from oscillatory, categorical, and partition frameworks.
    \textbf{(A)} Oscillatory state counting: for a system with $M = 3$ independent modes, 
    each with $n = 4$ quantum states, the total number of microstates is 
    $W_{\text{osc}} = n^M = 4^3 = 64$.
    \textbf{(B)} Categorical state counting: a categorical space with $M = 2$ dimensions, 
    each having $n = 4$ levels, contains $|\mathcal{C}| = n^M = 4^2 = 16$ distinguishable 
    states (shown as circles $C_{i,j}$).
    \textbf{(C)} Partition path counting: a partition tree with $M = 2$ levels and 
    branching factor $n = 3$ has $n^M = 3^2 = 9$ unique paths from root to leaves 
    (one path highlighted in red).
    \textbf{(D)} Boltzmann's fundamental relation $S = k_B \ln W$ combined with the 
    microstate count $W = n^M$ yields the unified entropy formula $S = k_B M \ln n$ 
    through logarithmic expansion.
    \textbf{(E)} Convergence diagram: all three frameworks (oscillatory, categorical, 
    partition) independently yield $W = n^M$, which through Boltzmann's relation gives 
    the same entropy formula.
    \textbf{(F)} Entropy scaling plot showing how $S/k_B = M \ln n$ varies with 
    degrees of freedom $M$ and states per DOF $n$. Entropy scales linearly with $M$ 
    (horizontal direction) and logarithmically with $n$ (vertical direction). 
    A simple pendulum ($M = 1$, $n = 4$) and a gas molecule ($M = 6$, $n = 3$) 
    are marked as reference points.}
    \label{fig:entropy_derivation}
\end{figure}

%=============================================================================
% FIGURE 3: Categorical Enthalpy - H = U + Σn_a Φ_a → U + PV
%=============================================================================
\begin{figure}[htbp]
    \centering
    \includegraphics[width=\textwidth]{../shared-figures/figures/panel3_categorical_enthalpy.pdf}
    \caption{\textbf{Categorical Enthalpy: From Aperture Selectivity to Classical Thermodynamics.}
    This panel derives classical enthalpy $H = U + PV$ as the coarse-grained limit of 
    categorical aperture dynamics.
    \textbf{(A)} Aperture concept: a membrane with an aperture selectively allows 
    passage of molecules based on their categorical properties. Small molecules (green) 
    pass through; large molecules (red) are blocked. The selectivity $s_a = \Omega_{\text{pass}}/\Omega_{\text{total}}$ 
    ranges from $s = 0$ (impermeable) to $s = 1$ (fully permeable).
    \textbf{(B)} Categorical potential vs.\ selectivity: the potential 
    $\Phi_a = -k_B T \ln s_a$ diverges as $s_a \to 0$ (impermeable barrier) and 
    vanishes as $s_a \to 1$ (no barrier). The curve shows how selective apertures 
    create entropic barriers.
    \textbf{(C)} Categorical enthalpy definition: for a system with internal energy $U$ 
    and multiple apertures on its boundary, the categorical enthalpy is 
    $\mathcal{H} = U + \sum_a n_a \Phi_a$, where $n_a$ is the number of type-$a$ 
    apertures and $\Phi_a$ is their categorical potential.
    \textbf{(D)} Classical limit: as aperture selectivity approaches unity ($s_a \to 1$) 
    and aperture density increases ($n_a \to \infty$), the discrete sum of aperture 
    potentials becomes a surface integral, yielding $\sum_a n_a \Phi_a \to \int_{\partial\Omega} P \, dA = PV$.
    \textbf{(E)} Pressure emergence: pressure is defined as the coarse-grained limit 
    of aperture potential density, $P = \lim_{s_a \to 1} \rho_a \cdot (-k_B T \ln s_a)$, 
    where $\rho_a$ is the aperture density per unit area.
    \textbf{(F)} Enthalpy hierarchy: the categorical enthalpy 
    $\mathcal{H} = U + \int_{\partial\Omega} \sigma(x) \cdot \phi(x) \, dA$ (fundamental) 
    reduces to classical enthalpy $H = U + PV$ (coarse-grained) when local selectivity 
    $\sigma(x) \to 1$ and local potential $\phi(x) \to P$. Classical thermodynamics 
    emerges as the coarse-grained limit of categorical aperture dynamics.}
    \label{fig:categorical_enthalpy}
\end{figure}

