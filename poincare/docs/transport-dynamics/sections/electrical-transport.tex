%==============================================================================
\section{Electrical Transport}
\label{sec:electrical}
%==============================================================================

\subsection{Electron-Lattice Partition Dynamics}

In metallic conductors, conduction electrons form a dense Fermi sea with density $n \sim 10^{28}$ m$^{-3}$ \citep{ashcroft1976}. Under an applied electric field $\mathbf{E}$, electrons acquire a drift velocity superimposed on their thermal motion. Resistance arises from scattering events that randomise the drift momentum.

\begin{definition}[Electron-Lattice Partition]
\label{def:electron_lattice}
An electron-lattice partition operation occurs when a conduction electron interacts with a lattice ion, phonon, impurity, or defect. The partition distinguishes pre-scattering and post-scattering electron states, with the scattering time $\tau_s$ serving as the partition lag.
\end{definition}

For electrical transport, the carriers are electrons with charge $e$ and density $n$. The normalisation factor is $\mathcal{N} = ne^2$, giving:

\begin{theorem}[Electrical Resistivity]
\label{thm:resistivity}
The electrical resistivity of a conductor is:
\begin{equation}
\rho = \frac{1}{ne^2} \sum_{i,j} \tau_{s,ij} g_{ij}
\label{eq:resistivity_partition}
\end{equation}
where the sum is over electron-scatterer pairs, $\tau_{s,ij}$ is the scattering partition lag, and $g_{ij}$ is the electron-scatterer coupling.
\end{theorem}

\begin{proof}
From the universal transport formula~\eqref{eq:universal_transport} with $\Tcoeff = \rho$ and $\mathcal{N} = ne^2$, the result follows directly. The normalisation $ne^2$ arises from the current density $\mathbf{J} = ne\mathbf{v}_d$ and the Drude relation $\mathbf{v}_d = e\mathbf{E}\tau/m$, giving $\rho = m/(ne^2\tau)$ in the single-$\tau$ limit. \qed
\end{proof}

\subsection{Scattering Mechanisms}

Multiple scattering mechanisms contribute to the total resistivity through Matthiessen's rule \citep{matthiessen1858}:
\begin{equation}
\rho_{\text{total}} = \rho_{\text{phonon}} + \rho_{\text{impurity}} + \rho_{\text{defect}} + \rho_{e-e}
\label{eq:matthiessen}
\end{equation}

Each contribution has characteristic partition lag:

\textbf{Phonon scattering}: Electrons scatter from lattice vibrations. The partition lag increases with temperature as phonon population grows:
\begin{equation}
\tau_{\text{phonon}}(T) = \frac{\tau_0 \Theta_D}{T} \quad (T > \Theta_D)
\label{eq:tau_phonon_high}
\end{equation}
where $\Theta_D$ is the Debye temperature \citep{ziman1960}. This gives $\rho_{\text{phonon}} \propto T$, matching the observed linear temperature dependence in metals above $\Theta_D$ \citep{white1959}.

\textbf{Impurity scattering}: Electrons scatter from substitutional impurities and foreign atoms. The partition lag is temperature-independent for elastic scattering:
\begin{equation}
\tau_{\text{impurity}} = \text{const}
\label{eq:tau_impurity}
\end{equation}
This produces the residual resistivity $\rho_0$ observed as $T \to 0$ \citep{nordheim1931}.

\textbf{Electron-electron scattering}: In Fermi liquids, electron-electron scattering is suppressed by Pauli blocking. The available phase space gives:
\begin{equation}
\tau_{e-e}(T) \propto T^{-2}
\label{eq:tau_ee_fermi}
\end{equation}
contributing $\rho_{e-e} \propto T^2$ at low temperatures \citep{abrikosov1963}.

\subsection{The Newton's Cradle Mechanism}

Electrical current propagates through conductors not by individual electron drift but by collective displacement, analogous to momentum transfer in a Newton's cradle \citep{drude1900a}.

\begin{proposition}[Velocity Disparity]
\label{prop:velocity_disparity}
The drift velocity $v_d \sim 10^{-4}$ m/s is twelve orders of magnitude smaller than the signal velocity $v_s \sim c/\sqrt{\varepsilon_r}$.
\end{proposition}

\begin{proof}
For a copper wire with current $I = 1$ A, cross-section $A = 1$ mm$^2$, and electron density $n = 8.5 \times 10^{28}$ m$^{-3}$:
\begin{equation}
v_d = \frac{I}{neA} = \frac{1}{8.5 \times 10^{28} \times 1.6 \times 10^{-19} \times 10^{-6}} \approx 7.4 \times 10^{-5} \text{ m/s}
\end{equation}
The signal propagates at electromagnetic speed $v_s = c/\sqrt{\varepsilon_r \mu_r} \sim 10^8$ m/s. \qed
\end{proof}

Current propagation occurs through sequential electron displacement:
\begin{equation}
e_1 \to e_2 \to e_3 \to \cdots \to e_N
\end{equation}
where each $e_i \to e_{i+1}$ represents a displacement event mediated by Coulomb repulsion, not physical electron transport between sites.

\subsection{Resistance from Partition Accumulation}

The total resistance of a conductor of length $L$ and cross-section $A$ is:
\begin{equation}
R = \rho \frac{L}{A} = \frac{L}{Ane^2} \sum_{i,j} \tau_{s,ij} g_{ij}
\label{eq:resistance_total}
\end{equation}

This can be interpreted as accumulated partition lag along the conductor:
\begin{equation}
R = \frac{1}{Ane^2} \int_0^L \sum_{i,j} \tau_{s,ij}(x) g_{ij}(x) \, dx
\label{eq:resistance_integral}
\end{equation}

For a uniform conductor, $\tau_s$ and $g$ are position-independent, recovering $R = \rho L/A$.

\subsection{Ohm's Law}

\begin{theorem}[Ohm's Law]
\label{thm:ohm}
For a conductor with resistance $R$ carrying current $I$ under potential difference $V$:
\begin{equation}
V = IR
\label{eq:ohm}
\end{equation}
\end{theorem}

\begin{proof}
The current density is $J = \sigma E = E/\rho$, where $\sigma = 1/\rho$ is conductivity. For a uniform field $E = V/L$:
\begin{equation}
I = JA = \frac{VA}{L\rho} = \frac{V}{R}
\end{equation}
Rearranging gives $V = IR$. \qed
\end{proof}

Ohm's law follows from the linear relation between flux (current) and driving force (electric field), which holds when partition operations are independent and partition lag is flux-independent.

\subsection{Power Dissipation}

The power dissipated in a resistor is:
\begin{equation}
P = I^2 R = \frac{V^2}{R} = IV
\label{eq:joule_heating}
\end{equation}

From the partition perspective, each scattering event dissipates energy through entropy production:
\begin{equation}
\dot{Q} = T \sum_{i,j} \Gamma_{ij} \Delta S_{ij}
\end{equation}

For current $I$, the scattering rate is $\Gamma = I/(e \cdot N_{\text{scatter}})$ where $N_{\text{scatter}} = L/\lambda$ is the number of scattering events per electron. This yields:
\begin{equation}
P = I^2 R
\end{equation}
confirming that Joule heating is the macroscopic manifestation of partition entropy production.

