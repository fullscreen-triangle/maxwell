%==============================================================================
\section{Viscous Transport}
\label{sec:viscous}
%==============================================================================

\subsection{Molecular Collision Partition Dynamics}

In fluids, momentum transport occurs through molecular collisions. When adjacent fluid layers move at different velocities, molecules crossing between layers carry momentum, producing a shear stress proportional to the velocity gradient \citep{batchelor1967,landau1987}.

\begin{definition}[Molecular Collision Partition]
\label{def:molecular_collision}
A molecular collision partition operation occurs when two molecules interact, exchanging momentum and creating a categorical distinction between pre-collision and post-collision states. The collision time $\tau_c$ serves as the partition lag.
\end{definition}

For viscous transport, the carriers are molecules with mass $m$ and number density $n$. The normalisation factor is $\mathcal{N} = 1$ for viscosity in standard units, giving:

\begin{theorem}[Dynamic Viscosity]
\label{thm:viscosity}
The dynamic viscosity of a fluid is:
\begin{equation}
\mu = \sum_{i,j} \tau_{c,ij} g_{ij}
\label{eq:viscosity_partition}
\end{equation}
where the sum is over molecular pairs, $\tau_{c,ij}$ is the collision partition lag, and $g_{ij}$ is the molecular coupling strength.
\end{theorem}

\subsection{Kinetic Theory Connection}

For a dilute gas, kinetic theory gives the viscosity as \citep{chapman1970}:
\begin{equation}
\mu = \frac{1}{3} n m \bar{v} \lambda
\label{eq:viscosity_kinetic}
\end{equation}
where $\bar{v} = \sqrt{8\kB T/\pi m}$ is the mean molecular speed and $\lambda = 1/(n\sigma)$ is the mean free path with collision cross-section $\sigma$.

The collision time is:
\begin{equation}
\tau_c = \frac{\lambda}{\bar{v}} = \frac{1}{n\sigma\bar{v}}
\label{eq:tau_collision}
\end{equation}

Identifying $\tau_c$ with partition lag and $g = nm\bar{v}^2 \sigma/3$ with coupling:
\begin{equation}
\mu = \tau_c \cdot g = \frac{1}{n\sigma\bar{v}} \cdot \frac{nm\bar{v}^2 \sigma}{3} = \frac{m\bar{v}}{3} = \frac{1}{3}\sqrt{\frac{8m\kB T}{\pi}}
\label{eq:viscosity_derived}
\end{equation}

This reproduces the Chapman-Enskog result for hard-sphere molecules.

\subsection{Temperature Dependence}

For gases, viscosity increases with temperature:
\begin{equation}
\mu(T) \propto \sqrt{T}
\label{eq:viscosity_gas_T}
\end{equation}

This arises because $\bar{v} \propto \sqrt{T}$ while $\lambda$ is approximately temperature-independent for hard spheres. More realistic potentials (Lennard-Jones, Sutherland) modify this dependence slightly \citep{hirschfelder1964}.

For liquids, viscosity decreases with temperature:
\begin{equation}
\mu(T) = \mu_0 \exp\left(\frac{E_a}{\kB T}\right)
\label{eq:viscosity_liquid_T}
\end{equation}

The Arrhenius form reflects activated molecular motion over energy barriers $E_a$ \citep{eyring1936}. In partition terms, the partition lag decreases exponentially with temperature as thermal activation overcomes barriers.

\subsection{Momentum Transfer Mechanism}

Viscous momentum transfer in fluids parallels current propagation in conductors. Consider a shear flow with velocity gradient $\partial v_x/\partial y$:

\begin{enumerate}
\item Molecules in faster-moving layer collide with molecules in slower-moving layer
\item Collisions transfer momentum from fast to slow layer
\item Net momentum flux produces shear stress $\tau_{xy} = \mu \, \partial v_x/\partial y$
\end{enumerate}

The molecular collisions are partition operations. Each collision creates undetermined residue (molecular states not sharply defined during collision), and this residue manifests as viscous dissipation.

\subsection{The Navier-Stokes Equation}

The momentum equation for a viscous fluid is \citep{navier1822,stokes1845}:
\begin{equation}
\rho \left( \frac{\partial \mathbf{v}}{\partial t} + (\mathbf{v} \cdot \nabla)\mathbf{v} \right) = -\nabla p + \mu \nabla^2 \mathbf{v} + \mathbf{f}
\label{eq:navier_stokes}
\end{equation}

The viscous term $\mu \nabla^2 \mathbf{v}$ represents momentum diffusion driven by partition operations between adjacent fluid elements. The viscosity $\mu = \sum \tau_c g$ sets the rate of this diffusion.

\subsection{Reynolds Number and Turbulence}

The Reynolds number $\text{Re} = \rho v L/\mu$ compares inertial to viscous forces. At high Re, viscous partition operations cannot keep pace with inertial momentum transport, and the flow becomes turbulent \citep{reynolds1883}.

In partition terms:
\begin{equation}
\text{Re} = \frac{\rho v L}{\sum \tau_c g} = \frac{\text{inertial momentum flux}}{\text{partition-limited momentum flux}}
\label{eq:reynolds_partition}
\end{equation}

Turbulence onset at $\text{Re} \gtrsim 2000$ marks the regime where partition operations no longer dominate momentum transport.

\subsection{Energy Dissipation}

Viscous dissipation per unit volume is:
\begin{equation}
\Phi = \mu \left( \frac{\partial v_i}{\partial x_j} + \frac{\partial v_j}{\partial x_i} \right)^2
\label{eq:viscous_dissipation}
\end{equation}

This equals the entropy production rate from molecular collision partitions:
\begin{equation}
T\dot{S} = \sum_{i,j} \Gamma_{ij} \kB T \ln n_{\text{res},ij}
\end{equation}

Viscous heating is the thermal manifestation of partition entropy production, precisely analogous to Joule heating in electrical transport.

