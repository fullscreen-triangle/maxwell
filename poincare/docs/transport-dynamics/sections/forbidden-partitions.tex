%==============================================================================
\section{Dissipationless States: Forbidden Partition Regimes}
\label{sec:forbidden}
%==============================================================================

\subsection{Superconductivity}

\subsubsection{BCS Theory and Partition Extinction}

BCS theory describes superconductivity through Cooper pairing of electrons mediated by phonon exchange \citep{bardeen1957}. In the partition framework, Cooper pairing is categorical unification: two electrons become a single categorical entity.

\begin{theorem}[Superconducting Resistivity]
\label{thm:superconducting}
Below the critical temperature $T_c$, the electrical resistivity of a superconductor is:
\begin{equation}
\rho(T < T_c) = 0
\label{eq:rho_super}
\end{equation}
exactly, not approximately.
\end{theorem}

\begin{proof}
Cooper pairs form when $\kB T < \Delta$, where $\Delta$ is the superconducting gap. Below $T_c$, all conduction electrons form Cooper pairs. The pairs are categorically unified---partition operations between them are undefined.

From the resistivity formula $\rho = (ne^2)^{-1} \sum \taulag g$:
\begin{itemize}
\item Above $T_c$: electrons are distinguishable, partition occurs, $\taulag > 0$, $\rho > 0$
\item Below $T_c$: Cooper pairs are unified, partition forbidden, $\taulag = 0$, $\rho = 0$
\end{itemize}
\qed
\end{proof}

\subsubsection{The BCS Gap Relation}

The BCS theory predicts $\Delta = 1.76\kB T_c$ for weak-coupling superconductors \citep{bardeen1957}. This relation follows from the partition framework as the condition for categorical unification.

The factor 1.76 arises from the density of states at the Fermi surface and the logarithmic dependence of the gap equation. In partition terms, it represents the ratio of phase-locking energy to thermal energy at the transition.

\subsubsection{Meissner Effect}

Superconductors expel magnetic fields (Meissner effect) \citep{meissner1933}. In the partition framework:

\begin{proposition}[Meissner Effect]
\label{prop:meissner}
Magnetic fields create Landau levels that partition electron momentum space. Cooper pairs resist partition. The field is expelled to preserve categorical unity.
\end{proposition}

The critical field $H_c$ is the field strength at which the magnetic partition energy equals the pairing energy:
\begin{equation}
\mu_0 H_c^2/2 \sim n\Delta
\label{eq:critical_field}
\end{equation}
Above $H_c$, magnetic partition overwhelms pairing, and superconductivity is destroyed.

\subsubsection{Persistent Currents}

In a superconducting ring, current flows indefinitely without decay. This is direct evidence of zero partition lag: with no scattering to dissipate momentum, current persists.

The current is quantised in units of the flux quantum $\Phi_0 = h/(2e)$, reflecting the single categorical state of the condensate.

\subsection{Superfluidity}

\subsubsection{Helium-4 Below $T_\lambda$}

Liquid helium-4 becomes superfluid below the $\lambda$-transition at $T_\lambda = 2.17$ K \citep{kapitza1938,allen1938}. The viscosity drops to exactly zero.

\begin{theorem}[Superfluid Viscosity]
\label{thm:superfluid}
Below $T_\lambda$, the viscosity of helium-4 is:
\begin{equation}
\mu(T < T_\lambda) = 0
\label{eq:mu_super}
\end{equation}
\end{theorem}

\begin{proof}
Helium-4 atoms are bosons. Below $T_\lambda$, a macroscopic fraction condenses into the ground state. Condensed atoms are categorically unified---they form a single categorical entity.

From the viscosity formula $\mu = \sum \taulag g$:
\begin{itemize}
\item Above $T_\lambda$: atoms are distinguishable, collisions partition, $\taulag > 0$, $\mu > 0$
\item Below $T_\lambda$: condensed atoms are unified, partition forbidden, $\taulag = 0$, $\mu = 0$
\end{itemize}
\qed
\end{proof}

\subsubsection{Two-Fluid Model}

Landau's two-fluid model describes superfluid helium as a mixture of superfluid (zero viscosity) and normal fluid (finite viscosity) components \citep{landau1941}. In the partition framework:

\begin{itemize}
\item \textbf{Superfluid component}: Atoms in the condensate (categorically unified)
\item \textbf{Normal component}: Atoms in excited states (categorically distinguishable)
\end{itemize}

The superfluid fraction increases as $T \to 0$:
\begin{equation}
\frac{\rho_s}{\rho} = 1 - \left(\frac{T}{T_\lambda}\right)^{\alpha}
\label{eq:superfluid_fraction}
\end{equation}
where $\alpha \approx 5.6$ near $T_\lambda$ \citep{donnelly1998}.

\subsubsection{Wall Climbing and Fountain Effect}

Superfluid helium climbs container walls and can empty an open container \citep{rollin1936}. This occurs because:

\begin{enumerate}
\item Zero viscosity means no partition barrier to flow
\item The superfluid redistributes to minimise total energy
\item A thin film forms on walls and flows to equalise pressure
\item If the container is open, the film can flow up and over the rim
\end{enumerate}

The fountain effect---superfluid flowing upward when heated---demonstrates that normal fluid is created by heating (breaking categorical unity), and the resulting pressure gradient drives superfluid flow.

\subsubsection{Quantised Vortices}

Rotation in a superfluid is quantised. Vortices carry circulation in units of $\kappa = h/m_4 = 9.97 \times 10^{-8}$ m$^2$/s \citep{vinen1961}.

This quantisation reflects the single categorical state of the condensate. Classical vorticity is continuous; superfluid vorticity is discrete because the condensate cannot support arbitrary phase gradients.

\subsection{Bose-Einstein Condensation}

\subsubsection{Dilute Atomic Gases}

In dilute atomic gases cooled to nanokelvin temperatures, atoms condense into the ground state \citep{anderson1995,davis1995}. The condensate is a macroscopic occupation of a single quantum state.

\begin{theorem}[BEC Transition]
\label{thm:BEC}
Below $T_{\text{BEC}}$, a macroscopic fraction of atoms occupy the ground state:
\begin{equation}
\frac{N_0}{N} = 1 - \left(\frac{T}{T_{\text{BEC}}}\right)^{3/2}
\label{eq:condensate_fraction}
\end{equation}
\end{theorem}

\begin{proof}
From Bose-Einstein statistics, the number of atoms in excited states is:
\begin{equation}
N - N_0 = \int_0^\infty \frac{g(\varepsilon)}{e^{\varepsilon/\kB T} - 1} d\varepsilon \propto T^{3/2}
\end{equation}

At $T = T_{\text{BEC}}$, $N_0 = 0$. Below $T_{\text{BEC}}$, $N_0/N = 1 - (T/T_{\text{BEC}})^{3/2}$. \qed
\end{proof}

\subsubsection{Partition Interpretation}

In a classical gas, atoms are distinguishable. Collisions between atoms are partition operations that randomise trajectories.

In a BEC, condensed atoms are indistinguishable. They occupy the same categorical state. Partition operations between them are undefined.

The normal-to-BEC transition is partition extinction: above $T_{\text{BEC}}$, atoms scatter; below $T_{\text{BEC}}$, condensed atoms form a single categorical entity that cannot self-scatter.

\subsubsection{Coherence and Interference}

BEC exhibits macroscopic coherence---all condensed atoms share the same quantum phase. Interference between two BECs produces fringes \citep{andrews1997}, demonstrating the single categorical state of each condensate.

This coherence is the signature of categorical unification. Distinguishable atoms would not interfere; unified atoms produce coherent interference patterns.

\subsection{Unified Perspective}

\begin{table}[h]
\centering
\caption{Comparison of dissipationless states}
\label{tab:dissipationless}
\begin{tabular}{lccc}
\toprule
Property & Superconductor & Superfluid He-4 & BEC \\
\midrule
Carriers & Electrons & He-4 atoms & Various atoms \\
Pairing & Cooper pairs & None (bosons) & None (bosons) \\
Critical temp & $T_c$ & $T_\lambda = 2.17$ K & $T_{\text{BEC}} \sim$ nK \\
Zero coefficient & Resistivity $\rho$ & Viscosity $\mu$ & Diffusivity$^{-1}$ \\
Quantisation & Flux $\Phi_0$ & Circulation $\kappa$ & Matter waves \\
\midrule
Mechanism & \multicolumn{3}{c}{Partition extinction} \\
\bottomrule
\end{tabular}
\end{table}

Despite their different physical manifestations, superconductivity, superfluidity, and Bose-Einstein condensation share a common origin: the extinction of partition operations when carriers become categorically unified.

In each case:
\begin{enumerate}
\item Carriers that were distinguishable become indistinguishable
\item Partition operations that were defined become undefined
\item Transport coefficients that were finite become exactly zero
\item Macroscopic quantum coherence emerges
\end{enumerate}

The dissipationless states are not anomalies requiring special explanation. They are the natural consequence of partition dynamics when partition becomes impossible. Transport without partition is transport without dissipation.

