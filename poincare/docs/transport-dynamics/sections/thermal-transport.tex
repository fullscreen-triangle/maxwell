%==============================================================================
\section{Thermal Transport}
\label{sec:thermal}
%==============================================================================

\subsection{Heat Carrier Partition Dynamics}

Thermal transport is the flow of heat driven by temperature gradients. Heat is carried by phonons (lattice vibrations) in insulators and by both phonons and electrons in metals \citep{kittel2005,ziman1960}.

\begin{definition}[Thermal Partition]
\label{def:thermal_partition}
A thermal partition operation occurs when a heat carrier (phonon or electron) scatters, randomising its direction and equilibrating its energy with the local temperature. The scattering time $\tau_\kappa$ serves as the partition lag.
\end{definition}

For thermal transport, the heat flux $\mathbf{q} = -\kappa \nabla T$ relates to temperature gradient, where $\kappa$ is thermal conductivity. The transport coefficient is $\Tcoeff = \kappa^{-1}$.

\begin{theorem}[Thermal Conductivity]
\label{thm:thermal_conductivity}
The thermal conductivity of a material is:
\begin{equation}
\kappa^{-1} \propto \sum_{i,j} \tau_{\kappa,ij} g_{ij}
\label{eq:thermal_partition}
\end{equation}
where the sum is over carrier-scatterer pairs.
\end{theorem}

\subsection{Phonon Thermal Conductivity}

In insulators, heat is carried by phonons. The kinetic theory expression is \citep{debye1914}:
\begin{equation}
\kappa_{\text{ph}} = \frac{1}{3} C_v v_s \lambda
\label{eq:kappa_phonon}
\end{equation}
where $C_v$ is the heat capacity per unit volume, $v_s$ is the sound velocity, and $\lambda$ is the phonon mean free path.

The phonon partition lag is:
\begin{equation}
\tau_{\text{ph}} = \frac{\lambda}{v_s}
\label{eq:tau_phonon_th}
\end{equation}

With coupling $g = C_v v_s^2$:
\begin{equation}
\kappa = \frac{g}{\tau} = \frac{C_v v_s^2 \cdot \lambda}{v_s} = \frac{1}{3} C_v v_s \lambda
\label{eq:kappa_derived}
\end{equation}

reproducing the kinetic theory result.

\subsection{Phonon Scattering Mechanisms}

Phonon mean free path is limited by several scattering mechanisms:

\textbf{Umklapp scattering}: Phonon-phonon scattering that does not conserve crystal momentum \citep{peierls1929}:
\begin{equation}
\tau_U^{-1} \propto T^3 \exp\left(-\frac{\Theta_D}{bT}\right)
\label{eq:umklapp}
\end{equation}
where $b \approx 2$--$3$. This dominates at high temperatures.

\textbf{Boundary scattering}: In samples of finite size $L$, phonons scatter at boundaries:
\begin{equation}
\lambda_{\text{boundary}} \approx L
\label{eq:boundary_scatter}
\end{equation}
This dominates at low temperatures where intrinsic scattering is weak.

\textbf{Impurity scattering}: Mass defects scatter phonons with rate \citep{klemens1955}:
\begin{equation}
\tau_{\text{imp}}^{-1} \propto \omega^4
\label{eq:impurity_phonon}
\end{equation}
where $\omega$ is phonon frequency.

\subsection{Temperature Dependence of Phonon Conductivity}

The thermal conductivity of insulators shows characteristic temperature dependence:

\textbf{High $T$ ($T > \Theta_D$)}: Umklapp scattering dominates, $\kappa \propto T^{-1}$.

\textbf{Low $T$ ($T \ll \Theta_D$)}: Heat capacity $C_v \propto T^3$ (Debye model), boundary scattering dominates ($\lambda = L$), giving $\kappa \propto T^3$.

\textbf{Peak at $T \sim \Theta_D/10$}: Competition between increasing $C_v$ and decreasing $\lambda$ produces a conductivity maximum.

\subsection{Electronic Thermal Conductivity}

In metals, electrons carry heat as well as charge. The electronic thermal conductivity is \citep{sommerfeld1928}:
\begin{equation}
\kappa_e = \frac{\pi^2}{3} \frac{\kB^2 T}{e^2} \sigma = \frac{\pi^2}{3} \frac{\kB^2 T}{e^2 \rho}
\label{eq:kappa_electron}
\end{equation}

\subsection{Wiedemann-Franz Law}

The ratio of electronic thermal conductivity to electrical conductivity is \citep{wiedemann1853}:
\begin{equation}
\frac{\kappa_e}{\sigma} = LT
\label{eq:wiedemann_franz}
\end{equation}
where $L = \pi^2 \kB^2/(3e^2) = 2.44 \times 10^{-8}$ W$\cdot\Omega$/K$^2$ is the Lorenz number.

\begin{theorem}[Partition Origin of Wiedemann-Franz]
\label{thm:wf_partition}
The Wiedemann-Franz law follows from the common partition structure of electrical and thermal transport by electrons.
\end{theorem}

\begin{proof}
Both $\sigma$ and $\kappa_e$ involve the same carrier (electrons) and the same scattering mechanisms (phonons, impurities). The partition lag $\tau$ is identical for both:
\begin{align}
\sigma &= \frac{ne^2\tau}{m} \\
\kappa_e &= \frac{\pi^2 \kB^2 T}{3} \frac{n\tau}{m}
\end{align}

The ratio:
\begin{equation}
\frac{\kappa_e}{\sigma} = \frac{\pi^2 \kB^2 T}{3e^2} = LT
\end{equation}
is independent of $\tau$, material properties, and scattering details. \qed
\end{proof}

The Wiedemann-Franz law holds when the same partition operations limit both electrical and thermal transport. Deviations occur when different scattering mechanisms have different energy dependencies \citep{kumar1993}.

\subsection{Fourier's Law}

\begin{theorem}[Fourier's Law]
\label{thm:fourier}
The heat flux is proportional to the temperature gradient:
\begin{equation}
\mathbf{q} = -\kappa \nabla T
\label{eq:fourier}
\end{equation}
\end{theorem}

This is the thermal analogue of Ohm's law and Fick's law. All three constitutive relations have the same partition structure: flux proportional to gradient, with transport coefficient determined by partition lag.

\subsection{Phonon Diversity: Sizes, Shapes, and Phases}

The treatment above uses a single phonon mean free path $\lambda$, but this obscures the rich internal structure of phonon transport. Phonons are not a homogeneous population---they vary in ``size'' (energy capacity), ``shape'' (polarization and wavevector), and phase.

\subsubsection{Phonon Spectrum}

A crystal with $N$ atoms per unit cell supports $3N$ phonon branches \citep{ashcroft1976}:

\begin{enumerate}
\item \textbf{Acoustic branches} (3): Atoms in a unit cell move in phase. Low frequency, $\omega \to 0$ as $\mathbf{k} \to 0$.
\item \textbf{Optical branches} ($3N-3$): Atoms in a unit cell move out of phase. Finite frequency at $\mathbf{k} = 0$.
\end{enumerate}

Each branch further divides by polarization:
\begin{itemize}
\item \textbf{Longitudinal}: Displacement parallel to propagation (compression waves)
\item \textbf{Transverse}: Displacement perpendicular to propagation (shear waves)
\end{itemize}

The phonon ``size''---its energy capacity---is $\hbar\omega(\mathbf{k})$, which varies from zero (long-wavelength acoustic) to $\sim k_B \Theta_D$ (zone boundary).

\subsubsection{Mode-Dependent Transport}

Different phonon modes carry heat with vastly different efficiencies:

\begin{theorem}[Mode-Dependent Thermal Conductivity]
\label{thm:mode_kappa}
The total thermal conductivity is a sum over all phonon modes:
\begin{equation}
\kappa = \sum_{\lambda} \int \frac{d^3k}{(2\pi)^3} \, c_\lambda(\mathbf{k}) \, v_\lambda(\mathbf{k})^2 \, \tau_\lambda(\mathbf{k})
\label{eq:mode_kappa}
\end{equation}
where $\lambda$ labels branches, $c_\lambda$ is mode heat capacity, $v_\lambda$ is group velocity, and $\tau_\lambda$ is mode lifetime.
\end{theorem}

The mode lifetime $\tau_\lambda(\mathbf{k})$ varies by orders of magnitude:
\begin{itemize}
\item Long-wavelength acoustic phonons: $\tau \sim 10^{-9}$ s (weak scattering)
\item Zone-boundary phonons: $\tau \sim 10^{-12}$ s (strong Umklapp)
\item Optical phonons: $\tau \sim 10^{-13}$ s (rapid decay to acoustic modes)
\end{itemize}

\subsubsection{Heat Transfer Chain: A to B to C}

Consider heat flowing through a chain of molecules A $\to$ B $\to$ C. The vibration energy passed from A to B is \textit{not} the same as that transmitted from B to C:

\begin{proposition}[Non-Uniform Energy Transfer]
\label{prop:nonuniform}
Each pair of coupled oscillators has different frequency matching and coupling strength. Energy transfer depends on:
\begin{enumerate}
\item Mode overlap: Which frequencies A and B share
\item Coupling strength: How strongly those modes interact
\item Phase relationship: Whether A and B are in phase
\end{enumerate}
\end{proposition}

The energy in mode $\omega_1$ at site A may transfer to mode $\omega_2$ at site B (mode conversion), then to mode $\omega_3$ at site C. The path through phonon space is tortuous, not direct.

\subsubsection{Phase Incoherence}

At finite temperature, each oscillator vibrates with random phase relative to its neighbours. This phase incoherence is fundamental:

\begin{equation}
\langle e^{i(\phi_A - \phi_B)} \rangle = 0 \quad (T > 0)
\label{eq:phase_incoherence}
\end{equation}

Phase-matched transfer (constructive interference) occurs only transiently. Most transfer events are phase-mismatched, reducing efficiency.

This contrasts sharply with electrical conduction, where the electromagnetic signal imposes global phase coherence. In heat conduction, there is no analogous coordinating field---each oscillator has its own phase.

\subsubsection{Selective Excitation}

Not all modes are equally accessible:

\begin{enumerate}
\item \textbf{Threshold effects}: Optical phonons require $k_B T > \hbar\omega_{\text{opt}}$ for significant population
\item \textbf{Symmetry selection}: Some modes couple strongly to certain excitations, others weakly
\item \textbf{Resonance conditions}: Energy transfer peaks when $\omega_A \approx \omega_B$
\end{enumerate}

At low temperature, only long-wavelength acoustic phonons are populated. As temperature rises, higher-frequency modes become available, but they also scatter more strongly.

\subsection{Thermal Transport as Chromatography}

\begin{definition}[Thermal Chromatography]
\label{def:thermal_chrom}
Heat flow through a material is analogous to chromatography: a mixture of phonon modes (the ``analyte'') propagates through a scattering medium (the ``column''), with different modes experiencing different partition and retention.
\end{definition}

The analogy is precise:

\begin{table}[h]
\centering
\caption{Chromatography-thermal transport correspondence}
\label{tab:chromatography}
\begin{tabular}{ll}
\toprule
\textbf{Chromatography} & \textbf{Thermal Transport} \\
\midrule
Analyte mixture & Phonon population \\
Different molecular species & Different phonon modes ($\omega$, $\mathbf{k}$, $\lambda$) \\
Mobile phase & Propagating phonons \\
Stationary phase & Lattice (scattering centres) \\
Partition coefficient & Mode-dependent scattering rate $\tau^{-1}(\omega)$ \\
Retention time & Mode mean free path $\lambda(\omega) = v(\omega)\tau(\omega)$ \\
Elution profile & Spectral heat flux $q(\omega)$ \\
Column efficiency & Thermal conductivity $\kappa$ \\
\bottomrule
\end{tabular}
\end{table}

\subsubsection{Mode Separation}

Just as chromatography separates molecules by their differential affinity for the stationary phase, thermal transport ``separates'' phonon modes by their differential scattering:

\begin{itemize}
\item \textbf{Long-wavelength acoustic}: Weak scattering, long mean free path, ``elutes'' quickly (carries heat far)
\item \textbf{High-frequency acoustic}: Strong Umklapp, short mean free path, ``retained'' (carries heat short distances)
\item \textbf{Optical}: Very short lifetime, ``stuck'' at injection point (carries almost no heat)
\end{itemize}

This explains why thermal conductivity depends so sensitively on material structure. The ``column'' (crystal structure, defects, boundaries) determines how each mode is partitioned.

\subsubsection{The Phonon Spectrum as Analyte}

A temperature gradient injects a spectrum of phonons at the hot end. This spectrum is not thermal equilibrium---it contains an excess of phonons at all frequencies:

\begin{equation}
\Delta n(\omega) = \frac{\partial n_{\text{BE}}}{\partial T} \Delta T = \frac{\hbar\omega}{k_B T^2} \frac{e^{\hbar\omega/k_B T}}{(e^{\hbar\omega/k_B T} - 1)^2} \Delta T
\label{eq:injected_spectrum}
\end{equation}

This ``injected'' population propagates through the crystal. Each mode scatters at its own rate. The ``elution profile''---the spectrum of phonons arriving at the cold end---is depleted in high-frequency modes relative to low-frequency modes.

\subsubsection{Why Thermal Conductivity is Harder Than Electrical}

The chromatographic picture explains why thermal transport is fundamentally more complex than electrical:

\begin{enumerate}
\item \textbf{Single carrier vs. spectrum}: Electrical current involves one carrier type (electrons) with one relaxation time (approximately). Thermal current involves a continuous spectrum of phonon modes, each with its own dynamics.

\item \textbf{Global phase vs. incoherence}: The electromagnetic field coordinates electron motion globally. Phonons have no such coordinator---each mode propagates independently.

\item \textbf{Conserved charge vs. non-conserved phonons}: Electrons are conserved (charge conservation). Phonons are created and destroyed (thermal equilibration). The ``analyte'' changes composition during transit.

\item \textbf{Simple partition vs. mode conversion}: Electrons scatter but remain electrons. Phonons can convert between modes, transferring energy across the spectrum.
\end{enumerate}

This is why the Wiedemann-Franz law is so remarkable: it says that \textit{for electrons}, thermal and electrical partition are the same. For phonons, no such simplification exists.

\subsubsection{Nanostructure as Column Engineering}

The chromatographic perspective suggests a design principle: engineer the ``column'' to control phonon separation.

\begin{itemize}
\item \textbf{Nanoparticle inclusions}: Scatter high-frequency phonons (mass mismatch), transmit low-frequency (wavelength larger than particle)
\item \textbf{Superlattices}: Create phonon bandgaps, block certain frequency ranges
\item \textbf{Grain boundaries}: Scatter phonons with mean free path $> $ grain size
\item \textbf{Isotope disorder}: Scatter high-frequency phonons preferentially ($\tau^{-1} \propto \omega^4$)
\end{itemize}

This is thermoelectric engineering: reduce $\kappa$ while preserving $\sigma$ by scattering phonons that don't carry charge.

\subsection{Partition Structure of Phonon Transport}

From the partition framework, each phonon mode represents a distinct partition channel:

\begin{equation}
\kappa^{-1} = \sum_{\text{modes}} \frac{\tau_\omega g_\omega}{\mathcal{N}_\omega}
\label{eq:kappa_partition_modes}
\end{equation}

where the sum runs over all phonon modes, each with its own partition lag $\tau_\omega$, coupling $g_\omega$, and normalisation $\mathcal{N}_\omega$.

The chromatographic analogy makes clear that thermal transport is \textit{not} a single phenomenon but a superposition of many parallel partition processes, each with its own characteristics. Understanding thermal conductivity requires understanding this entire spectrum of partition channels---which is why it remains one of the most challenging transport properties to predict from first principles.

\subsection{Post-Hoc Phonon Characterisation}

A fundamental feature of phonon transport emerges from the partition framework: phonons can only be characterised \textit{after} the experiment, not before.

\subsubsection{Measurement-Defined Phonons}

\begin{theorem}[Phonon Measurement Identity]
\label{thm:phonon_measurement}
A phonon is not a pre-existing entity that is subsequently measured. The phonon population at any point is defined by the act of measurement---the categorical partition that distinguishes phonon states.
\end{theorem}

This aligns with the Measurement-Partition Identity established in prior work \citep{sachikonye2024loschmidt}. Before measurement, there is no fact about ``which phonons are present.'' The question is undefined because no partition has occurred to distinguish phonon states.

In practice, phonon spectroscopy (Raman, neutron scattering, etc.) creates the partition that defines the phonon population. The result depends on:
\begin{itemize}
\item The measurement technique (what partition operation is performed)
\item The measurement location (where the partition occurs)
\item The measurement timing (when the partition occurs)
\end{itemize}

Different measurements yield different phonon characterisations---not because they reveal different aspects of the same underlying reality, but because they perform different partition operations.

\subsubsection{Most Probable Pathways}

Heat does not flow through the ``optimal'' pathway---it flows through the \textit{most probable} pathway:

\begin{definition}[Thermal Path Probability]
\label{def:path_prob}
The probability of heat flowing through a particular sequence of phonon modes is:
\begin{equation}
P[\text{path}] = \prod_{\text{steps}} p(\omega_{i+1} | \omega_i, T_i)
\label{eq:path_probability}
\end{equation}
where $p(\omega_{i+1} | \omega_i, T_i)$ is the conditional probability of transitioning from mode $\omega_i$ to mode $\omega_{i+1}$ at local temperature $T_i$.
\end{definition}

The transition probability is determined by:
\begin{itemize}
\item Mode overlap (phonon density of states matching)
\item Coupling strength (anharmonic matrix elements)
\item Phase space availability (Bose-Einstein statistics)
\item Energy conservation (within thermal fluctuations)
\end{itemize}

The actual path taken is sampled from this probability distribution. It is not the path of minimum resistance, nor the path of maximum efficiency, but a typical sample from the ensemble of possible paths.

\subsubsection{Sequential Most-Probable-State Computation}

\begin{theorem}[Discretised Thermal Transport]
\label{thm:discrete_thermal}
Thermal transport can be computed as a sequence of most-probable-state determinations:
\begin{enumerate}
\item Divide the spatial domain into increments $\Delta x$
\item At each increment, given the incoming phonon spectrum and local conditions, compute the most probable outgoing spectrum
\item The heat flux is the energy carried by this most-probable spectrum
\end{enumerate}
\end{theorem}

For increment $i$ with incoming spectrum $n_{\text{in}}(\omega)$, the outgoing spectrum maximises the entropy subject to energy conservation:

\begin{equation}
n_{\text{out}}(\omega) = \argmax_{n} S[n] \quad \text{subject to} \quad \int \hbar\omega \, n(\omega) \, d\omega = Q_i
\label{eq:max_entropy_spectrum}
\end{equation}

where $Q_i$ is the heat flux through increment $i$ and $S[n]$ is the entropy of the phonon distribution.

The solution is a Bose-Einstein distribution at the local temperature $T_i$:
\begin{equation}
n_{\text{out}}(\omega) = \frac{1}{e^{\hbar\omega/k_B T_i} - 1}
\label{eq:local_equilibrium}
\end{equation}

However, this local equilibrium is only achieved if the increment $\Delta x$ exceeds the phonon mean free path $\lambda$. For $\Delta x < \lambda$, non-equilibrium distributions persist.

\subsubsection{The Cascade Picture}

Heat transport through a material is a cascade of partition events:

\begin{equation}
\text{Hot} \xrightarrow{\tau_1} \text{State}_1 \xrightarrow{\tau_2} \text{State}_2 \xrightarrow{\tau_3} \cdots \xrightarrow{\tau_n} \text{Cold}
\label{eq:thermal_cascade}
\end{equation}

Each arrow represents a partition operation with lag $\tau_i$. The state at each step is the most probable state given:
\begin{enumerate}
\item The preceding state
\item The local temperature
\item The local lattice structure
\item The scattering mechanisms present
\end{enumerate}

This cascade picture explains several phenomena:

\textbf{Thermal relaxation}: The system ``forgets'' the initial phonon distribution over a relaxation length $\ell_{\text{relax}} \sim \lambda$. Beyond this distance, only the total energy (temperature) is remembered, not the spectral details.

\textbf{Ballistic-to-diffusive transition}: For $L < \lambda$, phonons traverse the sample without scattering (ballistic). For $L \gg \lambda$, many partition events occur (diffusive). The transition occurs when sample size matches mean free path.

\textbf{Kapitza resistance}: At interfaces, the lattice structure changes abruptly. The most-probable-state on each side may not match, creating thermal resistance even without bulk scattering.

\subsubsection{Connection to Virtual Instruments}

The post-hoc characterisation principle connects directly to virtual instrumentation. Since phonons are defined by measurement:

\begin{proposition}[Virtual Phonon Spectroscopy]
\label{prop:virtual_phonon}
Phonon spectra can be determined virtually by computing the most probable categorical state at each measurement point, using hardware oscillations as the partition mechanism.
\end{proposition}

We do not simulate phonon trajectories---we compute the most probable phonon population given the boundary conditions and local structure. The hardware oscillation (CPU cycles, memory access patterns) performs the categorical partition that defines the result.

This is not an approximation to ``real'' phonon dynamics. It \textit{is} phonon measurement, performed categorically rather than through physical probes.

\subsubsection{Implications for Thermal Conductivity Prediction}

The sequential most-probable-state picture suggests a computational approach:

\begin{enumerate}
\item Discretise the material into cells of size comparable to phonon mean free path
\item For each cell, characterise the local lattice structure and scattering mechanisms
\item Compute the most probable phonon spectrum in each cell given the neighbours
\item Extract the heat flux from the spectral flow between cells
\item Sum to obtain total thermal conductivity
\end{enumerate}

This approach naturally handles:
\begin{itemize}
\item Inhomogeneous materials (different cell properties)
\item Nanostructures (cell size comparable to feature size)
\item Interfaces (boundary conditions between dissimilar cells)
\item Non-equilibrium effects (cells not at local equilibrium)
\end{itemize}

The key insight is that we need not track individual phonon trajectories. We need only find the most probable state at each spatial increment---a much more tractable problem that aligns with the categorical measurement framework.

