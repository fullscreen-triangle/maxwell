\section{Forces from Cross-Scale Oscillatory Coupling}

\subsection{Hierarchical Mode Coupling}

Forces emerge from the coupling between oscillatory modes at different hierarchical levels.

\begin{definition}[Mode Coupling]
Two oscillatory modes $\omega_a$ and $\omega_b$ are \textbf{coupled} if the interaction Hamiltonian contains terms:
\begin{equation}
\mathcal{H}_{\text{int}} = g_{ab} A_a B_b + \text{h.c.}
\end{equation}
where $A_a$, $B_b$ are operators for modes $a$ and $b$, and $g_{ab}$ is the coupling strength.
\end{definition}

\begin{theorem}[Resonance Enhancement]\label{thm:resonance}
Mode coupling is enhanced when frequencies are commensurate:
\begin{equation}
n\omega_a = m\omega_b \quad (n, m \in \mathbb{Z}^+)
\end{equation}
Off-resonant coupling is suppressed by the frequency mismatch.
\end{theorem}

\begin{proof}
The transition amplitude between modes involves:
\begin{equation}
\langle b | \mathcal{H}_{\text{int}} | a \rangle \sim \int_0^T e^{i(\omega_b - \omega_a)t} dt
\end{equation}

For $\omega_a \neq \omega_b$:
\begin{equation}
\left| \int_0^T e^{i\Delta\omega t} dt \right| = \left| \frac{e^{i\Delta\omega T} - 1}{i\Delta\omega} \right| = \frac{2|\sin(\Delta\omega T/2)|}{|\Delta\omega|} \leq \frac{2}{|\Delta\omega|}
\end{equation}

For $\omega_a = \omega_b$ (resonance):
\begin{equation}
\left| \int_0^T dt \right| = T
\end{equation}

Resonant coupling grows linearly with time; off-resonant coupling is bounded. This explains why interactions are strongest between modes of matching or harmonically related frequencies. \qed
\end{proof}

\subsection{Effective Force from Mode Interaction}

\begin{definition}[Effective Force]
The \textbf{effective force} between oscillatory configurations is the gradient of the interaction energy:
\begin{equation}
\mathbf{F} = -\nabla \langle \mathcal{H}_{\text{int}} \rangle
\end{equation}
\end{definition}

\begin{theorem}[Force from Coupling]\label{thm:force-coupling}
The force between configurations at separation $r$ depends on the mode overlap:
\begin{equation}
F(r) \propto \int d\omega \, g^2(\omega) |\psi_a(r)|^2 |\psi_b(r)|^2
\end{equation}
where $\psi_a$, $\psi_b$ are the mode wavefunctions.
\end{theorem}

\begin{proof}
The interaction energy between two configurations is:
\begin{equation}
E_{\text{int}} = \sum_{\omega} g(\omega) n_a(\omega) n_b(\omega) V(r)
\end{equation}
where $n_a(\omega)$, $n_b(\omega)$ are occupation numbers and $V(r)$ is the position-dependent potential.

Taking the gradient:
\begin{equation}
F(r) = -\frac{\partial E_{\text{int}}}{\partial r} = -\sum_\omega g(\omega) n_a n_b \frac{\partial V}{\partial r}
\end{equation}

The force depends on which modes are occupied and how they overlap spatially. \qed
\end{proof}

\subsection{Electromagnetic Interaction}

\begin{theorem}[Electromagnetic Force from Charge Coupling]
Configurations with non-zero net mode occupation (``charge'') interact via:
\begin{equation}
F_{\text{EM}} = \frac{q_1 q_2}{4\pi\epsilon_0 r^2}
\end{equation}
where $q = Ne$ is the net mode occupation.
\end{theorem}

\begin{proof}
Define ``charge'' as the net asymmetry in mode occupation:
\begin{equation}
q = e \sum_n (N_n^+ - N_n^-)
\end{equation}
where $N_n^+$ and $N_n^-$ are occupation numbers for positive and negative chirality modes.

The coupling between charges is mediated by photonic modes (electromagnetic oscillations). In the low-frequency limit, the interaction reduces to Coulomb's law:
\begin{equation}
V(r) = \frac{q_1 q_2}{4\pi\epsilon_0 r}
\end{equation}

The $1/r$ dependence follows from the inverse-square law for massless mediators in three dimensions:
\begin{equation}
\nabla^2 V = -\frac{\rho}{\epsilon_0} \implies V \sim \frac{1}{r}
\end{equation}
\qed
\end{proof}

\begin{remark}
The electromagnetic force emerges from mode coupling mediated by photonic oscillations. The inverse-square law is a consequence of three-dimensional geometry derived earlier.
\end{remark}

\subsection{Strong and Weak Interactions}

\begin{theorem}[Short-Range Forces from Massive Mediators]
Forces mediated by massive oscillatory modes have finite range:
\begin{equation}
V(r) \sim \frac{e^{-mr/\hbar}}{r}
\end{equation}
where $m$ is the mediator mass.
\end{theorem}

\begin{proof}
Massive mediators satisfy the Klein-Gordon equation:
\begin{equation}
(\Box + m^2/\hbar^2)\phi = 0
\end{equation}

The static solution is the Yukawa potential:
\begin{equation}
\phi(r) = \frac{g}{4\pi r} e^{-mr/\hbar}
\end{equation}

The range $\lambda = \hbar/mc$ (Compton wavelength) determines the force reach. For heavy mediators, the force is short-ranged. \qed
\end{proof}

\begin{remark}
The strong nuclear force ($\lambda \sim 1$ fm) and weak force ($\lambda \sim 10^{-3}$ fm) emerge from massive mediator exchange. Their short range is a consequence of mediator mass, not fundamental asymmetry.
\end{remark}

\subsection{Gravitational Interaction}

\begin{theorem}[Gravity from Mode Distribution]
All oscillatory modes couple to the energy-momentum distribution, producing universal attraction:
\begin{equation}
F_{\text{grav}} = -\frac{G m_1 m_2}{r^2}
\end{equation}
\end{theorem}

\begin{proof}
Every oscillatory mode carries energy $E = \hbar\omega$ and thus mass $m = E/c^2$. Energy-momentum creates curvature in the mode space metric.

For weak fields, the metric perturbation satisfies:
\begin{equation}
\nabla^2 h_{00} = \frac{16\pi G}{c^4} T_{00} = \frac{16\pi G}{c^4} \rho c^2
\end{equation}

The solution gives Newtonian potential:
\begin{equation}
\Phi = -\frac{Gm}{r}
\end{equation}

Gravity is universal because all modes carry energy, and energy couples universally to geometry. \qed
\end{proof}

\subsection{Coupling Hierarchy}

\begin{theorem}[Force Hierarchy from Frequency Scales]
The relative strengths of forces scale with the characteristic frequencies of their mediators:
\begin{equation}
\frac{\alpha_1}{\alpha_2} \sim \frac{\omega_1^2}{\omega_2^2}
\end{equation}
where $\alpha$ is the coupling constant and $\omega$ is the mediator frequency.
\end{theorem}

\begin{proof}
Coupling strength depends on the overlap integral between matter modes and mediator modes. High-frequency mediators probe small distance scales with stronger coupling; low-frequency mediators probe large scales with weaker coupling.

For electromagnetic coupling $\alpha_{\text{EM}} \approx 1/137$ and gravitational coupling $\alpha_G \sim 10^{-39}$ (for protons):
\begin{equation}
\frac{\alpha_{\text{EM}}}{\alpha_G} \sim 10^{37} \sim \left(\frac{m_p}{m_{\text{Planck}}}\right)^2
\end{equation}

This hierarchy reflects the frequency scales of electromagnetic vs. gravitational mediators. \qed
\end{proof}

\subsection{Summary}

We have established:

\begin{enumerate}
    \item Forces emerge from oscillatory mode coupling
    \item Resonance enhances coupling between commensurate frequencies
    \item Electromagnetic force follows from charge (mode occupation asymmetry)
    \item Short-range forces arise from massive mediators
    \item Gravity couples universally to energy-momentum
    \item Force hierarchy reflects mediator frequency scales
\end{enumerate}

All fundamental forces emerge from the cross-scale structure of oscillatory mode coupling.

