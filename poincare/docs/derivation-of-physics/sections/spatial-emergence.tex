\section{Spatial Structure from Partition Geometry}

\subsection{The Emergence Problem}

Traditional physics treats space as a primitive---an arena in which dynamics occurs. We demonstrate that three-dimensional spatial structure emerges from partition geometry rather than being independently postulated.

\subsection{Angular Coordinates and Spherical Structure}

\begin{theorem}[Three-Dimensional Emergence]\label{thm:3d-emergence}
The angular partition coordinates $(l, m)$ with $l \in \{0, 1, ..., n-1\}$ and $m \in \{-l, ..., +l\}$ generate three-dimensional spherical structure.
\end{theorem}

\begin{proof}
The coordinate $l$ labels irreducible representations of $SO(3)$, the rotation group in three dimensions. The coordinate $m$ labels basis vectors within each representation.

The spherical harmonics $Y_l^m(\theta, \phi)$ form a complete orthonormal basis for functions on the 2-sphere $S^2$:
\begin{equation}
\int_0^{2\pi} d\phi \int_0^{\pi} \sin\theta \, d\theta \, Y_l^m(\theta, \phi)^* Y_{l'}^{m'}(\theta, \phi) = \delta_{ll'} \delta_{mm'}
\end{equation}

The angular coordinates $(\theta, \phi)$ parameterise $S^2$. Combined with the radial coordinate $r$ (related to partition depth $n$), this generates $\mathbb{R}^3$:
\begin{equation}
(r, \theta, \phi) \in [0, \infty) \times [0, \pi] \times [0, 2\pi) \cong \mathbb{R}^3
\end{equation}

The partition coordinates $(n, l, m)$ thus implicitly contain three-dimensional spatial structure through the angular momentum algebra. \qed
\end{proof}

\subsection{Radial Extension}

\begin{definition}[Radial Coordinate]
The \textbf{radial coordinate} $r$ is related to partition depth $n$ by:
\begin{equation}
r_n = a_0 n^2 / Z
\end{equation}
where $a_0$ is the characteristic length scale (analogous to Bohr radius) and $Z$ is an effective charge parameter.
\end{definition}

\begin{theorem}[Discrete Radial Structure]
The radial coordinate takes discrete values determined by partition depth:
\begin{equation}
r \in \{r_1, r_2, r_3, ...\} \quad \text{with} \quad r_n \propto n^2
\end{equation}
\end{theorem}

\begin{proof}
In hydrogen-like systems, the most probable radius scales as:
\begin{equation}
\langle r \rangle_{n,l} = \frac{a_0}{Z} \left[ \frac{3n^2 - l(l+1)}{2} \right]
\end{equation}

For fixed $l$, the dominant scaling is $\langle r \rangle \propto n^2$.

This quadratic scaling arises from the balance between kinetic and potential energy:
\begin{equation}
E \sim -\frac{1}{n^2} \implies r \sim n^2
\end{equation}
(using the virial theorem $E \sim -1/r$). \qed
\end{proof}

\begin{remark}
The $n^2$ scaling of orbital radii is a well-known feature of hydrogen-like atoms. Here it emerges from energy-radius relationships rather than being solved from the Schr\"{o}dinger equation.
\end{remark}

\subsection{Metric Structure}

\begin{definition}[Partition Metric]
The \textbf{partition metric} on the space of categorical states is:
\begin{equation}
d((n, l, m, s), (n', l', m', s')) = \sqrt{\alpha_n (n - n')^2 + \alpha_l (l - l')^2 + \alpha_m (m - m')^2 + \alpha_s (s - s')^2}
\end{equation}
where $\alpha_n, \alpha_l, \alpha_m, \alpha_s$ are scale factors.
\end{definition}

\begin{theorem}[Metric Consistency]
The partition metric induces a spatial metric consistent with three-dimensional Euclidean geometry.
\end{theorem}

\begin{proof}
In the continuum limit where partition coordinates become continuous:
\begin{align}
n &\to r/a_0 \\
l &\to L/\hbar \\
m &\to L_z/\hbar
\end{align}

The metric becomes:
\begin{equation}
ds^2 = dr^2 + r^2(d\theta^2 + \sin^2\theta \, d\phi^2)
\end{equation}

This is the standard Euclidean metric in spherical coordinates. The partition coordinate structure naturally generates three-dimensional Euclidean geometry. \qed
\end{proof}

\subsection{Dimensionality from Partition Constraints}

\begin{theorem}[Dimensional Necessity]\label{thm:dimensionality}
The constraint structure of partition coordinates $(n, l, m, s)$ uniquely determines spatial dimensionality $D = 3$.
\end{theorem}

\begin{proof}
Consider generalisation to arbitrary dimension $D$. The angular momentum in $D$ dimensions is characterised by $\lfloor D/2 \rfloor$ quantum numbers.

For $D = 1$: No angular momentum (trivial rotation group).

For $D = 2$: One angular quantum number $m \in \mathbb{Z}$ (rotation in a plane).

For $D = 3$: Two angular quantum numbers $(l, m)$ with $l \geq 0$ and $-l \leq m \leq l$ (full $SO(3)$ structure).

For $D = 4$: Three angular quantum numbers, corresponding to $SO(4) \cong SU(2) \times SU(2)$.

The partition coordinate structure $(n, l, m, s)$ has exactly two angular parameters $(l, m)$ with the constraint $-l \leq m \leq l$. This is the unique signature of $D = 3$.

Furthermore, the chirality $s = \pm 1/2$ corresponds to $\pi_1(SO(3)) = \mathbb{Z}_2$, which is specific to $D = 3$ (for $D > 3$, $\pi_1(SO(D)) = \mathbb{Z}_2$ as well, but the representation structure differs).

Therefore, the constraint structure uniquely specifies $D = 3$. \qed
\end{proof}

\begin{remark}
This result is striking: we did not assume three-dimensional space but derived it from the partition coordinate constraints. The dimensionality of space is not a free parameter but a consequence of the oscillatory partition structure.
\end{remark}

\subsection{Locality and Spatial Separation}

\begin{definition}[Spatial Separation]
Two categorical states $(n_1, l_1, m_1, s_1)$ and $(n_2, l_2, m_2, s_2)$ have \textbf{spatial separation}:
\begin{equation}
|\Delta \mathbf{r}| = |r_{n_1} - r_{n_2}|
\end{equation}
when $l_1 = l_2$ and $m_1 = m_2$ (same angular orientation).
\end{definition}

\begin{theorem}[Locality Principle]
Interactions between categorical states decrease with spatial separation:
\begin{equation}
V(|\Delta \mathbf{r}|) \to 0 \quad \text{as} \quad |\Delta \mathbf{r}| \to \infty
\end{equation}
for any finite-range interaction.
\end{theorem}

\begin{proof}
Interactions are mediated by overlap integrals of the form:
\begin{equation}
\langle n_1, l_1, m_1 | V | n_2, l_2, m_2 \rangle = \int d^3r \, \psi_{n_1 l_1 m_1}^*(\mathbf{r}) V(\mathbf{r}) \psi_{n_2 l_2 m_2}(\mathbf{r})
\end{equation}

For states with $|n_1 - n_2| \gg 1$, the radial wavefunctions are localised at different radii ($r_1 \sim n_1^2$, $r_2 \sim n_2^2$), and the overlap is exponentially suppressed:
\begin{equation}
\langle n_1 | n_2 \rangle \sim e^{-|n_1 - n_2|/\xi}
\end{equation}
where $\xi$ is a coherence length.

This spatial locality emerges from the radial structure of partition coordinates, not from an independent locality postulate. \qed
\end{proof}

\subsection{Connection to General Relativity}

\begin{remark}
While this paper focuses on flat spatial structure, the partition geometry framework suggests a path to curved spacetime. If the partition depth $n$ and angular parameters $(l, m)$ vary smoothly with position, the induced metric can deviate from Euclidean:
\begin{equation}
g_{\mu\nu} = \eta_{\mu\nu} + h_{\mu\nu}(n, l, m)
\end{equation}
where $h_{\mu\nu}$ encodes curvature arising from variations in the partition structure. This connection to general relativity remains to be developed in full detail.
\end{remark}

\subsection{Summary}

We have established:

\begin{enumerate}
    \item Three-dimensional structure emerges from angular coordinates $(l, m)$
    \item Radial extension follows $r \propto n^2$ from energy-radius relations
    \item The partition metric reduces to Euclidean in the continuum limit
    \item Dimensionality $D = 3$ is uniquely determined by constraint structure
    \item Locality follows from radial separation of partition shells
\end{enumerate}

Space is not a primitive arena but emerges from the geometry of bounded oscillatory partitions.

