\section{Spatial Structure from Partition Geometry}
\label{sec:spatial}

\subsection{The Emergence Problem}

Traditional physics treats space as a primitive—an arena in which dynamics occur. We now demonstrate that three-dimensional spatial structure emerges from partition geometry rather than being independently postulated.

\begin{remark}
This is a profound shift in perspective. We are not asking, "What happens in space?" but rather, "Why does space exist, and why does it have three dimensions?"

The answer: space is the geometric structure induced by partition coordinates $(n, l, m, s)$.
\end{remark}

\subsection{Angular Coordinates and Spherical Structure}

\begin{theorem}[Three-Dimensional Emergence]
\label{thm:3d-emergence}
The angular partition coordinates $(l, m)$ with constraints $0 \leq l \leq n-1$ and $-l \leq m \leq l$ necessarily generate a three-dimensional spherical structure.
\end{theorem}

\begin{proof}
The coordinate $l$ labels irreducible representations of the rotation group. The coordinate $m$ labels basis vectors within each representation.

The key question is: Which rotation group?

The constraint $m \in \{-l, \ldots, +l\}$ giving $(2l+1)$ values is the signature of $SO(3)$, the rotation group in three dimensions.

More precisely:
\begin{itemize}
    \item In $D=2$: rotations form $SO(2) \cong U(1)$, with representations labelled by a single integer $m \in \mathbb{Z}$
    \item In $D=3$: rotations form $SO(3)$, with representations labelled by $l \in \mathbb{Z}_{\geq 0}$ and $m \in \{-l, \ldots, +l\}$
    \item In $D=4$: rotations form $SO(4) \cong SU(2) \times SU(2)$, requiring two independent labels $(l_1, l_2)$
\end{itemize}

The partition coordinate structure $(l, m)$ with constraint $|m| \leq l$ is unique to $SO(3)$.

The angular functions are spherical harmonics $Y_l^m(\theta, \phi)$, which form a complete orthonormal basis for functions on the 2-sphere $S^2$:
\begin{equation}
\int_0^{2\pi} d\phi \int_0^{\pi} \sin\theta \, d\theta \, Y_l^m(\theta, \phi)^* Y_{l'}^{m'}(\theta, \phi) = \delta_{ll'} \delta_{mm'}
\end{equation}

The angular coordinates $(\theta, \phi)$ parameterize $S^2$, the surface of a sphere in three-dimensional space.

Combined with the radial coordinate $r$ (related to partition depth $n$), this generates $\mathbb{R}^3$:
\begin{equation}
(r, \theta, \phi) \in [0, \infty) \times [0, \pi] \times [0, 2\pi) \cong \mathbb{R}^3 \setminus \{0\}
\end{equation}

Therefore, the partition coordinates $(n, l, m)$ implicitly encode three-dimensional spatial structure through the angular momentum algebra. \qed
\end{proof}

\begin{remark}
This is a remarkable result: we did not assume three-dimensional space. We derived it from:
\begin{enumerate}
    \item Bounded oscillatory dynamics (Section~\ref{sec:oscillatory})
    \item Categorical partitioning (Section~\ref{sec:categorical})
    \item Angular constraint $l < n$ (Theorem~\ref{thm:angular-constraint})
    \item Orientation count $(2l+1)$ (Theorem~\ref{thm:orientation-count})
\end{enumerate}

The dimensionality of space is not a free parameter but a logical consequence of partition geometry.
\end{remark}

\begin{figure}[htbp]
\centering
\includegraphics[width=\textwidth]{figures/fig3_partition_spatial.png}
\caption{\textbf{Derivation of Three-Dimensional Euclidean Space from Partition Constraints.}
\textbf{(A)} Partition coordinate system $(n, l, m, s)$ represents discrete states in nested oscillatory structure. The 3D scatter plot shows states organized by depth $n$ (radial axis), angular complexity $l$ (angular axis), and orientation $m$ (azimuthal axis), with colors indicating different shells and chirality $s$ implicit in each point.
\textbf{(B)} Geometric constraints arise from nested boundary structure: depth $n \in \mathbb{Z}^+$ (at least one partition), angular quantum number $l \in \{0, 1, \ldots, n-1\}$ (angular complexity limited by depth), orientation $m \in \{-l, \ldots, +l\}$ (orientation range determined by angular structure), and chirality $s \in \{\pm\frac{1}{2}\}$ (boundary phase). These constraints yield exact capacity $2n^2$ distinguishable states per shell, producing the sequence $2, 8, 18, 32, 50, \ldots$ with no adjustable parameters.
\textbf{(C)} Angular structure emerges from spherical harmonic $Y_2^1(\theta, \phi)$ showing spatial probability distribution. The blue (positive) and red (negative) lobes demonstrate how quantum numbers $(l,m)$ naturally produce three-dimensional angular patterns, with nodal structure determined by angular momentum constraints.
\textbf{(D)} Mapping from partition coordinates to spatial structure: $l \in \{0,1,\ldots,n-1\}$ labels SO(3) irreducible representations, $m \in \{-l,\ldots,+l\}$ gives $(2l+1)$ orientation states forming spherical harmonics $Y_l^m(\theta,\phi)$, and radial coordinate scales as $\langle r \rangle \propto n^2$. Together, $(l,m)$ specify angular coordinates $(\theta,\phi)$ and $n$ determines radial extension $r$, producing complete 3D spatial parameterization.
\textbf{(E)} Radial extension follows $\langle r \rangle \propto n^2$ scaling, analogous to Bohr model but derived from partition depth. Concentric circles show shells for $n = 1, 2, 3, 4$ with radii $r \propto 1, 4, 9, 16$, demonstrating that spatial distance emerges from categorical depth rather than being assumed a priori.
\textbf{(F)} Dimensionality theorem: the constraint structure with exactly two angular quantum numbers $(l,m)$ uniquely specifies SO(3) rotational symmetry, which is the symmetry group of three-dimensional Euclidean space. This derives $D = 3$ as a mathematical necessity from partition constraints rather than assuming dimensionality, showing that three-dimensionality is the unique signature of nested oscillatory boundaries with angular structure.}
\label{fig:partition_spatial}
\end{figure}

\subsection{Radial Structure}

\begin{definition}[Radial Coordinate]
The \textbf{radial coordinate} $r$ is related to partition depth $n$ by:
\begin{equation}
r_n = a_0 \frac{n^2}{Z}
\end{equation}
where $a_0$ is a characteristic length scale and $Z$ is an effective charge parameter.
\end{definition}

\begin{theorem}[Discrete Radial Structure]
\label{thm:radial-structure}
The radial coordinate takes discrete values determined by partition depth:
\begin{equation}
r \in \{r_1, r_2, r_3, \ldots\} \quad \text{with} \quad r_n \propto n^2
\end{equation}
\end{theorem}

\begin{proof}
For oscillatory systems in a central potential $V(r) \sim -1/r$, the energy scales as:
\begin{equation}
E_n \sim -\frac{1}{n^2}
\end{equation}

By the virial theorem for Coulomb-like potentials:
\begin{equation}
\langle T \rangle = -\frac{1}{2}\langle V \rangle, \quad E = \langle T \rangle + \langle V \rangle = \frac{1}{2}\langle V \rangle
\end{equation}

Therefore:
\begin{equation}
E_n \sim -\frac{1}{\langle r \rangle_n} \implies \langle r \rangle_n \sim \frac{1}{|E_n|} \sim n^2
\end{equation}

The most probable radius for state $(n, l)$ is:
\begin{equation}
r_{n,l}^{\text{max}} = \frac{a_0}{Z} \left[ n^2 - \frac{l(l+1)}{2} + \mathcal{O}(1) \right]
\end{equation}

The dominant scaling is $r \propto n^2$. \qed
\end{proof}

\begin{remark}
The $n^2$ scaling of orbital radii is well-known in atomic physics. Here it emerges from:
\begin{enumerate}
    \item Energy quantization $E_n \sim 1/n^2$ (from oscillatory necessity)
    \item Virial theorem (from potential structure)
\end{enumerate}

We have derived it from partition geometry, not from solving the Schrödinger equation.
\end{remark}

\subsection{Metric Structure}

The partition coordinates induce a metric on the space of categorical states.

\begin{definition}[Partition Metric]
The \textbf{partition metric} on categorical state space is:
\begin{equation}
d^2((n, l, m, s), (n', l', m', s')) = \alpha_n (n - n')^2 + \alpha_l (l - l')^2 + \alpha_m (m - m')^2 + \alpha_s (s - s')^2
\end{equation}
where $\{\alpha_i\}$ are scale factors with dimensions of length squared.
\end{definition}

\begin{theorem}[Euclidean Emergence]
\label{thm:euclidean-emergence}
In the continuum limit, the partition metric reduces to the Euclidean metric in three-dimensional spherical coordinates.
\end{theorem}

\begin{proof}
In the continuum limit, partition coordinates become continuous:
\begin{align}
n &\to r/a_0 \quad (\text{radial coordinate}) \\
l &\to L/\hbar \quad (\text{angular momentum magnitude}) \\
m &\to L_z/\hbar \quad (\text{angular momentum projection})
\end{align}

The angular momentum components relate to angles via:
\begin{equation}
L = \hbar\sqrt{l(l+1)} \approx \hbar l \quad \text{for large } l
\end{equation}

The classical relation between angular momentum and angular velocity gives:
\begin{equation}
L = mr^2\omega \implies \omega = \frac{L}{mr^2}
\end{equation}

Angular displacements are:
\begin{equation}
d\theta \sim \frac{dL}{mr^2}, \quad d\phi \sim \frac{dL_z}{mr^2\sin\theta}
\end{equation}

Substituting into the partition metric:
\begin{align}
ds^2 &= \alpha_n (dn)^2 + \alpha_l (dl)^2 + \alpha_m (dm)^2 \\
&= \alpha_n \left(\frac{dr}{a_0}\right)^2 + \alpha_l \left(\frac{mr^2 d\theta}{\hbar}\right)^2 + \alpha_m \left(\frac{mr^2\sin\theta d\phi}{\hbar}\right)^2
\end{align}

Choosing scale factors:
\begin{equation}
\alpha_n = a_0^2, \quad \alpha_l = \alpha_m = \frac{\hbar^2}{m^2r^4}
\end{equation}

gives:
\begin{equation}
ds^2 = dr^2 + r^2 d\theta^2 + r^2\sin^2\theta \, d\phi^2
\end{equation}

This is the Euclidean metric in spherical coordinates. \qed
\end{proof}

\begin{remark}
This is extraordinary: Euclidean geometry emerges from the partition coordinate structure. We did not assume a metric—it arose from the natural distance measure on categorical states.

Space is not a background structure but rather the geometric realisation of partition coordinates.
\end{remark}

\subsection{Dimensional Uniqueness}

We now prove that the dimensionality of space is uniquely determined.

\begin{theorem}[Dimensional Necessity]
\label{thm:dimensionality}
The constraint structure of partition coordinates $(n, l, m, s)$ uniquely determines spatial dimensionality $D = 3$.
\end{theorem}

\begin{proof}
Consider the general structure of angular momentum in $D$ dimensions.

For $D = 1$: No rotations are possible. No angular quantum numbers.

For $D = 2$: Rotations form $SO(2) \cong U(1)$. One angular quantum number $m \in \mathbb{Z}$ (winding number). No constraint on $m$.

For $D = 3$: Rotations form $SO(3)$. Two angular quantum numbers $(l, m)$ with a constraint $|m| \leq l$. This gives $(2l+1)$-dimensional representations.

For $D = 4$: Rotations form $SO(4) \cong SU(2) \times SU(2)$. Two independent angular momenta $(j_1, j_2)$ with no constraints between them. Representations have dimension $(2j_1+1)(2j_2+1)$.

For $D \geq 5$: Rotations form $SO(D)$ with $\lfloor D/2 \rfloor$ Casimir operators, requiring multiple quantum numbers.

The partition coordinate structure has:
\begin{itemize}
    \item Exactly two angular parameters: $l$ and $m$
    \item A constraint between them: $|m| \leq l$
    \item Representation dimension $(2l+1)$ (not a product)
\end{itemize}

This structure is unique to $SO(3)$, hence $D = 3$.

Furthermore, the chirality parameter $s = \pm 1/2$ arises from:
\begin{equation}
\pi_1(SO(3)) = \mathbb{Z}_2
\end{equation}

For $D = 2$: $\pi_1(SO(2)) = \mathbb{Z}$ (infinite fundamental group)

For $D = 4$: $\pi_1(SO(4)) = \mathbb{Z}_2$, but the representation structure differs (product of two $SU(2)$ representations)

For $D \geq 5$: $\pi_1(SO(D)) = \mathbb{Z}_2$, but the representation structure is different

The combined structure $(l, m, s)$ with constraints:
\begin{equation}
0 \leq l < n, \quad |m| \leq l, \quad s = \pm 1/2
\end{equation}
uniquely specifies $D = 3$. \qed
\end{proof}

\begin{remark}
This theorem answers one of the deepest questions in physics: Why does space have three dimensions?

The answer is not anthropic ("because we exist") or contingent ("it just does"). It is logical: the partition coordinate structure that emerges from bounded oscillatory dynamics necessarily generates three-dimensional space.

Higher or lower dimensions are not merely "different possibilities"—they are inconsistent with the partition coordinate constraints derived in Section~\ref{sec:partition}.
\end{remark}


\begin{figure}[htbp]
\centering
\includegraphics[width=\textwidth]{figures/spatial_matter_panel.png}
\caption{\textbf{Spatial Structure, Matter, and Energy from Partition Coordinates.}
\textbf{Top row, left:} 3D structure from angular quantum numbers $(l,m)$ showing spherical harmonic $Y_{l}^{m}(\theta,\phi)$. Cyan wireframe demonstrates that $(l,m)$ produce spherical coordinates through SO(3) representation.
\textbf{Top row, center-left:} Radial extension scaling $r \propto n^2$ (Bohr scaling). Concentric circles for $n = 1,2,3,4,5,6,7$ show $\langle r \rangle = n^2 a_0$, establishing that spatial distance emerges from partition depth.
\textbf{Top row, center-right:} Dimensionality from partition constraints uniquely determines $D = 3$. Bar chart shows angular quantum numbers: $D = 1$ (1 number), $D = 2$ (2 numbers), $D = 3$ (2 numbers with unique structure). Constraint $l \in \{0,\ldots,n-1\}$ and $m \in \{-l,\ldots,+l\}$ uniquely specifies SO(3) symmetry.
\textbf{Top row, right:} Locality principle shows exponential decay of overlap $\langle n_1 | n_2 \rangle$ from $10^0$ to $10^{-4}$ as separation $|n_1 - n_2|$ increases from 0 to 5.
\textbf{Second row, left:} Mode occupation shows 6 occupied states out of 100 (6\%). Blue squares represent occupied modes with energy $E = \hbar\omega$; sparse occupation matches cosmic matter fraction ($\sim 5\%$), establishing visible matter as occupied modes.
\textbf{Second row, center-left:} Exclusion principle enforces maximum two fermions per orbital with opposite spin. Diagram shows electron distribution in $1s$, $2s$, $2p$, $3s$ orbitals; Pauli constraint $n_i \leq 2$ per state $(n,l,m)$ arises from antisymmetry.
\textbf{Second row, center-right:} Mass-frequency identity $m = \hbar\omega/c^2$ shows linear relationship. Blue line demonstrates mass increases from electron ($\sim 10^{-30}$ kg) through muon to proton ($\sim 10^{-27}$ kg) with frequency $\omega$.
\textbf{Second row, right:} Cosmic mode occupation shows 95\%/5\% split: dark energy 68\% (dark gray), dark matter 27\% (purple), ordinary matter 5\% (blue).
\textbf{Third row, left:} Wave-particle duality shows mode (blue sinusoidal wave) versus occupation (red dashed line with filled circle at $x \sim 5$), representing oscillatory structure versus localized particle.
\textbf{Third row, center-left:} Energy conservation $dE/dt = 0$ shows kinetic $E_k$ (blue) and potential $E_p$ (red) oscillating out of phase while total $E = E_k + E_p$ (black) remains constant at $\sim 2.5$ units, following from time-translation symmetry via Noether's theorem.
\textbf{Third row, center-right:} Mode occupation statistics: Fermi-Dirac $f_{FD}(E) = 1/(e^{(E-\mu)/k_BT} + 1)$ (blue, maximum $f = 1$) versus Bose-Einstein $f_{BE}(E) = 1/(e^{(E-\mu)/k_BT} - 1)$ (red, allows $f > 1$). Dashed line shows $\mu = 2.0$; fermions obey exclusion, bosons permit multiple occupation.
\textbf{Third row, right:} Vacuum energy from unoccupied modes. Purple curve increases from 0 to $\sim 500$ as frequency increases to 10, while occupied modes (blue) contribute only $\sim 50$ units. Vacuum dominates because mode density $\rho(\omega) \propto \omega^2$ grows quadratically.}
\label{fig:spatial_matter_panel}
\end{figure}

\subsection{Locality and Spatial Separation}

\begin{definition}[Spatial Separation]
Two categorical states $(n_1, l_1, m_1, s_1)$ and $(n_2, l_2, m_2, s_2)$ have spatial separation:
\begin{equation}
|\Delta \mathbf{r}| = |r_{n_1} - r_{n_2}|
\end{equation}
when they share the same angular orientation: $l_1 = l_2$ and $m_1 = m_2$.
\end{definition}

\begin{theorem}[Locality Principle]
\label{thm:locality}
Interactions between categorical states decrease with spatial separation:
\begin{equation}
V(|\Delta \mathbf{r}|) \to 0 \quad \text{as} \quad |\Delta \mathbf{r}| \to \infty
\end{equation}
for any finite-range interaction.
\end{theorem}

\begin{proof}
Interactions are mediated by overlap integrals:
\begin{equation}
\langle n_1, l_1, m_1 | \hat{V} | n_2, l_2, m_2 \rangle = \int d^3r \, \psi_{n_1 l_1 m_1}^*(\mathbf{r}) V(\mathbf{r}) \psi_{n_2 l_2 m_2}(\mathbf{r})
\end{equation}

For states with $\Delta n = |n_1 - n_2| \gg 1$, the radial wavefunctions are localised at different radii:
\begin{equation}
r_1 \sim n_1^2 a_0, \quad r_2 \sim n_2^2 a_0
\end{equation}

The spatial separation is:
\begin{equation}
|\Delta r| = |r_1 - r_2| \sim (n_1^2 - n_2^2) a_0 \sim 2n\Delta n \, a_0 \quad \text{for } \Delta n \ll n
\end{equation}

The radial overlap decays exponentially:
\begin{equation}
\langle \psi_{n_1} | \psi_{n_2} \rangle \sim e^{-|\Delta r|/\xi}
\end{equation}
where $\xi$ is the coherence length.

Therefore:
\begin{equation}
V(|\Delta \mathbf{r}|) \sim e^{-|\Delta \mathbf{r}|/\xi} \to 0 \quad \text{as} \quad |\Delta \mathbf{r}| \to \infty
\end{equation}

Locality emerges from the radial structure of partition coordinates. \qed
\end{proof}

\begin{remark}
Locality is often postulated as a fundamental principle (e.g., in quantum field theory). Here it emerges from partition geometry: states with large $\Delta n$ have negligible overlap because they occupy different spatial regions.

The "spooky action at a distance" problem dissolves: distant states don't interact not because of a locality postulate, but because they have exponentially suppressed overlap in the partition coordinate basis.
\end{remark}

\subsection{Connection to Curved Spacetime}

\begin{remark}
The analysis above assumes a flat (Euclidean) spatial structure. However, the partition geometry framework naturally extends to curved spacetime.

If the partition parameters $(n, l, m)$ vary smoothly with position, the induced metric can deviate from Euclidean:
\begin{equation}
g_{\mu\nu}(x) = \eta_{\mu\nu} + h_{\mu\nu}(n(x), l(x), m(x))
\end{equation}

where $h_{\mu\nu}$ encodes the curvature arising from variations in the partition structure.

This suggests a path to deriving general relativity from partition geometry: spacetime curvature emerges from gradients in the categorical state structure.

The Einstein field equations would then express how matter (categorical states) determines the partition structure, which in turn determines the metric.

This connection remains to be developed in full detail, but the framework is in place:
\begin{equation}
\text{Matter} \to \text{Partition structure} \to \text{Metric} \to \text{Curvature}
\end{equation}
\end{remark}

\subsection{Summary and Implications}

We have established:

\begin{enumerate}
    \item Three-dimensional structure emerges from angular coordinates $(l, m)$ (Theorem~\ref{thm:3d-emergence})
    \item Radial extension follows $r \propto n^2$ from energy-radius relations (Theorem~\ref{thm:radial-structure})
    \item The partition metric reduces to Euclidean in the continuum limit (Theorem~\ref{thm:euclidean-emergence})
    \item Dimensionality $D = 3$ is uniquely determined by constraint structure (Theorem~\ref{thm:dimensionality})
    \item Locality emerges from radial separation of partition shells (Theorem~\ref{thm:locality})
\end{enumerate}

Space is not a primitive arena but emerges from the geometry of bounded oscillatory partitions.

The profound implications:
\begin{itemize}
    \item Dimensionality is not arbitrary—$D=3$ is the unique consistent choice
    \item Euclidean geometry emerges from partition coordinate structure
    \item Locality emerges from spatial separation in partition space
    \item Curved spacetime may emerge from position-dependent partition structure
\end{itemize}

The question now becomes: How do multiple categorical states interact? This is addressed in the following section.
