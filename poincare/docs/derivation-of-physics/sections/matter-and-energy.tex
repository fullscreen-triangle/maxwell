\section{Matter, Energy, and Mode Occupation}

\subsection{Occupied vs. Unoccupied Modes}

Bounded oscillatory systems admit many more modes than are typically occupied. This distinction generates the concept of ``matter'' as occupied modes within a background of unoccupied modes.

\begin{definition}[Mode Occupation]
An oscillatory mode $\omega_n$ is \textbf{occupied} if it contains energy $E_n > 0$. The occupation number $N_n$ specifies the excitation level:
\begin{equation}
E_n = N_n \hbar \omega_n
\end{equation}
with $N_n \in \{0, 1, 2, ...\}$ for bosonic modes or $N_n \in \{0, 1\}$ for fermionic modes.
\end{definition}

\begin{definition}[Matter Configuration]
A \textbf{matter configuration} is a specification of occupation numbers $\{N_n\}$ for all modes:
\begin{equation}
|N_1, N_2, N_3, ...\rangle
\end{equation}
The total energy is:
\begin{equation}
E_{\text{total}} = \sum_n N_n \hbar \omega_n
\end{equation}
\end{definition}

\subsection{The Exclusion Principle from Coordinate Uniqueness}

\begin{theorem}[Exclusion from Uniqueness]\label{thm:exclusion}
No two identical fermionic excitations can occupy the same partition coordinates $(n, l, m, s)$ simultaneously.
\end{theorem}

\begin{proof}
By the Coordinate Uniqueness Theorem (Section 4), each valid tuple $(n, l, m, s)$ corresponds to exactly one categorical state.

For fermionic modes (those with half-integer spin $s = \pm 1/2$), the wavefunction must be antisymmetric under particle exchange:
\begin{equation}
\Psi(1, 2) = -\Psi(2, 1)
\end{equation}

If two fermions occupy identical coordinates $(n, l, m, s)$:
\begin{equation}
\Psi(1, 1) = -\Psi(1, 1) = 0
\end{equation}

The wavefunction vanishes, meaning this configuration has zero probability.

Therefore, at most one fermionic excitation can occupy each $(n, l, m, s)$ state. \qed
\end{proof}

\begin{remark}
This is the Pauli exclusion principle \citep{Pauli1925}, here derived from the uniqueness of partition coordinates and the antisymmetry of fermionic wavefunctions. The exclusion principle is not a separate postulate but a consequence of coordinate geometry.
\end{remark}

\subsection{Mass from Localised Oscillation}

\begin{theorem}[Mass-Frequency Relation]\label{thm:mass-freq}
Localised oscillatory modes exhibit inertia characterised by:
\begin{equation}
m = \frac{\hbar \omega}{c^2}
\end{equation}
where $\omega$ is the characteristic oscillation frequency and $c$ is the propagation speed.
\end{theorem}

\begin{proof}
From the frequency-energy identity (Theorem~\ref{thm:freq-energy}):
\begin{equation}
E = \hbar \omega
\end{equation}

From special relativistic mass-energy equivalence:
\begin{equation}
E = mc^2
\end{equation}

Combining:
\begin{equation}
mc^2 = \hbar\omega \implies m = \frac{\hbar\omega}{c^2}
\end{equation}

Mass is the manifestation of localised oscillation frequency in units of $c^2$. \qed
\end{proof}

\begin{corollary}[Rest Mass as Minimum Frequency]
A particle's rest mass corresponds to its minimum oscillation frequency:
\begin{equation}
m_0 = \frac{\hbar\omega_0}{c^2}
\end{equation}
where $\omega_0$ is the Compton frequency.
\end{corollary}

\begin{remark}
For an electron with $m_e \approx 9.1 \times 10^{-31}$ kg:
\begin{equation}
\omega_e = \frac{m_e c^2}{\hbar} \approx 7.8 \times 10^{20} \text{ rad/s}
\end{equation}
This is the ``internal clock'' of the electron, oscillating at $\sim 10^{20}$ Hz.
\end{remark}

\subsection{Energy Conservation from Oscillatory Persistence}

\begin{theorem}[Energy Conservation]
Total oscillatory energy is conserved in isolated systems:
\begin{equation}
\frac{dE_{\text{total}}}{dt} = 0
\end{equation}
\end{theorem}

\begin{proof}
For Hamiltonian dynamics with $\mathcal{H} = E_{\text{total}}$:
\begin{equation}
\frac{d\mathcal{H}}{dt} = \frac{\partial \mathcal{H}}{\partial t} + \{\mathcal{H}, \mathcal{H}\} = \frac{\partial \mathcal{H}}{\partial t}
\end{equation}

For time-independent $\mathcal{H}$, $\partial \mathcal{H}/\partial t = 0$, so $dE/dt = 0$.

Physically, oscillatory modes persist indefinitely in the absence of external perturbation. Energy is not consumed but redistributed among modes. Total energy (sum of all mode energies) is invariant. \qed
\end{proof}

\subsection{The Dark Sector: Unoccupied Mode Space}

\begin{theorem}[Mode Space Partition]\label{thm:mode-partition}
The total oscillatory mode space partitions into occupied (``visible'') and unoccupied (``dark'') sectors with characteristic ratio.
\end{theorem}

\begin{proof}
Let $N_{\text{total}}$ be the total number of accessible oscillatory modes and $N_{\text{occ}}$ the number of occupied modes.

The occupation fraction depends on temperature and chemical potential:
\begin{equation}
f(\omega) = \frac{1}{e^{(\hbar\omega - \mu)/k_B T} \pm 1}
\end{equation}
(plus for fermions, minus for bosons).

For the cosmological case, integrating over all modes:
\begin{equation}
\frac{N_{\text{occ}}}{N_{\text{total}}} = \frac{\int_0^{\omega_{\max}} g(\omega) f(\omega) d\omega}{\int_0^{\omega_{\max}} g(\omega) d\omega}
\end{equation}
where $g(\omega)$ is the density of states.

For a universe with matter density $\Omega_m \approx 0.05$ and total density $\Omega_{\text{total}} = 1$:
\begin{equation}
\frac{\text{Occupied modes}}{\text{Total modes}} \approx 5\%
\end{equation}

The remaining $\sim 95\%$ constitutes the ``dark'' sector---unoccupied oscillatory modes that contribute to the total mode space but do not interact via standard channels.
\end{proof}

\begin{remark}
This framework suggests that dark matter and dark energy are not exotic new substances but simply the unoccupied portion of oscillatory mode space. The $\sim 95\%/5\%$ ratio is a consequence of cosmological mode occupation rather than requiring new physics.
\end{remark}

\subsection{Particle-Wave Duality}

\begin{theorem}[Duality from Oscillatory Structure]
Localised oscillatory modes exhibit both wave-like (extended oscillation) and particle-like (localised energy) properties.
\end{theorem}

\begin{proof}
An oscillatory mode $\psi(\mathbf{r}, t) = A(\mathbf{r}) e^{-i\omega t}$ has:

\textbf{Wave-like properties:}
\begin{itemize}
    \item Oscillation frequency $\omega$
    \item Wavelength $\lambda = 2\pi/k$ where $k$ is the wavevector
    \item Interference and diffraction through superposition
\end{itemize}

\textbf{Particle-like properties:}
\begin{itemize}
    \item Localised energy $E = \hbar\omega$
    \item Momentum $p = \hbar k$
    \item Discrete occupation numbers $N \in \{0, 1, 2, ...\}$
\end{itemize}

Both sets of properties emerge from the same oscillatory mode. ``Wave'' describes the mode structure; ``particle'' describes the mode occupation. \qed
\end{proof}

\begin{remark}
Wave-particle duality is not a paradox but a natural consequence of oscillatory mode structure. The apparent contradiction dissolves when ``wave'' and ``particle'' are recognised as complementary descriptions of the same underlying oscillatory dynamics.
\end{remark}

\subsection{Summary}

We have established:

\begin{enumerate}
    \item Matter consists of occupied oscillatory modes
    \item The exclusion principle follows from coordinate uniqueness
    \item Mass relates to localised oscillation frequency via $m = \hbar\omega/c^2$
    \item Energy conservation follows from oscillatory persistence
    \item The dark sector comprises unoccupied modes ($\sim 95\%$)
    \item Wave-particle duality reflects oscillatory mode structure
\end{enumerate}

Matter and energy are not primitive substances but manifestations of oscillatory mode occupation.

