\section{Maxwell's Equations from S-Coordinate Dynamics}
\label{sec:maxwell}

\subsection{From Circuits to Fields}

Sections 4-5 derived Ohm's law ($V = IR$) and Kirchhoff's laws for steady-state DC circuits. These laws assume:
\begin{itemize}
\item Time-independent fields: $\partial/\partial t = 0$
\item Lumped elements: circuit dimensions $L \ll \lambda$ (wavelength)
\item Quasi-static approximation: signal propagation time negligible
\end{itemize}

Real electromagnetic phenomena violate these assumptions. Electromagnetic waves have wavelengths comparable to system size. Fields vary rapidly in time. Signal propagation delays become important. We need the full frequency-dependent theory: Maxwell's equations.

Classical electromagnetism is governed by four equations:
\begin{align}
\nabla \cdot \mathbf{E} &= \frac{\rho}{\varepsilon_0} \quad &&\text{(Gauss's law)} \label{eq:gauss_classical} \\
\nabla \cdot \mathbf{B} &= 0 \quad &&\text{(No magnetic monopoles)} \label{eq:no_monopoles_classical} \\
\nabla \times \mathbf{E} &= -\frac{\partial \mathbf{B}}{\partial t} \quad &&\text{(Faraday's law)} \label{eq:faraday_classical} \\
\nabla \times \mathbf{B} &= \mu_0 \mathbf{J} + \mu_0\varepsilon_0 \frac{\partial \mathbf{E}}{\partial t} \quad &&\text{(Ampère-Maxwell law)} \label{eq:ampere_maxwell_classical}
\end{align}

These equations are typically postulated from experimental observations:
\begin{itemize}
\item Gauss's law: from Coulomb's law for static charges
\item No monopoles: from observation that magnetic field lines close
\item Faraday's law: from Faraday's induction experiments (1831)
\item Ampère-Maxwell law: from Ampère's law plus Maxwell's displacement current (1861)
\end{itemize}

Can these equations be derived from the S-coordinate framework? This section shows that all four Maxwell equations emerge from S-coordinate dynamics—specifically, from the behavior of S-potential gradients and S-coordinate curls under time evolution.

\subsection{S-Potential and the Electric Field}

In Section 4, we defined the S-potential $\Phi_S(\Svec)$ as a scalar function mapping the S-coordinate to a potential value. The voltage across a conductor was:
\begin{equation}
V = \Phi_S(\Svec_{\text{in}}) - \Phi_S(\Svec_{\text{out}})
\label{eq:voltage_reminder}
\end{equation}

For spatially varying fields, we generalise this to a continuous S-potential field $\Phi_S(\mathbf{r}, t)$ at each point $\mathbf{r}$ and time $t$.

\begin{definition}[Electric Field from S-Potential]
\label{def:electric_field_s}
The electric field is the negative gradient of the S-potential:
\begin{equation}
\mathbf{E}(\mathbf{r}, t) = -\nabla \Phi_S(\mathbf{r}, t)
\label{eq:e_from_s}
\end{equation}
\end{definition}

This definition is consistent with the classical relation $\mathbf{E} = -\nabla \Phi$ between electric field and electric potential. The S-potential $\Phi_S$ plays the role of the electric potential $\Phi$ in classical electromagnetism.

\textbf{Physical interpretation:} The electric field measures how the S-potential changes in space. Regions where $\Phi_S$ changes rapidly have strong electric fields. The negative sign ensures that the field points from high to low potential—the direction of categorical state flow.

\begin{figure*}[htbp]
\centering
\includegraphics[width=\textwidth]{figures/panel_electric_field_mechanics.png}
\caption{\textbf{Electromagnetic Field Mechanics and Electron Trajectories.} 
\textit{Top row:} (\textbf{Left}) Electric field configuration around two opposite charges ($+$ and $-$). Field lines (blue arrows) emanate from positive charges and terminate on negative charges. The cyan circle shows an equipotential surface. (\textbf{Center}) Magnetic field (wire cross-section) around a current-carrying wire. The magnetic field forms concentric circles (blue arrows) around the wire (yellow circle at center). (\textbf{Right}) Electron trajectories in three-dimensional space under combined electric and magnetic fields. Colored trajectories show helical motion characteristic of charged particles in crossed fields. 
\textit{Middle row:} (\textbf{Left}) Newton's cradle model showing resistance as damping. Three curves show wave propagation in superconductor ($R = 0$, green, no damping), medium-$R$ material (yellow, moderate damping), and high-$R$ material (red, strong damping). Resistance damps the wave amplitude exponentially. (\textbf{Center}) Potential landscape showing the S-coordinate potential $\Phi(x, y)$ as a three-dimensional surface. Peaks represent high-potential regions; valleys represent low-potential regions. Current flows from high to low potential. (\textbf{Right}) Material resistance comparison on a logarithmic scale. Superconductors (germanium) have resistivity $\rho < 10^{-100}~\Omega\cdot$m below $T_c$. Metals (copper, aluminum, tungsten) have $\rho \sim 10^{-8}$ to $10^{-5}~\Omega\cdot$m. Nichrome (resistor alloy) has $\rho \sim 10^{-6}~\Omega\cdot$m. The resistivity spans over 100 orders of magnitude.}
\label{fig:em_field_mechanics}
\end{figure*}

\subsection{Gauss's Law from S-Potential Sources}

The S-potential is not arbitrary—it is determined by the charge distribution. Charges create S-potential variations, just as masses create gravitational potential variations.

The relationship between charge density $\rho(\mathbf{r})$ and S-potential is Poisson's equation:
\begin{equation}
\nabla^2 \Phi_S = -\frac{\rho}{\varepsilon_0}
\label{eq:poisson_s}
\end{equation}

\textbf{Justification:} Charge is a manifestation of mode occupation asymmetry in the categorical framework (see prior work on forces). Positive charge corresponds to excess forward-time modes; negative charge to excess backward-time modes. The S-potential is sourced by this asymmetry:
\begin{equation}
\rho(\mathbf{r}) = e[n_+(\mathbf{r}) - n_-(\mathbf{r})]
\label{eq:charge_density_modes}
\end{equation}
where $n_+$ and $n_-$ are the densities of forward and backward modes.

The S-potential satisfies Poisson's equation because the categorical state must adjust to accommodate the mode asymmetry. The adjustment propagates through space according to the Laplacian $\nabla^2$.

Taking the divergence of the electric field:
\begin{equation}
\nabla \cdot \mathbf{E} = \nabla \cdot (-\nabla \Phi_S) = -\nabla^2 \Phi_S
\label{eq:div_e_intermediate}
\end{equation}

Substituting Poisson's equation:
\begin{equation}
\nabla \cdot \mathbf{E} = -\left(-\frac{\rho}{\varepsilon_0}\right) = \frac{\rho}{\varepsilon_0}
\label{eq:gauss_derived}
\end{equation}

This is Gauss's law.

\begin{theorem}[Gauss's Law from S-Potential Structure]
\label{thm:gauss}
The electric field satisfies Gauss's law:
\begin{equation}
\nabla \cdot \mathbf{E} = \frac{\rho}{\varepsilon_0}
\end{equation}
where $\rho$ is the charge density.
\end{theorem}

\begin{proof}
The S-potential $\Phi_S$ satisfies Poisson's equation (Equation \ref{eq:poisson_s}). The electric field is $\mathbf{E} = -\nabla \Phi_S$. Taking the divergence:
\begin{equation}
\nabla \cdot \mathbf{E} = -\nabla^2 \Phi_S = \frac{\rho}{\varepsilon_0}
\end{equation}
\qed
\end{proof}

\textbf{Physical interpretation:} Gauss's law states that electric field lines originate from positive charges and terminate on negative charges. In the S-framework, charges create S-potential sources. The divergence of $\mathbf{E}$ measures the "source strength" at each point. Positive charges are sources ($\nabla \cdot \mathbf{E} > 0$); negative charges are sinks ($\nabla \cdot \mathbf{E} < 0$).

\subsection{Vector Potential and the Magnetic Field}

The electric field is related to the S-potential gradient. What about the magnetic field? In classical electromagnetism, the magnetic field is defined through the vector potential $\mathbf{A}$:
\begin{equation}
\mathbf{B} = \nabla \times \mathbf{A}
\label{eq:b_from_a_classical}
\end{equation}

We introduce an analogous structure in the S-framework.

\begin{definition}[S-Vector Potential]
\label{def:s_vector_potential}
The S-vector potential $\mathbf{A}_S(\mathbf{r}, t)$ is a vector field encoding the current structure at each point. It is related to the S-coordinate by:
\begin{equation}
\mathbf{A}_S = \alpha \Svec
\label{eq:as_from_s}
\end{equation}
where $\alpha$ is a proportionality constant with dimensions of action (energy × time).
\end{definition}

\begin{definition}[Magnetic Field from S-Vector Potential]
\label{def:magnetic_field_s}
The magnetic field is the curl of the S-vector potential:
\begin{equation}
\mathbf{B}(\mathbf{r}, t) = \nabla \times \mathbf{A}_S(\mathbf{r}, t)
\label{eq:b_from_as}
\end{equation}
\end{definition}

\textbf{Physical interpretation:} The magnetic field measures the "rotation" or "curl" of the S-coordinate field in space. While the electric field arises from S-potential gradients (how $\Phi_S$ changes), the magnetic field arises from S-coordinate curls (how $\Svec$ rotates).

This is analogous to fluid flow:
\begin{itemize}
\item \textbf{Irrotational flow:} $\nabla \times \mathbf{v} = 0$ (no vorticity, pure potential flow)
\item \textbf{Rotational flow:} $\nabla \times \mathbf{v} \neq 0$ (vortices present)
\end{itemize}

Similarly for electromagnetic fields:
\begin{itemize}
\item \textbf{Electrostatic:} $\nabla \times \mathbf{E} = 0$ (conservative field, pure gradient)
\item \textbf{Magnetostatic:} $\nabla \times \mathbf{B} \neq 0$ (curl present, currents flowing)
\end{itemize}

\subsection{No Magnetic Monopoles from Curl Structure}

Taking the divergence of the magnetic field:
\begin{equation}
\nabla \cdot \mathbf{B} = \nabla \cdot (\nabla \times \mathbf{A}_S)
\label{eq:div_b_intermediate}
\end{equation}

By a fundamental vector identity, the divergence of any curl is identically zero:
\begin{equation}
\nabla \cdot (\nabla \times \mathbf{F}) = 0 \quad \text{for any vector field } \mathbf{F}
\label{eq:div_curl_identity}
\end{equation}

Therefore:
\begin{equation}
\nabla \cdot \mathbf{B} = 0
\label{eq:no_monopoles_derived}
\end{equation}

This is the "no magnetic monopoles" equation.

\begin{theorem}[No Magnetic Monopoles from S-Curl Structure]
\label{thm:no_monopoles}
The magnetic field has zero divergence:
\begin{equation}
\nabla \cdot \mathbf{B} = 0
\end{equation}
\end{theorem}

\begin{proof}
The magnetic field is defined as $\mathbf{B} = \nabla \times \mathbf{A}_S$. By the vector identity $\nabla \cdot (\nabla \times \mathbf{F}) = 0$:
\begin{equation}
\nabla \cdot \mathbf{B} = \nabla \cdot (\nabla \times \mathbf{A}_S) = 0
\end{equation}
\qed
\end{proof}

\textbf{Physical interpretation:} Magnetic field lines never begin or end—they always form closed loops. This is a mathematical necessity, not an empirical observation. By defining $\mathbf{B}$ as a curl, we guarantee $\nabla \cdot \mathbf{B} = 0$. 

The absence of magnetic monopoles is a consequence of the S-coordinate curl structure. If magnetic monopoles existed, we would need to modify the definition of $\mathbf{B}$ to include a gradient term (analogous to how $\mathbf{E}$ includes a gradient). The fact that $\mathbf{B} = \nabla \times \mathbf{A}_S$ suffices to describe all observed magnetic phenomena indicates that monopoles do not exist.

\subsection{Faraday's Law from S-Vector Potential Evolution}

Faraday's law describes electromagnetic induction: changing magnetic fields produce electric fields. This is the principle behind electric generators, transformers, and inductors.

In the S-framework, Faraday's law emerges from the time evolution of the S-vector potential.

The electric field has two contributions:
\begin{equation}
\mathbf{E} = -\nabla \Phi_S - \frac{\partial \mathbf{A}_S}{\partial t}
\label{eq:e_full}
\end{equation}

The first term ($-\nabla \Phi_S$) is the electrostatic contribution from charges. The second term ($-\partial \mathbf{A}_S/\partial t$) is the induced contribution from changing magnetic fields.

Taking the curl of Equation \ref{eq:e_full}:
\begin{equation}
\nabla \times \mathbf{E} = -\nabla \times (\nabla \Phi_S) - \nabla \times \left(\frac{\partial \mathbf{A}_S}{\partial t}\right)
\label{eq:curl_e_intermediate}
\end{equation}

The curl of a gradient is zero:
\begin{equation}
\nabla \times (\nabla \Phi_S) = 0
\label{eq:curl_grad_zero}
\end{equation}

For smooth fields, we can interchange curl and time derivative:
\begin{equation}
\nabla \times \left(\frac{\partial \mathbf{A}_S}{\partial t}\right) = \frac{\partial}{\partial t}(\nabla \times \mathbf{A}_S) = \frac{\partial \mathbf{B}}{\partial t}
\label{eq:curl_time_interchange}
\end{equation}

Substituting into Equation \ref{eq:curl_e_intermediate}:
\begin{equation}
\nabla \times \mathbf{E} = -\frac{\partial \mathbf{B}}{\partial t}
\label{eq:faraday_derived}
\end{equation}

This is Faraday's law.

\begin{theorem}[Faraday's Law from S-Vector Potential Evolution]
\label{thm:faraday}
The curl of the electric field equals the negative time derivative of the magnetic field:
\begin{equation}
\nabla \times \mathbf{E} = -\frac{\partial \mathbf{B}}{\partial t}
\end{equation}
\end{theorem}

\begin{proof}
The electric field is $\mathbf{E} = -\nabla \Phi_S - \partial \mathbf{A}_S/\partial t$. Taking the curl and using $\nabla \times (\nabla \Phi_S) = 0$:
\begin{equation}
\nabla \times \mathbf{E} = -\frac{\partial}{\partial t}(\nabla \times \mathbf{A}_S) = -\frac{\partial \mathbf{B}}{\partial t}
\end{equation}
\qed
\end{proof}

\textbf{Physical interpretation:} Faraday's law states that a changing magnetic field produces a circulating electric field. In the S-framework, this is a consequence of S-vector potential evolution. When $\mathbf{A}_S$ changes with time, it induces an electric field contribution $-\partial \mathbf{A}_S/\partial t$. The curl of this contribution equals $-\partial \mathbf{B}/\partial t$.

The integral form of Faraday's law is:
\begin{equation}
\oint_C \mathbf{E} \cdot d\mathbf{l} = -\frac{d\Phi_B}{dt}
\label{eq:faraday_integral}
\end{equation}
where $\Phi_B = \int_S \mathbf{B} \cdot d\mathbf{A}$ is the magnetic flux through surface $S$ bounded by curve $C$.

This is the generalization of Kirchhoff's Voltage Law to time-varying fields. In the quasi-static limit ($d\Phi_B/dt \to 0$), Faraday's law reduces to KVL: $\oint \mathbf{E} \cdot d\mathbf{l} = 0$.

\begin{figure*}[htbp]
\centering
\includegraphics[width=\textwidth]{figures/panel_maxwell_equations.png}
\caption{\textbf{Maxwell's Equations Derived from Categorical S-Dynamics.} 
(\textbf{A}) Gauss's law: The electric field $\mathbf{E} = -\nabla \Phi_S$ emerges from the S-gradient around a positive charge (red). Blue arrows show field lines radiating outward from the charge. Dashed circles represent equipotential surfaces where $\Phi_S = \text{const}$. The field strength decreases as $1/r^2$ with distance from the charge. 
(\textbf{B}) Ampère's law: The magnetic field $\mathbf{B} = \nabla \times \mathbf{A}_S$ emerges from the S-curl around a current-carrying wire (gray circle with current into page, marked $\otimes$). Green arrows show magnetic field lines forming concentric circles around the wire. The field strength decreases as $1/r$ with distance from the wire. 
(\textbf{C}) Coupled E-B oscillation: Electromagnetic wave propagation showing electric field (blue) and magnetic field (green) oscillating perpendicular to each other with a 90° phase shift. The fields are perpendicular to the propagation direction, forming a transverse wave. The wavelength and amplitude are indicated. 
(\textbf{D}) Speed of light from S-dynamics: The wave equation $\nabla^2 \mathbf{E} = \mu_0 \varepsilon_0 \partial^2 \mathbf{E}/\partial t^2$ emerges from S-transformation dynamics. The speed of light is $c = 1/\sqrt{\mu_0 \varepsilon_0} = 299{,}792{,}458$ m/s, determined by the vacuum partition-coupling structure. The S-transformation rate equals the wave velocity, establishing that electromagnetic waves are propagating S-transformations in the vacuum field.}
\label{fig:maxwell_equations}
\end{figure*}

\subsection{Ampère-Maxwell Law from Current and Displacement}

Ampère's original law (1826) related the magnetic field curl to current density:
\begin{equation}
\nabla \times \mathbf{B} = \mu_0 \mathbf{J}
\label{eq:ampere_original}
\end{equation}

This law works for steady currents but fails for time-varying fields. The problem is revealed by taking the divergence:
\begin{equation}
\nabla \cdot (\nabla \times \mathbf{B}) = \mu_0 \nabla \cdot \mathbf{J}
\label{eq:ampere_divergence}
\end{equation}

The left side is zero (divergence of curl). Therefore Ampère's law requires:
\begin{equation}
\nabla \cdot \mathbf{J} = 0
\label{eq:current_divergence_zero}
\end{equation}

But charge conservation (continuity equation) requires:
\begin{equation}
\frac{\partial \rho}{\partial t} + \nabla \cdot \mathbf{J} = 0
\label{eq:continuity}
\end{equation}

For time-varying charge density ($\partial \rho/\partial t \neq 0$), Equations \ref{eq:current_divergence_zero} and \ref{eq:continuity} are inconsistent. Ampère's law must be modified.

Maxwell (1861) resolved this by adding the displacement current term. Using Gauss's law $\rho = \varepsilon_0 \nabla \cdot \mathbf{E}$:
\begin{equation}
\frac{\partial \rho}{\partial t} = \varepsilon_0 \frac{\partial}{\partial t}(\nabla \cdot \mathbf{E}) = \varepsilon_0 \nabla \cdot \left(\frac{\partial \mathbf{E}}{\partial t}\right)
\label{eq:charge_rate_from_e}
\end{equation}

Substituting into the continuity equation:
\begin{equation}
\nabla \cdot \mathbf{J} = -\frac{\partial \rho}{\partial t} = -\varepsilon_0 \nabla \cdot \left(\frac{\partial \mathbf{E}}{\partial t}\right)
\label{eq:current_divergence_from_e}
\end{equation}

Therefore:
\begin{equation}
\nabla \cdot \left(\mathbf{J} + \varepsilon_0 \frac{\partial \mathbf{E}}{\partial t}\right) = 0
\label{eq:total_current_divergence}
\end{equation}

The quantity $\mathbf{J} + \varepsilon_0 \partial \mathbf{E}/\partial t$ has zero divergence, so it can be the curl of a vector field. The modified Ampère's law is:
\begin{equation}
\nabla \times \mathbf{B} = \mu_0 \left(\mathbf{J} + \varepsilon_0 \frac{\partial \mathbf{E}}{\partial t}\right)
\label{eq:ampere_maxwell_derived}
\end{equation}

This is the Ampère-Maxwell law.

\begin{theorem}[Ampère-Maxwell Law from Charge Conservation]
\label{thm:ampere_maxwell}
The curl of the magnetic field is:
\begin{equation}
\nabla \times \mathbf{B} = \mu_0 \mathbf{J} + \mu_0 \varepsilon_0 \frac{\partial \mathbf{E}}{\partial t}
\end{equation}
where $\mathbf{J}$ is the current density and $\varepsilon_0 \partial \mathbf{E}/\partial t$ is the displacement current.
\end{theorem}

\begin{proof}
Ampère's original law $\nabla \times \mathbf{B} = \mu_0 \mathbf{J}$ is inconsistent with charge conservation for time-varying fields. Taking the divergence and using continuity equation $\nabla \cdot \mathbf{J} = -\partial \rho/\partial t$ and Gauss's law $\rho = \varepsilon_0 \nabla \cdot \mathbf{E}$:
\begin{equation}
\nabla \cdot \left(\mathbf{J} + \varepsilon_0 \frac{\partial \mathbf{E}}{\partial t}\right) = 0
\end{equation}

Therefore the modified law is:
\begin{equation}
\nabla \times \mathbf{B} = \mu_0 \left(\mathbf{J} + \varepsilon_0 \frac{\partial \mathbf{E}}{\partial t}\right)
\end{equation}
\qed
\end{proof}

\textbf{Physical interpretation:} The Ampère-Maxwell law has two source terms for magnetic fields:

\begin{enumerate}
\item \textbf{Conduction current} $\mathbf{J}$: Physical charge flow (electrons moving through wires). This is the S-gradient propagation studied in Sections 4-5.

\item \textbf{Displacement current} $\varepsilon_0 \partial \mathbf{E}/\partial t$: Changing electric field (no physical charge flow). This is the S-transformation rate—the rate at which categorical states evolve.
\end{enumerate}

The displacement current is crucial for electromagnetic wave propagation. In vacuum ($\mathbf{J} = 0$), only the displacement current exists. Changing $\mathbf{E}$ produces $\mathbf{B}$ (Ampère-Maxwell); changing $\mathbf{B}$ produces $\mathbf{E}$ (Faraday). This creates self-sustaining oscillations that propagate as electromagnetic waves.

\subsection{Electromagnetic Waves}

Combining Faraday's law and Ampère-Maxwell law produces the wave equation. In vacuum ($\mathbf{J} = 0$, $\rho = 0$):
\begin{align}
\nabla \times \mathbf{E} &= -\frac{\partial \mathbf{B}}{\partial t} \label{eq:faraday_vacuum} \\
\nabla \times \mathbf{B} &= \mu_0 \varepsilon_0 \frac{\partial \mathbf{E}}{\partial t} \label{eq:ampere_vacuum}
\end{align}

Taking the curl of Equation \ref{eq:faraday_vacuum}:
\begin{equation}
\nabla \times (\nabla \times \mathbf{E}) = -\frac{\partial}{\partial t}(\nabla \times \mathbf{B})
\label{eq:curl_curl_e}
\end{equation}

Using the vector identity $\nabla \times (\nabla \times \mathbf{F}) = \nabla(\nabla \cdot \mathbf{F}) - \nabla^2 \mathbf{F}$:
\begin{equation}
\nabla(\nabla \cdot \mathbf{E}) - \nabla^2 \mathbf{E} = -\frac{\partial}{\partial t}(\nabla \times \mathbf{B})
\label{eq:curl_curl_expanded}
\end{equation}

In a vacuum, $\nabla \cdot \mathbf{E} = 0$ (Gauss's law with $\rho = 0$):
\begin{equation}
-\nabla^2 \mathbf{E} = -\frac{\partial}{\partial t}(\nabla \times \mathbf{B})
\label{eq:laplacian_e}
\end{equation}

Substituting Equation \ref{eq:ampere_vacuum}:
\begin{equation}
\nabla^2 \mathbf{E} = \frac{\partial}{\partial t}\left(\mu_0 \varepsilon_0 \frac{\partial \mathbf{E}}{\partial t}\right) = \mu_0 \varepsilon_0 \frac{\partial^2 \mathbf{E}}{\partial t^2}
\label{eq:wave_equation_e}
\end{equation}

This is the wave equation for the electric field.

Similarly, taking the curl of Equation \ref{eq:ampere_vacuum} and using Equation \ref{eq:faraday_vacuum}:
\begin{equation}
\nabla^2 \mathbf{B} = \mu_0 \varepsilon_0 \frac{\partial^2 \mathbf{B}}{\partial t^2}
\label{eq:wave_equation_b}
\end{equation}

\begin{theorem}[Electromagnetic Wave Equation]
\label{thm:em_wave}
In vacuum, the electric and magnetic fields satisfy wave equations:
\begin{align}
\nabla^2 \mathbf{E} &= \mu_0 \varepsilon_0 \frac{\partial^2 \mathbf{E}}{\partial t^2} \\
\nabla^2 \mathbf{B} &= \mu_0 \varepsilon_0 \frac{\partial^2 \mathbf{B}}{\partial t^2}
\end{align}
with wave speed:
\begin{equation}
c = \frac{1}{\sqrt{\mu_0 \varepsilon_0}} = 299{,}792{,}458 \text{ m/s}
\label{eq:speed_of_light}
\end{equation}
\end{theorem}

\begin{proof}
Combining Faraday's law and Ampère-Maxwell law in vacuum produces the wave equations (Equations \ref{eq:wave_equation_e} and \ref{eq:wave_equation_b}). The wave speed is:
\begin{equation}
c = \frac{1}{\sqrt{\mu_0 \varepsilon_0}}
\end{equation}
\qed
\end{proof}

\textbf{Physical interpretation:} Electromagnetic waves are self-sustaining oscillations of electric and magnetic fields. The fields propagate through space at speed $c$, carrying energy and momentum. The wave speed is determined by the vacuum permittivity $\varepsilon_0$ and permeability $\mu_0$—fundamental constants that characterise the electromagnetic properties of space.

Plane wave solutions have the form:
\begin{align}
\mathbf{E}(\mathbf{r}, t) &= \mathbf{E}_0 \cos(\mathbf{k} \cdot \mathbf{r} - \omega t + \phi) \\
\mathbf{B}(\mathbf{r}, t) &= \mathbf{B}_0 \cos(\mathbf{k} \cdot \mathbf{r} - \omega t + \phi)
\end{align}
where $\mathbf{k}$ is the wave vector, $\omega$ is the angular frequency, and $\phi$ is the phase. The dispersion relation is:
\begin{equation}
\omega = c|\mathbf{k}|
\label{eq:dispersion}
\end{equation}

The fields are perpendicular to each other and to the direction of propagation:
\begin{equation}
\mathbf{E} \perp \mathbf{B} \perp \mathbf{k}
\label{eq:transverse}
\end{equation}

The field amplitudes are related by:
\begin{equation}
|\mathbf{B}_0| = \frac{|\mathbf{E}_0|}{c}
\label{eq:field_ratio}
\end{equation}

\subsection{Speed of Light from Partition-Coupling Structure}

The speed of light $c = 1/\sqrt{\mu_0 \varepsilon_0}$ has a deep interpretation in the S-framework. The constants $\mu_0$ and $\varepsilon_0$ are not arbitrary—they characterise the partition-coupling structure of the electromagnetic vacuum.

Recall from the universal transport coefficient (prior work):
\begin{equation}
\Xi = \frac{1}{\mathcal{N}} \sum_{i,j} \tau_{p,ij} \cdot g_{ij}
\label{eq:transport_coefficient_reminder}
\end{equation}

For electromagnetic wave propagation, we identify:
\begin{align}
\mu_0 &\sim \text{electromagnetic partition lag (field inertia)} \\
\varepsilon_0 &\sim \text{vacuum coupling (field flexibility)}
\end{align}

The speed of light is then:
\begin{equation}
c = \frac{1}{\sqrt{\mu_0 \varepsilon_0}} = \frac{1}{\sqrt{\tau_p^{(\text{EM})} \cdot g^{(\text{EM})}}}
\label{eq:c_from_partition}
\end{equation}

\textbf{Physical interpretation:}

\begin{itemize}
\item \textbf{$\mu_0 = 4\pi \times 10^{-7}$ H/m (permeability):} Measures how long magnetic field changes take to propagate. High $\mu_0$ means slow propagation (large partition lag). This is the electromagnetic analogue of mass—it represents field inertia.

\item \textbf{$\varepsilon_0 = 8.854 \times 10^{-12}$ F/m (permittivity):} Measures how strongly the vacuum responds to changes in the electric field. High $\varepsilon_0$ means strong response (strong coupling). This is the electromagnetic analogue of compliance—it represents field flexibility.
\end{itemize}

The speed of light is not arbitrary. It is determined by the partition-coupling structure of space itself. The vacuum has intrinsic electromagnetic properties ($\mu_0$, $\varepsilon_0$) that determine how fast categorical states can propagate.

This provides a partition-geometric interpretation of fundamental constants. Just as resistivity is $\rho = m_e/(ne^2\tau_s)$ (determined by scattering partition lag), the speed of light is $c = 1/\sqrt{\mu_0\varepsilon_0}$ (determined by electromagnetic partition lag).

\subsection{Quasi-Static Limit and Circuit Theory}

In the limit of low frequencies ($\omega \to 0$) and small systems ($L \ll \lambda$), Maxwell's equations reduce to circuit theory.

\textbf{Low frequency limit:}

For $\omega \to 0$, time derivatives become negligible:
\begin{align}
\frac{\partial \mathbf{B}}{\partial t} &\to 0 \quad \text{(Faraday's law)} \\
\frac{\partial \mathbf{E}}{\partial t} &\to 0 \quad \text{(Ampère-Maxwell law)}
\end{align}

Maxwell's equations reduce to:
\begin{align}
\nabla \cdot \mathbf{E} &= \frac{\rho}{\varepsilon_0} \quad &&\text{(electrostatics)} \\
\nabla \times \mathbf{E} &= 0 \quad &&\text{(conservative field)} \\
\nabla \times \mathbf{B} &= \mu_0 \mathbf{J} \quad &&\text{(magnetostatics)}
\end{align}

\textbf{Small system limit:}

For $L \ll \lambda$, the system is much smaller than the wavelength. Retardation effects are negligible—signals propagate instantaneously across the system. This is the lumped element approximation.

In this limit:
\begin{itemize}
\item $\nabla \times \mathbf{E} = 0$ implies $\mathbf{E} = -\nabla \Phi$ (electrostatic potential)
\item Integrating around a loop: $\oint \mathbf{E} \cdot d\mathbf{l} = 0$ (Kirchhoff's Voltage Law)
\item $\nabla \cdot \mathbf{J} = 0$ at each node (charge conservation)
\item Integrating over a node: $\sum_k I_k = 0$ (Kirchhoff's Current Law)
\end{itemize}

Ohm's law $\mathbf{J} = \sigma \mathbf{E}$ becomes $I = V/R$ when integrated over a conductor.

\begin{theorem}[Quasi-Static Limit]
\label{thm:quasi_static}
In the limit $\omega \to 0$ and $L \ll \lambda$:
\begin{align}
\text{Maxwell's equations} &\implies \text{Kirchhoff's laws} \\
\mathbf{J} = \sigma \mathbf{E} &\implies V = IR
\end{align}
\end{theorem}

\textbf{Physical interpretation:} Circuit theory is the low-frequency approximation to Maxwell's equations. Kirchhoff's laws are valid when fields don't vary rapidly and systems are small compared to wavelengths. For high frequencies or large systems, the full Maxwell equations are required.

\subsection{Summary}

We have derived all four Maxwell equations from S-coordinate dynamics:

\begin{center}
\begin{tabular}{lll}
\toprule
\textbf{Equation} & \textbf{Form} & \textbf{S-Dynamics Origin} \\
\midrule
Gauss (E) & $\nabla \cdot \mathbf{E} = \rho/\varepsilon_0$ & S-potential Poisson equation \\
No monopoles & $\nabla \cdot \mathbf{B} = 0$ & S-curl structure ($\nabla \cdot (\nabla \times) = 0$) \\
Faraday & $\nabla \times \mathbf{E} = -\partial\mathbf{B}/\partial t$ & S-vector potential evolution \\
Ampère-Maxwell & $\nabla \times \mathbf{B} = \mu_0\mathbf{J} + \mu_0\varepsilon_0\partial\mathbf{E}/\partial t$ & Current + S-transformation rate \\
\bottomrule
\end{tabular}
\end{center}

Key results:

\begin{enumerate}
\item Electric field: $\mathbf{E} = -\nabla \Phi_S$ (S-potential gradient)
\item Magnetic field: $\mathbf{B} = \nabla \times \mathbf{A}_S$ (S-vector potential curl)
\item Electromagnetic waves: $\nabla^2 \mathbf{E} = (1/c^2)\partial^2\mathbf{E}/\partial t^2$ with $c = 1/\sqrt{\mu_0\varepsilon_0}$
\item Speed of light: $c = 1/\sqrt{\tau_p^{(\text{EM})} \cdot g^{(\text{EM})}}$ (partition-coupling structure)
\item Quasi-static limit: Maxwell → Kirchhoff + Ohm
\end{enumerate}

All of classical electromagnetism emerges from the S-coordinate framework. The next section applies this framework to specific phenomena: superconductivity, quantum Hall effect, and topological materials.
