%==============================================================================
% SECTION 9: MAXWELL'S EQUATIONS FROM S-DYNAMICS
%==============================================================================

\section{Maxwell's Equations from S-Dynamics}
\label{sec:maxwell}

\subsection{From Circuit Theory to Field Theory}

Ohm's law and Kirchhoff's laws are valid in the quasi-static limit where time derivatives are negligible. Maxwell's equations are the full frequency-dependent generalisation, valid at all frequencies including electromagnetic radiation.

\begin{theorem}[Quasi-Static Limit]
\label{thm:quasi_static}
In the limit of low frequencies and small systems:
\begin{align}
\text{Maxwell's equations} &\xrightarrow[\omega \to 0]{L \ll \lambda} \text{Kirchhoff's laws} \\
\mathbf{J} = \sigma \mathbf{E} &\xrightarrow[\text{integrate}]{} V = IR
\end{align}
\end{theorem}

\subsection{S-Potential and Electric Field}

\begin{definition}[S-Potential and Electric Field]
\label{def:s_field}
The electric field is the gradient of the S-potential:
\begin{equation}
\mathbf{E} = -\nabla \Phi_S
\label{eq:e_from_s}
\end{equation}
where $\Phi_S$ is the S-potential encoding categorical state information.
\end{definition}

\begin{theorem}[Gauss's Law from Mode Asymmetry]
\label{thm:gauss_law}
The divergence of the electric field equals the charge density:
\begin{equation}
\nabla \cdot \mathbf{E} = \frac{\rho}{\varepsilon_0}
\label{eq:gauss_law}
\end{equation}
where charge density $\rho = e(n_+ - n_-)$ is the mode occupation asymmetry.
\end{theorem}

\begin{proof}
Charge is mode occupation asymmetry (see prior work on forces):
\begin{equation}
q = e \sum_n (N_n^+ - N_n^-)
\end{equation}

The S-potential is sourced by this asymmetry:
\begin{equation}
\nabla^2 \Phi_S = -\frac{\rho}{\varepsilon_0}
\end{equation}

Taking the divergence of $\mathbf{E} = -\nabla \Phi_S$:
\begin{equation}
\nabla \cdot \mathbf{E} = -\nabla^2 \Phi_S = \frac{\rho}{\varepsilon_0}
\end{equation}
\qed
\end{proof}

\subsection{Magnetic Field from S-Curl}

\begin{definition}[S-Curl and Magnetic Field]
\label{def:s_curl}
The magnetic field is the curl of the S-potential vector:
\begin{equation}
\mathbf{B} = \nabla \times \mathbf{A}_S
\label{eq:b_from_s}
\end{equation}
where $\mathbf{A}_S$ is the S-potential vector encoding current structure.
\end{definition}

\begin{theorem}[Gauss's Law for Magnetism]
\label{thm:gauss_magnetism}
The divergence of the magnetic field vanishes:
\begin{equation}
\nabla \cdot \mathbf{B} = 0
\label{eq:gauss_magnetism}
\end{equation}
\end{theorem}

\begin{proof}
By vector identity, the divergence of a curl vanishes:
\begin{equation}
\nabla \cdot \mathbf{B} = \nabla \cdot (\nabla \times \mathbf{A}_S) = 0
\end{equation}

Physically: there are no magnetic monopoles. The S-curl structure is topologically closed. \qed
\end{proof}

\subsection{Faraday's Law from Categorical Completion}

\begin{theorem}[Faraday's Law]
\label{thm:faraday}
A changing magnetic field induces an electric field:
\begin{equation}
\nabla \times \mathbf{E} = -\frac{\partial \mathbf{B}}{\partial t}
\label{eq:faraday}
\end{equation}
\end{theorem}

\begin{proof}
The S-potential satisfies the consistency relation:
\begin{equation}
\nabla \times (-\nabla \Phi_S) = -\frac{\partial}{\partial t}(\nabla \times \mathbf{A}_S)
\end{equation}

For static fields, $\nabla \times \mathbf{E} = 0$ (conservative field). For time-varying fields, the S-curl evolves:
\begin{equation}
\mathbf{B} = \nabla \times \mathbf{A}_S \implies \frac{\partial \mathbf{B}}{\partial t} = \nabla \times \frac{\partial \mathbf{A}_S}{\partial t}
\end{equation}

The induced electric field is:
\begin{equation}
\mathbf{E}_{\text{induced}} = -\frac{\partial \mathbf{A}_S}{\partial t}
\end{equation}

Taking the curl:
\begin{equation}
\nabla \times \mathbf{E} = -\nabla \times \frac{\partial \mathbf{A}_S}{\partial t} = -\frac{\partial}{\partial t}(\nabla \times \mathbf{A}_S) = -\frac{\partial \mathbf{B}}{\partial t}
\end{equation}
\qed
\end{proof}

\begin{remark}
Faraday's law is KVL with time-varying flux:
\begin{equation}
\oint \mathbf{E} \cdot d\mathbf{l} = -\frac{d\Phi_B}{dt}
\end{equation}
The loop integral of electric field equals the rate of magnetic flux change through the loop. In the quasi-static limit ($d\Phi_B/dt \to 0$), this reduces to KVL: $\sum V = 0$.
\end{remark}

\subsection{Ampère-Maxwell Law from S-Transformation Rate}

\begin{theorem}[Ampère-Maxwell Law]
\label{thm:ampere_maxwell}
A changing electric field and current density produce magnetic field:
\begin{equation}
\nabla \times \mathbf{B} = \mu_0 \mathbf{J} + \mu_0 \varepsilon_0 \frac{\partial \mathbf{E}}{\partial t}
\label{eq:ampere_maxwell}
\end{equation}
\end{theorem}

\begin{proof}
\textbf{Step 1: Current density term.}

Moving charges produce magnetic field:
\begin{equation}
\nabla \times \mathbf{B} = \mu_0 \mathbf{J}
\end{equation}

This is Ampère's original law.

\textbf{Step 2: Displacement current.}

Taking the divergence of Ampère's law:
\begin{equation}
\nabla \cdot (\nabla \times \mathbf{B}) = \mu_0 \nabla \cdot \mathbf{J}
\end{equation}

The left side vanishes (divergence of curl). For consistency:
\begin{equation}
\nabla \cdot \mathbf{J} = 0
\end{equation}

But charge conservation requires:
\begin{equation}
\frac{\partial \rho}{\partial t} + \nabla \cdot \mathbf{J} = 0 \implies \nabla \cdot \mathbf{J} = -\frac{\partial \rho}{\partial t}
\end{equation}

Using Gauss's law, $\rho = \varepsilon_0 \nabla \cdot \mathbf{E}$:
\begin{equation}
\nabla \cdot \mathbf{J} = -\varepsilon_0 \frac{\partial}{\partial t}(\nabla \cdot \mathbf{E}) = -\nabla \cdot \left( \varepsilon_0 \frac{\partial \mathbf{E}}{\partial t} \right)
\end{equation}

Therefore:
\begin{equation}
\nabla \cdot \left( \mathbf{J} + \varepsilon_0 \frac{\partial \mathbf{E}}{\partial t} \right) = 0
\end{equation}

The modified Ampère's law is:
\begin{equation}
\nabla \times \mathbf{B} = \mu_0 \left( \mathbf{J} + \varepsilon_0 \frac{\partial \mathbf{E}}{\partial t} \right)
\end{equation}
\qed
\end{proof}

\begin{remark}
The displacement current $\varepsilon_0 \partial\mathbf{E}/\partial t$ represents the \textbf{S-transformation rate}. Even without physical charge flow, a changing electric field represents categorical completion---the S-state evolving in time. This produces magnetic field just as physical current does.

This is the key insight enabling electromagnetic wave propagation: changing E produces B (Ampère-Maxwell), changing B produces E (Faraday), creating self-sustaining oscillation.
\end{remark}

\subsection{Wave Equation and Speed of Light}

\begin{theorem}[Electromagnetic Wave Equation]
\label{thm:em_wave}
In vacuum, the electric field satisfies:
\begin{equation}
\nabla^2 \mathbf{E} = \mu_0 \varepsilon_0 \frac{\partial^2 \mathbf{E}}{\partial t^2}
\label{eq:em_wave}
\end{equation}
with wave speed:
\begin{equation}
c = \frac{1}{\sqrt{\mu_0 \varepsilon_0}} = 299792458 \text{ m/s}
\label{eq:speed_of_light}
\end{equation}
\end{theorem}

\begin{proof}
Taking the curl of Faraday's law:
\begin{equation}
\nabla \times (\nabla \times \mathbf{E}) = -\frac{\partial}{\partial t}(\nabla \times \mathbf{B})
\end{equation}

Using the vector identity $\nabla \times (\nabla \times \mathbf{E}) = \nabla(\nabla \cdot \mathbf{E}) - \nabla^2 \mathbf{E}$:

In vacuum, $\nabla \cdot \mathbf{E} = 0$:
\begin{equation}
-\nabla^2 \mathbf{E} = -\frac{\partial}{\partial t}(\nabla \times \mathbf{B})
\end{equation}

Using Ampère-Maxwell law (with $\mathbf{J} = 0$ in vacuum):
\begin{equation}
\nabla \times \mathbf{B} = \mu_0 \varepsilon_0 \frac{\partial \mathbf{E}}{\partial t}
\end{equation}

Substituting:
\begin{equation}
\nabla^2 \mathbf{E} = \mu_0 \varepsilon_0 \frac{\partial^2 \mathbf{E}}{\partial t^2}
\end{equation}

This is the wave equation with speed:
\begin{equation}
c = \frac{1}{\sqrt{\mu_0 \varepsilon_0}}
\end{equation}
\qed
\end{proof}

\subsection{Speed of Light from Partition-Coupling Structure}

\begin{theorem}[Speed of Light from S-Dynamics]
\label{thm:c_from_s}
The speed of light is determined by electromagnetic partition lag and vacuum coupling:
\begin{equation}
c = \frac{1}{\sqrt{\tau_p^{(\text{EM})} \cdot g^{(\text{EM})}}}
\label{eq:c_from_partition}
\end{equation}
where $\tau_p^{(\text{EM})} = \mu_0$ (electromagnetic partition lag) and $g^{(\text{EM})} = \varepsilon_0$ (vacuum coupling).
\end{theorem}

\begin{proof}
By analogy with the universal transport coefficient:
\begin{equation}
\Xi = \frac{1}{\mathcal{N}} \sum_{i,j} \tau_{p,ij} \cdot g_{ij}
\end{equation}

For electromagnetic wave propagation, the effective impedance is:
\begin{equation}
Z_0 = \sqrt{\frac{\mu_0}{\varepsilon_0}}
\end{equation}

The speed of propagation is:
\begin{equation}
c = \frac{1}{\sqrt{\mu_0 \varepsilon_0}}
\end{equation}

Identifying $\mu_0$ as the electromagnetic partition lag (field inertia) and $\varepsilon_0$ as the vacuum coupling (field flexibility):
\begin{equation}
c = \frac{1}{\sqrt{\tau_p^{(\text{EM})} \cdot g^{(\text{EM})}}}
\end{equation}
\qed
\end{proof}

\begin{remark}
This provides a partition-geometric interpretation of the speed of light:
\begin{itemize}
\item $\mu_0 = 4\pi \times 10^{-7}$ H/m: electromagnetic partition lag (how long field changes take to propagate)
\item $\varepsilon_0 = 8.854 \times 10^{-12}$ F/m: vacuum coupling (how strongly the vacuum responds to field changes)
\end{itemize}

The speed of light is not arbitrary but determined by the partition-coupling structure of the electromagnetic vacuum.
\end{remark}

\subsection{Summary: Maxwell's Equations}

\begin{center}
\begin{tabular}{lll}
\toprule
\textbf{Equation} & \textbf{Differential Form} & \textbf{S-Dynamics Interpretation} \\
\midrule
Gauss (E) & $\nabla \cdot \mathbf{E} = \rho/\varepsilon_0$ & S-potential sourced by mode asymmetry \\
Gauss (B) & $\nabla \cdot \mathbf{B} = 0$ & S-curl has no sources \\
Faraday & $\nabla \times \mathbf{E} = -\partial\mathbf{B}/\partial t$ & S-gradient curl = categorical rate \\
Ampère-Maxwell & $\nabla \times \mathbf{B} = \mu_0\mathbf{J} + \mu_0\varepsilon_0\partial\mathbf{E}/\partial t$ & S-curl from current + transformation \\
\bottomrule
\end{tabular}
\end{center}

\begin{remark}
Maxwell's equations are the S-dynamics of electromagnetic fields:
\begin{itemize}
\item Gauss's laws describe S-potential sources
\item Faraday's law describes S-transformation inducing fields
\item Ampère-Maxwell law describes currents and S-transformation rates
\end{itemize}

Ohm's law and Kirchhoff's laws are the low-frequency limits where $\partial/\partial t \to 0$.
\end{remark}

