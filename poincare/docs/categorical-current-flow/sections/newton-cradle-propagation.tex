\section{Current Flow as Categorical Propagation}
\label{sec:current_propagation}

\subsection{The Velocity Paradox}

A fundamental paradox exists in electrical conduction: when you flip a light switch, the bulb illuminates almost instantaneously, yet individual electrons move extraordinarily slowly through the wire.

Consider a copper wire carrying $I = 1$ A through a cross-sectional area of $A = 1$ mm$^2$. The drift velocity of conduction electrons is:
\begin{equation}
v_d = \frac{I}{neA} = \frac{1}{(8.5 \times 10^{28})(1.6 \times 10^{-19})(10^{-6})} \approx 7.4 \times 10^{-5} \text{ m/s}
\label{eq:drift_velocity}
\end{equation}
where $n = 8.5 \times 10^{28}$ m$^{-3}$ is the electron density in copper. At this velocity, an individual electron would take approximately 4 hours to traverse 1 metre of wire.

Yet the electrical signal propagates at a significant fraction of the speed of light:
\begin{equation}
v_s = \frac{c}{\sqrt{\varepsilon_r \mu_r}} \approx 0.5c \text{ to } 0.9c
\label{eq:signal_velocity}
\end{equation}
where $\varepsilon_r$ and $\mu_r$ are the relative permittivity and permeability of the insulating medium surrounding the conductor. For typical values, $v_s \sim 2 \times 10^8$ m/s.

The ratio of signal velocity to drift velocity is:
\begin{equation}
\frac{v_s}{v_d} \sim \frac{2 \times 10^8}{7.4 \times 10^{-5}} \sim 10^{12}
\label{eq:velocity_ratio}
\end{equation}

This twelve-order-of-magnitude disparity demonstrates that current propagation cannot be explained by electron transport. The signal arrives $10^{12}$ times faster than electrons could carry it. Current must propagate through some mechanism other than particle flow.

\subsection{Newton's Cradle Mechanism}

The resolution lies in recognising that current is a \emph{collective phenomenon}, not individual particle transport. The appropriate analogy is Newton's cradle.

In Newton's cradle, when one ball strikes the chain, momentum transfers through successive collisions to the ball at the opposite end, which swings outward. The momentum propagates through the chain at the speed of sound in the material—much faster than any individual ball moves. Crucially, the intermediate balls barely move at all; they oscillate slightly but remain essentially stationary. Momentum transfer occurs without mass transport.

Electrical current operates by the same mechanism. When a potential difference is applied across a conductor:

\begin{enumerate}
\item Electron $e_1$ at the negative terminal shifts position by a small distance $\delta x$ due to the electric field
\item This shift displaces electron $e_2$ through Coulomb repulsion
\item Electron $e_2$ displaces electron $e_3$, and so on
\item The displacement propagates through the electron gas as a collective wave
\item Electron $e_N$ at the positive terminal shifts, constituting current flow
\end{enumerate}

The propagation occurs at electromagnetic speed $v_s \sim c$, not at the drift velocity $v_d$. Individual electrons oscillate locally with a small amplitude $\delta x \sim 10^{-10}$ m, while the displacement wave traverses the entire length of the conductor $L$ in time $t = L/v_s$.

This is \emph{categorical propagation}: the categorical state "electron displaced" propagates through the network, while individual electrons remain essentially stationary. Current is the rate of categorical state change, not the rate of particle transport.

\begin{figure*}[htbp]
\centering
\includegraphics[width=\textwidth]{figures/panel_newton_cradle.png}
\caption{\textbf{Newton's Cradle Model: Current as Categorical State Propagation.} 
(\textbf{A}) Wire cross-section showing electron chain: Fixed lattice ions (red $+$ symbols) form a periodic array. Mobile electrons (blue circles) form a chain between the ions. The electrons are confined to move along the wire axis but can displace slightly in response to applied fields. 
(\textbf{B}) Newton's cradle displacement propagation: At $t = 0$, an electron is pushed at the left end (red arrow). At $t = dt$, the displacement propagates through the chain as each electron pushes its neighbor. At $t = 2dt$, the signal exits at the right end (green arrow). The signal propagates at speed $\sim c$ while individual electrons barely move. 
(\textbf{C}) Speed comparison: Signal speed ($\sim 3 \times 10^8$ m/s, red bar) versus drift velocity ($\sim 10^{-4}$ m/s, blue bar) on a logarithmic scale. The ratio is approximately $3 \times 10^{12}$, demonstrating that current propagation is fundamentally different from electron drift. 
(\textbf{D}) Current as categorical state propagation: 
\textit{Classical view} (left, marked WRONG): Electrons flow like water, with each electron physically moving from source to destination. This picture cannot explain the rapid establishment of current. 
\textit{Categorical view} (right, marked CORRECT): Categorical states $|0\rangle$, $|1\rangle$ propagate through the electron network (gray circles). Green arrows show state propagation. Individual electrons remain nearly stationary while states propagate rapidly. This resolves the paradox between slow drift and fast signal propagation.}
\label{fig:newton_cradle}
\end{figure*}

\subsection{The Electron Phase-Lock Network}

The Newton's cradle mechanism requires that electrons be strongly coupled—they must form a dense network where each electron immediately affects its neighbours. This is precisely the structure of conduction electrons in metals.

Conduction electrons in metals are delocalised across the lattice with Fermi wavelength:
\begin{equation}
\lambda_F = \frac{h}{\sqrt{2m_e E_F}} \sim 0.5 \text{ nm}
\label{eq:fermi_wavelength}
\end{equation}
where $E_F \sim 5$ eV is the Fermi energy. This wavelength is comparable to the lattice spacing $a \sim 0.3$ nm, meaning each electron's wavefunction overlaps with $\sim 10$ nearest neighbours.

The overlapping wavefunctions create Coulomb coupling between electrons. The coupling strength between electrons $i$ and $j$ separated by distance $r_{ij}$ is:
\begin{equation}
g_{ij} = \frac{e^2}{4\pi\varepsilon_0 r_{ij}}
\label{eq:coulomb_coupling}
\end{equation}

For nearest neighbours with $r_{ij} \sim a$, the coupling energy is:
\begin{equation}
g_{ij} \sim \frac{(1.6 \times 10^{-19})^2}{4\pi(8.85 \times 10^{-12})(3 \times 10^{-10})} \sim 10^{-18} \text{ J} \sim 6 \text{ eV}
\label{eq:coupling_energy}
\end{equation}

This coupling energy is comparable to the Fermi energy, indicating strong coupling. The characteristic coupling time is:
\begin{equation}
\tau_c = \frac{\hbar}{g_{ij}} \sim \frac{10^{-34}}{10^{-18}} \sim 10^{-16} \text{ s}
\label{eq:coupling_time}
\end{equation}

This is much shorter than the scattering time $\tau_s \sim 10^{-14}$ s. Electrons are phase-locked on timescales relevant to current flow.

We can formalise this structure as a phase-lock network $\mathcal{G} = (V, E)$ where:
\begin{itemize}
\item \textbf{Vertices} $V$: electron oscillatory modes (one per conduction electron)
\item \textbf{Edges} $E$: phase-lock relationships through Coulomb interaction
\item \textbf{Edge weights} $g_{ij}$: coupling strength between electrons $i$ and $j$
\end{itemize}

The network density is:
\begin{equation}
\rho_{\mathcal{G}} = \frac{|E|}{|V|(|V|-1)/2}
\label{eq:network_density}
\end{equation}

For metals, where each electron couples to $\sim 10$ neighbours and there are $N \sim 10^{23}$ electrons per cubic centimetre, the network is essentially complete within the coherence length: $\rho_{\mathcal{G}} \approx 1$.

This dense network structure is the defining characteristic of metallic conduction. In insulators, electrons are localised to specific atoms and do not form overlapping wave functions. The network density is $\rho_{\mathcal{G}} \ll 1$, and categorical propagation cannot occur. This is why insulators do not conduct.

\subsection{Categorical State Propagation}

We can now provide a precise definition of current in terms of categorical propagation.

\begin{definition}[Current as Categorical Propagation]
\label{def:current_categorical}
Electric current is the rate of categorical state change through the electron phase-lock network:
\begin{equation}
I = e \cdot \frac{d}{dt}\left[\sum_i C_i(t)\right]
\label{eq:current_categorical}
\end{equation}
where $C_i(t) \in \{0, 1\}$ denotes the categorical state of electron $i$ at time $t$ (0 = original position, 1 = displaced position), and the sum represents the total number of displaced electrons.
\end{definition}

The key insight is that $\sum_i C_i(t)$ can change rapidly even though individual electrons move slowly. When electron $e_1$ shifts from state 0 to state 1, it immediately causes electron $e_2$ to shift through phase-lock coupling. The categorical state change propagates through the network at electromagnetic speed, even though each individual electron displaces by only $\delta x \sim 10^{-10}$ m.

The speed of categorical propagation equals the signal velocity:
\begin{equation}
v_{\text{categorical}} = v_s = \frac{c}{\sqrt{\varepsilon_r \mu_r}}
\label{eq:categorical_speed}
\end{equation}

This follows because categorical state changes propagate through electromagnetic coupling, which is mediated by photons traveling at speed $c/\sqrt{\varepsilon_r \mu_r}$ in the medium.

\begin{figure*}[htbp]
\centering
\includegraphics[width=\textwidth]{figures/panel_aperture_carriers.png}
\caption{\textbf{Transport Through Apertures: Carrier-Dependent Mechanisms.} 
(\textbf{Top left}) Electrons through lattice apertures: A 10 $\times$ 8 lattice with fixed ions (blue circles) and a mobile electron chain (cyan trajectory). The electron navigates through apertures between lattice sites. Transverse displacement shows scattering events where the electron changes direction due to lattice interactions. 
(\textbf{Top right}) Phonons through mode-matching apertures: Spectral density of transmitted phonons as a function of frequency $\omega$ (THz). Source phonons (brown) encounter apertures (green) that selectively transmit certain frequencies. Only phonons matching the aperture resonance frequencies pass through (transmitted, white curve). The selectivity creates distinct transmission peaks at characteristic frequencies. 
(\textbf{Bottom left}) Viscous fluid through collision apertures: Molecular trajectories in a dense fluid where molecules (green circles) undergo frequent collisions. Yellow arrows indicate velocity vectors. The red dashed line marks a boundary where flow is impeded by collision apertures. Molecules must navigate through gaps between neighbors, creating viscous resistance. 
(\textbf{Bottom right}) Ideal gas through sparse collision apertures: Molecular trajectories in a dilute gas where molecules (magenta circles) travel long distances between collisions. Cyan arrows show velocity vectors, with one molecule's mean free path highlighted in yellow ($\lambda_{\text{mfp}}$). Collisions are rare, so molecules pass through apertures with minimal resistance. 
The four panels demonstrate that aperture transport is universal across carrier types: electrons in solids, phonons in crystals, molecules in fluids, and atoms in gases all experience selective transmission through categorical apertures.}
\label{fig:aperture_carriers}
\end{figure*}

\subsection{Dimensional Reduction from Phase-Lock Structure}

The phase-lock network structure enables the dimensional reduction stated in Theorem \ref{thm:conductor_reduction}. A three-dimensional conductor with $\sim 10^{23}$ electrons appears to require $\sim 10^{23}$ degrees of freedom—one for each electron's position. But the phase-lock coupling imposes constraints.

In a dense phase-lock network ($\rho_{\mathcal{G}} \approx 1$), electrons cannot move independently. When one electron shifts, all phase-locked neighbours must shift to maintain coherence. The network responds collectively, not individually.

Consider a cross-section of the conductor at position $x$. This cross-section contains $N_\perp = nA$ electrons, where $n$ is the electron density and $A$ is the cross-sectional area. Without phase-locking, these electrons would have $N_\perp$ independent degrees of freedom. But phase-locking imposes the constraint:
\begin{equation}
C_1(x,t) = C_2(x,t) = \cdots = C_{N_\perp}(x,t) \equiv C_\perp(x,t)
\label{eq:cross_section_constraint}
\end{equation}

All electrons in the cross-section must be in the same categorical state. This reduces the $N_\perp$ degrees of freedom to a single collective degree of freedom $C_\perp(x,t)$.

The remaining degree of freedom is the longitudinal variation $C_\perp(x,t)$ as a function of position $x$ along the conductor. This is the one-dimensional S-transformation. The three-dimensional conductor thus reduces to:
\begin{equation}
\text{3D Conductor} = \underbrace{\text{0D Cross-Section}}_{\text{fixed } C_\perp} \times \underbrace{\text{1D S-Transformation}}_{\text{varying } x}
\label{eq:dimensional_reduction_mechanism}
\end{equation}

This reduction is exact for uniform conductors where the cross-section is constant along the length. It is the foundation for deriving Ohm's law in the next section.

\subsection{Summary}

Current flow in conductors operates by the Newton's cradle mechanism:
\begin{itemize}
\item Individual electrons have a drift velocity of $v_d \sim 10^{-4}$ m/s
\item Categorical state changes propagate at $v_s \sim 10^8$ m/s
\item The $10^{12}$-fold speed difference arises because the current is categorical propagation, not particle transport
\item Conduction electrons form a dense phase-lock network ($\rho_{\mathcal{G}} \approx 1$) enabling collective response
\item Phase-locking reduces the 3D conductor to 0D cross-section + 1D S-transformation
\end{itemize}

This framework resolves the velocity paradox and provides the foundation for deriving Ohm's law from first principles.
