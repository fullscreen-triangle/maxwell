%==============================================================================
% SECTION 2: NEWTON'S CRADLE PROPAGATION
%==============================================================================

\section{The Newton's Cradle Model of Current Flow}
\label{sec:newton_cradle}

\subsection{Electron Drift versus Signal Propagation}

A fundamental distinction exists between the velocity of individual electrons and the velocity of current signals in conductors.

\begin{definition}[Drift Velocity]
\label{def:drift_velocity}
The drift velocity $v_d$ is the average velocity of conduction electrons under an applied electric field:
\begin{equation}
v_d = \frac{I}{neA}
\label{eq:drift_velocity}
\end{equation}
where $I$ is current, $n$ is electron density, $e$ is electron charge, and $A$ is cross-sectional area.
\end{definition}

\begin{example}[Copper Wire]
For a copper wire with $I = 1$ A, $A = 1$ mm$^2$, and $n = 8.5 \times 10^{28}$ m$^{-3}$:
\begin{equation}
v_d = \frac{1}{8.5 \times 10^{28} \times 1.6 \times 10^{-19} \times 10^{-6}} \approx 7.4 \times 10^{-5} \text{ m/s}
\end{equation}
At this velocity, an electron would take approximately 4 hours to traverse 1 metre of wire.
\end{example}

\begin{definition}[Signal Velocity]
\label{def:signal_velocity}
The signal velocity $v_s$ is the speed at which electromagnetic disturbances propagate along the conductor:
\begin{equation}
v_s = \frac{c}{\sqrt{\varepsilon_r \mu_r}} \approx 0.5c \text{ to } 0.9c
\label{eq:signal_velocity}
\end{equation}
where $\varepsilon_r$ and $\mu_r$ are relative permittivity and permeability of the medium.
\end{definition}

\begin{theorem}[Velocity Disparity]
\label{thm:velocity_disparity}
The ratio of signal velocity to drift velocity is:
\begin{equation}
\frac{v_s}{v_d} \sim 10^{12}
\label{eq:velocity_ratio}
\end{equation}
Current propagation cannot be explained by electron transport.
\end{theorem}

\begin{proof}
From the definitions:
\begin{equation}
\frac{v_s}{v_d} = \frac{c/\sqrt{\varepsilon_r \mu_r}}{I/(neA)} = \frac{neAc}{I\sqrt{\varepsilon_r \mu_r}}
\end{equation}

For typical values ($n \sim 10^{28}$ m$^{-3}$, $A \sim 10^{-6}$ m$^2$, $I \sim 1$ A):
\begin{equation}
\frac{v_s}{v_d} \sim \frac{10^{28} \times 10^{-19} \times 10^{-6} \times 10^8}{1} \sim 10^{12}
\end{equation}

The signal arrives $10^{12}$ times faster than electrons could carry it. \qed
\end{proof}

\subsection{The Newton's Cradle Mechanism}

\begin{definition}[Newton's Cradle Propagation]
\label{def:newton_cradle}
In a Newton's cradle, momentum transfers through a chain of objects without individual objects traversing the chain:
\begin{equation}
\text{Ball}_1 \xrightarrow{\text{collision}} \text{Ball}_2 \xrightarrow{\text{collision}} \cdots \xrightarrow{\text{collision}} \text{Ball}_N
\end{equation}
The momentum transfer time $\tau_{\text{transfer}} \ll L/v_{\text{ball}}$, where $L$ is the chain length.
\end{definition}

\begin{theorem}[Electron Chain Transfer]
\label{thm:electron_chain}
Electrical current propagates through successive electron displacement:
\begin{equation}
e_1 \to e_2 \to e_3 \to \cdots \to e_N
\end{equation}
where each $e_i \to e_{i+1}$ represents a displacement event, not physical electron motion between sites.
\end{theorem}

\begin{proof}
Consider a conductor of length $L$ with electron density $n$. The total number of electrons is $N = nAL$.

When a potential difference is applied:
\begin{enumerate}
\item Electron $e_1$ at position $x_1$ shifts by $\delta x$ due to the field
\item This shift displaces $e_2$ at position $x_2$ through Coulomb repulsion
\item The displacement propagates through the electron gas at speed $v_s \sim c$
\item Electron $e_N$ at the far end shifts, constituting current flow
\end{enumerate}

The propagation time is:
\begin{equation}
t_{\text{propagate}} = \frac{L}{v_s} \sim \frac{L}{c}
\end{equation}

The time for a single electron to traverse:
\begin{equation}
t_{\text{drift}} = \frac{L}{v_d} \sim \frac{L}{10^{-4} \text{ m/s}}
\end{equation}

Since $t_{\text{propagate}} \ll t_{\text{drift}}$, current propagation is a collective displacement, not individual transport. \qed
\end{proof}

\subsection{Phase-Lock Network of Conduction Electrons}

\begin{definition}[Electron Phase-Lock Network]
\label{def:electron_network}
The conduction electrons in a metal form a phase-lock network $\mathcal{G} = (V, E)$ where:
\begin{itemize}
\item Vertices $V$: electron oscillatory modes
\item Edges $E$: phase-lock relationships through Coulomb interaction
\end{itemize}
Edge weight $g_{ij}$ quantifies the coupling strength between electrons $i$ and $j$.
\end{definition}

\begin{theorem}[Network Density in Metals]
\label{thm:metal_network_density}
Conduction electrons in metals form a dense phase-lock network with density:
\begin{equation}
\rho_{\mathcal{G}} = \frac{|E|}{|V|(|V|-1)/2} \approx 1
\label{eq:metal_network_density}
\end{equation}
This is the defining characteristic of metallic conduction.
\end{theorem}

\begin{proof}
In metals, conduction electrons are delocalised across the lattice. The Fermi energy $E_F \sim 5$ eV corresponds to wavelength:
\begin{equation}
\lambda_F = \frac{h}{\sqrt{2m_e E_F}} \sim 0.5 \text{ nm}
\end{equation}

This is comparable to the lattice spacing $a \sim 0.3$ nm. Each electron's wavefunction overlaps with $\sim 10$ nearest neighbours. The resulting Coulomb coupling creates phase-lock edges between all overlapping electrons.

For $N \sim 10^{28}$ electrons per m$^3$, essentially all pairs within the coherence length are coupled, giving $\rho_{\mathcal{G}} \approx 1$. \qed
\end{proof}

\begin{corollary}[Conductor versus Insulator]
\label{cor:conductor_insulator}
The distinction between conductors and insulators is:
\begin{align}
\text{Conductor:} \quad &\rho_{\mathcal{G}} \approx 1 \quad \text{(dense electron network)} \\
\text{Insulator:} \quad &\rho_{\mathcal{G}} \ll 1 \quad \text{(sparse electron network)}
\end{align}
\end{corollary}

\subsection{Categorical Interpretation}

\begin{definition}[Current as Categorical Propagation]
\label{def:current_categorical}
Electric current is the propagation of categorical state changes through the electron phase-lock network:
\begin{equation}
I = \frac{dQ}{dt} = e \cdot \frac{d}{dt}\left[\sum_i C_i\right]
\label{eq:current_categorical}
\end{equation}
where $C_i$ denotes the categorical state of electron $i$ and the sum represents the total categorical displacement.
\end{definition}

\begin{theorem}[Categorical Current Speed]
\label{thm:categorical_speed}
The speed of categorical propagation through the electron network equals the signal velocity:
\begin{equation}
v_{\text{categorical}} = v_s = \frac{c}{\sqrt{\varepsilon_r \mu_r}}
\label{eq:categorical_speed}
\end{equation}
\end{theorem}

\begin{proof}
Categorical state changes propagate through S-window adjacency. In a dense phase-lock network ($\rho_{\mathcal{G}} \approx 1$), every electron is within S-window distance of its neighbours. The S-transformation rate is limited only by the electromagnetic coupling speed.

The electromagnetic field mediating electron-electron interaction propagates at:
\begin{equation}
v_{\text{EM}} = \frac{c}{\sqrt{\varepsilon_r \mu_r}}
\end{equation}

Therefore, categorical propagation occurs at this speed. \qed
\end{proof}

\begin{remark}
The Newton's cradle model resolves the apparent paradox of slow electrons producing fast signals. Current is not electron flow; it is categorical state propagation. The dense electron network enables this propagation at electromagnetic speeds.
\end{remark}

