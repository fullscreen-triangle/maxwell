\section{Derivation of Ohm's Law from S-Transformations}
\label{sec:ohms_law_derivation}

\subsection{From Discrete Displacements to Continuous Current}

Section 2 established that current propagates through successive electron displacements—the Newton's cradle mechanism. Section 3 showed that this reduces to a one-dimensional S-transformation along the length of the conductor. We now derive Ohm's law by analyzing how discrete S-transformations produce continuous current flow.

The key insight is that current is not a single displacement but a continuous sequence of displacements. Each electron undergoes three types of interactions:

\begin{enumerate}
\item \textbf{Scattering:} Collisions with lattice vibrations (phonons), impurities, and defects
\item \textbf{Drift:} Acceleration under the applied electric field
\item \textbf{Coupling:} Interactions with neighbouring electrons through Coulomb repulsion
\end{enumerate}

These interactions transform the S-coordinate at each position along the conductor. The transformation is described by the S-transformation operator $\hat{T}_s$, which we now construct from first principles.

\subsection{The Scattering Operator: Resistance from Partition Lag}

When an electron scatters from a lattice vibration, it undergoes a categorical state change. The electron's momentum changes direction, and its phase relationship with neighboring electrons is disrupted. This disruption creates a \emph{partition lag}—a delay in the propagation of categorical states through the network.

The partition lag is characterized by the scattering time $\tau_s$, the average time between scattering events. During this time, the electron's S-coordinate relaxes toward the lattice equilibrium state $\Svec_{\text{lattice}}$.

The scattering transformation over time interval $\Delta t$ is:
\begin{equation}
\hat{T}_{\text{scatter}}[\Svec] = \Svec - \frac{\Delta t}{\tau_s}(\Svec - \Svec_{\text{lattice}})
\label{eq:scattering_operator}
\end{equation}

This is an exponential relaxation with rate $1/\tau_s$. The S-coordinate approaches the lattice state with characteristic time $\tau_s$.

\textbf{Physical interpretation:} Scattering opposes the electron's driven motion. The electric field tries to push electrons in one direction; scattering randomises their motion. The competition between driving and scattering determines the steady-state drift velocity—and hence the resistance.

The scattering time depends on the scattering mechanisms present:
\begin{equation}
\frac{1}{\tau_s} = \frac{1}{\tau_{\text{phonon}}} + \frac{1}{\tau_{\text{impurity}}} + \frac{1}{\tau_{\text{defect}}}
\label{eq:matthiessen_rule}
\end{equation}

This is \emph{Matthiessen's rule}: scattering rates add because scattering events are independent. Each mechanism contributes to the total resistance.

\begin{itemize}
\item \textbf{Phonon scattering:} Dominates at high temperature. Phonon density increases with temperature, so $\tau_{\text{phonon}}^{-1} \propto T$ for $T > \Theta_D$ (Debye temperature). This produces the linear temperature dependence of resistivity in metals: $\rho \propto T$.

\item \textbf{Impurity scattering:} Temperature-independent. Impurities create static potential variations that scatter electrons. The rate depends on impurity concentration: $\tau_{\text{impurity}}^{-1} \propto n_{\text{imp}}$.

\item \textbf{Defect scattering:} Includes grain boundaries, dislocations, and surface roughness. Also temperature-independent.
\end{itemize}

At low temperature, phonon scattering becomes negligible and impurity scattering dominates. This produces the \emph{residual resistivity} $\rho_0$—the resistance that remains as $T \to 0$.

\subsection{The Drift Operator: Acceleration Under Electric Field}

The electric field $\mathbf{E} = -\nabla \Phi$ applies force $\mathbf{F} = -e\mathbf{E}$ to each electron (taking $e > 0$ as the elementary charge magnitude). Between scattering events, electrons accelerate under this force.

After time $\tau_s$, the electron reaches drift velocity:
\begin{equation}
v_d = \frac{|F| \tau_s}{m_e} = \frac{e E \tau_s}{m_e} \equiv \mu E
\label{eq:drift_velocity_formula}
\end{equation}
where:
\begin{equation}
\mu = \frac{e \tau_s}{m_e}
\label{eq:mobility}
\end{equation}
is the \emph{mobility}—the proportionality constant between drift velocity and electric field.

The drift operator shifts the S-coordinate in the direction of current flow:
\begin{equation}
\hat{T}_{\text{drift}}[\Svec](x, t) = \Svec(x - v_d \Delta t, t)
\label{eq:drift_operator}
\end{equation}

This is a spatial translation: the S-coordinate at position $x$ at time $t + \Delta t$ equals the S-coordinate that was at position $x - v_d \Delta t$ at time $t$.

\textbf{Physical interpretation:} Drift represents the collective motion of the electron gas under the applied field. Individual electrons move slowly (drift velocity $v_d \sim 10^{-4}$ m/s), but the categorical state propagates rapidly through the phase-lock network.

The mobility $\mu$ determines how easily electrons respond to the field. High mobility means low resistance; low mobility means high resistance. The mobility is inversely proportional to the scattering time: more scattering → lower mobility → higher resistance.

\begin{figure*}[htbp]
\centering
\includegraphics[width=\textwidth]{figures/panel_scattering_apertures.png}
\caption{\textbf{Lattice Scattering as Categorical Apertures in Momentum Space.} 
(\textbf{A}) Scattering apertures in k-space: The Fermi surface (blue circle) in two-dimensional momentum space $(k_x, k_y)$. Electrons occupy states inside the Fermi surface. Impurity apertures (orange) and phonon apertures (red) scatter electrons across the Fermi surface. The green arrow shows an electron trajectory in k-space. Scattering events redirect the electron's momentum, creating resistance. 
(\textbf{B}) Scattering types and selectivities: Table showing five scattering mechanisms with their selectivities $s$, temperature dependences, and characteristic length scales $\lambda$. Phonon scattering has $s \sim 0.1$ and $\lambda \sim 10$--100 nm, with selectivity $\propto T$. Impurity scattering has $s \sim 0.01$ and is temperature-independent. Electron-electron scattering has $s \sim 0.5$ and $\propto T^2$. Grain boundary scattering has $s \sim 0.001$ and weak temperature dependence. Surface scattering has $s \sim 0.1$ with complex temperature dependence and $\lambda \sim$ film thickness. 
(\textbf{C}) Mean free path from aperture density: Mean free path $\lambda = 1/(n\sigma)$ versus scatterer density $n$ (m$^{-3}$) on a log-log plot. The blue line shows $\lambda \propto 1/n$. Horizontal dashed lines mark typical values for copper ($\lambda \sim 40$ nm, green) and iron ($\lambda \sim 5$ nm, orange). At low density ($n \sim 10^{18}$ m$^{-3}$), mean free paths are macroscopic ($\lambda \sim 10^{10}$ nm). At high density ($n \sim 10^{24}$ m$^{-3}$), mean free paths are atomic ($\lambda \sim 10^4$ nm). 
(\textbf{D}) Resistance as aperture barrier sum: Resistance $R = \sum_a \Phi_a/e = (1/e) \sum_a 1/(s_a \tau_s)$. Each scatterer (red X) acts as an aperture barrier. The total resistance is the sum of individual aperture potentials along the conductor (blue bar). This formula unifies microscopic scattering with macroscopic resistance.}
\label{fig:scattering_apertures}
\end{figure*}

\subsection{The Coupling Operator: Electron-Electron Interactions}

Electrons interact through Coulomb repulsion. When one electron shifts position, neighboring electrons respond through the phase-lock coupling (Section 2.3). This coupling spreads the S-coordinate disturbance through the electron gas.

The spreading is described by a diffusion process with a diffusion coefficient $D_e$:
\begin{equation}
\hat{T}_{\text{couple}}[\Svec] = \Svec + D_e \nabla^2 \Svec \, \Delta t
\label{eq:coupling_operator}
\end{equation}

The diffusion coefficient is related to mobility by the \emph{Einstein relation}:
\begin{equation}
D_e = \frac{k_B T}{e} \mu
\label{eq:einstein_relation}
\end{equation}

\begin{proof}[Derivation of Einstein Relation]
At thermal equilibrium, electrons have mean kinetic energy $\frac{1}{2}m_e \langle v^2 \rangle = \frac{3}{2}k_B T$, giving the mean-square velocity:
\begin{equation}
\langle v^2 \rangle = \frac{3k_B T}{m_e}
\end{equation}

For a random walk with step time $\tau_s$, the diffusion coefficient is:
\begin{equation}
D_e = \frac{\langle v^2 \rangle \tau_s}{3} = \frac{k_B T \tau_s}{m_e}
\end{equation}

Using the mobility $\mu = e\tau_s/m_e$:
\begin{equation}
D_e = \frac{k_B T}{e} \mu
\end{equation}
This is the Einstein relation. \qed
\end{proof}

\textbf{Physical interpretation:} The coupling operator smooths out S-coordinate variations. If one region has higher electron density, coupling spreads electrons to neighboring regions. This prevents charge accumulation and maintains current continuity.

\subsection{The Complete S-Transformation}

The complete S-transformation combines all three operators:
\begin{equation}
\hat{T}_s = \hat{T}_{\text{scatter}} \circ \hat{T}_{\text{drift}} \circ \hat{T}_{\text{couple}}
\label{eq:complete_operator}
\end{equation}

Applied sequentially over time interval $\Delta t$:

\textbf{Step 1 (Coupling):} Spread S-coordinate variations
\begin{equation}
\Svec^{(1)} = \Svec + D_e \nabla^2 \Svec \, \Delta t
\end{equation}

\textbf{Step 2 (Drift):} Shift in the direction of the field
\begin{equation}
\Svec^{(2)} = \Svec^{(1)}(x - v_d \Delta t)
\end{equation}

\textbf{Step 3 (Scattering):} Relax toward lattice state
\begin{equation}
\Svec^{(3)} = \Svec^{(2)} - \frac{\Delta t}{\tau_s}(\Svec^{(2)} - \Svec_{\text{lattice}})
\end{equation}

Combining these and converting to spatial derivative using $\Delta t = \Delta x / v_d$:
\begin{equation}
\Svec(x + \Delta x) = \Svec(x) - \frac{\Delta x}{\lambda}(\Svec - \Svec_{\text{eq}}) + \frac{D_e}{v_d} \frac{\partial^2 \Svec}{\partial x^2} \Delta x - \frac{\partial \Svec}{\partial x} \Delta x
\label{eq:discrete_transformation}
\end{equation}
where $\lambda = v_d \tau_s$ is the mean free path.

In the continuum limit $\Delta x \to 0$, this becomes:
\begin{equation}
\frac{\partial \Svec}{\partial t} + v_d \frac{\partial \Svec}{\partial x} = D_e \frac{\partial^2 \Svec}{\partial x^2} - \frac{v_d}{\lambda}(\Svec - \Svec_{\text{eq}})
\label{eq:s_transport_equation}
\end{equation}

This is the \emph{S-transport equation}—the fundamental equation governing current flow in the S-coordinate framework.

\subsection{Steady-State Solution and Ohm's Law}

In steady state, $\partial \Svec/\partial t = 0$. For a uniform conductor with constant electric field $E = V/L$, the S-coordinate reaches a constant gradient:
\begin{equation}
\frac{d\Svec}{dx} = \text{constant}
\label{eq:steady_state_gradient}
\end{equation}

The steady-state S-transport equation becomes:
\begin{equation}
v_d \frac{d\Svec}{dx} = D_e \frac{d^2\Svec}{dx^2} - \frac{v_d}{\lambda}(\Svec - \Svec_{\text{eq}})
\label{eq:steady_state_equation}
\end{equation}

For a constant gradient, $d^2\Svec/dx^2 = 0$, so:
\begin{equation}
v_d \frac{d\Svec}{dx} = -\frac{v_d}{\lambda}(\Svec - \Svec_{\text{eq}})
\end{equation}

This gives:
\begin{equation}
\Svec - \Svec_{\text{eq}} = -\lambda \frac{d\Svec}{dx}
\label{eq:s_deviation}
\end{equation}

The S-coordinate deviates from equilibrium in proportion to its gradient, with the proportionality constant equal to the mean free path $\lambda$.

The current density is proportional to the S-potential gradient:
\begin{equation}
J = -\sigma \frac{d\Phi_S}{dx}
\label{eq:current_density_s}
\end{equation}

For the dominant S-component (knowledge deficit $S_k$), the S-potential is $\Phi_S \approx \alpha S_k$, so:
\begin{equation}
J = -\sigma \alpha \frac{dS_k}{dx}
\label{eq:current_from_sk}
\end{equation}

The gradient $dS_k/dx$ is determined by the applied voltage. For a conductor of length $L$ with voltage $V$:
\begin{equation}
\frac{dS_k}{dx} = \frac{S_k(0) - S_k(L)}{L} = -\frac{V}{\alpha L}
\label{eq:sk_gradient}
\end{equation}

Therefore:
\begin{equation}
J = \sigma \frac{V}{L}
\label{eq:current_density_ohm}
\end{equation}

Integrating over the cross-sectional area $A$:
\begin{equation}
I = JA = \sigma A \frac{V}{L}
\label{eq:current_ohm}
\end{equation}

Rearranging:
\begin{equation}
V = \frac{L}{\sigma A} I = RI
\label{eq:ohms_law}
\end{equation}

This is \textbf{Ohm's law}, with resistance:
\begin{equation}
R = \frac{L}{\sigma A} = \frac{\rho L}{A}
\label{eq:resistance}
\end{equation}
where $\rho = 1/\sigma$ is the resistivity.

We have derived Ohm's law from the S-transformation structure without invoking empirical relations. The resistance emerges from the scattering partition lag $\tau_s$ through the mobility $\mu = e\tau_s/m_e$ and conductivity $\sigma = ne\mu$.

\begin{figure*}[htbp]
\centering
\includegraphics[width=\textwidth]{figures/panel_transformation_operator.png}
\caption{\textbf{The S-Transformation Operator: Decomposition and Experimental Validation.} 
(\textbf{A}) Operator decomposition: An initial S-profile (black dashed) undergoes three sequential transformations: advection $\mathcal{T}_{\text{adv}}$ (blue), diffusion $\mathcal{T}_{\text{diff}}$ (green), and partition $\mathcal{T}_{\text{part}}$ (red, final). Each operator modifies the S-coordinate distribution in a characteristic way. 
(\textbf{B}) Partition operator equilibration: S-coordinate evolution toward stationary state $S_{\text{stat}}$ (red dashed line). Four trajectories with initial values $S_0 = 1.0, 3.0, 7.0, 9.0$ all converge to $S_{\text{stat}} = 5.0$. The approach is exponential with time constant $\tau_p$. 
(\textbf{C}) Diffusion operator S-spreading: S-density profiles at times $t = 0, 0.5, 1.0, 2.0, 4.0$. An initially sharp peak (purple, $t = 0$) spreads according to $\sigma = \sqrt{2 D_S t}$. The peak height decreases and width increases while conserving total S-coordinate. 
(\textbf{D}) Advection operator S-translation: S-profiles at times $t = 0, 1, 2, 3, 4$ (cyan to magenta). The profile translates rigidly with velocity $v = 2.0$ (purple arrow). The profile shape is preserved during translation. 
(\textbf{E}) Operator composition: Relative error in $\mathcal{T}_{0 \to x} = \mathcal{T}_{dx}^{(x/dx)}$ versus number of steps. The error decreases exponentially, reaching $< 0.1\%$ after 100 steps. This validates the composition property of S-transformations. 
(\textbf{F}) Partition coefficient: $K = K_0 \exp(-d_S/\sigma_S)$ versus S-distance $d_S$ for four values of $\sigma_S = 0.5, 1.0, 2.0, 3.0$ (red to yellow). Larger $\sigma_S$ gives slower decay, indicating weaker selectivity. The partition coefficient quantifies how readily a system transitions between categorical states separated by S-distance $d_S$.}
\label{fig:transformation_operator}
\end{figure*}

\subsection{Resistivity from Scattering Partition Lag}

The conductivity can be expressed in terms of microscopic parameters:
\begin{equation}
\sigma = ne\mu = ne \cdot \frac{e\tau_s}{m_e} = \frac{ne^2 \tau_s}{m_e}
\label{eq:conductivity_microscopic}
\end{equation}

The resistivity is:
\begin{equation}
\rho = \frac{1}{\sigma} = \frac{m_e}{ne^2 \tau_s}
\label{eq:resistivity_microscopic}
\end{equation}

This is the \emph{Drude formula} for resistivity, but derived here from partition lag rather than from kinetic theory.

The scattering time $\tau_s$ is the partition lag—the time delay between when an electron begins to respond to the field and when it scatters, losing its directed momentum. This lag creates resistance.

For a conductor with multiple scattering mechanisms:
\begin{equation}
\rho = \frac{m_e}{ne^2} \left(\frac{1}{\tau_{\text{phonon}}} + \frac{1}{\tau_{\text{impurity}}} + \frac{1}{\tau_{\text{defect}}}\right)
\label{eq:resistivity_components}
\end{equation}

Each scattering mechanism contributes additively to the resistivity. This is why:
\begin{itemize}
\item Pure metals have low resistivity (only phonon scattering)
\item Alloys have high resistivity (additional impurity scattering)
\item Annealing reduces resistivity (removes defects)
\end{itemize}

\subsection{Temperature Dependence}

The temperature dependence of resistivity arises from phonon scattering. At high temperature ($T > \Theta_D$, where $\Theta_D$ is the Debye temperature), the phonon density is proportional to temperature:
\begin{equation}
n_{\text{phonon}} \propto T
\end{equation}

The scattering rate is proportional to phonon density:
\begin{equation}
\frac{1}{\tau_{\text{phonon}}} \propto n_{\text{phonon}} \propto T
\end{equation}

Therefore:
\begin{equation}
\rho(T) = \rho_0 + \alpha T
\label{eq:resistivity_temperature}
\end{equation}
where:
\begin{itemize}
\item $\rho_0$: residual resistivity (from impurities and defects, temperature-independent)
\item $\alpha T$: phonon contribution (linear in temperature)
\end{itemize}

This linear temperature dependence is observed in all metals at room temperature. It is a direct consequence of the phonon scattering partition lag increasing with temperature.

At low temperature ($T \ll \Theta_D$), phonon scattering becomes negligible and the resistivity approaches the residual value:
\begin{equation}
\lim_{T \to 0} \rho(T) = \rho_0
\label{eq:residual_resistivity}
\end{equation}

The ratio $\rho(300\text{ K})/\rho_0$ is called the \emph{residual resistivity ratio} (RRR). High-purity metals have RRR $> 100$, indicating that phonon scattering dominates at room temperature.

\subsection{Comparison with Drude Model}

The Drude model (1900) treats electrons as classical particles undergoing collisions with a characteristic time $\tau$. It predicts:
\begin{equation}
\sigma_{\text{Drude}} = \frac{ne^2 \tau}{m_e}
\end{equation}

Our S-transformation framework gives the same formula, but with a crucial difference in interpretation:

\begin{center}
\begin{tabular}{lcc}
\toprule
\textbf{Aspect} & \textbf{Drude Model} & \textbf{S-Transformation} \\
\midrule
Current mechanism & Particle transport & Categorical propagation \\
Electron velocity & Drift velocity $v_d$ & Phase-lock velocity $v_s$ \\
Collision time & Empirical parameter & Partition lag $\tau_s$ \\
Resistance origin & Momentum loss & Categorical state disruption \\
Temperature dependence & Assumed & Derived from phonon lag \\
\bottomrule
\end{tabular}
\end{center}

The Drude model treats $\tau$ as an empirical parameter fitted to experimental data. The S-transformation framework derives $\tau_s$ from the partition structure of electron-lattice interactions.

Moreover, the Drude model cannot explain why the signal velocity ($\sim 10^8$ m/s) is so much faster than the drift velocity ($\sim 10^{-4}$ m/s). The S-transformation framework resolves this: current is categorical propagation at phase-lock velocity $v_s$, not particle transport at drift velocity $v_d$.

\subsection{Summary}

We have derived Ohm's law from first principles:

\begin{enumerate}
\item The S-transformation operator combines scattering, drift, and coupling: $\hat{T}_s = \hat{T}_{\text{scatter}} \circ \hat{T}_{\text{drift}} \circ \hat{T}_{\text{couple}}$

\item In steady state, the S-coordinate reaches a constant gradient determined by the applied voltage

\item Current density is proportional to the S-gradient: $J = \sigma V/L$

\item This gives Ohm's law: $V = IR$ with $R = \rho L/A$

\item Resistivity is determined by scattering partition lag: $\rho = m_e/(ne^2\tau_s)$

\item Temperature dependence arises from phonon scattering: $\rho(T) = \rho_0 + \alpha T$
\end{enumerate}

The next section extends this framework to circuit networks, deriving Kirchhoff's laws from categorical state conservation.
