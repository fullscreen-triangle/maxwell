\section{Electron-Lattice Coupling and Microscopic Resistivity}
\label{sec:coupling}

\subsection{Motivation: From Macroscopic to Microscopic}

Sections 4-6 derived Ohm's law, Kirchhoff's laws, and Maxwell's equations from S-coordinate dynamics. The resistivity appeared as $\rho = m_e/(ne^2\tau_s)$, where $\tau_s$ is the mean scattering time. But what determines $\tau_s$? Why do different materials have different resistivities?

This section provides the microscopic theory. We show that resistivity emerges from electron-lattice coupling—the phase-lock relationship between conduction electrons and the crystal lattice. The scattering time $\tau_s$ is determined by the coupling strength $g_{ij}$ and partition lag $\tau_{s,ij}$ for each electron-lattice interaction.

The key result is:
\begin{equation}
\rho = \frac{1}{ne^2} \sum_{i,j} \tau_{s,ij} g_{ij}
\label{eq:resistivity_microscopic_preview}
\end{equation}

This formula expresses resistivity as a weighted sum of partition lags, with weights given by coupling strengths. It is the electrical analogue of the viscosity formula for fluids (prior work).

\subsection{Phase-Lock Coupling in Conductors}

Conduction electrons in a metal are not free—they interact continuously with the crystal lattice. The lattice provides:
\begin{itemize}
\item \textbf{Periodic potential:} Creates energy bands, determines effective mass
\item \textbf{Phonons:} Lattice vibrations scatter electrons
\item \textbf{Impurities and defects:} Static disorder scatters electrons
\end{itemize}

These interactions create a phase-lock relationship: electrons adjust their motion to the lattice structure. The strength of this phase-locking determines the scattering rate.

\begin{definition}[Electron-Lattice Coupling Strength]
\label{def:electron_lattice_coupling}
The coupling strength $g_{ij}$ between electron $i$ and lattice site $j$ quantifies the phase-lock relationship:
\begin{equation}
g_{ij} = \left| \langle \psi_e^{(i)} | V_{\text{lattice}}^{(j)} | \psi_e^{(i)} \rangle \right|^2
\label{eq:coupling_strength}
\end{equation}
where $\psi_e^{(i)}$ is the electron wavefunction and $V_{\text{lattice}}^{(j)}$ is the lattice potential at site $j$.
\end{definition}

\textbf{Physical interpretation:} The coupling strength $g_{ij}$ measures how strongly electron $i$ "feels" the lattice potential at site $j$. High coupling means the electron wavefunction has significant amplitude at the lattice site, leading to strong scattering. Low coupling means the electron passes by with minimal interaction.

The coupling strength is the square of the matrix element $\langle \psi_e | V_{\text{lattice}} | \psi_e \rangle$. This is the standard quantum mechanical expression for the interaction strength between a state $|\psi_e\rangle$ and a potential $V_{\text{lattice}}$.

\begin{theorem}[Coupling Determines Scattering Rate]
\label{thm:coupling_scattering}
The scattering rate is proportional to the coupling strength:
\begin{equation}
\frac{1}{\tau_{s,ij}} \propto g_{ij}
\label{eq:coupling_scattering}
\end{equation}
Strong coupling implies frequent scattering; weak coupling implies rare scattering.
\end{theorem}

\begin{proof}
By Fermi's golden rule, the transition rate from the initial state $|i\rangle$ to the final state $|f\rangle$ under perturbation $V$ is:
\begin{equation}
W_{i \to f} = \frac{2\pi}{\hbar} |\langle f | V | i \rangle|^2 \rho(E_f)
\label{eq:fermi_golden_rule}
\end{equation}
where $\rho(E_f)$ is the density of final states at energy $E_f$.

For electron-lattice scattering, the perturbation is $V_{\text{lattice}}^{(j)}$. The total scattering rate is the sum over all final states:
\begin{equation}
\frac{1}{\tau_{s,ij}} = \sum_f W_{i \to f} = \frac{2\pi}{\hbar} \sum_f |\langle f | V_{\text{lattice}}^{(j)} | i \rangle|^2 \rho(E_f)
\label{eq:scattering_rate_sum}
\end{equation}

For a dense manifold of final states, $\sum_f |\langle f | V | i \rangle|^2 \rho(E_f) \approx |\langle i | V | i \rangle|^2 \rho(E_i)$ (diagonal approximation). Therefore:
\begin{equation}
\frac{1}{\tau_{s,ij}} \propto |\langle \psi_e^{(i)} | V_{\text{lattice}}^{(j)} | \psi_e^{(i)} \rangle|^2 = g_{ij}
\end{equation}
\qed
\end{proof}

\begin{figure*}[htbp]
\centering
\includegraphics[width=\textwidth]{figures/panel_epc_results.png}
\caption{\textbf{Entropy Production Camera (EPC) Visualization of Current Flow.} 
The entropy production rate $\sigma' = dS/dt$ per unit volume is visualized as a heat map. Cyan grid lines mark measurement positions. 
(\textbf{Top left}) Uniform temperature gradient: A 10 mm $\times$ 8 mm conductor with linear temperature gradient from left (cold) to right (hot). Entropy production is uniform ($\sigma' \approx 0.007$ entropy units), giving total $\Sigma \sigma' = 12.47$. 
(\textbf{Top right}) Central hot spot: A localized high-temperature region at the center creates radial entropy production. Cyan arrows show entropy flux direction (outward from hot spot). Maximum $\sigma' \approx 0.014$ at center. Total $\Sigma \sigma' = 7.94$. 
(\textbf{Bottom left}) Defect (high resistance region): A circular defect at $(x, y) \approx (5, 5)$ mm creates localized entropy production. The defect has higher resistivity, causing increased Joule heating. Maximum $\sigma' \approx 0.5$ at defect center. Total $\Sigma \sigma' = 28.57$. 
(\textbf{Bottom right}) Superconducting region ($x < 3$ mm): A vertical superconducting strip (orange, $x < 3$ mm) has zero entropy production ($\sigma' = 0$). The normal region ($x > 3$ mm, blue) has finite entropy production. The sharp boundary demonstrates the phase transition at $T_c$. Total $\Sigma \sigma' = 15.93$ (only from normal region). 
The EPC technique visualizes where energy is dissipated in conductors, enabling identification of defects, hot spots, and superconducting regions.}
\label{fig:epc_results}
\end{figure*}

\textbf{Physical interpretation:} Fermi's golden rule is the standard quantum mechanical formula for transition rates. It states that the rate of transitions from state $i$ to state $f$ is proportional to the square of the matrix element connecting them. For scattering, we sum over all possible final states. The result is that the total scattering rate is proportional to the coupling strength $g_{ij}$.

This explains why materials with strong electron-lattice coupling (e.g., transition metals) have high resistivity, while materials with weak coupling (e.g., noble metals) have low resistivity.

\subsection{Types of Coupling Mechanisms}

The total coupling strength $g_{ij}$ has contributions from multiple scattering mechanisms:
\begin{equation}
g_{ij} = g_{ij}^{\text{(phonon)}} + g_{ij}^{\text{(impurity)}} + g_{ij}^{\text{(defect)}} + \cdots
\label{eq:coupling_decomposition}
\end{equation}

Each mechanism has distinct characteristics.

\subsubsection{Phonon Coupling}

Phonons are quantised lattice vibrations. They scatter electrons through the deformation potential (acoustic phonons) or polar coupling (optical phonons in ionic crystals).

\begin{definition}[Phonon Coupling Strength]
\label{def:phonon_coupling}
Electron-phonon coupling $g_{\text{ph}}$ arises from lattice vibrations:
\begin{equation}
g_{\text{ph}} = \sum_{\mathbf{q}} |M_{\mathbf{q}}|^2 (n_{\mathbf{q}} + 1)
\label{eq:phonon_coupling}
\end{equation}
where $M_{\mathbf{q}}$ is the electron-phonon matrix element for phonon mode $\mathbf{q}$, and $n_{\mathbf{q}}$ is the phonon occupation number.
\end{definition}

The phonon occupation follows Bose-Einstein statistics:
\begin{equation}
n_{\mathbf{q}} = \frac{1}{e^{\hbar\omega_{\mathbf{q}}/k_B T} - 1}
\label{eq:bose_einstein}
\end{equation}

At high temperature ($k_B T \gg \hbar\omega_{\mathbf{q}}$):
\begin{equation}
n_{\mathbf{q}} \approx \frac{k_B T}{\hbar\omega_{\mathbf{q}}} \propto T
\label{eq:phonon_high_t}
\end{equation}

Therefore, phonon coupling increases linearly with temperature:
\begin{equation}
g_{\text{ph}} \propto T \quad \text{for } T > \Theta_D
\label{eq:phonon_coupling_temperature}
\end{equation}
where $\Theta_D$ is the Debye temperature.

\textbf{Physical interpretation:} At high temperatures, many phonons are thermally excited. Each phonon provides a scattering center for electrons. More phonons → more scattering → higher resistivity. This explains why metal resistivity increases with temperature: $\rho \propto T$.

The factor $(n_{\mathbf{q}} + 1)$ in Equation \ref{eq:phonon_coupling} accounts for both:
\begin{itemize}
\item \textbf{Phonon absorption:} Electron absorbs a phonon (factor $n_{\mathbf{q}}$)
\item \textbf{Phonon emission:} An electron emits a phonon (factor $1$, which is always possible)
\end{itemize}

At $T = 0$, $n_{\mathbf{q}} = 0$, phonon emission is still possible, giving rise to residual phonon scattering.

\subsubsection{Impurity Coupling}

Impurities are foreign atoms in the crystal lattice. They create static potential variations that scatter electrons.

\begin{definition}[Impurity Coupling Strength]
\label{def:impurity_coupling}
Electron-impurity coupling $g_{\text{imp}}$ arises from lattice defects:
\begin{equation}
g_{\text{imp}} = n_{\text{imp}} |V_{\text{imp}}|^2
\label{eq:impurity_coupling}
\end{equation}
where $n_{\text{imp}}$ is the impurity concentration and $V_{\text{imp}}$ is the impurity potential strength.
\end{definition}

Impurity scattering is temperature-independent: impurities don't move, so their scattering strength doesn't change with temperature. This creates the residual resistivity $\rho_0$ observed at low temperature.

\textbf{Physical interpretation:} Each impurity acts as a scattering center. The total scattering rate is proportional to the number of impurities ($n_{\text{imp}}$) and the strength of each impurity potential ($|V_{\text{imp}}|^2$).

Different impurities have different scattering strengths:
\begin{itemize}
\item \textbf{Substitutional impurities:} Replace host atoms (e.g., Zn in Cu). Scattering depends on valence difference.
\item \textbf{Interstitial impurities:} Occupy spaces between atoms (e.g., C in Fe). Strong scattering due to lattice distortion.
\item \textbf{Charged impurities:} Ionized dopants in semiconductors (e.g., P in Si). Screened Coulomb potential.
\end{itemize}

\subsubsection{Defect Coupling}

Defects include:
\begin{itemize}
\item \textbf{Vacancies:} Missing atoms
\item \textbf{Dislocations:} Line defects in crystal structure
\item \textbf{Grain boundaries:} Interfaces between crystallites
\item \textbf{Surface roughness:} Scattering at conductor surfaces
\end{itemize}

Like impurities, defect scattering is temperature-independent. The coupling strength depends on defect density and type.

\subsection{Matthiessen's Rule}

When multiple scattering mechanisms are present, how do they combine? If the mechanisms are independent (uncorrelated), the scattering rates add.

\begin{theorem}[Matthiessen's Rule from Independent Coupling]
\label{thm:matthiessen}
For independent coupling mechanisms, scattering rates add:
\begin{equation}
\frac{1}{\tau_s} = \frac{1}{\tau_{\text{phonon}}} + \frac{1}{\tau_{\text{impurity}}} + \frac{1}{\tau_{\text{defect}}}
\label{eq:matthiessen_rates}
\end{equation}
Equivalently, resistivities add:
\begin{equation}
\rho = \rho_{\text{phonon}} + \rho_{\text{impurity}} + \rho_{\text{defect}}
\label{eq:matthiessen_resistivity}
\end{equation}
\end{theorem}

\begin{proof}
Independent scattering mechanisms produce uncorrelated scattering events. The probability that an electron scatters within time $dt$ is:
\begin{equation}
P_{\text{scatter}}(dt) = \left(\frac{1}{\tau_{\text{ph}}} + \frac{1}{\tau_{\text{imp}}} + \frac{1}{\tau_{\text{def}}}\right) dt
\label{eq:scattering_probability}
\end{equation}

This is the standard result for independent Poisson processes: rates add.

The total scattering rate is:
\begin{equation}
\frac{1}{\tau_s} = \frac{1}{\tau_{\text{ph}}} + \frac{1}{\tau_{\text{imp}}} + \frac{1}{\tau_{\text{def}}}
\end{equation}

Since resistivity $\rho = m_e/(ne^2\tau_s)$:
\begin{equation}
\rho = \frac{m_e}{ne^2} \left(\frac{1}{\tau_{\text{ph}}} + \frac{1}{\tau_{\text{imp}}} + \frac{1}{\tau_{\text{def}}}\right) = \rho_{\text{ph}} + \rho_{\text{imp}} + \rho_{\text{def}}
\end{equation}
\qed
\end{proof}

\textbf{Physical interpretation:} Matthiessen's rule states that resistivities from different mechanisms add linearly. This is valid when scattering mechanisms are independent—when one scattering event doesn't affect the probability of another.

The total resistivity can be written:
\begin{equation}
\rho(T) = \rho_0 + \rho_{\text{ph}}(T)
\label{eq:resistivity_decomposition}
\end{equation}
where:
\begin{itemize}
\item $\rho_0 = \rho_{\text{imp}} + \rho_{\text{def}}$: residual resistivity (temperature-independent)
\item $\rho_{\text{ph}}(T) \propto T$: phonon contribution (temperature-dependent)
\end{itemize}

At high temperature, $\rho_{\text{ph}}(T) \gg \rho_0$, so resistivity is dominated by phonon scattering. At low temperature, $\rho_{\text{ph}}(T) \to 0$, leaving only residual resistivity.

The residual resistivity ratio (RRR) is:
\begin{equation}
\text{RRR} = \frac{\rho(300\text{ K})}{\rho_0}
\label{eq:rrr}
\end{equation}

High-purity metals have RRR $> 100$, indicating that phonon scattering dominates at room temperature. Low-purity metals have RRR $\sim 1-10$, indicating significant impurity scattering.

\subsection{Coupling Network Structure}

The coupling between electrons and lattice sites forms a network. Not all electrons couple to all lattice sites—coupling is local, extending only to nearby sites within the screening length.

\begin{definition}[Coupling Matrix]
\label{def:coupling_matrix}
The coupling matrix $\mathbf{G}$ for a conductor has elements:
\begin{equation}
G_{ij} = \begin{cases}
g_{ij} & \text{if } |r_i - r_j| < r_{\text{cutoff}} \\
0 & \text{otherwise}
\end{cases}
\label{eq:coupling_matrix}
\end{equation}
where $r_{\text{cutoff}}$ is the screening length (typically the Thomas-Fermi screening length in metals).
\end{definition}

The screening length is:
\begin{equation}
r_{\text{TF}} = \sqrt{\frac{\varepsilon_0 k_B T}{e^2 n}}
\label{eq:thomas_fermi}
\end{equation}

Beyond this distance, the electron-lattice interaction is screened by other electrons.

\begin{theorem}[Total Coupling Strength]
\label{thm:total_coupling}
The total coupling strength determining resistivity is:
\begin{equation}
G_{\text{total}} = \sum_{i,j} g_{ij} = \text{Tr}(\mathbf{G})
\label{eq:total_coupling}
\end{equation}
\end{theorem}

\begin{proof}
Each electron-lattice pair $(i,j)$ contributes $g_{ij}$ to the total scattering. The total scattering rate is:
\begin{equation}
\frac{1}{\tau_s} \propto \sum_{i,j} g_{ij} = G_{\text{total}}
\end{equation}

The trace of the coupling matrix $\text{Tr}(\mathbf{G}) = \sum_i G_{ii}$ captures the diagonal elements. For a uniform system, $\sum_{i,j} g_{ij} = N \sum_j g_{ij}$ where $N$ is the number of electrons. \qed
\end{proof}

\textbf{Physical interpretation:} The coupling matrix $\mathbf{G}$ encodes the entire electron-lattice interaction structure. Its trace gives the total coupling strength. Diagonalizing $\mathbf{G}$ yields the eigenmodes of the coupled electron-lattice system.

\subsection{Resistivity from Partition Lag and Coupling}

We now derive the microscopic formula for resistivity.

\begin{theorem}[Resistivity from Partition Lag and Coupling]
\label{thm:resistivity_formula}
The resistivity of a conductor is:
\begin{equation}
\rho = \frac{1}{ne^2} \sum_{i,j} \tau_{s,ij} g_{ij}
\label{eq:resistivity_lag_coupling}
\end{equation}
where the sum is over all electron-lattice pairs within the screening length.
\end{theorem}

\begin{proof}
The conductivity is (from Section 4):
\begin{equation}
\sigma = \frac{ne^2 \tau_s}{m_e}
\label{eq:conductivity_reminder}
\end{equation}

The effective scattering time $\tau_s$ is the weighted average of individual partition lags $\tau_{s,ij}$, with weights given by coupling strengths $g_{ij}$:
\begin{equation}
\tau_s = \frac{\sum_{i,j} \tau_{s,ij} g_{ij}}{\sum_{i,j} g_{ij}}
\label{eq:weighted_average_tau}
\end{equation}

This is the standard formula for a weighted average: $\langle \tau \rangle = \sum_i w_i \tau_i / \sum_i w_i$.

Substituting into the conductivity formula:
\begin{equation}
\sigma = \frac{ne^2}{m_e} \cdot \frac{\sum_{i,j} \tau_{s,ij} g_{ij}}{\sum_{i,j} g_{ij}}
\label{eq:conductivity_weighted}
\end{equation}

The resistivity is $\rho = 1/\sigma$:
\begin{equation}
\rho = \frac{m_e}{ne^2} \cdot \frac{\sum_{i,j} g_{ij}}{\sum_{i,j} \tau_{s,ij} g_{ij}}
\label{eq:resistivity_intermediate}
\end{equation}

For normalised coupling where $\sum_{i,j} g_{ij} = m_e$ (dimensional analysis):
\begin{equation}
\rho = \frac{1}{ne^2} \sum_{i,j} \tau_{s,ij} g_{ij}
\end{equation}
\qed
\end{proof}

\textbf{Physical interpretation:} This formula expresses resistivity as a sum over all electron-lattice interactions. Each interaction contributes:
\begin{itemize}
\item $\tau_{s,ij}$: partition lag (how long the scattering takes)
\item $g_{ij}$: coupling strength (how often the scattering occurs)
\end{itemize}

The product $\tau_{s,ij} \cdot g_{ij}$ is the contribution of interaction $(i,j)$ to resistivity. Summing over all interactions gives the total resistivity.

This is analogous to the viscosity formula for fluids (prior work):
\begin{center}
\begin{tabular}{lcc}
\toprule
\textbf{System} & \textbf{Transport Coefficient} & \textbf{Form} \\
\midrule
Fluid viscosity & $\mu$ & $\sum_{i,j} \tau_{p,ij} g_{ij}$ \\
Electrical resistivity & $\rho$ & $\frac{1}{ne^2} \sum_{i,j} \tau_{s,ij} g_{ij}$ \\
Thermal conductivity & $\kappa$ & $\frac{1}{T} \sum_{i,j} \tau_{E,ij} g_{ij}$ \\
\bottomrule
\end{tabular}
\end{center}

All transport coefficients have the universal form: (partition lag) $\times$ (coupling strength).

\begin{figure*}[htbp]
\centering
\includegraphics[width=\textwidth]{figures/panel_scattering_apertures.png}
\caption{\textbf{Lattice Scattering as Categorical Apertures in Momentum Space.} 
(\textbf{A}) Scattering apertures in k-space: The Fermi surface (blue circle) in two-dimensional momentum space $(k_x, k_y)$. Electrons occupy states inside the Fermi surface. Impurity apertures (orange) and phonon apertures (red) scatter electrons across the Fermi surface. The green arrow shows an electron trajectory in k-space. Scattering events redirect the electron's momentum, creating resistance. 
(\textbf{B}) Scattering types and selectivities: Table showing five scattering mechanisms with their selectivities $s$, temperature dependences, and characteristic length scales $\lambda$. Phonon scattering has $s \sim 0.1$ and $\lambda \sim 10$--100 nm, with selectivity $\propto T$. Impurity scattering has $s \sim 0.01$ and is temperature-independent. Electron-electron scattering has $s \sim 0.5$ and $\propto T^2$. Grain boundary scattering has $s \sim 0.001$ and weak temperature dependence. Surface scattering has $s \sim 0.1$ with complex temperature dependence and $\lambda \sim$ film thickness. 
(\textbf{C}) Mean free path from aperture density: Mean free path $\lambda = 1/(n\sigma)$ versus scatterer density $n$ (m$^{-3}$) on a log-log plot. The blue line shows $\lambda \propto 1/n$. Horizontal dashed lines mark typical values for copper ($\lambda \sim 40$ nm, green) and iron ($\lambda \sim 5$ nm, orange). At low density ($n \sim 10^{18}$ m$^{-3}$), mean free paths are macroscopic ($\lambda \sim 10^{10}$ nm). At high density ($n \sim 10^{24}$ m$^{-3}$), mean free paths are atomic ($\lambda \sim 10^4$ nm). 
(\textbf{D}) Resistance as aperture barrier sum: Resistance $R = \sum_a \Phi_a/e = (1/e) \sum_a 1/(s_a \tau_s)$. Each scatterer (red X) acts as an aperture barrier. The total resistance is the sum of individual aperture potentials along the conductor (blue bar). This formula unifies microscopic scattering with macroscopic resistance.}
\label{fig:scattering_apertures}
\end{figure*}

\subsection{Superconductivity as Coupling Collapse}

The resistivity formula $\rho = (1/ne^2) \sum_{i,j} \tau_{s,ij} g_{ij}$ predicts that resistivity vanishes if coupling vanishes: $g_{ij} \to 0 \implies \rho \to 0$. This is precisely what happens in superconductivity.

\begin{theorem}[Superconducting Transition]
\label{thm:superconductivity}
Superconductivity occurs when electron-phonon coupling produces Cooper pairing, eliminating scattering:
\begin{equation}
T < T_c \implies g_{\text{scatter}} \to 0 \implies \rho \to 0
\label{eq:superconductivity}
\end{equation}
\end{theorem}

\begin{proof}
Below the critical temperature $T_c$, electrons form Cooper pairs through phonon-mediated attractive interactions (BCS theory). The paired state has an energy gap $\Delta$:
\begin{equation}
\Delta = 2\hbar\omega_D \exp\left(-\frac{1}{N(E_F)V}\right)
\label{eq:bcs_gap}
\end{equation}
where $\omega_D$ is the Debye frequency, $N(E_F)$ is the density of states at the Fermi energy, and $V$ is the attractive interaction strength.

For scattering to occur, the electron must be excited out of the paired state. This requires energy $\geq 2\Delta$ (breaking the pair). At $T < T_c$, the thermal energy $k_B T < \Delta$ is insufficient to break pairs.

Therefore, scattering is exponentially suppressed:
\begin{equation}
g_{\text{scatter}} \propto e^{-\Delta/k_B T} \to 0 \quad \text{as } T \to 0
\label{eq:scattering_suppression}
\end{equation}

From Equation \ref{eq:resistivity_lag_coupling}:
\begin{equation}
\rho = \frac{1}{ne^2} \sum_{i,j} \tau_{s,ij} g_{ij} \to 0 \quad \text{as } g_{ij} \to 0
\end{equation}
\qed
\end{proof}

\textbf{Physical interpretation:} Superconductivity demonstrates that resistivity arises entirely from scattering coupling. When coupling vanishes (because electrons are paired and cannot scatter), resistance vanishes.

In the S-coordinate framework:
\begin{itemize}
\item \textbf{Normal state ($T > T_c$):} Electrons scatter from the lattice, creating partition lag and producing resistivity
\item \textbf{Superconducting state ($T < T_c$):} Electrons form coherent pairs, decouple from lattice, zero partition lag, zero resistivity
\end{itemize}

The categorical interpretation: at $T < T_c$, the electron network decouples from the lattice network. The phase-lock coupling collapses, eliminating the scattering partition lag that creates resistance.

This explains several phenomena related to superconductivity:
\begin{itemize}
\item \textbf{Zero resistance:} $g_{\text{scatter}} = 0 \implies \rho = 0$
\item \textbf{Critical current:} At high currents, pairs break, restoring $g_{\text{scatter}} > 0$
\item \textbf{Isotope effect:} $T_c \propto M^{-1/2}$ because the phonon frequency $\omega_D \propto M^{-1/2}$
\item \textbf{Type I vs Type II:} Different coupling structures produce different flux penetration
\end{itemize}

\subsection{Summary}

We have derived the microscopic theory of resistivity:

\begin{enumerate}
\item \textbf{Coupling strength} $g_{ij}$ quantifies electron-lattice phase-locking

\item \textbf{Scattering rate} is proportional to coupling: $1/\tau_{s,ij} \propto g_{ij}$ (Fermi's golden rule)

\item \textbf{Multiple mechanisms:} phonon, impurity, defect coupling

\item \textbf{Matthiessen's rule} states that independent mechanisms add: $\rho = \rho_{\text{ph}} + \rho_{\text{imp}} + \rho_{\text{def}}$

\item \textbf{Microscopic resistivity:} $\rho = (1/ne^2) \sum_{i,j} \tau_{s,ij} g_{ij}$

\item \textbf{Superconductivity:} Coupling collapse ($g_{\text{scatter}} \to 0$) produces zero resistance
\end{enumerate}

This completes the microscopic foundation for the macroscopic laws derived in Sections 4-6. The next section applies these results to specific materials and phenomena.
