%==============================================================================
% SECTION 3: DIMENSIONAL REDUCTION FOR CONDUCTORS
%==============================================================================

\section{Dimensional Reduction for Conductors}
\label{sec:dimensional_reduction}

\subsection{Conductor Geometry}

\begin{definition}[Conductor Parameters]
\label{def:conductor_parameters}
A conductor is characterised by:
\begin{itemize}
\item Length $L$ along the current flow direction
\item Cross-sectional area $A$ perpendicular to flow
\item Lattice spacing $a_0$ (typically 0.2--0.4 nm for metals)
\item Electron density $n$ (typically $10^{28}$--$10^{29}$ m$^{-3}$ for metals)
\end{itemize}
\end{definition}

\begin{definition}[Parallel Conduction Paths]
\label{def:parallel_paths}
The number of parallel conduction paths through a conductor is:
\begin{equation}
N_\parallel = \frac{A}{a_0^2}
\label{eq:parallel_paths}
\end{equation}
Each path represents an independent chain of electron displacement events.
\end{definition}

\begin{example}[Copper Wire]
For a copper wire with $A = 1$ mm$^2$ and $a_0 = 0.36$ nm:
\begin{equation}
N_\parallel = \frac{10^{-6}}{(3.6 \times 10^{-10})^2} \approx 7.7 \times 10^{12}
\end{equation}
The wire contains approximately $10^{13}$ parallel conduction paths.
\end{example}

\subsection{The Reduction Theorem}

\begin{theorem}[Conductor Dimensional Reduction]
\label{thm:conductor_dimensional_reduction}
A three-dimensional conductor reduces to:
\begin{equation}
\text{3D Conductor}(L, A) = \text{0D Cross-Section}(N_\parallel) \times \text{1D S-Transformation}(L)
\label{eq:conductor_reduction}
\end{equation}
The cross-section is characterised entirely by $N_\parallel$; the S-transformation along $L$ determines resistance.
\end{theorem}

\begin{proof}
Consider a conductor with uniform cross-section $A$ and length $L$.

\textbf{Step 1: Transverse uniformity.}

For a uniform conductor, all points at the same position $x$ along the length have identical S-coordinates:
\begin{equation}
\Svec(x, y, z) = \Svec(x) \quad \forall (y, z) \in A
\end{equation}

This follows from the translational symmetry of the conductor perpendicular to current flow.

\textbf{Step 2: Cross-section characterisation.}

The cross-section at position $x$ is characterised by:
\begin{itemize}
\item Number of parallel paths: $N_\parallel = A/a_0^2$
\item Total conduction electrons: $N_e = nA\Delta x$ in slice of thickness $\Delta x$
\item Categorical state: single S-vector $\Svec(x)$
\end{itemize}

All electrons in the cross-section share the same categorical state because they are phase-locked through the dense network (Theorem~\ref{thm:metal_network_density}).

\textbf{Step 3: Longitudinal S-transformation.}

The S-state at position $x + \Delta x$ is determined by the S-transformation:
\begin{equation}
\Svec(x + \Delta x) = \Toperator_{\Delta x}[\Svec(x)]
\label{eq:conductor_s_transform}
\end{equation}

The transformation encodes scattering, drift, and coupling effects over the interval $\Delta x$.

\textbf{Step 4: Dimensional count.}

The 3D conductor has $N_x \times N_y \times N_z$ spatial points. Under reduction:
\begin{itemize}
\item Cross-section: 1 number ($N_\parallel$)
\item S-transformation: $N_x$ values of $\Svec(x)$, each with 3 components
\end{itemize}

Total degrees of freedom: $3N_x + 1$, compared to $3N_x N_y N_z$ without reduction.

The reduction factor is:
\begin{equation}
\text{Reduction} = \frac{3N_x N_y N_z}{3N_x + 1} \approx N_y N_z = \frac{A}{\Delta x^2}
\end{equation}

For typical conductors, this is $\sim 10^{12}$ or greater. \qed
\end{proof}

\subsection{S-State of the Cross-Section}

\begin{definition}[Cross-Section S-State]
\label{def:cross_section_s}
The S-state of a conductor cross-section at position $x$ is:
\begin{equation}
\Svec(x) = (S_k(x), S_t(x), S_e(x))
\label{eq:cross_section_s}
\end{equation}
where:
\begin{align}
S_k(x) &= -\log_2 P_{\text{config}}(x) \quad \text{(configuration knowledge deficit)} \\
S_t(x) &= \log_{10}(\tau(x)/\tau_0) \quad \text{(temporal position)} \\
S_e(x) &= -\sum_i p_i(x) \log_2 p_i(x) \quad \text{(entropy constraint)}
\end{align}
\end{definition}

\begin{theorem}[S-State Continuity]
\label{thm:s_continuity}
For a uniform conductor, the S-state varies continuously along the length:
\begin{equation}
\frac{d\Svec}{dx} = -\frac{1}{\lambda} (\Svec - \Svec_{\text{eq}})
\label{eq:s_continuity}
\end{equation}
where $\lambda$ is the mean free path and $\Svec_{\text{eq}}$ is the equilibrium S-state.
\end{theorem}

\begin{proof}
Electrons in the conductor undergo scattering events with characteristic length $\lambda$. Between scattering events, the S-state drifts toward equilibrium.

The rate of approach to equilibrium is:
\begin{equation}
\frac{d\Svec}{dt} = -\frac{1}{\tau_s}(\Svec - \Svec_{\text{eq}})
\end{equation}
where $\tau_s$ is the scattering time.

Converting to spatial dependence using $dx = v_d \, dt$:
\begin{equation}
\frac{d\Svec}{dx} = \frac{1}{v_d} \frac{d\Svec}{dt} = -\frac{1}{v_d \tau_s}(\Svec - \Svec_{\text{eq}}) = -\frac{1}{\lambda}(\Svec - \Svec_{\text{eq}})
\end{equation}
where $\lambda = v_d \tau_s$ is the mean free path. \qed
\end{proof}

\subsection{Boundary Conditions}

\begin{definition}[Conductor Boundary Conditions]
\label{def:boundary_conditions}
The S-state boundary conditions for a conductor of length $L$ are:
\begin{align}
\Svec(0) &= \Svec_{\text{in}} \quad \text{(input cross-section)} \\
\Svec(L) &= \Svec_{\text{out}} \quad \text{(output cross-section)}
\end{align}
The potential difference is:
\begin{equation}
V = \Phi_S(\Svec_{\text{in}}) - \Phi_S(\Svec_{\text{out}})
\label{eq:potential_difference}
\end{equation}
where $\Phi_S$ is the S-potential.
\end{definition}

\begin{theorem}[Current from S-Gradient]
\label{thm:current_s_gradient}
The current through the conductor is:
\begin{equation}
I = -\sigma A \frac{d\Phi_S}{dx} = \sigma A \frac{V}{L}
\label{eq:current_s_gradient}
\end{equation}
where $\sigma$ is the conductivity.
\end{theorem}

\begin{proof}
Current density is proportional to the S-potential gradient:
\begin{equation}
\mathbf{J} = -\sigma \nabla \Phi_S
\end{equation}

For a uniform conductor with $\nabla \Phi_S = (V/L)\hat{x}$:
\begin{equation}
J = \sigma \frac{V}{L}
\end{equation}

Integrating over cross-section:
\begin{equation}
I = JA = \sigma A \frac{V}{L}
\end{equation}

This is Ohm's law with $R = L/(\sigma A)$. \qed
\end{proof}

\subsection{Comparison with Fluid Dimensional Reduction}

\begin{remark}
The conductor dimensional reduction is a special case of the general fluid reduction:

\begin{center}
\begin{tabular}{lcc}
\toprule
\textbf{Property} & \textbf{Fluid} & \textbf{Conductor} \\
\midrule
Cross-section & 2D ($N_y \times N_z$ points) & 0D ($N_\parallel$ paths) \\
S-transformation & 1D along flow & 1D along length \\
Reduction factor & $N_y N_z$ & $N_y N_z$ \\
Driving force & Pressure gradient & Potential gradient \\
Transport coefficient & Viscosity $\mu$ & Resistivity $\rho$ \\
\bottomrule
\end{tabular}
\end{center}

The conductor case is simpler because the cross-section is uniform (all electrons share the same S-state), eliminating the need to track 2D cross-section structure.
\end{remark}

