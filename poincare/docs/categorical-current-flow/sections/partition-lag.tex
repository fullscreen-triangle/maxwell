%==============================================================================
% SECTION 5: PARTITION LAG IN ELECTRON TRANSPORT
%==============================================================================

\section{Partition Lag in Electron Transport}
\label{sec:partition_lag}

\subsection{Definition of Scattering Partition Lag}

\begin{definition}[Scattering Partition Lag]
\label{def:scattering_lag}
The scattering partition lag $\tau_{s,ij}$ is the irreducible temporal interval between an electron $i$ initiating a scattering event with lattice site $j$ and the establishment of the scattered state:
\begin{equation}
\tau_{s,ij} = t_{\text{scattered}} - t_{\text{incident}} > 0
\label{eq:scattering_lag}
\end{equation}
\end{definition}

\begin{theorem}[Positive Scattering Time]
\label{thm:positive_scattering}
Scattering operations require positive time: $\tau_{s,ij} > 0$ for all scattering events.
\end{theorem}

\begin{proof}
Scattering distinguishes between incident and scattered categorical states. Distinguishing requires information acquisition about the lattice potential. Information acquisition in physical systems requires finite time by causality constraints. Hence $\tau_{s,ij} > 0$. \qed
\end{proof}

\subsection{Mean Free Path and Scattering Time}

\begin{definition}[Mean Free Path]
\label{def:mean_free_path}
The mean free path $\lambda$ is the average distance between scattering events:
\begin{equation}
\lambda = v_F \tau_s
\label{eq:mean_free_path}
\end{equation}
where $v_F$ is the Fermi velocity and $\tau_s$ is the mean scattering time.
\end{definition}

\begin{theorem}[Scattering Time from Partition Lag]
\label{thm:scattering_time}
The mean scattering time is the average of individual partition lags:
\begin{equation}
\tau_s = \langle \tau_{s,ij} \rangle = \frac{1}{N_{\text{scatter}}} \sum_{i,j} \tau_{s,ij}
\label{eq:mean_scattering_time}
\end{equation}
\end{theorem}

\begin{proof}
Scattering events are statistically distributed across electron-lattice pairs. The mean scattering time is the expectation over all possible scattering interactions. Since each interaction has partition lag $\tau_{s,ij}$, the mean is:
\begin{equation}
\tau_s = \mathbb{E}[\tau_{s,ij}] = \frac{1}{N_{\text{scatter}}} \sum_{i,j} \tau_{s,ij}
\end{equation}
\qed
\end{proof}

\subsection{Temperature Dependence of Partition Lag}

\begin{theorem}[Phonon Partition Lag]
\label{thm:phonon_lag}
The phonon-mediated partition lag increases with temperature:
\begin{equation}
\tau_{s,\text{phonon}}^{-1} = A_{\text{ph}} T \quad \text{for } T > \Theta_D
\label{eq:phonon_lag}
\end{equation}
where $A_{\text{ph}}$ is a material-dependent constant and $\Theta_D$ is the Debye temperature.
\end{theorem}

\begin{proof}
Phonon occupation number follows Bose-Einstein statistics:
\begin{equation}
n_{\text{ph}}(\omega) = \frac{1}{e^{\hbar\omega/\kB T} - 1}
\end{equation}

For $T > \Theta_D$, $\hbar\omega \ll \kB T$, so:
\begin{equation}
n_{\text{ph}} \approx \frac{\kB T}{\hbar\omega} \propto T
\end{equation}

Scattering rate is proportional to phonon occupation:
\begin{equation}
\frac{1}{\tau_{s,\text{phonon}}} \propto n_{\text{ph}} \propto T
\end{equation}
\qed
\end{proof}

\begin{corollary}[Residual Partition Lag]
\label{cor:residual_lag}
At $T \to 0$, phonon scattering vanishes, leaving only impurity and defect scattering:
\begin{equation}
\lim_{T \to 0} \frac{1}{\tau_s} = \frac{1}{\tau_{\text{impurity}}} + \frac{1}{\tau_{\text{defect}}} = \frac{1}{\tau_0}
\label{eq:residual_lag}
\end{equation}
This gives the residual resistivity $\rho_0 = m_e/(ne^2\tau_0)$.
\end{corollary}

\subsection{Undetermined Residue in Scattering}

\begin{definition}[Scattering Undetermined Residue]
\label{def:scattering_residue}
During scattering partition lag $\tau_{s,ij}$, the electron exists in undetermined superposition across possible scattered states. The undetermined residue $n_{\text{res}}^{(s)}$ counts states not assignable to either incident or scattered outcome.
\end{definition}

\begin{theorem}[Scattering Entropy Production]
\label{thm:scattering_entropy}
Each scattering event produces entropy:
\begin{equation}
\Delta S_{\text{scatter}} = \kB \ln n_{\text{res}}^{(s)} > 0
\label{eq:scattering_entropy}
\end{equation}
\end{theorem}

\begin{proof}
Undetermined residue represents states that cannot be classified during $\tau_{s,ij}$. These states contribute $\ln n_{\text{res}}^{(s)}$ to entropy. By Theorem~\ref{thm:positive_scattering}, $\tau_{s,ij} > 0$, hence $n_{\text{res}}^{(s)} > 1$, hence $\Delta S_{\text{scatter}} > 0$. \qed
\end{proof}

\begin{remark}
This entropy production is the microscopic origin of Joule heating. Energy dissipated as heat is:
\begin{equation}
Q = T \Delta S_{\text{scatter}} = \kB T \ln n_{\text{res}}^{(s)}
\end{equation}
per scattering event. Summed over all scatterings, this gives $P = I^2 R$.
\end{remark}

\subsection{Partition Lag Statistics}

\begin{definition}[Partition Lag Distribution]
\label{def:lag_distribution}
The partition lag distribution $P(\tau_s)$ describes the probability of scattering with lag $\tau_s$:
\begin{equation}
P(\tau_s) = \frac{1}{\langle \tau_s \rangle} e^{-\tau_s / \langle \tau_s \rangle}
\label{eq:lag_distribution}
\end{equation}
This exponential distribution follows from Poisson scattering statistics.
\end{definition}

\begin{theorem}[Partition Lag Variance]
\label{thm:lag_variance}
The variance of partition lag is:
\begin{equation}
\text{Var}(\tau_s) = \langle \tau_s \rangle^2
\label{eq:lag_variance}
\end{equation}
The coefficient of variation is unity: $\text{CV} = 1$.
\end{theorem}

\begin{proof}
For exponential distribution with mean $\mu = \langle \tau_s \rangle$:
\begin{equation}
\text{Var}(\tau_s) = \mu^2 = \langle \tau_s \rangle^2
\end{equation}

The standard deviation equals the mean:
\begin{equation}
\sigma_{\tau_s} = \langle \tau_s \rangle \implies \text{CV} = \frac{\sigma_{\tau_s}}{\langle \tau_s \rangle} = 1
\end{equation}
\qed
\end{proof}

\begin{remark}
The unit coefficient of variation indicates that scattering is highly stochastic. Individual electrons experience widely varying partition lags. Only ensemble averages produce deterministic transport behaviour (Ohm's law).
\end{remark}

