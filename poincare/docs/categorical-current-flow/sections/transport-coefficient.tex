%==============================================================================
% SECTION 8: UNIFIED TRANSPORT COEFFICIENTS
%==============================================================================

\section{Unified Transport Coefficients}
\label{sec:transport}

\subsection{The Universal Form}

\begin{theorem}[Universal Transport Coefficient]
\label{thm:universal_transport}
All transport coefficients have the form:
\begin{equation}
\Xi = \frac{1}{\mathcal{N}} \sum_{i,j} \tau_{p,ij} \cdot g_{ij}
\label{eq:universal_transport}
\end{equation}
where $\tau_{p,ij}$ is the partition lag, $g_{ij}$ is the coupling strength, and $\mathcal{N}$ is a normalisation factor.
\end{theorem}

\begin{proof}
Transport occurs through S-transformation across a medium. Each transformation step involves:
\begin{enumerate}
\item Partition lag $\tau_{p,ij}$: time to complete the categorical transition
\item Coupling strength $g_{ij}$: probability of the transition occurring
\end{enumerate}

The total transport resistance is the sum over all pairs:
\begin{equation}
\text{Resistance} \propto \sum_{i,j} \tau_{p,ij} \cdot g_{ij}
\end{equation}

The transport coefficient (inverse resistance or direct coefficient) is:
\begin{equation}
\Xi = \frac{1}{\mathcal{N}} \sum_{i,j} \tau_{p,ij} \cdot g_{ij}
\end{equation}
where $\mathcal{N}$ ensures correct units. \qed
\end{proof}

\subsection{Transport Coefficient Table}

\begin{center}
\begin{tabular}{llll}
\toprule
\textbf{System} & \textbf{Coefficient} & \textbf{$\tau_p$} & \textbf{$g$} \\
\midrule
Fluid flow & Viscosity $\mu$ & Molecular relaxation & Phase-lock coupling \\
Current flow & Resistivity $\rho$ & Scattering time & Electron-lattice coupling \\
Heat conduction & Thermal resistance $1/k$ & Phonon scattering & Phonon coupling \\
Mass diffusion & Inverse diffusivity $1/D$ & Collision time & Molecular coupling \\
EM propagation & Impedance $\sqrt{\mu_0/\varepsilon_0}$ & Field relaxation & Vacuum coupling \\
\bottomrule
\end{tabular}
\end{center}

\subsection{Specific Forms}

\begin{definition}[Fluid Viscosity]
\label{def:fluid_viscosity}
Fluid viscosity is:
\begin{equation}
\mu = \sum_{i,j} \tau_{p,ij}^{(\text{mol})} \cdot g_{ij}^{(\text{mol})}
\label{eq:fluid_viscosity}
\end{equation}
where the sum is over molecular pairs.
\end{definition}

\begin{definition}[Electrical Resistivity]
\label{def:electrical_resistivity}
Electrical resistivity is:
\begin{equation}
\rho = \frac{1}{ne^2} \sum_{i,j} \tau_{s,ij} \cdot g_{ij}^{(e-\text{lat})}
\label{eq:electrical_resistivity}
\end{equation}
where the sum is over electron-lattice pairs.
\end{definition}

\begin{definition}[Thermal Resistance]
\label{def:thermal_resistance}
Thermal resistance (inverse conductivity) is:
\begin{equation}
\frac{1}{k} = \frac{1}{C_v v_{\text{ph}}} \sum_{i,j} \tau_{ph,ij} \cdot g_{ij}^{(\text{ph})}
\label{eq:thermal_resistance}
\end{equation}
where the sum is over phonon pairs, $C_v$ is heat capacity, and $v_{\text{ph}}$ is phonon velocity.
\end{definition}

\subsection{Wiedemann-Franz Law}

\begin{theorem}[Wiedemann-Franz Law]
\label{thm:wiedemann_franz}
The ratio of thermal to electrical conductivity is:
\begin{equation}
\frac{k}{\sigma} = L T
\label{eq:wiedemann_franz}
\end{equation}
where $L = \pi^2 \kB^2/(3e^2) = 2.44 \times 10^{-8}$ W$\Omega$/K$^2$ is the Lorenz number.
\end{theorem}

\begin{proof}
In metals, both electrical and thermal conduction are dominated by electrons.

Electrical conductivity:
\begin{equation}
\sigma = \frac{ne^2 \tau_s}{m_e}
\end{equation}

Thermal conductivity (electronic):
\begin{equation}
k_e = \frac{1}{3} C_e v_F^2 \tau_s = \frac{1}{3} \frac{\pi^2 n \kB^2 T}{m_e} \tau_s
\end{equation}
using $C_e = \pi^2 n \kB^2 T / (2E_F)$ and $v_F^2 = 2E_F/m_e$.

The ratio:
\begin{equation}
\frac{k_e}{\sigma} = \frac{\frac{\pi^2 n \kB^2 T}{3m_e} \tau_s}{\frac{ne^2 \tau_s}{m_e}} = \frac{\pi^2 \kB^2 T}{3e^2} = LT
\end{equation}
\qed
\end{proof}

\begin{remark}
The Wiedemann-Franz law follows because both conductivities share the same partition lag $\tau_s$ and coupling structure $g_{ij}$. When electrons carry both charge and heat, their transport coefficients are proportional.

Deviations from Wiedemann-Franz occur when:
\begin{itemize}
\item Phonons contribute to thermal conduction (insulators)
\item Inelastic scattering violates $\tau_{\text{electrical}} = \tau_{\text{thermal}}$
\item Strong electron correlations modify coupling structure
\end{itemize}
\end{remark}

\subsection{Dimensional Analysis}

\begin{theorem}[Dimensional Consistency]
\label{thm:dimensional_consistency}
All transport coefficients satisfy dimensional consistency under the universal form.
\end{theorem}

\begin{proof}
The partition lag has dimension [time]. The coupling strength has dimension dependent on the transport mode:

\begin{center}
\begin{tabular}{lllll}
\toprule
\textbf{Coefficient} & \textbf{Dimension} & \textbf{$[\tau_p]$} & \textbf{$[g]$} & \textbf{$[\mathcal{N}]$} \\
\midrule
$\mu$ & Pa$\cdot$s & s & Pa & 1 \\
$\rho$ & $\Omega\cdot$m & s & 1 & A$^2$/m$^3$ \\
$1/k$ & m$\cdot$K/W & s & 1/(m$^2\cdot$K) & J/(m$^3\cdot$s$\cdot$K) \\
\bottomrule
\end{tabular}
\end{center}

In each case:
\begin{equation}
[\Xi] = \frac{[\tau_p] \cdot [g]}{[\mathcal{N}]}
\end{equation}
giving the correct dimensions. \qed
\end{proof}

\subsection{Temperature Dependence}

\begin{theorem}[Universal Temperature Scaling]
\label{thm:temperature_scaling}
Transport coefficients scale with temperature according to:
\begin{equation}
\Xi(T) = \Xi_0 + A T^n
\label{eq:temperature_scaling}
\end{equation}
where $\Xi_0$ is the residual (impurity) contribution, $A$ is a material constant, and $n$ depends on the dominant scattering mechanism.
\end{theorem}

\begin{proof}
The partition lag depends on temperature through scattering mechanisms:

\begin{center}
\begin{tabular}{lll}
\toprule
\textbf{Mechanism} & \textbf{$\tau_p(T)$} & \textbf{$n$} \\
\midrule
Impurity/defect & Constant & 0 \\
Phonon (high $T$) & $\propto 1/T$ & 1 \\
Phonon (low $T$) & $\propto 1/T^5$ & 5 \\
Electron-electron & $\propto 1/T^2$ & 2 \\
\bottomrule
\end{tabular}
\end{center}

Since $\Xi \propto 1/\tau_p$ (for conductivity-type) or $\Xi \propto \tau_p$ (for resistivity-type), the temperature dependence follows. The residual term $\Xi_0$ comes from temperature-independent impurity scattering. \qed
\end{proof}

