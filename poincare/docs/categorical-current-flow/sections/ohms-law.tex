%==============================================================================
% SECTION 7: OHM'S LAW AND KIRCHHOFF'S LAWS
%==============================================================================

\section{Ohm's Law from Partition Lag}
\label{sec:ohms_law}

\subsection{Derivation of Ohm's Law}

\begin{theorem}[Ohm's Law]
\label{thm:ohms_law}
For a conductor of length $L$ and cross-sectional area $A$ under potential difference $V$:
\begin{equation}
V = IR
\label{eq:ohms_law}
\end{equation}
where the resistance is:
\begin{equation}
R = \frac{\rho L}{A} = \frac{L}{A} \cdot \frac{1}{ne^2} \sum_{i,j} \tau_{s,ij} g_{ij}
\label{eq:resistance}
\end{equation}
\end{theorem}

\begin{proof}
\textbf{Step 1: S-potential gradient.}

The electric field in the conductor is uniform:
\begin{equation}
E = \frac{V}{L}
\end{equation}

In S-coordinate form, this is the S-potential gradient:
\begin{equation}
\frac{d\Phi_S}{dx} = -E = -\frac{V}{L}
\end{equation}

\textbf{Step 2: Current density from drift.}

Under the S-gradient, electrons drift with velocity:
\begin{equation}
v_d = \mu E = \frac{e\tau_s}{m_e} E
\end{equation}

Current density is:
\begin{equation}
J = nev_d = \frac{ne^2\tau_s}{m_e} E = \sigma E
\end{equation}

where conductivity $\sigma = ne^2\tau_s/m_e$.

\textbf{Step 3: Total current.}

Integrating over cross-section:
\begin{equation}
I = JA = \sigma A E = \sigma A \frac{V}{L}
\end{equation}

Rearranging:
\begin{equation}
V = \frac{L}{\sigma A} I = RI
\end{equation}

where $R = L/(\sigma A) = \rho L/A$.

\textbf{Step 4: Resistivity from partition lag.}

From Theorem~\ref{thm:resistivity_formula}:
\begin{equation}
\rho = \frac{1}{\sigma} = \frac{m_e}{ne^2\tau_s} = \frac{1}{ne^2} \sum_{i,j} \tau_{s,ij} g_{ij}
\end{equation}

Therefore:
\begin{equation}
R = \frac{L}{A} \cdot \frac{1}{ne^2} \sum_{i,j} \tau_{s,ij} g_{ij}
\end{equation}
\qed
\end{proof}

\subsection{Resistance Interpretation}

\begin{remark}
Resistance has a clear partition-geometric interpretation:

\begin{center}
\begin{tabular}{ll}
\toprule
\textbf{Factor} & \textbf{Interpretation} \\
\midrule
$L$ & Number of S-transformation steps \\
$1/A$ & Inverse of parallel conduction paths \\
$\tau_{s,ij}$ & Partition lag per scattering event \\
$g_{ij}$ & Electron-lattice coupling strength \\
$1/(ne^2)$ & Normalisation by carrier density \\
\bottomrule
\end{tabular}
\end{center}

Resistance measures the total partition lag accumulated over the conductor length.
\end{remark}

\subsection{Kirchhoff's Current Law}

\begin{theorem}[Kirchhoff's Current Law (KCL)]
\label{thm:kcl}
At any junction in a circuit:
\begin{equation}
\sum_k I_k = 0
\label{eq:kcl}
\end{equation}
where currents entering are positive and currents leaving are negative.
\end{theorem}

\begin{proof}
Current is the rate of categorical state propagation (Definition~\ref{def:current_categorical}):
\begin{equation}
I = e \cdot \frac{d}{dt}\left[\sum_i C_i\right]
\end{equation}

Categorical states are conserved: they are neither created nor destroyed at junctions, only redirected.

At a junction, the total rate of categorical states entering equals the total rate leaving:
\begin{equation}
\sum_{k \in \text{in}} \frac{dC_k}{dt} = \sum_{k \in \text{out}} \frac{dC_k}{dt}
\end{equation}

Defining entering currents as positive and leaving as negative:
\begin{equation}
\sum_{k \in \text{in}} I_k - \sum_{k \in \text{out}} I_k = 0 \implies \sum_k I_k = 0
\end{equation}
\qed
\end{proof}

\begin{remark}
KCL is categorical state conservation. It is the electrical analogue of the continuity equation:
\begin{equation}
\frac{\partial \rho}{\partial t} + \nabla \cdot \mathbf{J} = 0
\end{equation}
At a junction in steady state ($\partial \rho/\partial t = 0$), $\nabla \cdot \mathbf{J} = 0$ implies $\sum I = 0$.
\end{remark}

\subsection{Kirchhoff's Voltage Law}

\begin{theorem}[Kirchhoff's Voltage Law (KVL)]
\label{thm:kvl}
Around any closed loop in a circuit:
\begin{equation}
\sum_k V_k = 0
\label{eq:kvl}
\end{equation}
where voltage rises are positive and drops are negative.
\end{theorem}

\begin{proof}
The S-potential $\Phi_S$ must be single-valued: returning to the same point yields the same potential.

Around a closed loop:
\begin{equation}
\oint d\Phi_S = 0
\end{equation}

The integral decomposes into segments:
\begin{equation}
\sum_k \int_{a_k}^{b_k} d\Phi_S = \sum_k (\Phi_S(b_k) - \Phi_S(a_k)) = \sum_k V_k = 0
\end{equation}

where $V_k = \Phi_S(b_k) - \Phi_S(a_k)$ is the potential difference across segment $k$. \qed
\end{proof}

\begin{remark}
KVL is S-potential single-valuedness. In the presence of time-varying magnetic fields:
\begin{equation}
\oint \mathbf{E} \cdot d\mathbf{l} = -\frac{d\Phi_B}{dt}
\end{equation}
KVL becomes $\sum V = -d\Phi_B/dt$ (Faraday's law). The quasi-static approximation $d\Phi_B/dt \approx 0$ recovers the standard KVL.
\end{remark}

\subsection{Series and Parallel Resistance}

\begin{corollary}[Series Resistance]
\label{cor:series_resistance}
For resistors in series:
\begin{equation}
R_{\text{series}} = R_1 + R_2 + \cdots + R_n
\label{eq:series_resistance}
\end{equation}
\end{corollary}

\begin{proof}
By KVL around the series loop:
\begin{equation}
V = V_1 + V_2 + \cdots + V_n = IR_1 + IR_2 + \cdots + IR_n = I(R_1 + R_2 + \cdots + R_n)
\end{equation}
Therefore $R_{\text{series}} = \sum_k R_k$. \qed
\end{proof}

\begin{corollary}[Parallel Resistance]
\label{cor:parallel_resistance}
For resistors in parallel:
\begin{equation}
\frac{1}{R_{\text{parallel}}} = \frac{1}{R_1} + \frac{1}{R_2} + \cdots + \frac{1}{R_n}
\label{eq:parallel_resistance}
\end{equation}
\end{corollary}

\begin{proof}
By KCL at the parallel junction:
\begin{equation}
I = I_1 + I_2 + \cdots + I_n = \frac{V}{R_1} + \frac{V}{R_2} + \cdots + \frac{V}{R_n} = V\left(\frac{1}{R_1} + \frac{1}{R_2} + \cdots + \frac{1}{R_n}\right)
\end{equation}
Therefore $1/R_{\text{parallel}} = \sum_k 1/R_k$. \qed
\end{proof}

\begin{remark}
Series resistance adds partition lags: each segment contributes its share of scattering events. Parallel resistance adds conductances (inverse partition lags): each path provides an independent conduction channel.
\end{remark}

\subsection{Power Dissipation}

\begin{theorem}[Joule Heating]
\label{thm:joule_heating}
Power dissipated in a resistor is:
\begin{equation}
P = I^2 R = \frac{V^2}{R} = IV
\label{eq:joule_heating}
\end{equation}
\end{theorem}

\begin{proof}
Power is the rate of energy dissipation. Each scattering event produces entropy (Theorem~\ref{thm:scattering_entropy}):
\begin{equation}
\Delta S_{\text{scatter}} = \kB \ln n_{\text{res}}^{(s)}
\end{equation}

Energy dissipated per scattering event:
\begin{equation}
\Delta E = T \Delta S = \kB T \ln n_{\text{res}}^{(s)}
\end{equation}

Scattering rate for current $I$:
\begin{equation}
\Gamma = \frac{I}{e} \cdot \frac{L}{\lambda} = \frac{IL}{e\lambda}
\end{equation}

Power:
\begin{equation}
P = \Gamma \cdot \Delta E = \frac{IL}{e\lambda} \cdot \kB T \ln n_{\text{res}}^{(s)}
\end{equation}

For metallic conduction, $\Delta E = eV/N_{\text{scatter}} = eV\lambda/L$:
\begin{equation}
P = \frac{IL}{e\lambda} \cdot \frac{eV\lambda}{L} = IV = I^2 R
\end{equation}
\qed
\end{proof}

\begin{remark}
Joule heating is entropy production from scattering partition lag. Each scattering event produces undetermined residue; resolving this residue dissipates energy as heat.
\end{remark}

