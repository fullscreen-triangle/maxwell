\documentclass[12pt,a4paper]{article}
\usepackage[utf8]{inputenc}
\usepackage[T1]{fontenc}
\usepackage{amsmath,amssymb,amsfonts,amsthm}
\usepackage{mathtools}
\usepackage{geometry}
\usepackage{graphicx}
\usepackage{float}
\usepackage{booktabs}
\usepackage{array}
\usepackage{tikz}
\usepackage{pgfplots}
\usepackage{hyperref}
\usepackage{cite}
\usepackage{natbib}
\usepackage{physics}
\usepackage{siunitx}
\usepackage{import}

\geometry{margin=1in}
\pgfplotsset{compat=1.17}

% Theorem environments
\newtheorem{theorem}{Theorem}[section]
\newtheorem{lemma}[theorem]{Lemma}
\newtheorem{corollary}[theorem]{Corollary}
\newtheorem{definition}[theorem]{Definition}
\newtheorem{proposition}[theorem]{Proposition}
\newtheorem{axiom}[theorem]{Axiom}

\theoremstyle{remark}
\newtheorem{remark}[theorem]{Remark}

% Custom commands
\newcommand{\kB}{k_{\mathrm{B}}}
\newcommand{\Sspace}{\mathcal{S}}
\newcommand{\Cspace}{\mathcal{C}}
\newcommand{\Toperator}{\mathbf{T}}
\newcommand{\Svec}{\mathbf{S}}

\title{\textbf{Categorical Current Flow: Derivation of Ohm's Law and Maxwell's Equations from S-Entropy Transformations in Bounded Conductors}}

\author{
Kundai Farai Sachikonye\\
\texttt{kundai.sachikonye@wzw.tum.de}
}

\date{\today}

\begin{document}

\maketitle

\begin{abstract}
We derive electrical current flow from the partition-oscillation-category equivalence. The derivation proceeds from the identity $S = \kB M \ln n$, which holds for oscillatory systems with $M$ modes and $n$ accessible states. We prove that a three-dimensional conductor reduces to a zero-dimensional cross-section state (characterised by radius $r$) combined with a one-dimensional S-transformation along the conductor length. This dimensional reduction follows from the electron phase-lock network: conduction electrons form a dense categorical network where current propagates through successive displacement rather than individual electron drift.

We demonstrate that Ohm's law $V = IR$ emerges as a continuum limit of discrete S-transformations. The resistance $R = \rho L/A$ follows with resistivity $\rho = \sum_{i,j} \tau_{s,ij} g_{ij} / (ne^2)$, where $\tau_{s,ij}$ is the scattering partition lag between electron-lattice pairs and $g_{ij}$ is the phase-lock coupling strength. Kirchhoff's current law $\sum I = 0$ follows from categorical state conservation at junctions. Kirchhoff's voltage law $\sum V = 0$ follows from S-potential single-valuedness around closed loops.

We extend the framework to time-varying fields, deriving Maxwell's equations as the full frequency-dependent generalisation. The continuity equation $\partial\rho/\partial t + \nabla \cdot \mathbf{J} = 0$ follows from charge conservation. Faraday's law $\nabla \times \mathbf{E} = -\partial\mathbf{B}/\partial t$ emerges from S-curl dynamics. The displacement current $\varepsilon_0 \partial\mathbf{E}/\partial t$ in Ampère's law represents the S-transformation rate. The speed of light $c = 1/\sqrt{\mu_0\varepsilon_0}$ emerges from electromagnetic partition lag and vacuum coupling.

Experimental validation on conductor measurements yields resistivity predictions with mean absolute error of 2.8\% relative to measured values. The framework reproduces the temperature dependence of resistance, the skin effect at high frequencies, and the transition to superconductivity at critical temperatures.

\textbf{Keywords:} categorical current flow, S-entropy coordinates, Ohm's law, Kirchhoff's laws, Maxwell's equations, electron scattering, resistivity
\end{abstract}

\tableofcontents
\newpage

%==============================================================================
% INTRODUCTION
%==============================================================================

\section{Introduction}
\label{sec:introduction}

\subsection{Circuit Theory and Electromagnetic Theory}

Classical electrical theory operates at two levels. Circuit theory, developed by Ohm \cite{ohm1827} and Kirchhoff \cite{kirchhoff1845}, treats conductors as discrete elements characterised by resistance, capacitance, and inductance. Electromagnetic field theory, developed by Maxwell \cite{maxwell1865}, treats fields as continuous distributions governed by partial differential equations.

The relationship between these levels is well understood: circuit theory emerges as the low-frequency, quasi-static limit of electromagnetic theory. Kirchhoff's current law follows from charge conservation when $\partial\rho/\partial t \approx 0$. Kirchhoff's voltage law follows from Faraday's law when $\partial\mathbf{B}/\partial t \approx 0$. Ohm's law emerges from the linear response $\mathbf{J} = \sigma\mathbf{E}$ integrated over conductor geometry.

Both frameworks treat their fundamental quantities as empirical inputs. Resistivity $\rho$ is measured and tabulated. Permittivity $\varepsilon_0$ and permeability $\mu_0$ are fundamental constants. The question of why conductors conduct, why resistance has the values it does, and why electromagnetic fields propagate at speed $c$ remains outside the scope of these theories.

\subsection{The Categorical Framework}

We approach electrical phenomena from the partition-oscillation-category equivalence established in prior work \cite{sachikonye2024partition}. The identity $S = \kB M \ln n$ holds across oscillatory, categorical, and partition formulations. This equivalence enables the derivation of transport phenomena from geometric principles rather than empirical parameterisation.

The key insight is that current flow in conductors is fundamentally a categorical phenomenon. Electrons do not flow freely through conductors; they propagate through a dense phase-lock network of electron-lattice interactions. Each electron's motion is constrained by its categorical relationships with neighbouring electrons and lattice ions. Current emerges as the propagation of categorical state changes through this network—analogous to a Newton's cradle where momentum transfers through successive collisions without individual ball displacement.

\subsection{Dimensional Reduction for Conductors}

For conductors, the dimensional reduction takes a simpler form than for general fluids:

\begin{theorem}[Conductor Dimensional Reduction]
\label{thm:conductor_reduction}
A conductor of length $L$ and cross-sectional area $A$ reduces to:
\begin{equation}
\text{3D Conductor} = \text{0D Cross-Section} \times \text{1D S-Transformation}
\end{equation}
The cross-section is characterised by the number of parallel conduction paths $N_\parallel = A/a_0^2$, where $a_0$ is the lattice spacing.
\end{theorem}

This reduction is simpler than the fluid case because conductors have uniform cross-sections and current flows along a single direction. The S-transformation along the conductor length captures all relevant physics: scattering, drift, and electron-electron coupling.

\subsection{Paper Structure}

Section~\ref{sec:newton_cradle} establishes the Newton's cradle model of electron propagation. Section~\ref{sec:dimensional_reduction} proves the dimensional reduction theorem for conductors. Section~\ref{sec:transformation} defines the S-transformation operator for current flow. Section~\ref{sec:ohms_law} derives Ohm's law and resistivity from partition lag. Section~\ref{sec:kirchhoff} derives Kirchhoff's laws from categorical conservation. Section~\ref{sec:maxwell} extends the framework to derive Maxwell's equations. Section~\ref{sec:transport} unifies transport coefficients across systems. Section~\ref{sec:current_cross_sectional_validation} validates the framework through cross-sectional S-coordinate measurements along conductors.

%==============================================================================
% SECTION IMPORTS
%==============================================================================

\import{sections/}{newton-cradle-propagation.tex}
\import{sections/}{dimensional-reduction.tex}
\import{sections/}{st-stellas-transform-operator.tex}
\import{sections/}{partition-lag.tex}
\import{sections/}{coupling.tex}
\import{sections/}{ohms-law.tex}
\import{sections/}{transport-coefficient.tex}
\import{sections/}{maxwells-equations.tex}
\import{sections/}{cross-sectional-validation.tex}

%==============================================================================
% DISCUSSION
%==============================================================================

\section{Discussion}
\label{sec:discussion}

\subsection{Summary of Derivations}

The partition-oscillation-category equivalence provides the foundation for deriving electrical phenomena from geometric principles. The identity $S = \kB M \ln n$ holds across oscillatory, categorical, and partition formulations, enabling unified treatment of current flow and electromagnetic fields.

The Newton's cradle model (Section~\ref{sec:newton_cradle}) establishes that current propagates through electron displacement chains rather than individual electron drift. The drift velocity $v_d \sim 10^{-4}$ m/s is negligible compared to signal propagation at $\sim c$, confirming that current is a categorical phenomenon.

The dimensional reduction theorem (Theorem~\ref{thm:conductor_reduction}) establishes that three-dimensional conductors reduce to zero-dimensional cross-sections and one-dimensional S-transformations. The cross-section determines the number of parallel paths; the S-transformation determines the resistance per unit length.

Ohm's law emerges with resistivity $\rho = \sum \tau_{s,ij} g_{ij} / (ne^2)$, where $\tau_s$ is the scattering partition lag and $g$ is the electron-lattice coupling. This formula explains why metals with longer mean free paths have lower resistivity and why resistivity increases with temperature (increased scattering).

Kirchhoff's laws follow from categorical conservation. The current law $\sum I = 0$ expresses conservation of categorical states at junctions. The voltage law $\sum V = 0$ expresses S-potential single-valuedness around loops.

Maxwell's equations emerge as the full frequency-dependent generalisation. Ohm's law and Kirchhoff's laws are the low-frequency limits where time derivatives vanish. The displacement current $\varepsilon_0 \partial\mathbf{E}/\partial t$ represents the rate of S-transformation in the electromagnetic field.

\subsection{The Speed of Light}

The speed of light emerges from the framework as:
\begin{equation}
c = \frac{1}{\sqrt{\mu_0 \varepsilon_0}} = \frac{1}{\sqrt{\tau_p^{(\text{EM})} \cdot g^{(\text{EM})}}}
\end{equation}
where $\tau_p^{(\text{EM})}$ is the electromagnetic partition lag and $g^{(\text{EM})}$ is the vacuum field coupling. This provides a partition-geometric interpretation of why light has the speed it does.

\subsection{Relationship to Standard Formulations}

The categorical formulation does not contradict standard electromagnetic theory but provides a geometric foundation for it. Maxwell's equations remain valid as the field equations of electromagnetism. Ohm's law and Kirchhoff's laws remain valid as circuit equations.

The categorical formulation explains \emph{why} these equations take their particular form. Resistance arises from scattering partition lag. Conservation laws arise from categorical state conservation. The displacement current arises from S-transformation dynamics. The speed of light arises from vacuum partition-coupling structure.

%==============================================================================
% CONCLUSIONS
%==============================================================================

\section{Conclusions}
\label{sec:conclusions}

We derived electrical current flow and electromagnetic field equations from the partition-oscillation-category equivalence. The derivation established six principal results:

\begin{enumerate}
\item The Newton's cradle model: current propagates through electron displacement chains at signal speed $\sim c$, not through electron drift at $v_d \sim 10^{-4}$ m/s.

\item The dimensional reduction theorem: three-dimensional conductors decompose into zero-dimensional cross-sections (parallel paths) and one-dimensional S-transformations (resistance per length).

\item Ohm's law: $V = IR$ with resistivity $\rho = \sum \tau_{s,ij} g_{ij} / (ne^2)$ from scattering partition lag and electron-lattice coupling.

\item Kirchhoff's current law: $\sum I = 0$ from categorical state conservation at junctions.

\item Kirchhoff's voltage law: $\sum V = 0$ from S-potential single-valuedness around loops.

\item Maxwell's equations: emerge as frequency-dependent generalisation with displacement current $\varepsilon_0 \partial\mathbf{E}/\partial t$ representing S-transformation rate.
\end{enumerate}

Experimental validation on conductor measurements yielded resistivity predictions with 2.8\% mean absolute error. The framework reproduces temperature dependence, frequency dependence (skin effect), and critical phenomena (superconductivity).

The categorical formulation reveals that circuit theory and electromagnetic theory share a common geometric foundation in partition-oscillation-category equivalence. Ohm's law and Maxwell's equations are not independent empirical discoveries but necessary consequences of categorical dynamics in bounded oscillatory systems.

%==============================================================================
% BIBLIOGRAPHY
%==============================================================================

\bibliographystyle{unsrt}
\bibliography{references}

\end{document}

