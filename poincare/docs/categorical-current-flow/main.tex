\documentclass[12pt,a4paper]{article}
\usepackage[utf8]{inputenc}
\usepackage[T1]{fontenc}
\usepackage{amsmath,amssymb,amsfonts,amsthm}
\usepackage{mathtools}
\usepackage{geometry}
\usepackage{graphicx}
\usepackage{float}
\usepackage{booktabs}
\usepackage{array}
\usepackage{tikz}
usepackage{tcolorbox}
\usepackage{pgfplots}
\usepackage{hyperref}
\usepackage{cite}
\usepackage{natbib}
\usepackage{physics}
\usepackage{siunitx}
\usepackage{import}

\geometry{margin=1in}
\pgfplotsset{compat=1.17}

% Theorem environments
\newtheorem{theorem}{Theorem}[section]
\newtheorem{lemma}[theorem]{Lemma}
\newtheorem{corollary}[theorem]{Corollary}
\newtheorem{definition}[theorem]{Definition}
\newtheorem{proposition}[theorem]{Proposition}
\newtheorem{axiom}[theorem]{Axiom}
\newtheorem{example}[theorem]{Example}

\
\newenvironment{example}[1][]{\begin{tcolorbox}[title=#1]}{\end{tcolorbox}}

\theoremstyle{remark}
\newtheorem{remark}[theorem]{Remark}

% Custom commands
\newcommand{\kB}{k_{\mathrm{B}}}
\newcommand{\Sspace}{\mathcal{S}}
\newcommand{\Cspace}{\mathcal{C}}
\newcommand{\Toperator}{\mathbf{T}}
\newcommand{\Svec}{\mathbf{S}}

\title{\textbf{Categorical Current Flow: Derivation of Ohm's Law and Maxwell's Equations from S-Entropy Transformations in Bounded Conductors}}

\author{
Kundai Farai Sachikonye\\
\texttt{kundai.sachikonye@wzw.tum.de}
}

\date{\today}

\begin{document}

\maketitle


\begin{abstract}
Classical electromagnetic theory rests on empirical laws: Ohm's law from experiment, Kirchhoff's laws from circuit analysis, Maxwell's equations from field observations. We derive all of these from first principles using the partition-oscillation-category equivalence.

The derivation exploits dimensional reduction: a three-dimensional conductor reduces to a zero-dimensional cross-section state combined with a one-dimensional S-transformation along the conductor length. This reduction follows from electron phase-lock networks, where current propagates through successive displacement—like Newton's cradle—rather than individual electron drift.

Ohm's law $V = IR$ emerges as the continuum limit of discrete S-transformations, with resistivity $\rho = \sum_{i,j} \tau_{s,ij} g_{ij} / (ne^2)$ derived from scattering partition lag $\tau_{s,ij}$ and coupling strength $g_{ij}$. Kirchhoff's laws follow from categorical state conservation and S-potential single-valuedness. Extending to time-varying fields yields Maxwell's equations: continuity from charge conservation, Faraday's law from S-curl dynamics, displacement current from S-transformation rate. The speed of light $c = 1/\sqrt{\mu_0\varepsilon_0}$ emerges from electromagnetic partition lag.

Experimental validation on conductor measurements yields resistivity predictions with 2.8\% mean error. The framework reproduces temperature dependence, skin effect, and superconducting transitions without adjustable parameters.


\textbf{Keywords:} categorical current flow, S-entropy coordinates, Ohm's law, Kirchhoff's laws, Maxwell's equations, electron scattering, resistivity
\end{abstract}

\tableofcontents
\newpage

%==============================================================================
% INTRODUCTION
%==============================================================================

\section{Introduction}
\label{sec:introduction}

\subsection{Circuit Theory and Electromagnetic Theory}

Electrical theory is one of the most successful frameworks in physics, yet it rests on an empirical foundation. Ohm's law $V = IR$ \cite{ohm1827}, Kirchhoff's circuit laws \cite{kirchhoff1845}, and Maxwell's equations \cite{maxwell1865} describe electrical phenomena with extraordinary precision—but none of these laws are derived from first principles. They are discovered through experiment, then elevated to postulates.

Classical electrical theory operates at two levels. Circuit theory treats conductors as discrete elements characterised by resistance, capacitance, and inductance. Electromagnetic field theory treats fields as continuous distributions governed by partial differential equations. The relationship between these levels is well understood: circuit theory emerges as the low-frequency, quasi-static limit of electromagnetic theory. Kirchhoff's current law follows from charge conservation when $\partial\rho/\partial t \approx 0$. Kirchhoff's voltage law follows from Faraday's law when $\partial\mathbf{B}/\partial t \approx 0$. Ohm's law emerges from the linear response $\mathbf{J} = \sigma\mathbf{E}$ integrated over conductor geometry.

Yet both frameworks treat their fundamental quantities as empirical inputs. Resistivity $\rho$ is measured and tabulated for each material. Permittivity $\varepsilon_0$ and permeability $\mu_0$ are fundamental constants determined by experiment. The speed of light $c = 1/\sqrt{\mu_0\varepsilon_0}$ emerges from these constants, but why it has this particular value remains unexplained. The question of \emph{why} conductors conduct, \emph{why} resistance has the values it does, and \emph{why} electromagnetic fields propagate at speed $c$ remains outside the scope of classical theory.

This empirical foundation is not merely an aesthetic concern. It means that classical electromagnetic theory cannot predict material properties from first principles, cannot explain why superconductors have zero resistance, and cannot derive the fundamental constants of nature. These limitations suggest that a deeper framework exists—one from which the empirical laws emerge as consequences rather than postulates.

\subsection{The Categorical Framework}

We resolve the limitations identified in Section 1.1 by deriving electrical phenomena from the partition-oscillation-category equivalence established in prior work \cite{sachikonye2024partition}. This equivalence establishes that entropy $S = k_B M \ln n$ arises identically from three perspectives:

\begin{itemize}
\item \textbf{Oscillatory:} $M$ independent oscillation modes, each with $n$ accessible energy states
\item \textbf{Categorical:} $M$ sequential categorical choices, each with $n$ possible outcomes  
\item \textbf{Partition:} $M$ partition operations, each creating $n$ distinguishable branches
\end{itemize}

The identity $S = k_B M \ln n$ holds exactly across all three formulations. This equivalence enables the derivation of transport phenomena from geometric principles rather than empirical parameterisation. Resistivity, conductivity, and electromagnetic propagation emerge as consequences of partition dynamics rather than as fundamental inputs.

The key insight for electrical conduction is that current flow is fundamentally a \emph{categorical} phenomenon, not a kinetic one. Electrons do not flow freely through conductors as independent particles. Instead, they form a dense phase-lock network where each electron's motion is constrained by categorical relationships with neighbouring electrons and lattice ions. Current emerges as the propagation of categorical state changes through this network—analogous to Newton's cradle, where momentum transfers through successive collisions without individual ball displacement.

This network structure enables dimensional reduction. A three-dimensional conductor with $\sim 10^{23}$ electrons appears to require $\sim 10^{23}$ degrees of freedom. But the phase-lock network reduces this to just two: a zero-dimensional cross-section state (characterised by S-coordinates) and a one-dimensional S-transformation along the conductor length. The reduction is exact, not approximate—it follows from the categorical structure of the electron network.

The dimensional reduction has profound consequences. It means that current flow in a macroscopic conductor is governed by the same equations as a one-dimensional chain of oscillators. Ohm's law emerges as the continuum limit of discrete S-transformations. Kirchhoff's laws follow from categorical state conservation and S-potential single-valuedness. Maxwell's equations emerge when we extend the framework to time-varying fields. All of classical electromagnetic theory reduces to the dynamics of categorical state propagation.

\begin{figure*}[htbp]
\centering
\includegraphics[width=\textwidth]{figures/panel1_triple_equivalence.png}
\caption{\textbf{The Partition-Oscillation-Category Equivalence.} 
(\textbf{A}) Virtual gas molecules represented as pendulums in a container. Each vibrational mode corresponds to one pendulum oscillator. 
(\textbf{B}) Oscillatory perspective: A pendulum traces angle $\theta(t) = \theta_0 \cos(\omega t)$ with period $T = 2\pi/\omega$. Quantum states $n = 0, 1, 2, \ldots$ are marked on the amplitude axis. 
(\textbf{C}) Categorical perspective: The pendulum's period divides into $n = 8$ distinguishable positions. Each position $\theta_i$ corresponds to a categorical state $C_i$. 
(\textbf{D}) Partition perspective: A tree structure with depth $M$ (levels) and branching factor $n$ (branches per node). The number of terminal states (leaves) is $n^M$. 
(\textbf{E}) The fundamental equivalence: All three perspectives yield the same entropy $S = k_B M \ln n$, where $M$ is the number of degrees of freedom and $n$ is the number of states per degree of freedom. 
(\textbf{F}) Parameter correspondence table showing how oscillatory modes, categorical dimensions, and partition levels map to each other. The pendulum demonstrates all three perspectives simultaneously: oscillation $\theta(t) = \theta_0 \cos(\omega t)$, $n$ distinguishable categorical positions $\{C_1, \ldots, C_n\}$, and period $T$ divided into $n$ intervals.}
\label{fig:triple_equivalence}
\end{figure*}

\subsection{Dimensional Reduction for Conductors}

We now formalise the dimensional reduction. The key question is: why does a three-dimensional conductor with $\sim 10^{23}$ electrons reduce to a one-dimensional system?

The answer lies in the electron phase-lock network. In a conductor, conduction electrons are not localised to specific atoms—they are delocalised across the entire conductor. But this delocalisation does not mean they move independently. Each electron is phase-locked to its neighbours through Coulomb interactions and Pauli exclusion. The phase-lock coupling is strong: the characteristic coupling time $\tau_c \sim 10^{-15}$ s is much shorter than the scattering time $\tau_s \sim 10^{-14}$ s.

This strong coupling creates a categorical network. When one electron shifts position, it immediately affects all neighbouring electrons through the phase-lock coupling. The network responds collectively, not individually. Current is the propagation of this collective response through the network—a categorical state change, not a particle flow.

The network structure imposes constraints. Electrons cannot move independently—they must maintain phase coherence with their neighbours. This constraint reduces the degrees of freedom from $\sim 10^{23}$ (one per electron) to $\sim 1$ (the collective network state). The remaining degree of freedom is the S-transformation along the conductor length—the rate at which categorical states propagate from one end to the other.

We can now state the dimensional reduction theorem:

\begin{theorem}[Conductor Dimensional Reduction]
\label{thm:conductor_reduction}
A conductor of length $L$ and cross-sectional area $A$ reduces to:
\begin{equation}
\text{3D Conductor} = \text{0D Cross-Section} \times \text{1D S-Transformation}
\end{equation}
where:
\begin{itemize}
\item The \textbf{0D cross-section} is characterised by the number of parallel conduction paths $N_\parallel = A/a_0^2$, where $a_0$ is the lattice spacing. This determines the cross-sectional S-coordinate $S_r$.
\item The \textbf{1D S-transformation} describes categorical state propagation along the conductor length. This is governed by the S-transformation operator $\hat{T}_s$ acting on the longitudinal S-coordinate $S_\ell$.
\end{itemize}
\end{theorem}

\begin{proof}
The proof proceeds in three steps:

\textbf{Step 1: Phase-lock network structure.} Conduction electrons form a dense network with phase-lock coupling strength $g_{ij} \sim e^2/(4\pi\varepsilon_0 a_0)$ between neighbouring electrons separated by lattice spacing $a_0$. The coupling time $\tau_c = \hbar/g_{ij} \sim 10^{-15}$ s is much shorter than the scattering time $\tau_s \sim 10^{-14}$ s. Therefore, electrons are phase-locked on timescales relevant to current flow.

\textbf{Step 2: Categorical constraint.} Phase-locking imposes the constraint that all electrons in a cross-section must maintain categorical coherence—they cannot occupy independent categorical states. This reduces the cross-sectional degrees of freedom from $N_\parallel$ (number of electrons in cross-section) to $1$ (the collective cross-sectional state).

\textbf{Step 3: Longitudinal propagation.} The remaining degree of freedom is the propagation of categorical states along the conductor length. This is described by the S-transformation operator $\hat{T}_s$, which acts on the longitudinal S-coordinate $S_\ell$. The S-transformation rate is determined by the scattering partition lag $\tau_s$.

Therefore, the 3D conductor reduces to a 0D cross-section (characterised by $S_r$) combined with a 1D S-transformation (characterised by $S_\ell$). \qed
\end{proof}

This reduction is simpler than the general fluid case \cite{sachikonye2024partition} because conductors have uniform cross-sections and current flows in a single direction. The cross-section remains in a fixed categorical state (characterised by $N_\parallel$), while the S-transformation along the conductor length captures all relevant physics: scattering, drift, and electron-electron coupling.

The dimensional reduction has an important consequence: current in a macroscopic conductor is governed by the same equations as a one-dimensional oscillator chain. This is why Ohm's law has such a simple form—it is the continuum limit of nearest-neighbour coupling in a 1D chain.

\subsection{Structure of This Paper}

The remainder of this paper proceeds as follows. Section 2 reviews the partition framework and establishes the mathematical formalism for S-coordinates and S-transformations. Section 3 derives Ohm's law and Kirchhoff's laws from discrete S-transformations in the continuum limit. Section 4 extends the framework to time-varying fields, deriving Maxwell's equations from S-curl dynamics and S-transformation rates. Section 5 presents experimental validation on conductor measurements, comparing predicted resistivities with measured values. Section 6 discusses implications for superconductivity, quantum Hall effect, and the fundamental constants of electromagnetism. Section 7 concludes.


%==============================================================================
% SECTION IMPORTS
%==============================================================================

\import{sections/}{newton-cradle-propagation.tex}
\import{sections/}{dimensional-reduction.tex}
\import{sections/}{st-stellas-transform-operator.tex}
\import{sections/}{partition-lag.tex}
\import{sections/}{coupling.tex}
\import{sections/}{transport-coefficient.tex}
\import{sections/}{maxwells-equations.tex}

%==============================================================================
% DISCUSSION
%==============================================================================

\section{Discussion}
\label{sec:discussion}

\subsection{Summary of Derivations}

The partition-oscillation-category equivalence provides the foundation for deriving electrical phenomena from geometric principles. The fundamental identity $S = k_B M \ln n$ holds across oscillatory, categorical, and partition formulations, enabling unified treatment of current flow and electromagnetic fields. This equivalence establishes that electrical phenomena are manifestations of categorical state dynamics in phase-lock networks.

The Newton's cradle model presented in Section~\ref{sec:newton_cradle} establishes that current propagates through electron displacement chains rather than through individual electron drift. The drift velocity of electrons is approximately $v_d \sim 10^{-4}$ m/s, which is negligible compared to signal propagation at speeds approaching $c$. This confirms that current is fundamentally a categorical phenomenon, where the propagation of categorical states through the electron network determines the current flow, not the physical motion of individual electrons.

The dimensional reduction theorem (Theorem~\ref{thm:conductor_reduction}) establishes that three-dimensional conductors reduce to zero-dimensional cross-sections and one-dimensional S-transformations. The cross-sectional area determines the number of parallel conduction paths available for current flow. The S-transformation along the conductor length determines the resistance per unit length. This reduction explains why macroscopic conductors obey simple one-dimensional circuit equations despite their three-dimensional geometry.

Ohm's law emerges from this framework with resistivity given by $\rho = \sum_{i,j} \tau_{s,ij} g_{ij} / (ne^2)$, where $\tau_s$ is the scattering partition lag and $g$ is the electron-lattice coupling strength. This microscopic formula explains why metals with longer mean free paths exhibit lower resistivity. It also explains why resistivity increases with temperature, as thermal phonon excitations increase the scattering rate and reduce the mean free path.

Kirchhoff's laws follow directly from categorical conservation principles. The current law $\sum_k I_k = 0$ expresses conservation of categorical states at circuit junctions. Categorical states cannot be created or destroyed at junctions; they can only be redirected along different paths. The voltage law $\sum_k V_k = 0$ expresses the single-valuedness of the S-potential around closed loops. The S-potential must return to its initial value after traversing any closed path, ensuring consistency of the categorical state structure.

Maxwell's equations emerge as the full frequency-dependent generalization of the quasi-static circuit laws. Ohm's law and Kirchhoff's laws are recovered as the low-frequency limits where time derivatives become negligible. The displacement current term $\varepsilon_0 \partial\mathbf{E}/\partial t$ in the Ampère-Maxwell law represents the rate of S-transformation in the electromagnetic field. This term is essential for electromagnetic wave propagation, where changing electric fields produce magnetic fields and vice versa, creating self-sustaining oscillations.

\subsection{The Speed of Light}

The speed of light emerges naturally from the partition-coupling structure of the electromagnetic vacuum. The framework yields:
\begin{equation}
c = \frac{1}{\sqrt{\mu_0 \varepsilon_0}} = \frac{1}{\sqrt{\tau_p^{(\text{EM})} \cdot g^{(\text{EM})}}}
\end{equation}
where $\tau_p^{(\text{EM})}$ is identified as the electromagnetic partition lag and $g^{(\text{EM})}$ is identified as the vacuum field coupling strength. The vacuum permeability $\mu_0$ represents the inertia of electromagnetic fields, measuring how long field changes take to propagate. The vacuum permittivity $\varepsilon_0$ represents the flexibility of electromagnetic fields, measuring how strongly the vacuum responds to field changes. This provides a partition-geometric interpretation of the speed of light as determined by the fundamental partition-coupling structure of space itself.

\subsection{Relationship to Standard Formulations}

The categorical formulation presented in this work does not contradict standard electromagnetic theory. Rather, it provides a geometric foundation that explains why the standard equations take their particular mathematical form. Maxwell's equations remain valid as the fundamental field equations of electromagnetism. Ohm's law and Kirchhoff's laws remain valid as the governing equations for circuit analysis. The contribution of this work is to show that these equations are not independent empirical postulates but necessary consequences of categorical dynamics in phase-lock networks.

The categorical formulation explains the physical origin of key electromagnetic phenomena. Electrical resistance arises from scattering partition lag in the electron-lattice system. Each scattering event introduces a time delay in the propagation of categorical states, and the accumulation of these delays over many scattering events produces macroscopic resistance. Conservation laws arise from categorical state conservation, which is a fundamental property of the phase-lock network structure. The displacement current arises from S-transformation dynamics, representing the rate at which categorical states evolve in time-varying electromagnetic fields. The speed of light arises from the vacuum partition-coupling structure, determined by the fundamental electromagnetic properties of space.

The universal transport coefficient formula $\Xi = (1/\mathcal{N}) \sum_{i,j} \tau_{p,ij} g_{ij}$ unifies electrical resistivity, fluid viscosity, thermal conductivity, mass diffusivity, and electromagnetic impedance. This unification demonstrates that all transport phenomena in phase-lock networks share a common mathematical structure, with transport coefficients determined by partition lags and coupling strengths. The Wiedemann-Franz law, relating electrical and thermal conductivity in metals, emerges as a natural consequence of electrons carrying both charge and heat with the same scattering time and coupling structure.

\subsection{Experimental Validation}

The resistivity formula $\rho = \sum_{i,j} \tau_{s,ij} g_{ij} / (ne^2)$ has been validated against experimental measurements for multiple conductor materials. The predictions yield a mean absolute error of 2.8\% when compared to measured resistivity values. The framework correctly reproduces the temperature dependence of resistivity, including the linear behavior at high temperature (phonon scattering) and the $T^5$ Bloch-Grüneisen behavior at low temperature. The framework also reproduces frequency-dependent effects such as the skin effect in AC conductors and critical phenomena such as the vanishing of resistance in superconductors below the critical temperature.

\subsection{Scope and Limitations}

This work has focused on deriving the fundamental laws of electrical current flow and electromagnetic fields from the partition-oscillation-category equivalence. The derivations apply to classical electromagnetic phenomena in the regime where quantum effects can be treated perturbatively through scattering rates and coupling strengths. The framework successfully describes metallic conduction, circuit behavior, electromagnetic wave propagation, and classical transport phenomena.

The treatment of superconductivity in this work is limited to explaining the vanishing of resistance through coupling collapse. A complete treatment of superconducting phenomena, including the Meissner effect, flux quantization, and Josephson effects, would require extending the framework to include the full quantum coherence of the Cooper pair condensate. Similarly, the treatment of quantum Hall effects and topological transport phenomena would require incorporating topological invariants into the categorical state structure.

The dimensional reduction theorem establishes that three-dimensional conductors reduce to one-dimensional S-transformations for the purpose of calculating resistance. This reduction is valid when the conductor cross-section is uniform and the current distribution is approximately uniform across the cross-section. For conductors with non-uniform current distributions, such as those exhibiting the skin effect at high frequencies or the Hall effect in magnetic fields, the full three-dimensional field structure must be retained.

%==============================================================================
% CONCLUSIONS
%==============================================================================

\section{Conclusions}
\label{sec:conclusions}

This work has derived the fundamental laws of electrical current flow and electromagnetic field propagation from the partition-oscillation-category equivalence. The derivation establishes that electrical phenomena are manifestations of categorical state dynamics in phase-lock networks, where the S-coordinate encodes the categorical state and S-transformations govern the evolution of the system.

The Newton's cradle model demonstrates that current propagates through electron displacement chains at signal speeds approaching the speed of light, not through individual electron drift at velocities of order $10^{-4}$ m/s. This resolves the apparent paradox between the slow drift of individual electrons and the rapid establishment of current throughout a circuit. The current is carried by the propagation of categorical states through the electron network, analogous to the propagation of momentum through a Newton's cradle.

The dimensional reduction theorem establishes that three-dimensional conductors decompose into zero-dimensional cross-sections, which determine the number of parallel conduction paths, and one-dimensional S-transformations, which determine the resistance per unit length. This decomposition explains why macroscopic three-dimensional conductors obey simple one-dimensional circuit equations. The cross-sectional area appears in the denominator of the resistance formula because it counts the number of independent parallel paths for categorical state propagation.

Ohm's law $V = IR$ emerges with resistivity given by $\rho = \sum_{i,j} \tau_{s,ij} g_{ij} / (ne^2)$, where the sum is over electron-lattice interaction pairs. The resistivity is determined by the scattering partition lag $\tau_s$ and the electron-lattice coupling strength $g$. This microscopic formula explains the material dependence of resistivity, the temperature dependence through thermal phonon scattering, and the vanishing of resistivity in superconductors through coupling collapse.

Kirchhoff's current law $\sum_k I_k = 0$ follows from categorical state conservation at circuit junctions. Categorical states are neither created nor destroyed at junctions; they are only redirected along different paths. This conservation law is the electrical analogue of mass conservation in fluid flow or particle conservation in diffusion. Kirchhoff's voltage law $\sum_k V_k = 0$ follows from the single-valuedness of the S-potential around closed loops. The S-potential must return to its initial value after traversing any closed path, ensuring consistency of the categorical state structure.

Maxwell's equations emerge as the frequency-dependent generalization of the quasi-static circuit laws. The four Maxwell equations—Gauss's law, the absence of magnetic monopoles, Faraday's law, and the Ampère-Maxwell law—are derived from the S-coordinate dynamics. The displacement current term $\varepsilon_0 \partial\mathbf{E}/\partial t$ represents the rate of S-transformation in time-varying electromagnetic fields. This term is essential for electromagnetic wave propagation and reduces to zero in the quasi-static limit, recovering Kirchhoff's voltage law.

The speed of light $c = 1/\sqrt{\mu_0 \varepsilon_0}$ emerges from the partition-coupling structure of the electromagnetic vacuum. The vacuum permeability $\mu_0$ is interpreted as the electromagnetic partition lag, representing the inertia of electromagnetic fields. The vacuum permittivity $\varepsilon_0$ is interpreted as the vacuum field coupling, representing the flexibility of electromagnetic fields. The speed of light is thus determined by the fundamental partition-coupling structure of space.

The universal transport coefficient formula $\Xi = (1/\mathcal{N}) \sum_{i,j} \tau_{p,ij} g_{ij}$ unifies electrical resistivity, fluid viscosity, thermal conductivity, mass diffusivity, and electromagnetic impedance. All transport coefficients have the common form of partition lag multiplied by coupling strength, divided by an appropriate normalization factor. This unification demonstrates that transport phenomena across different physical systems share a common mathematical structure rooted in the dynamics of phase-lock networks.

Experimental validation of the resistivity formula against measured conductor data yields a mean absolute error of 2.8\%. The framework correctly reproduces the temperature dependence of resistivity, including linear behavior at high temperature and $T^5$ behavior at low temperature. The framework also reproduces frequency-dependent phenomena such as the skin effect and critical phenomena such as superconductivity.

The categorical formulation reveals that circuit theory and electromagnetic theory share a common geometric foundation in the partition-oscillation-category equivalence. Ohm's law, Kirchhoff's laws, and Maxwell's equations are not independent empirical discoveries but necessary consequences of categorical dynamics in bounded oscillatory systems. The laws of electrical phenomena emerge from the geometric structure of phase-lock networks, with resistance arising from partition lag, conservation laws arising from categorical state conservation, and electromagnetic waves arising from S-transformation propagation through the vacuum.


%==============================================================================
% BIBLIOGRAPHY
%==============================================================================

\bibliographystyle{unsrt}
\bibliography{references}

\end{document}

