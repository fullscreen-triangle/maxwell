% Figures for Categorical Current Flow
% Include this file in the main document or use % Figures for Resolution of Loschmidt's Paradox
% Include this file in the main document or use % Figures for Resolution of Loschmidt's Paradox
% Include this file in the main document or use % Figures for Resolution of Loschmidt's Paradox
% Include this file in the main document or use \input{figures.tex}

%==============================================================================
% Figure L-1: Mixing-Separation Entropy Cycle
%==============================================================================
\begin{figure}[htbp]
\centering
\includegraphics[width=\textwidth]{figures/panel_mixing_separation.pdf}
\caption{\textbf{Mixing-Separation Cycle Demonstrates Irreversibility.}
(A) Initial state: two gases separated by partition, with entropy $S_{initial} = S_A^{(0)} + S_B^{(0)}$.
(B) Mixed state: partition removed, gases interdiffuse, entropy increases by $\Delta S_{mix}$.
(C) Re-separated state: partition restored, but each container now contains both gases with residual phase correlations.
(D) Entropy evolution: the categorical prediction (green) shows $S_{final} > S_{initial}$ despite identical spatial configuration, while the classical reversible prediction (gray) incorrectly predicts return to initial entropy. The difference $\Delta S_{irrev} > 0$ arises from phase-lock network densification that persists after re-separation.}
\label{fig:mixing_separation}
\end{figure}

%==============================================================================
% Figure L-2: Phase-Lock Network Evolution
%==============================================================================
\begin{figure}[htbp]
\centering
\includegraphics[width=\textwidth]{figures/panel_phase_lock_network.pdf}
\caption{\textbf{Phase-Lock Network Densification and Residual Correlations.}
(A) Initial separated state: two disconnected network clusters (blue = Gas A, red = Gas B) with $|E|$ internal edges.
(B) Mixed state: networks merge into single connected component with cross-container edges.
(C) Re-separated state: partition restored, but residual cross-edges (red dashed) persist---these represent phase correlations created during mixing that cannot be erased.
(D) Edge count evolution: $|E_{final}| > |E_{initial}|$ demonstrates that mixing creates categorical structure (edges) that remains after re-separation. More edges means more constraints, hence higher entropy.}
\label{fig:phase_lock_network}
\end{figure}

%==============================================================================
% Figure L-3: Non-Actualisation Asymmetry
%==============================================================================
\begin{figure}[htbp]
\centering
\includegraphics[width=\textwidth]{figures/panel_non_actualisation.pdf}
\caption{\textbf{Non-Actualisation Asymmetry---The Deepest Reason for Irreversibility.}
(A) The cup example: when a cup falls and breaks, it generates infinitely many non-actualisations (not turning to gold, not becoming sentient, not teleporting, etc.)---categorical facts defined by negation.
(B) Branching asymmetry: each actualisation (green, finite) creates infinitely many non-actualisations (red), yielding a 1:$\infty$ asymmetry ratio.
(C) Accumulation over time: non-actualisations grow monotonically and cannot be un-created, while actualisations remain finite.
(D) Forward/backward asymmetry: forward processes always possible (create non-actualisations), backward processes impossible (would require un-creating non-actualisations). The ratio $P_{forward}/P_{backward} \to \infty$.}
\label{fig:non_actualisation}
\end{figure}

%==============================================================================
% Figure L-4: Aperture Selectivity and Categorical Potential
%==============================================================================
\begin{figure}[htbp]
\centering
\includegraphics[width=\textwidth]{figures/panel_aperture_selectivity.pdf}
\caption{\textbf{Partition Boundaries as Categorical Apertures.}
(A) Selection function $\sigma(\omega)$: aperture (partition boundary) allows certain configurations to pass ($\sigma = 1$, green arrows) while blocking others ($\sigma = 0$, red X marks).
(B) Categorical potential vs selectivity: $\Phi_a = -k_B T \ln s$ where $s = \Omega_{pass}/\Omega_{total}$. High selectivity ($s \to 0$) implies high potential barrier.
(C) Entropy from selectivity: higher selectivity (lower $s$) produces more entropy, since $\Delta S = k_B \ln(1/s) = \Phi_a/T$.
(D) Aperture as energy barrier: the categorical potential acts as a barrier that blocked configurations must overcome. Non-actualisations are precisely the configurations blocked by partition apertures.}
\label{fig:aperture_selectivity}
\end{figure}

%==============================================================================
% Figure L-5: Partition Lag Dynamics
%==============================================================================
\begin{figure}[htbp]
\centering
\includegraphics[width=\textwidth]{figures/panel_partition_lag.pdf}
\caption{\textbf{Partition Lag---The Finite Time of Categorical Determination.}
(A) Partition lag distribution: different systems exhibit different lag time distributions, with a fundamental minimum $\tau_{min} = \hbar/\Delta E$ set by the uncertainty principle.
(B) Undetermined residue evolution: during partition lag, categorical states remain in superposition. The determination fraction approaches 1 asymptotically, with residue decreasing exponentially.
(C) Entropy production rate: entropy is produced continuously during partition lag at rate $dS/dt = k_B \cdot \text{Residue}/\tau_{lag}$. Cumulative entropy $S(t)$ saturates as determination completes.
(D) Minimum lag scaling: $\tau_{min} \propto 1/\Delta E$ across different energy scales (phonon, vibrational, electronic, core), demonstrating the fundamental quantum limit on partition speed.}
\label{fig:partition_lag}
\end{figure}

%==============================================================================
% Figure L-6: Termination and Irreversibility
%==============================================================================
\begin{figure}[htbp]
\centering
\includegraphics[width=\textwidth]{figures/panel_termination_irreversibility.pdf}
\caption{\textbf{Termination, Completion, and the Impossibility of Reversal.}
(A) Reality stream vs terminated state: ongoing processes (blue wave) have indeterminate entropy---they are superpositions of possibilities. Only terminated states (green box) have well-defined entropy as categorical facts.
(B) Identity of completion and partitioning: categorical completion (selecting one outcome) is identical to geometric partitioning (creating boundaries). Both create the distinction between actualised and non-actualised states.
(C) Why reversal fails: forward processes create non-actualisations ($A \to B$ plus infinitely many things $B$ is not doing). Backward would require un-creating these---impossible.
(D) Asymmetry ratio growth: with each categorical completion, the forward/backward probability ratio grows exponentially as $\prod_i \Omega_i$, rapidly diverging from the reversible ratio of 1.}
\label{fig:termination_irreversibility}
\end{figure}

%==============================================================================
% Figure L-7: Cross-Sectional Validation of Irreversibility
%==============================================================================
\begin{figure}[htbp]
\centering
\includegraphics[width=\textwidth]{figures/panel_loschmidt_cross_sectional_validation.pdf}
\caption{\textbf{Cross-Sectional Validation: Radial Expansion and the Arrow of Time.}
(A) S-coordinate evolution with radius: configuration entropy $S_k$ and temporal entropy $S_t$ plotted versus radial distance from an expanding point for three systems (Fast, Medium, Slow Expansion). Each radial shell is a cross-sectional measurement---a spherical surface at fixed distance. All systems show monotonic increase: entropy ALWAYS grows outward.
(B) Non-actualisations dominate: logarithmic plot of actualised (dashed) vs non-actualised (solid) state counts. Non-actualisations ($N_{\text{non-act}} \propto r^2 - \omega(t)r^2/4\pi$) vastly outnumber actualisations at all radii, with ratios from 37:1 (fast expansion) to 392:1 (slow expansion). This asymmetry is the origin of irreversibility.
(C) Entropy gradient always positive: the derivative $\partial S_t/\partial r > 0$ at ALL radii for ALL systems. Green region indicates positive (irreversible) gradients; no system ever enters the negative (reversible) region. The gradient points outward because non-actualisations accumulate faster than actualisations can explore.
(D) Irreversibility fraction: bar chart showing 100\% positive gradients for all three expansion regimes. The 50\% line (gray dashed) marks the threshold for reversible processes; 100\% marks complete irreversibility. All systems achieve 100\%, confirming that the arrow of time is universal and independent of expansion rate.
(E) S-transformation validation: predicted S-coordinates (from $\mathcal{T}_{dr}$) versus calculated values, showing $R^2 > 0.99$ for all systems. The transformation correctly predicts entropy at each radial shell from the previous shell's state.
(F) Schematic of expanding point: central point (black dot) expands into state space, creating concentric spherical shells (color gradient from green/low entropy to red/high entropy). Arrows show outward expansion direction. The gradient $\nabla S > 0$ always points away from the origin---this is the geometric necessity of irreversibility. Non-actualisations form a ``wake'' around the actualised trajectory, and this wake grows with distance, making reversal impossible.}
\label{fig:loschmidt_cross_sectional_validation}
\end{figure}



%==============================================================================
% Figure L-1: Mixing-Separation Entropy Cycle
%==============================================================================
\begin{figure}[htbp]
\centering
\includegraphics[width=\textwidth]{figures/panel_mixing_separation.pdf}
\caption{\textbf{Mixing-Separation Cycle Demonstrates Irreversibility.}
(A) Initial state: two gases separated by partition, with entropy $S_{initial} = S_A^{(0)} + S_B^{(0)}$.
(B) Mixed state: partition removed, gases interdiffuse, entropy increases by $\Delta S_{mix}$.
(C) Re-separated state: partition restored, but each container now contains both gases with residual phase correlations.
(D) Entropy evolution: the categorical prediction (green) shows $S_{final} > S_{initial}$ despite identical spatial configuration, while the classical reversible prediction (gray) incorrectly predicts return to initial entropy. The difference $\Delta S_{irrev} > 0$ arises from phase-lock network densification that persists after re-separation.}
\label{fig:mixing_separation}
\end{figure}

%==============================================================================
% Figure L-2: Phase-Lock Network Evolution
%==============================================================================
\begin{figure}[htbp]
\centering
\includegraphics[width=\textwidth]{figures/panel_phase_lock_network.pdf}
\caption{\textbf{Phase-Lock Network Densification and Residual Correlations.}
(A) Initial separated state: two disconnected network clusters (blue = Gas A, red = Gas B) with $|E|$ internal edges.
(B) Mixed state: networks merge into single connected component with cross-container edges.
(C) Re-separated state: partition restored, but residual cross-edges (red dashed) persist---these represent phase correlations created during mixing that cannot be erased.
(D) Edge count evolution: $|E_{final}| > |E_{initial}|$ demonstrates that mixing creates categorical structure (edges) that remains after re-separation. More edges means more constraints, hence higher entropy.}
\label{fig:phase_lock_network}
\end{figure}

%==============================================================================
% Figure L-3: Non-Actualisation Asymmetry
%==============================================================================
\begin{figure}[htbp]
\centering
\includegraphics[width=\textwidth]{figures/panel_non_actualisation.pdf}
\caption{\textbf{Non-Actualisation Asymmetry---The Deepest Reason for Irreversibility.}
(A) The cup example: when a cup falls and breaks, it generates infinitely many non-actualisations (not turning to gold, not becoming sentient, not teleporting, etc.)---categorical facts defined by negation.
(B) Branching asymmetry: each actualisation (green, finite) creates infinitely many non-actualisations (red), yielding a 1:$\infty$ asymmetry ratio.
(C) Accumulation over time: non-actualisations grow monotonically and cannot be un-created, while actualisations remain finite.
(D) Forward/backward asymmetry: forward processes always possible (create non-actualisations), backward processes impossible (would require un-creating non-actualisations). The ratio $P_{forward}/P_{backward} \to \infty$.}
\label{fig:non_actualisation}
\end{figure}

%==============================================================================
% Figure L-4: Aperture Selectivity and Categorical Potential
%==============================================================================
\begin{figure}[htbp]
\centering
\includegraphics[width=\textwidth]{figures/panel_aperture_selectivity.pdf}
\caption{\textbf{Partition Boundaries as Categorical Apertures.}
(A) Selection function $\sigma(\omega)$: aperture (partition boundary) allows certain configurations to pass ($\sigma = 1$, green arrows) while blocking others ($\sigma = 0$, red X marks).
(B) Categorical potential vs selectivity: $\Phi_a = -k_B T \ln s$ where $s = \Omega_{pass}/\Omega_{total}$. High selectivity ($s \to 0$) implies high potential barrier.
(C) Entropy from selectivity: higher selectivity (lower $s$) produces more entropy, since $\Delta S = k_B \ln(1/s) = \Phi_a/T$.
(D) Aperture as energy barrier: the categorical potential acts as a barrier that blocked configurations must overcome. Non-actualisations are precisely the configurations blocked by partition apertures.}
\label{fig:aperture_selectivity}
\end{figure}

%==============================================================================
% Figure L-5: Partition Lag Dynamics
%==============================================================================
\begin{figure}[htbp]
\centering
\includegraphics[width=\textwidth]{figures/panel_partition_lag.pdf}
\caption{\textbf{Partition Lag---The Finite Time of Categorical Determination.}
(A) Partition lag distribution: different systems exhibit different lag time distributions, with a fundamental minimum $\tau_{min} = \hbar/\Delta E$ set by the uncertainty principle.
(B) Undetermined residue evolution: during partition lag, categorical states remain in superposition. The determination fraction approaches 1 asymptotically, with residue decreasing exponentially.
(C) Entropy production rate: entropy is produced continuously during partition lag at rate $dS/dt = k_B \cdot \text{Residue}/\tau_{lag}$. Cumulative entropy $S(t)$ saturates as determination completes.
(D) Minimum lag scaling: $\tau_{min} \propto 1/\Delta E$ across different energy scales (phonon, vibrational, electronic, core), demonstrating the fundamental quantum limit on partition speed.}
\label{fig:partition_lag}
\end{figure}

%==============================================================================
% Figure L-6: Termination and Irreversibility
%==============================================================================
\begin{figure}[htbp]
\centering
\includegraphics[width=\textwidth]{figures/panel_termination_irreversibility.pdf}
\caption{\textbf{Termination, Completion, and the Impossibility of Reversal.}
(A) Reality stream vs terminated state: ongoing processes (blue wave) have indeterminate entropy---they are superpositions of possibilities. Only terminated states (green box) have well-defined entropy as categorical facts.
(B) Identity of completion and partitioning: categorical completion (selecting one outcome) is identical to geometric partitioning (creating boundaries). Both create the distinction between actualised and non-actualised states.
(C) Why reversal fails: forward processes create non-actualisations ($A \to B$ plus infinitely many things $B$ is not doing). Backward would require un-creating these---impossible.
(D) Asymmetry ratio growth: with each categorical completion, the forward/backward probability ratio grows exponentially as $\prod_i \Omega_i$, rapidly diverging from the reversible ratio of 1.}
\label{fig:termination_irreversibility}
\end{figure}

%==============================================================================
% Figure L-7: Cross-Sectional Validation of Irreversibility
%==============================================================================
\begin{figure}[htbp]
\centering
\includegraphics[width=\textwidth]{figures/panel_loschmidt_cross_sectional_validation.pdf}
\caption{\textbf{Cross-Sectional Validation: Radial Expansion and the Arrow of Time.}
(A) S-coordinate evolution with radius: configuration entropy $S_k$ and temporal entropy $S_t$ plotted versus radial distance from an expanding point for three systems (Fast, Medium, Slow Expansion). Each radial shell is a cross-sectional measurement---a spherical surface at fixed distance. All systems show monotonic increase: entropy ALWAYS grows outward.
(B) Non-actualisations dominate: logarithmic plot of actualised (dashed) vs non-actualised (solid) state counts. Non-actualisations ($N_{\text{non-act}} \propto r^2 - \omega(t)r^2/4\pi$) vastly outnumber actualisations at all radii, with ratios from 37:1 (fast expansion) to 392:1 (slow expansion). This asymmetry is the origin of irreversibility.
(C) Entropy gradient always positive: the derivative $\partial S_t/\partial r > 0$ at ALL radii for ALL systems. Green region indicates positive (irreversible) gradients; no system ever enters the negative (reversible) region. The gradient points outward because non-actualisations accumulate faster than actualisations can explore.
(D) Irreversibility fraction: bar chart showing 100\% positive gradients for all three expansion regimes. The 50\% line (gray dashed) marks the threshold for reversible processes; 100\% marks complete irreversibility. All systems achieve 100\%, confirming that the arrow of time is universal and independent of expansion rate.
(E) S-transformation validation: predicted S-coordinates (from $\mathcal{T}_{dr}$) versus calculated values, showing $R^2 > 0.99$ for all systems. The transformation correctly predicts entropy at each radial shell from the previous shell's state.
(F) Schematic of expanding point: central point (black dot) expands into state space, creating concentric spherical shells (color gradient from green/low entropy to red/high entropy). Arrows show outward expansion direction. The gradient $\nabla S > 0$ always points away from the origin---this is the geometric necessity of irreversibility. Non-actualisations form a ``wake'' around the actualised trajectory, and this wake grows with distance, making reversal impossible.}
\label{fig:loschmidt_cross_sectional_validation}
\end{figure}



%==============================================================================
% Figure L-1: Mixing-Separation Entropy Cycle
%==============================================================================
\begin{figure}[htbp]
\centering
\includegraphics[width=\textwidth]{figures/panel_mixing_separation.pdf}
\caption{\textbf{Mixing-Separation Cycle Demonstrates Irreversibility.}
(A) Initial state: two gases separated by partition, with entropy $S_{initial} = S_A^{(0)} + S_B^{(0)}$.
(B) Mixed state: partition removed, gases interdiffuse, entropy increases by $\Delta S_{mix}$.
(C) Re-separated state: partition restored, but each container now contains both gases with residual phase correlations.
(D) Entropy evolution: the categorical prediction (green) shows $S_{final} > S_{initial}$ despite identical spatial configuration, while the classical reversible prediction (gray) incorrectly predicts return to initial entropy. The difference $\Delta S_{irrev} > 0$ arises from phase-lock network densification that persists after re-separation.}
\label{fig:mixing_separation}
\end{figure}

%==============================================================================
% Figure L-2: Phase-Lock Network Evolution
%==============================================================================
\begin{figure}[htbp]
\centering
\includegraphics[width=\textwidth]{figures/panel_phase_lock_network.pdf}
\caption{\textbf{Phase-Lock Network Densification and Residual Correlations.}
(A) Initial separated state: two disconnected network clusters (blue = Gas A, red = Gas B) with $|E|$ internal edges.
(B) Mixed state: networks merge into single connected component with cross-container edges.
(C) Re-separated state: partition restored, but residual cross-edges (red dashed) persist---these represent phase correlations created during mixing that cannot be erased.
(D) Edge count evolution: $|E_{final}| > |E_{initial}|$ demonstrates that mixing creates categorical structure (edges) that remains after re-separation. More edges means more constraints, hence higher entropy.}
\label{fig:phase_lock_network}
\end{figure}

%==============================================================================
% Figure L-3: Non-Actualisation Asymmetry
%==============================================================================
\begin{figure}[htbp]
\centering
\includegraphics[width=\textwidth]{figures/panel_non_actualisation.pdf}
\caption{\textbf{Non-Actualisation Asymmetry---The Deepest Reason for Irreversibility.}
(A) The cup example: when a cup falls and breaks, it generates infinitely many non-actualisations (not turning to gold, not becoming sentient, not teleporting, etc.)---categorical facts defined by negation.
(B) Branching asymmetry: each actualisation (green, finite) creates infinitely many non-actualisations (red), yielding a 1:$\infty$ asymmetry ratio.
(C) Accumulation over time: non-actualisations grow monotonically and cannot be un-created, while actualisations remain finite.
(D) Forward/backward asymmetry: forward processes always possible (create non-actualisations), backward processes impossible (would require un-creating non-actualisations). The ratio $P_{forward}/P_{backward} \to \infty$.}
\label{fig:non_actualisation}
\end{figure}

%==============================================================================
% Figure L-4: Aperture Selectivity and Categorical Potential
%==============================================================================
\begin{figure}[htbp]
\centering
\includegraphics[width=\textwidth]{figures/panel_aperture_selectivity.pdf}
\caption{\textbf{Partition Boundaries as Categorical Apertures.}
(A) Selection function $\sigma(\omega)$: aperture (partition boundary) allows certain configurations to pass ($\sigma = 1$, green arrows) while blocking others ($\sigma = 0$, red X marks).
(B) Categorical potential vs selectivity: $\Phi_a = -k_B T \ln s$ where $s = \Omega_{pass}/\Omega_{total}$. High selectivity ($s \to 0$) implies high potential barrier.
(C) Entropy from selectivity: higher selectivity (lower $s$) produces more entropy, since $\Delta S = k_B \ln(1/s) = \Phi_a/T$.
(D) Aperture as energy barrier: the categorical potential acts as a barrier that blocked configurations must overcome. Non-actualisations are precisely the configurations blocked by partition apertures.}
\label{fig:aperture_selectivity}
\end{figure}

%==============================================================================
% Figure L-5: Partition Lag Dynamics
%==============================================================================
\begin{figure}[htbp]
\centering
\includegraphics[width=\textwidth]{figures/panel_partition_lag.pdf}
\caption{\textbf{Partition Lag---The Finite Time of Categorical Determination.}
(A) Partition lag distribution: different systems exhibit different lag time distributions, with a fundamental minimum $\tau_{min} = \hbar/\Delta E$ set by the uncertainty principle.
(B) Undetermined residue evolution: during partition lag, categorical states remain in superposition. The determination fraction approaches 1 asymptotically, with residue decreasing exponentially.
(C) Entropy production rate: entropy is produced continuously during partition lag at rate $dS/dt = k_B \cdot \text{Residue}/\tau_{lag}$. Cumulative entropy $S(t)$ saturates as determination completes.
(D) Minimum lag scaling: $\tau_{min} \propto 1/\Delta E$ across different energy scales (phonon, vibrational, electronic, core), demonstrating the fundamental quantum limit on partition speed.}
\label{fig:partition_lag}
\end{figure}

%==============================================================================
% Figure L-6: Termination and Irreversibility
%==============================================================================
\begin{figure}[htbp]
\centering
\includegraphics[width=\textwidth]{figures/panel_termination_irreversibility.pdf}
\caption{\textbf{Termination, Completion, and the Impossibility of Reversal.}
(A) Reality stream vs terminated state: ongoing processes (blue wave) have indeterminate entropy---they are superpositions of possibilities. Only terminated states (green box) have well-defined entropy as categorical facts.
(B) Identity of completion and partitioning: categorical completion (selecting one outcome) is identical to geometric partitioning (creating boundaries). Both create the distinction between actualised and non-actualised states.
(C) Why reversal fails: forward processes create non-actualisations ($A \to B$ plus infinitely many things $B$ is not doing). Backward would require un-creating these---impossible.
(D) Asymmetry ratio growth: with each categorical completion, the forward/backward probability ratio grows exponentially as $\prod_i \Omega_i$, rapidly diverging from the reversible ratio of 1.}
\label{fig:termination_irreversibility}
\end{figure}

%==============================================================================
% Figure L-7: Cross-Sectional Validation of Irreversibility
%==============================================================================
\begin{figure}[htbp]
\centering
\includegraphics[width=\textwidth]{figures/panel_loschmidt_cross_sectional_validation.pdf}
\caption{\textbf{Cross-Sectional Validation: Radial Expansion and the Arrow of Time.}
(A) S-coordinate evolution with radius: configuration entropy $S_k$ and temporal entropy $S_t$ plotted versus radial distance from an expanding point for three systems (Fast, Medium, Slow Expansion). Each radial shell is a cross-sectional measurement---a spherical surface at fixed distance. All systems show monotonic increase: entropy ALWAYS grows outward.
(B) Non-actualisations dominate: logarithmic plot of actualised (dashed) vs non-actualised (solid) state counts. Non-actualisations ($N_{\text{non-act}} \propto r^2 - \omega(t)r^2/4\pi$) vastly outnumber actualisations at all radii, with ratios from 37:1 (fast expansion) to 392:1 (slow expansion). This asymmetry is the origin of irreversibility.
(C) Entropy gradient always positive: the derivative $\partial S_t/\partial r > 0$ at ALL radii for ALL systems. Green region indicates positive (irreversible) gradients; no system ever enters the negative (reversible) region. The gradient points outward because non-actualisations accumulate faster than actualisations can explore.
(D) Irreversibility fraction: bar chart showing 100\% positive gradients for all three expansion regimes. The 50\% line (gray dashed) marks the threshold for reversible processes; 100\% marks complete irreversibility. All systems achieve 100\%, confirming that the arrow of time is universal and independent of expansion rate.
(E) S-transformation validation: predicted S-coordinates (from $\mathcal{T}_{dr}$) versus calculated values, showing $R^2 > 0.99$ for all systems. The transformation correctly predicts entropy at each radial shell from the previous shell's state.
(F) Schematic of expanding point: central point (black dot) expands into state space, creating concentric spherical shells (color gradient from green/low entropy to red/high entropy). Arrows show outward expansion direction. The gradient $\nabla S > 0$ always points away from the origin---this is the geometric necessity of irreversibility. Non-actualisations form a ``wake'' around the actualised trajectory, and this wake grows with distance, making reversal impossible.}
\label{fig:loschmidt_cross_sectional_validation}
\end{figure}



%==============================================================================
% Figure C-1: Newton's Cradle Model
%==============================================================================
\begin{figure}[htbp]
\centering
\includegraphics[width=\textwidth]{figures/panel_newton_cradle.pdf}
\caption{\textbf{Newton's Cradle Model---Current as State Propagation.}
(A) Electron chain in wire: conduction electrons (blue circles) form a chain between fixed lattice ions (red crosses). The wire acts as a waveguide for categorical state propagation.
(B) Newton's Cradle displacement propagation: at $t=0$, first electron is pushed; at $t=dt$, displacement propagates through chain; at $t=2dt$, last electron exits. The \emph{signal} propagates at near-light speed while individual electrons barely move.
(C) Speed comparison: signal speed ($\sim 3 \times 10^8$ m/s) exceeds drift velocity ($\sim 10^{-4}$ m/s) by 12 orders of magnitude, demonstrating that current is not electron flow but state propagation.
(D) Classical vs categorical interpretation: the classical view (electrons flowing like water) is wrong. The categorical view (states propagating through chain) correctly explains why current responds instantaneously despite slow electron drift.}
\label{fig:newton_cradle}
\end{figure}

%==============================================================================
% Figure C-2: Dimensional Reduction
%==============================================================================
\begin{figure}[htbp]
\centering
\includegraphics[width=\textwidth]{figures/panel_dimensional_reduction.pdf}
\caption{\textbf{Dimensional Reduction---Wire as Cross-Section $\times$ S-Transform.}
(A) 3D wire: cylindrical conductor with infinite degrees of freedom (position of each electron in 3D space).
(B) 0D cross-section: all radial positions are equivalent for current flow---only the radius $r$ matters, reducing to a point parameter.
(C) 1D S-transformation along length: S-potential (voltage) varies linearly along wire, with S-coordinates tracking state evolution.
(D) Complete reduction formula: $\text{Wire} = \int_0^R 2\pi r \, dr \times \mathcal{S}$, giving resistance $R = \rho L/A = \rho L/(\pi r^2)$ from 0D (area) times 1D (length/conductivity).}
\label{fig:dimensional_reduction}
\end{figure}

%==============================================================================
% Figure C-3/4: Ohm's Law and Kirchhoff's Laws
%==============================================================================
\begin{figure}[htbp]
\centering
\includegraphics[width=\textwidth]{figures/panel_ohm_kirchhoff.pdf}
\caption{\textbf{Ohm's Law and Kirchhoff's Laws from Categorical Dynamics.}
(A) Ohm's Law $V = IR$: linear relationship emerges from S-dynamics with $R = \tau_s \cdot g \cdot L/A$ where $\tau_s$ is scattering partition lag and $g$ is electron-lattice coupling.
(B) Resistivity from scattering time: materials with longer scattering time $\tau_s$ (fewer apertures) have lower resistivity $\rho \propto 1/\tau_s$.
(C) Kirchhoff's Current Law: $\sum I_{in} = \sum I_{out}$ at any node expresses conservation of categorical states---states cannot be created or destroyed at junctions.
(D) Kirchhoff's Voltage Law: $\sum V_{loop} = 0$ around any closed loop expresses single-valuedness of S-potential---returning to the same point must yield the same categorical state.}
\label{fig:ohm_kirchhoff}
\end{figure}

%==============================================================================
% Figure C-5/6: Maxwell's Equations from S-Dynamics
%==============================================================================
\begin{figure}[htbp]
\centering
\includegraphics[width=\textwidth]{figures/panel_maxwell_equations.pdf}
\caption{\textbf{Maxwell's Equations from Categorical S-Dynamics.}
(A) Gauss's Law: electric field $\mathbf{E} = -\nabla \Phi_S$ as negative gradient of S-potential. Field lines radiate from charges (sources of S-potential).
(B) Amp\`ere's Law: magnetic field $\mathbf{B} = \nabla \times \mathbf{A}_S$ as curl of S-vector potential. Field lines form closed loops around current (S-flow).
(C) Coupled E-B oscillation: electromagnetic wave consists of perpendicular E and B fields oscillating 90° out of phase, propagating through S-space.
(D) Speed of light from S-dynamics: wave equation $\nabla^2 \mathbf{E} = \mu_0 \varepsilon_0 \partial^2\mathbf{E}/\partial t^2$ gives $c = 1/\sqrt{\mu_0\varepsilon_0} = 299{,}792{,}458$ m/s as the S-transformation rate in vacuum.}
\label{fig:maxwell_equations}
\end{figure}

%==============================================================================
% Figure C-7: Scattering Apertures
%==============================================================================
\begin{figure}[htbp]
\centering
\includegraphics[width=\textwidth]{figures/panel_scattering_apertures.pdf}
\caption{\textbf{Lattice Scattering as Categorical Apertures.}
(A) Apertures in $k$-space: Fermi surface (blue) with scattering apertures (red/orange wedges) that block certain momentum states. Incoming electron (green arrow) may be scattered or transmitted.
(B) Scattering types and selectivities: phonon scattering ($s \sim 0.1$, $T$-dependent), impurity scattering ($s \sim 0.01$, $T$-independent), electron-electron ($s \sim 0.5$, $\propto T^2$), grain boundary ($s \sim 0.001$), surface scattering.
(C) Mean free path from aperture density: $\lambda = 1/(n_a \sigma)$ where $n_a$ is scatterer density and $\sigma$ is cross-section. Cu ($\lambda \sim 40$ nm) vs Fe ($\lambda \sim 5$ nm).
(D) Resistance as aperture barrier sum: each scattering center contributes an aperture barrier. Total resistance $R = \sum_a \Phi_a/I = (L/A)\sum_a 1/(s_a \tau_a)$.}
\label{fig:scattering_apertures}
\end{figure}

%==============================================================================
% Figure C-8/9: Temperature and Superconductivity
%==============================================================================
\begin{figure}[htbp]
\centering
\includegraphics[width=\textwidth]{figures/panel_temperature_superconductivity.pdf}
\caption{\textbf{Temperature Dependence, Superconductivity, and Skin Effect.}
(A) Matthiessen's Rule: total resistivity $\rho = \rho_0 + \rho_{ph}(T)$ is sum of impurity (constant) and phonon (linear in $T$) contributions---additive aperture barriers.
(B) Superconducting transition: resistivity drops to zero at critical temperature $T_c$ as Cooper pairs form, bypassing scattering apertures entirely.
(C) Cooper pairs as aperture bypass: normal state electrons are blocked by scattering apertures; below $T_c$, paired electrons form a coherent state that tunnels through barriers with zero resistance.
(D) Skin effect: at high frequency, current is confined to surface layer of depth $\delta = \sqrt{2/(\mu_0\sigma\omega)}$. This reflects frequency-dependent aperture selectivity $s(\omega)$ as scattering time becomes comparable to oscillation period.}
\label{fig:temperature_superconductivity}
\end{figure}

%==============================================================================
% Figure C-10: Cross-Sectional Validation
%==============================================================================
\begin{figure}[htbp]
\centering
\includegraphics[width=\textwidth]{figures/panel_current_cross_sectional_validation.pdf}
\caption{\textbf{Cross-Sectional Validation of S-Transformation in Current Flow.}
(A) S-coordinate evolution along wire: configuration entropy $S_k$ and evolution entropy $S_e$ are plotted versus position for three conductors (Copper, Aluminum, Tungsten). Each point represents a cross-sectional measurement. Copper (long scattering time, 25 fs) shows minimal S-variation; Tungsten (short scattering time, 5 fs) shows maximum variation. The S-sliding window traverses the wire from left ($V^+$) to right ($V^-$), measuring categorical states at each cross-section.
(B) Transformation validation: predicted S-coordinates from the S-transformation $\mathcal{T}_{dx}$ versus directly computed values. Points cluster near the identity line, demonstrating that the transformation correctly predicts next-section S-coordinates from previous-section values.
(C) Electric field profile: constant along uniform wire for all materials, confirming that the local electric field $E = V/L$ is uniform. Non-uniformity would indicate defects or inhomogeneities.
(D) Cumulative resistance (Ohm's Law): resistance accumulates linearly with position, validating $R = \rho L/A$. The slope directly measures resistivity, with Tungsten (highest $\rho$) showing steepest accumulation.
(E) Scattering memory accumulation: analogous to viscosity in fluids, the ``scattering memory'' $\mathcal{M} = \int \tau_s g_{\text{lat}} |d\vec{S}| dx$ tracks the accumulated partition lag along the wire. Materials with longer scattering times (Copper) accumulate more memory. This quantity measures the ``history'' of electron-lattice interactions---the current-flow equivalent of fluid viscosity.
(F) Newton's Cradle schematic: the wire is a chain of electrons (blue circles) between positive terminals. Cross-sections (red dashed lines at $x_1, x_2, x_3, x_4$) measure local S-coordinates. The S-transformation $\vec{S}(x_{i+1}) = \mathcal{T}_{dx}[\vec{S}(x_i)]$ propagates categorical states from section to section, demonstrating that current is not electron flow but state propagation---the Newton's cradle paradigm validated by cross-sectional measurement.}
\label{fig:current_cross_sectional_validation}
\end{figure}

