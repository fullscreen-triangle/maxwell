\section{Solution as Recurrent Trajectory}
\label{sec:solution_trajectory}

This section establishes the central result: computational solutions correspond to recurrent trajectories in S-entropy space. We prove that Poincar\'{e} recurrence guarantees solution existence and derive bounds on recurrence time.

\subsection{Recurrence in Categorical Space}

\begin{theorem}[Poincar\'{e} Recurrence in $\Sspace$]
\label{thm:poincare_recurrence}
Let $(\Sspace, \mathcal{B}(\Sspace), \mu)$ be the S-entropy measure space and $\varphi_t: \Sspace \to \Sspace$ the flow generated by measure-preserving dynamics satisfying the modular condition \eqref{eq:modular_condition}. Then for $\mu$-almost every $\Scoord_0 \in \Sspace$ and every $\epsilon > 0$:
\begin{equation}
\liminf_{t \to \infty} \|\varphi_t(\Scoord_0) - \Scoord_0\| < \epsilon
\label{eq:recurrence_condition}
\end{equation}
\end{theorem}

\begin{proof}
This is a direct application of Poincar\'{e}'s recurrence theorem \citep{poincare1890probleme}. The conditions are:
\begin{enumerate}
    \item $\mu(\Sspace) = 1 < \infty$ (Theorem~\ref{thm:recurrence_preconditions})
    \item $\varphi_t$ is measure-preserving (Theorem~\ref{thm:measure_preservation})
\end{enumerate}
By the recurrence theorem, almost every orbit returns to any neighborhood of its initial point infinitely often.
\end{proof}

\subsection{Solution Definition}

\begin{definition}[Categorical Solution]
\label{def:categorical_solution}
For problem $P = (\Scoord_0, \mathcal{C}, \epsilon)$, a \textbf{solution} is a trajectory $\gamma: [0, T] \to \Sspace$ satisfying:
\begin{enumerate}
    \item Initial condition: $\gamma(0) = \Scoord_0$
    \item Recurrence: $\|\gamma(T) - \Scoord_0\| < \epsilon$
    \item Constraint satisfaction: $\mathcal{C}(\gamma) = \text{true}$
\end{enumerate}
The value $T$ is the \textbf{solution time}.
\end{definition}

\begin{theorem}[Solution-Recurrence Equivalence]
\label{thm:solution_recurrence}
A trajectory $\gamma$ solves problem $P = (\Scoord_0, \mathcal{C}, \epsilon)$ if and only if:
\begin{equation}
\gamma \text{ is } \epsilon\text{-recurrent from } \Scoord_0 \text{ and } \mathcal{C}(\gamma) = \text{true}
\end{equation}
\end{theorem}

\begin{proof}
Immediate from Definition~\ref{def:categorical_solution}.
\end{proof}

\subsection{Recurrence Time Bounds}

\begin{theorem}[Recurrence Time]
\label{thm:recurrence_time}
For the discretized phase space with $N = 3^k$ cells, the expected recurrence time satisfies:
\begin{equation}
\mathbb{E}[T_{\text{rec}}] = O(N) = O(3^k)
\end{equation}
\end{theorem}

\begin{proof}
Consider the discrete approximation where $\Sspace$ is partitioned into $N = 3^k$ cells of equal measure $\mu_i = 1/N$. By Kac's lemma \citep{kac1947notion}, the expected return time to a set $A$ is:
\begin{equation}
\mathbb{E}[\tau_A | \Scoord_0 \in A] = \frac{1}{\mu(A)}
\end{equation}
For a single cell with $\mu(A) = 1/N$:
\begin{equation}
\mathbb{E}[\tau] = N = 3^k
\end{equation}
\end{proof}

\begin{corollary}[Resolution-Time Tradeoff]
\label{cor:resolution_time}
For recurrence tolerance $\epsilon$, the expected recurrence time scales as:
\begin{equation}
\mathbb{E}[T_{\text{rec}}] = O\left( \epsilon^{-\log_3 e} \right) \approx O\left( \epsilon^{-0.91} \right)
\end{equation}
\end{corollary}

\begin{proof}
The tolerance $\epsilon$ corresponds to cell size $\epsilon \approx 3^{-k}$ in the hierarchical partition, giving $k \approx \log_3(1/\epsilon)$. The recurrence time is $O(3^k) = O(3^{\log_3(1/\epsilon)}) = O(1/\epsilon)$. The precise exponent is $\log_3 e \approx 0.91$.
\end{proof}

\subsection{Constraint Filtering}

Not all recurrent trajectories satisfy the constraints $\mathcal{C}$. We characterize the constraint-satisfying subset.

\begin{definition}[Admissible Trajectory Set]
\label{def:admissible_set}
The \textbf{admissible trajectory set} for problem $P = (\Scoord_0, \mathcal{C}, \epsilon)$ is:
\begin{equation}
\mathcal{A}(P) = \left\{ \gamma : \gamma(0) = \Scoord_0, \|\gamma(T) - \Scoord_0\| < \epsilon, \mathcal{C}(\gamma) = \text{true} \right\}
\end{equation}
\end{definition}

\begin{proposition}[Admissibility Measure]
\label{prop:admissibility_measure}
Let $\rho(\mathcal{C})$ denote the fraction of trajectories satisfying constraint $\mathcal{C}$. The expected number of recurrence attempts before finding an admissible trajectory is:
\begin{equation}
\mathbb{E}[N_{\text{attempts}}] = \frac{1}{\rho(\mathcal{C})}
\end{equation}
\end{proposition}

\begin{proof}
Each recurrence provides an independent Bernoulli trial with success probability $\rho(\mathcal{C})$. The expected number of trials until success is $1/\rho(\mathcal{C})$.
\end{proof}

\subsection{Solution Uniqueness}

\begin{theorem}[Solution Non-Uniqueness]
\label{thm:non_uniqueness}
For generic problems $P = (\Scoord_0, \mathcal{C}, \epsilon)$ with $\epsilon > 0$, the solution set $\mathcal{A}(P)$ contains uncountably many trajectories.
\end{theorem}

\begin{proof}
The recurrence condition $\|\gamma(T) - \Scoord_0\| < \epsilon$ defines an open ball of radius $\epsilon$ in $\Sspace$. For measure-preserving dynamics, trajectories returning to this ball do so from a set of positive measure in phase space. The set of initial velocities compatible with recurrence has positive measure, generating uncountably many distinct trajectories.
\end{proof}

\begin{corollary}[Solution Selection]
\label{cor:solution_selection}
Among multiple solutions, the solution with minimum recurrence time $T$ is distinguished:
\begin{equation}
\gamma^* = \arg\min_{\gamma \in \mathcal{A}(P)} T(\gamma)
\end{equation}
This is the \textbf{minimal solution}.
\end{corollary}

\subsection{Halting Equivalence}

\begin{theorem}[Recurrence-Halting Correspondence]
\label{thm:halting_correspondence}
For a computational problem encoded as $P = (\Scoord_0, \mathcal{C}, \epsilon)$:
\begin{equation}
\text{Problem } P \text{ has a solution} \iff \mathcal{A}(P) \neq \emptyset
\end{equation}
\end{theorem}

\begin{proof}
By definition, $P$ has a solution if and only if there exists an admissible trajectory, which is equivalent to $\mathcal{A}(P)$ being non-empty.
\end{proof}

\begin{remark}
The Poincar\'{e} recurrence theorem guarantees that recurrent trajectories exist (Theorem~\ref{thm:poincare_recurrence}), but does not guarantee that constraint-satisfying trajectories exist. The decidability of $\mathcal{A}(P) \neq \emptyset$ depends on the constraint structure $\mathcal{C}$.
\end{remark}

