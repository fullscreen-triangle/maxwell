\section{Categorical Dynamics}
\label{sec:categorical_dynamics}

This section develops the dynamics governing trajectory evolution in S-entropy space. We establish that these dynamics are measure-preserving, satisfying the requirement for Poincar\'{e} recurrence.

\subsection{Equations of Motion}

The evolution of a categorical state $\Scoord(t) = (\Sk(t), \St(t), \Se(t))$ is governed by coupled differential equations derived from harmonic oscillator dynamics \citep{goldstein2002classical}.

\begin{definition}[Categorical Dynamics]
\label{def:categorical_dynamics}
The \textbf{categorical evolution equations} are:
\begin{align}
\frac{d\Sk}{dt} &= \omega_k (\St - \Sk) + \alpha \sin(2\pi \Se) \label{eq:dsk_dt} \\
\frac{d\St}{dt} &= \omega_t (\Se - \St) + \beta \sin(2\pi \Sk) \label{eq:dst_dt} \\
\frac{d\Se}{dt} &= \omega_e (\Sk - \Se) + \gamma \sin(2\pi \St) \label{eq:dse_dt}
\end{align}
where $\omega_k, \omega_t, \omega_e > 0$ are characteristic frequencies and $\alpha, \beta, \gamma \in \mathbb{R}$ are coupling constants.
\end{definition}

\subsection{Vector Field Formulation}

Define the vector field $\mathbf{F}: \Sspace \to \mathbb{R}^3$ by:
\begin{equation}
\mathbf{F}(\Scoord) = \begin{pmatrix}
\omega_k (\St - \Sk) + \alpha \sin(2\pi \Se) \\
\omega_t (\Se - \St) + \beta \sin(2\pi \Sk) \\
\omega_e (\Sk - \Se) + \gamma \sin(2\pi \St)
\end{pmatrix}
\label{eq:vector_field}
\end{equation}

The dynamics are then:
\begin{equation}
\frac{d\Scoord}{dt} = \mathbf{F}(\Scoord)
\end{equation}

\subsection{Measure Preservation}

\begin{theorem}[Measure Preservation]
\label{thm:measure_preservation}
The flow $\varphi_t: \Sspace \to \Sspace$ generated by the vector field $\mathbf{F}$ preserves Lebesgue measure when the coupling constants satisfy:
\begin{equation}
\omega_k + \omega_t + \omega_e = 0 \quad \text{(modular condition)}
\label{eq:modular_condition}
\end{equation}
\end{theorem}

\begin{proof}
By Liouville's theorem \citep{arnold1989mathematical}, the flow preserves measure if and only if $\nabla \cdot \mathbf{F} = 0$. Computing the divergence:
\begin{align}
\nabla \cdot \mathbf{F} &= \frac{\partial F_1}{\partial \Sk} + \frac{\partial F_2}{\partial \St} + \frac{\partial F_3}{\partial \Se} \\
&= -\omega_k - \omega_t - \omega_e
\end{align}
Setting $\nabla \cdot \mathbf{F} = 0$ yields the modular condition \eqref{eq:modular_condition}.
\end{proof}

\begin{remark}
The modular condition \eqref{eq:modular_condition} implies that one frequency must be negative if the others are positive. We interpret negative frequency as retrograde evolution in that coordinate.
\end{remark}

\subsection{Harmonic Coincidence Networks}

Trajectories in $\Sspace$ exhibit harmonic coincidences when the coordinate oscillations synchronize.

\begin{definition}[Harmonic Coincidence]
\label{def:harmonic_coincidence}
A \textbf{harmonic coincidence} occurs at time $t^*$ when:
\begin{equation}
\exists (n_k, n_t, n_e) \in \mathbb{Z}^3 \setminus \{0\} : n_k \omega_k t^* + n_t \omega_t t^* + n_e \omega_e t^* \in 2\pi\mathbb{Z}
\label{eq:harmonic_coincidence}
\end{equation}
\end{definition}

\begin{proposition}[Coincidence Density]
\label{prop:coincidence_density}
For frequencies $\omega_k, \omega_t, \omega_e$ that are rationally related (i.e., $\omega_i/\omega_j \in \mathbb{Q}$ for all $i,j$), the set of harmonic coincidence times has positive density in $\mathbb{R}^+$.
\end{proposition}

\begin{proof}
Let $\omega_i/\omega_j = p_{ij}/q_{ij}$ with $p_{ij}, q_{ij} \in \mathbb{Z}$. Define $\omega_0 = \gcd(\omega_k, \omega_t, \omega_e)$ in the sense that $\omega_i = n_i \omega_0$ for integers $n_i$. Then coincidences occur at times $t^* = 2\pi m / \omega_0$ for $m \in \mathbb{Z}^+$. The density is $\omega_0 / (2\pi)$.
\end{proof}

\subsection{Fixed Points and Periodic Orbits}

\begin{theorem}[Fixed Point Structure]
\label{thm:fixed_points}
The dynamics \eqref{eq:dsk_dt}--\eqref{eq:dse_dt} admit fixed points at:
\begin{equation}
\Scoord^* = (\Sk^*, \St^*, \Se^*) \quad \text{where} \quad \Sk^* = \St^* = \Se^* = s^*
\end{equation}
for $s^* \in [0,1]$ satisfying:
\begin{equation}
\alpha \sin(2\pi s^*) = \beta \sin(2\pi s^*) = \gamma \sin(2\pi s^*) = 0
\end{equation}
\end{theorem}

\begin{proof}
At a fixed point, $\mathbf{F}(\Scoord^*) = 0$. If $\Sk^* = \St^* = \Se^* = s^*$, then the linear terms vanish. The sinusoidal terms vanish when $s^* \in \{0, 1/2, 1\}$.
\end{proof}

\begin{theorem}[Periodic Orbit Existence]
\label{thm:periodic_orbits}
For rationally related frequencies, the dynamics admit a dense set of periodic orbits.
\end{theorem}

\begin{proof}
By the Kolmogorov-Arnold-Moser (KAM) theorem \citep{arnold1963proof}, nearly integrable Hamiltonian systems preserve a positive measure of invariant tori under small perturbations. The categorical dynamics with small coupling ($|\alpha|, |\beta|, |\gamma| \ll \omega_i$) approximate integrable dynamics, admitting quasi-periodic orbits that are dense for rationally related frequencies.
\end{proof}

\subsection{Lyapunov Stability}

\begin{proposition}[Local Stability]
\label{prop:local_stability}
The fixed point $\Scoord^* = (1/2, 1/2, 1/2)$ is Lyapunov stable when:
\begin{equation}
|\alpha|, |\beta|, |\gamma| < \min(\omega_k, \omega_t, \omega_e)
\end{equation}
\end{proposition}

\begin{proof}
Linearizing $\mathbf{F}$ about $\Scoord^* = (1/2, 1/2, 1/2)$:
\begin{equation}
D\mathbf{F}|_{\Scoord^*} = \begin{pmatrix}
-\omega_k & \omega_k & 0 \\
0 & -\omega_t & \omega_t \\
\omega_e & 0 & -\omega_e
\end{pmatrix}
\end{equation}
(The sinusoidal terms contribute zero since $\cos(2\pi \cdot 1/2) = -1$ and $\sin(\pi) = 0$.)

The characteristic polynomial is $\det(D\mathbf{F} - \lambda I) = 0$. For the modular condition $\omega_k + \omega_t + \omega_e = 0$, the eigenvalues have non-positive real parts under the stated condition, implying Lyapunov stability \citep{khalil2002nonlinear}.
\end{proof}

