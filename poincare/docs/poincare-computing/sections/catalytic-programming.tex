%==============================================================================
\section{Catalytic Programming: Apertures as Computational Constraints}
\label{sec:catalytic}
%==============================================================================

The categorical framework admits a programming paradigm fundamentally distinct from instruction-based computation. Programs are not sequences of operations but \emph{catalytic structures}---partitions with apertures that constrain which categorical states can transition to which other states. The solution is the equilibrium state to which the gas dynamics converge.

\subsection{From Ball Game to Gas Dynamics}

Consider a partition with apertures separating two regions of categorical space. Let $n_A$ and $n_B$ denote the occupancy of each region, with total molecular count $N = n_A + n_B$ conserved. The apertures permit transitions only for molecules whose categorical configuration satisfies the geometric constraints.

\begin{definition}[Computational Aperture]
A \textbf{computational aperture} is a geometric constraint $\mathcal{A} \subset \Sspace$ that permits state transitions only for categorical states $\Scoord$ satisfying
\begin{equation}
\Scoord \in \mathcal{A} \implies \text{transition permitted}
\end{equation}
The aperture is defined by its geometric properties (size, shape, position in $\Sspace$), not by any temporal or velocity-dependent criterion.
\end{definition}

\begin{definition}[Catalytic Partition]
A \textbf{catalytic partition} is a hypersurface $\Pi \subset \Sspace$ containing one or more apertures $\{\mathcal{A}_1, \ldots, \mathcal{A}_m\}$ that divides the phase space into regions. The partition constrains categorical dynamics by permitting transitions only through apertures.
\end{definition}

\subsection{Programs as Catalyst Structures}

In this paradigm, a program is not a sequence of instructions but a configuration of catalytic partitions.

\begin{theorem}[Program-Catalyst Correspondence]
\label{thm:program_catalyst}
Let $\mathcal{P}$ be a computational problem with constraint set $\mathcal{C}$. There exists a catalytic structure $\{\Pi_1, \ldots, \Pi_n\}$ such that the gas dynamics on $\Sspace$ constrained by these partitions converge to an equilibrium state $\Scoord^*$ satisfying $\mathcal{C}$.
\end{theorem}

\begin{proof}
The constraint set $\mathcal{C}$ defines a subset $\mathcal{S}_{\mathcal{C}} \subset \Sspace$ of satisfying states. Construct partitions $\Pi_i$ such that:
\begin{enumerate}
    \item Apertures $\mathcal{A}_{ij} \subset \Pi_i$ permit transitions only toward $\mathcal{S}_{\mathcal{C}}$
    \item The partition configuration is conservative (total molecular count preserved)
    \item No aperture configuration permits ``complete victory'' (all molecules on one side)
\end{enumerate}

By the meaningless victory theorem, the dynamics must reach equilibrium. By construction of apertures, this equilibrium lies in $\mathcal{S}_{\mathcal{C}}$.
\end{proof}

\begin{corollary}[Declarative Programming]
Catalytic programming is inherently declarative: the programmer specifies aperture geometry (constraints), not execution sequence (instructions).
\end{corollary}

\subsection{The Equilibrium-Solution Equivalence}

The connection between catalysis and computation becomes precise through the identification of equilibrium with the penultimate state.

\begin{theorem}[Equilibrium-Penultimate Equivalence]
\label{thm:eq_penult}
The equilibrium state of a catalytic system and the penultimate state of Poincar\'e recurrence are mathematically identical: both are configurations exactly one categorical step from closure.
\end{theorem}

\begin{proof}
\textbf{Equilibrium characterization:} At equilibrium, the forward and reverse rates are equal:
\begin{equation}
\text{Rate}_{A \to B} = \text{Rate}_{B \to A}
\end{equation}
This occurs when neither side can achieve ``complete victory'' (Theorem~\ref{thm:meaningless_victory}), i.e., when both sides maintain non-zero occupancy.

\textbf{Penultimate state characterization:} The penultimate state $\Scoord_{\epsilon}$ satisfies $\|\Scoord_{\epsilon} - \Scoord_0\| < \epsilon$ with $\Scoord_{\epsilon} \neq \Scoord_0$. It is one categorical step from closure but cannot reach exact closure due to categorical irreversibility.

\textbf{Identification:} In both cases:
\begin{itemize}
    \item The system cannot reach ``completion'' (zero on one side / exact return)
    \item The system is dynamically stable (equal rates / trajectory closure)
    \item The configuration is recognized as the ``answer'' (equilibrium distribution / solution)
\end{itemize}

The mathematical structure is identical: a fixed point under the dynamics that is arbitrarily close to but distinct from the origin/completion.
\end{proof}

\subsection{Aperture Geometry as Problem Encoding}

The shape and position of apertures encode the problem to be solved.

\begin{definition}[Problem Encoding via Apertures]
A computational problem $P$ is encoded by a catalytic structure $\mathcal{K}(P) = \{\Pi_i, \mathcal{A}_{ij}\}$ where:
\begin{itemize}
    \item Each partition $\Pi_i$ corresponds to a constraint category in $P$
    \item Each aperture $\mathcal{A}_{ij}$ corresponds to a satisfying assignment for that constraint
    \item The aperture geometry (size, shape) encodes the constraint tightness
\end{itemize}
\end{definition}

\begin{proposition}[Aperture Size and Constraint Selectivity]
Smaller apertures correspond to tighter constraints:
\begin{equation}
|\mathcal{A}| \propto \frac{1}{\text{selectivity}}
\end{equation}
An infinitesimally small aperture represents an exactly-determined constraint; a large aperture represents a weakly-constrained variable.
\end{proposition}

\begin{theorem}[No Velocity Dependence]
\label{thm:no_velocity}
Aperture traversal depends only on categorical configuration, not on approach velocity, trajectory length, or temporal rate:
\begin{equation}
\Scoord \in \mathcal{A} \iff \text{traversal permitted, regardless of } \|\dot{\Scoord}\|
\end{equation}
\end{theorem}

\begin{proof}
Apertures are subsets of $\Sspace$ defined by geometric constraints on $(S_k, S_t, S_e)$. Velocity $\dot{\Scoord}$ is a tangent vector, not a point in $\Sspace$. Membership in $\mathcal{A}$ is determined by position, not velocity. Therefore aperture traversal is velocity-independent.
\end{proof}

\subsection{Conservation and Termination}

The conservation law explains why catalytic computation terminates (reaches equilibrium) rather than running indefinitely in one direction.

\begin{theorem}[Computational Termination from Conservation]
\label{thm:termination}
Let $N = n_A + n_B$ be the conserved molecular count across a catalytic partition. The computation terminates (reaches equilibrium) because:
\begin{enumerate}
    \item Complete conversion ($n_A = 0$ or $n_B = 0$) halts dynamics
    \item Halted dynamics force reversal
    \item Sustained dynamics require $n_A > 0$ and $n_B > 0$
    \item Equilibrium is the configuration satisfying (3) with equal rates
\end{enumerate}
\end{theorem}

\begin{proof}
Transition rates depend on occupancy:
\begin{align}
\text{Rate}_{A \to B} &= f(n_A, \mathcal{A}) \\
\text{Rate}_{B \to A} &= g(n_B, \mathcal{A}')
\end{align}
where $f, g$ are monotonic in occupancy.

If $n_A \to 0$: $\text{Rate}_{A \to B} \to 0$, dynamics halt in forward direction.

If $n_B \to 0$: $\text{Rate}_{B \to A} \to 0$, dynamics halt in reverse direction.

For sustained dynamics: $n_A > 0$ and $n_B > 0$.

By continuity, there exists a configuration $(n_A^*, n_B^*)$ with $n_A^* + n_B^* = N$ where $f(n_A^*, \mathcal{A}) = g(n_B^*, \mathcal{A}')$. This is the equilibrium = solution.
\end{proof}

\subsection{Multi-Partition Computation}

Complex computations involve multiple catalytic partitions in sequence or parallel.

\begin{definition}[Catalytic Network]
A \textbf{catalytic network} is a collection of partitions $\{\Pi_1, \ldots, \Pi_n\}$ with apertures that create a multi-compartment phase space. The network topology---which compartments connect to which via apertures---encodes the computational structure.
\end{definition}

\begin{proposition}[Serial Computation]
Sequential constraints are encoded by serial partition arrangement:
\begin{equation}
A \xrightarrow{\mathcal{A}_1} B \xrightarrow{\mathcal{A}_2} C \xrightarrow{\mathcal{A}_3} D
\end{equation}
Each aperture $\mathcal{A}_i$ represents a constraint that must be satisfied before proceeding.
\end{proposition}

\begin{proposition}[Parallel Computation]
Independent constraints are encoded by parallel partition arrangement:
\begin{equation}
A \xrightarrow[\mathcal{A}_2]{\mathcal{A}_1} \begin{cases} B_1 \\ B_2 \end{cases}
\end{equation}
Multiple apertures from a single compartment permit simultaneous exploration of independent solution branches.
\end{proposition}

\begin{theorem}[Autocatalytic Acceleration]
\label{thm:autocatalytic}
Catalytic computation exhibits positive feedback: each successful aperture traversal reduces resistance to subsequent traversals.
\end{theorem}

\begin{proof}
From the ball game analysis: when molecules accumulate on the receiving side, they create ``coverage burden'' that reduces the probability of blocking subsequent transits. Mathematically, resistance $R$ satisfies:
\begin{equation}
\frac{dR}{dt} = -\frac{k}{n_B^2} \cdot \frac{dn_B}{dt} < 0
\end{equation}
Each transit increases $n_B$, which decreases $R$, facilitating the next transit. This is autocatalytic positive feedback.
\end{proof}

\subsection{Comparison with Traditional Programming}

\begin{table}[htbp]
\centering
\caption{Comparison of programming paradigms}
\label{tab:paradigm_comparison}
\begin{tabular}{lll}
\toprule
\textbf{Aspect} & \textbf{Instruction-Based} & \textbf{Catalytic} \\
\midrule
Program & Instruction sequence & Aperture geometry \\
Execution & Step-by-step & Gas dynamics \\
Control flow & Imperative & Emergent \\
Termination & Halt instruction & Equilibrium \\
Speed measure & Operations/time & Irrelevant \\
Memory & Separate storage & Same phase space \\
Solution & Computed output & Equilibrium state \\
\bottomrule
\end{tabular}
\end{table}

\begin{theorem}[Catalyst Ignorance]
\label{thm:catalyst_ignorance}
A catalytic program does not ``know'' the solution. The aperture geometry constrains pathways; the gas dynamics find equilibrium. The solution emerges from dynamics, not from encoding.
\end{theorem}

\begin{proof}
The catalytic structure $\mathcal{K}(P)$ encodes constraints, not solutions. The apertures specify \emph{which transitions are permitted}, not \emph{what the answer is}. The equilibrium state---the solution---is determined by the dynamics subject to these constraints. The catalyst is ignorant of the specific equilibrium configuration; it only defines the rules governing approach to equilibrium.
\end{proof}

\subsection{Le Chatelier and Problem Perturbation}

Le Chatelier's principle provides the mechanism for problem modification and incremental computation.

\begin{theorem}[Problem Perturbation via Le Chatelier]
\label{thm:perturbation}
Modifying a problem corresponds to perturbing the catalytic system, with the response governed by Le Chatelier's principle:
\begin{equation}
\Delta P \to \Delta \mathcal{K} \to \text{shift to new equilibrium}
\end{equation}
\end{theorem}

\begin{proof}
A problem modification $\Delta P$ corresponds to:
\begin{itemize}
    \item Adding molecules: increases pressure toward apertures
    \item Removing molecules: decreases pressure from affected regions
    \item Changing aperture geometry: modifies permitted transitions
\end{itemize}
In each case, Le Chatelier's principle predicts the direction of equilibrium shift. The new solution emerges from the new equilibrium.
\end{proof}

\begin{corollary}[Incremental Computation]
Small problem changes produce small solution changes. The system does not restart; it adjusts from the current equilibrium to the new equilibrium.
\end{corollary}

\subsection{Summary}

The catalytic programming paradigm transforms computation from instruction execution to equilibrium seeking:

\begin{enumerate}
    \item Programs are catalytic structures (partitions with apertures)
    \item Execution is gas dynamics constrained by apertures
    \item Solutions are equilibrium states (penultimate configurations)
    \item Termination follows from conservation (meaningless victory theorem)
    \item Speed is categorically irrelevant (no velocity dependence)
    \item Catalysts are ignorant of solutions (constraints, not answers)
    \item Autocatalysis provides positive feedback (resistance reduction)
    \item Le Chatelier governs problem modification (incremental adjustment)
\end{enumerate}

This paradigm is declarative (specify constraints, not procedures), geometric (aperture shape, not instruction sequence), and equilibrium-based (solutions emerge from dynamics, not computation).

