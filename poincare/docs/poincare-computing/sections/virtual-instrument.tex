\section{Hardware Grounding: Virtual Instrumentation}
\label{sec:virtual_instrument}

The S-entropy coordinate space is instantiated through measurements from hardware oscillators. This section establishes the mapping from physical timing measurements to categorical states.

\subsection{Oscillator Sources}

Modern computing hardware contains multiple oscillatory processes with characteristic frequencies \citep{hennessy2017computer}:

\begin{table}[h]
\centering
\begin{tabular}{lll}
\toprule
\textbf{Source} & \textbf{Frequency Range} & \textbf{Precision} \\
\midrule
CPU clock & $10^9$--$10^{10}$ Hz & $\sim 10^{-10}$ s \\
Memory bus & $10^9$ Hz & $\sim 10^{-9}$ s \\
PCIe clock & $10^8$ Hz & $\sim 10^{-8}$ s \\
USB timing & $10^6$--$10^9$ Hz & $\sim 10^{-9}$ s \\
Power supply ripple & $10^4$--$10^6$ Hz & $\sim 10^{-6}$ s \\
\bottomrule
\end{tabular}
\caption{Hardware oscillator sources and characteristic frequencies.}
\label{tab:oscillator_sources}
\end{table}

\subsection{Timing Measurement}

Let $t_{\text{ref}}$ denote a reference timestamp from a high-precision clock (e.g., CPU timestamp counter) and $t_{\text{local}}$ denote a local measurement. The timing difference is:
\begin{equation}
\delta_p = t_{\text{ref}} - t_{\text{local}}
\label{eq:timing_difference}
\end{equation}

The timing jitter $\delta_p$ exhibits stochastic variation due to hardware non-determinism. We model this as:
\begin{equation}
\delta_p = \bar{\delta} + \eta(t)
\end{equation}
where $\bar{\delta}$ is the mean offset and $\eta(t)$ is a zero-mean stochastic process.

\begin{proposition}[Jitter Boundedness]
\label{prop:jitter_boundedness}
For a stable hardware oscillator, the timing jitter satisfies:
\begin{equation}
|\eta(t)| \leq \eta_{\max} < \infty
\end{equation}
with probability 1, where $\eta_{\max}$ depends on the oscillator quality factor \citep{leeson1966simple}.
\end{proposition}

\subsection{Coordinate Mapping Functions}

The timing difference $\delta_p$ maps to S-entropy coordinates through three functions.

\begin{definition}[Knowledge Entropy Mapping]
\label{def:phi_k}
The \textbf{knowledge entropy coordinate} is:
\begin{equation}
\Sk = \phi_k(\delta_p) = \frac{1}{2}\left(1 + \tanh\left(\frac{\delta_p - \mu_\delta}{\sigma_\delta}\right)\right)
\label{eq:phi_k}
\end{equation}
where $\mu_\delta$ and $\sigma_\delta$ are the mean and standard deviation of the timing distribution.
\end{definition}

\begin{definition}[Temporal Entropy Mapping]
\label{def:phi_t}
The \textbf{temporal entropy coordinate} is:
\begin{equation}
\St = \phi_t(\delta_p) = \left| \sin\left( 2\pi f_{\text{ref}} \delta_p \right) \right|
\label{eq:phi_t}
\end{equation}
where $f_{\text{ref}}$ is the reference oscillator frequency.
\end{definition}

\begin{definition}[Evolution Entropy Mapping]
\label{def:phi_e}
The \textbf{evolution entropy coordinate} is:
\begin{equation}
\Se = \phi_e(\delta_p) = \frac{1}{2}\left(1 + \cos\left( 2\pi f_{\text{beat}} \delta_p \right)\right)
\label{eq:phi_e}
\end{equation}
where $f_{\text{beat}} = |f_{\text{ref}} - f_{\text{local}}|$ is the beat frequency between oscillators.
\end{definition}

\begin{proposition}[Range Preservation]
\label{prop:range_preservation}
The coordinate functions satisfy $\phi_k, \phi_t, \phi_e: \mathbb{R} \to [0,1]$, ensuring that all timing measurements map to valid S-entropy coordinates.
\end{proposition}

\begin{proof}
For $\phi_k$: the hyperbolic tangent has range $(-1, 1)$, so $\phi_k \in (0, 1) \subset [0,1]$.
For $\phi_t$: $|\sin(\cdot)| \in [0, 1]$.
For $\phi_e$: $\cos(\cdot) \in [-1, 1]$, so $\phi_e \in [0, 1]$.
\end{proof}

\subsection{Categorical State Construction}

Given a timing measurement $\delta_p$, the categorical state is:
\begin{equation}
\Scoord = \Phi(\delta_p) = \left( \phi_k(\delta_p), \phi_t(\delta_p), \phi_e(\delta_p) \right)
\label{eq:full_mapping}
\end{equation}

\begin{theorem}[Deterministic Mapping]
\label{thm:deterministic_mapping}
The mapping $\Phi: \mathbb{R} \to \Sspace$ is deterministic: identical timing measurements produce identical categorical states.
\end{theorem}

\begin{proof}
Each component function $\phi_k$, $\phi_t$, $\phi_e$ is a composition of elementary functions (arithmetic, trigonometric, hyperbolic). Elementary functions are deterministic. Therefore $\Phi$ is deterministic.
\end{proof}

\subsection{Spectrometer-State Identity}

A fundamental property of the virtual instrumentation is the identity between measurement apparatus and measured state.

\begin{theorem}[Spectrometer-State Identity]
\label{thm:spectrometer_identity}
Let $\mathcal{I}_\omega$ denote a virtual spectrometer tuned to frequency $\omega$. The categorical state $\Scoord_\omega$ measurable by $\mathcal{I}_\omega$ satisfies:
\begin{equation}
\Scoord_\omega = \Phi(\omega^{-1})
\end{equation}
The spectrometer configuration and the measurable state are encoded by the same point in $\Sspace$.
\end{theorem}

\begin{proof}
The spectrometer $\mathcal{I}_\omega$ is defined by its sensitivity to oscillations at frequency $\omega$. This sensitivity corresponds to a timing resolution $\Delta t = \omega^{-1}$. A measurement at this resolution produces timing difference $\delta_p \approx \omega^{-1}$. Applying $\Phi$ yields $\Scoord_\omega = \Phi(\omega^{-1})$.

The only states that produce non-null measurements on $\mathcal{I}_\omega$ are those with characteristic frequency $\omega$, which map to the same $\Scoord_\omega$. Therefore the spectrometer configuration and measurable state coincide.
\end{proof}

\begin{corollary}[Zero Backaction]
\label{cor:zero_backaction}
Measurement by $\mathcal{I}_\omega$ does not perturb the categorical state $\Scoord_\omega$.
\end{corollary}

\begin{proof}
Since $\mathcal{I}_\omega$ and $\Scoord_\omega$ are identical in $\Sspace$, measurement is identity morphism. Identity morphisms do not perturb their domain.
\end{proof}

