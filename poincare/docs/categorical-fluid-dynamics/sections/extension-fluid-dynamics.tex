%==============================================================================
% SECTION 8: EXTENSION TO GENERAL FLUID DYNAMICS
%==============================================================================

\section{Extension to General Fluid Dynamics}
\label{sec:extension}

\subsection{Beyond Chromatography: General Flow Systems}

The framework developed for chromatography extends to general fluid dynamics. The key structures—dimensional reduction, S-transformation, and partition lag—apply universally.

\begin{theorem}[Generalisation to Multi-Component Systems]
\label{thm:generalisation}
For a fluid with $N$ species, the S-transformation extends to:
\begin{equation}
\Svec_{\text{total}} = \bigoplus_{i=1}^{N} \Svec_i
\label{eq:multicomponent_s}
\end{equation}
where $\oplus$ denotes the direct sum in S-space.
\end{theorem}

\begin{proof}
Each species has independent S-coordinates. The total S-state is the collection of individual S-states. Transformations operate on each component according to species-specific parameters.
\end{proof}

\subsection{Turbulence from Partition Lag Amplification}

\begin{definition}[Partition Lag Spectrum]
\label{def:lag_spectrum}
The partition lag spectrum $P(\tau_p)$ is the distribution of partition lags across all molecular pairs in the fluid.
\end{definition}

\begin{theorem}[Turbulence Criterion]
\label{thm:turbulence}
Turbulent flow occurs when the partition lag spectrum satisfies:
\begin{equation}
\frac{\max(\tau_p)}{\min(\tau_p)} > \text{Re}_c
\label{eq:turbulence_criterion}
\end{equation}
where $\text{Re}_c$ is the critical Reynolds number.
\end{theorem}

\begin{proof}
The Reynolds number measures inertial forces relative to viscous forces:
\begin{equation}
\text{Re} = \frac{\rho u L}{\mu}
\end{equation}

Viscosity $\mu = \sum \tau_{p,ij} g_{ij}$ (Theorem~\ref{thm:navier_stokes}). When partition lag variation is large, viscosity fluctuates spatially. Regions of low viscosity (short $\tau_p$) permit rapid S-changes; regions of high viscosity (long $\tau_p$) resist S-changes.

The ratio $\max(\tau_p) / \min(\tau_p)$ quantifies this fluctuation. When fluctuations exceed the critical ratio, laminar flow destabilises. Energy cascades through scales as partition lags amplify.
\end{proof}

\subsection{Boundary Layers from S-Gradient Steepening}

\begin{definition}[S-Gradient]
\label{def:s_gradient}
The S-gradient at position $\mathbf{x}$ is:
\begin{equation}
\nabla_S = \frac{\partial \Svec}{\partial \mathbf{x}}
\label{eq:s_gradient}
\end{equation}
\end{definition}

\begin{theorem}[Boundary Layer Formation]
\label{thm:boundary_layer}
A boundary layer forms where:
\begin{equation}
\|\nabla_S\| > \frac{1}{\delta}
\label{eq:boundary_layer_criterion}
\end{equation}
where $\delta$ is the boundary layer thickness.
\end{theorem}

\begin{proof}
At a solid boundary, the no-slip condition requires $\mathbf{v} = 0$. In the bulk, $\mathbf{v} \neq 0$. The velocity gradient creates an S-gradient:
\begin{equation}
\nabla_S = \frac{\partial \Svec}{\partial \mathbf{v}} \cdot \frac{\partial \mathbf{v}}{\partial y} \propto \frac{v_{\infty}}{\delta}
\end{equation}

A large S-gradient implies a rapid S-change over a short distance. The layer where this transition occurs is the boundary layer. Its thickness $\delta$ is determined by the balance between advection and diffusion:
\begin{equation}
\delta \sim \sqrt{\frac{\mu L}{\rho u}} = \frac{L}{\sqrt{\text{Re}}}
\end{equation}
\end{proof}

\begin{figure}[htbp]
\centering
\includegraphics[width=\textwidth]{figures/panel_extension.pdf}
\caption{\textbf{Extension to Complex Fluid Phenomena: Beyond Newtonian Flow.}
The aperture-memory framework extends naturally to complex fluids and flow regimes. (A) Turbulence as aperture hierarchy: at large scales, inertia opens non-sequential apertures; energy cascades to smaller scales where viscous (sequential) apertures dissipate kinetic energy. The Kolmogorov scale $\ell_K = (\nu^3/\epsilon)^{1/4}$ marks the transition from non-sequential to sequential aperture dominance. (B) Non-Newtonian fluids: strain-rate-dependent aperture reconfiguration produces shear-dependent viscosity. Shear-thinning (polymer solutions): high shear aligns molecules, reducing aperture resistance. Shear-thickening (cornstarch): high shear creates new apertures through jamming. (C) Multiphase flows: phase boundaries are categorical apertures with distinct selectivities. Liquid-gas interfaces have high selectivity (surface tension); liquid-liquid interfaces have intermediate selectivity (immiscibility). Coalescence and breakup are aperture creation/destruction events. (D) Reactive flows: chemical reactions are partition operations that transform S-coordinates. Combustion creates new apertures (products have different phase-lock networks than reactants); the reaction rate is the aperture traversal rate.}
\label{fig:extension}
\end{figure}

\subsection{Phase Transitions from S-Space Topology Change}

\begin{definition}[S-Space Topology]
\label{def:s_topology}
The S-space topology of a fluid is the structure of its S-landscape $\Phi_S$: the locations and types of critical points (minima, maxima, and saddles).
\end{definition}

\begin{theorem}[Phase Transition as Topological Change]
\label{thm:phase_transition}
A phase transition occurs when the S-space topology changes:
\begin{equation}
\chi(\Sspace_{\text{before}}) \neq \chi(\Sspace_{\text{after}})
\label{eq:topology_change}
\end{equation}
where $\chi$ is the Euler characteristic.
\end{theorem}

\begin{proof}
Before the transition, the system occupies the S-space region $\Sspace_1$. After transition, it occupies $\Sspace_2$. Phase boundaries are S-space barriers.

At the transition, the barrier between $\Sspace_1$ and $\Sspace_2$ either disappears or appears. This change alters the connectivity of accessible S-space. A change in connectivity implies a change in the Euler characteristic.
\end{proof}

\subsection{Mixtures and Solutions}

\begin{definition}[Mixture S-State]
\label{def:mixture_s}
A mixture of components $\{a, b, \ldots\}$ with mole fractions $\{x_a, x_b, \ldots\}$ has S-state:
\begin{equation}
\Svec_{\text{mix}} = \sum_i x_i \Svec_i + \Svec_{\text{mixing}}
\label{eq:mixture_s}
\end{equation}
where $\Svec_{\text{mixing}}$ is the entropy of the mixing contribution.
\end{definition}

\begin{theorem}[Entropy of Mixing in S-Coordinates]
\label{thm:mixing_entropy}
The mixing contribution is:
\begin{equation}
\Svec_{\text{mixing}} = (0, 0, -\sum_i x_i \ln x_i)
\label{eq:mixing_s}
\end{equation}
\end{theorem}

\begin{proof}
Mixing increases entropy but does not change the knowledge deficit or temporal position. The entropy increase is the configurational entropy:
\begin{equation}
\Delta S_{\text{mix}} = -\kB \sum_i x_i \ln x_i
\end{equation}

This contributes only to $S_e$ (entropy dimension):
\begin{equation}
\Svec_{\text{mixing}} = (0, 0, -\sum_i x_i \ln x_i)
\end{equation}
\end{proof}

\subsection{Heat Transfer}

\begin{theorem}[Heat Conduction in S-Coordinates]
\label{thm:heat_conduction}
Fourier's law of heat conduction:
\begin{equation}
\mathbf{q} = -k \nabla T
\label{eq:fourier}
\end{equation}
emerges from S-coordinate dynamics with thermal conductivity:
\begin{equation}
k = \frac{\kB}{m} \sum_{i,j} \omega_{ij} g_{ij}
\label{eq:thermal_conductivity}
\end{equation}
where $\omega_{ij}$ are oscillation frequencies and $g_{ij}$ are phase-lock couplings.
\end{theorem}

\begin{proof}
Heat flow results from energy transfer through a phase-lock network. The energy transfer rate is proportional to:
\begin{enumerate}
\item Oscillation frequency $\omega_{ij}$ (how fast modes oscillate)
\item Coupling strength $g_{ij}$ (how strongly modes interact)
\end{enumerate}

The sum $\sum \omega_{ij} g_{ij}$ gives the total energy transfer capacity. Dividing by molecular mass $m$ and multiplying by $\kB$ gives thermal conductivity.

The temperature  gradient drives energy flow from high-$T$ to low-$T$ regions. Flow rate proportional to gradient and conductivity gives Fourier's law.
\end{proof}

\subsection{Mass Transfer}

\begin{theorem}[Fick's Law in S-Coordinates]
\label{thm:ficks_law}
Fick's law of diffusion:
\begin{equation}
\mathbf{J} = -D \nabla c
\label{eq:fick}
\end{equation}
emerges from S-coordinate dynamics with diffusivity:
\begin{equation}
D = \frac{\kB T}{\sum_{i,j} g_{ij}}
\label{eq:diffusivity}
\end{equation}
\end{theorem}

\begin{proof}
Mass diffusion results from molecular random walks constrained by a phase-lock network. Diffusivity is:
\begin{equation}
D = \frac{\langle r^2 \rangle}{6t}
\end{equation}

Mean squared displacement $\langle r^2 \rangle \propto \kB T$ (thermal energy) and $t \propto \sum g_{ij}$ (constraint from the network). Thus:
\begin{equation}
D = \frac{\kB T}{\sum g_{ij}}
\end{equation}

The concentration gradient drives mass flow from high-$c$ to low-$c$ regions. The flow rate proportional to the gradient and diffusivity gives Fick's law.
\end{proof}

\subsection{Experimental Validation Protocol}

\begin{definition}[S-Coordinate Measurement Protocol]
\label{def:measurement_protocol}
S-coordinates are measured by:
\begin{enumerate}
\item Mass spectrometry Measures $S_k$ through mass-to-charge distribution
\item Retention time Measures $S_t$ through temporal position in separation
\item Calorimetry measures $S_e$ through heat capacity
\end{enumerate}
\end{definition}

\begin{theorem}[Experimental Validation Requirements]
\label{thm:validation_requirements}
The framework is validated when:
\begin{enumerate}
\item Retention time predictions match measurements: $|t_{R,\text{pred}} - t_{R,\text{obs}}| / t_{R,\text{obs}} < \epsilon_t$
\item Van Deemter coefficients match fits: $|A_{\text{pred}} - A_{\text{fit}}| / A_{\text{fit}} < \epsilon_A$; similarly for $B$ and $C$
\item Cross-platform consistency: S-coordinates computed from different platforms agree
\end{enumerate}
\end{theorem}

\begin{proof}
Each requirement tests a different aspect:
\begin{enumerate}
\item Retention time tests the S-transformation integral (Theorem~\ref{thm:retention_integral})
\item Van Deemter coefficients test the partition lag derivation (Theorem~\ref{thm:vandeemter})
\item Cross-platform consistency tests the categorical invariance of S-coordinates
\end{enumerate}

Validation requires all three to ensure the framework is physically accurate, not just mathematically self-consistent.
\end{proof}

