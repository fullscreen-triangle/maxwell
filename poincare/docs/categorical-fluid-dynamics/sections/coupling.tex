%==============================================================================
% SECTION: MOLECULAR COUPLING IN FLUIDS
%==============================================================================

\section{Molecular Coupling and Phase-Lock Networks}
\label{sec:coupling}

\subsection{Phase-Lock Coupling Strength}

\begin{definition}[Molecular Coupling Strength]
\label{def:molecular_coupling}
The coupling strength $g_{ij}$ between molecules $i$ and $j$ quantifies the phase-lock relationship through intermolecular forces:
\begin{equation}
g_{ij} = \left| \frac{\partial U_{ij}}{\partial r_{ij}} \right|_{r = r_{\text{eq}}}
\label{eq:molecular_coupling}
\end{equation}
where $U_{ij}$ is the pair potential and $r_{\text{eq}}$ is the equilibrium separation.
\end{definition}

\begin{theorem}[Coupling from Lennard-Jones Potential]
\label{thm:lj_coupling}
For Lennard-Jones interaction $U(r) = 4\varepsilon[(\sigma/r)^{12} - (\sigma/r)^6]$:
\begin{equation}
g_{ij} = \frac{72\varepsilon}{\sigma} \quad \text{at } r = 2^{1/6}\sigma
\label{eq:lj_coupling}
\end{equation}
\end{theorem}

\begin{proof}
The force is:
\begin{equation}
F = -\frac{dU}{dr} = 4\varepsilon \left[ \frac{12\sigma^{12}}{r^{13}} - \frac{6\sigma^6}{r^7} \right]
\end{equation}

At equilibrium $r_{\text{eq}} = 2^{1/6}\sigma$:
\begin{equation}
g = \left| \frac{dF}{dr} \right|_{r=r_{\text{eq}}} = 4\varepsilon \left[ \frac{156\sigma^{12}}{r^{14}} - \frac{42\sigma^6}{r^8} \right]_{r=2^{1/6}\sigma}
\end{equation}

Evaluating: $g = 72\varepsilon/\sigma$. \qed
\end{proof}

\begin{figure}[htbp]
\centering
\includegraphics[width=\textwidth]{figures/panel_coupling_networks.pdf}
\caption{\textbf{Phase-Lock Networks: The Molecular Basis of Viscosity.}
Fluid molecules are coupled pendulums, their oscillatory phases locked through intermolecular forces. These phase-locks form networks that encode viscosity as accumulated memory. (A) Network structure visualisation: nodes = molecules, edges = phase-lock couplings. Node colour indicates degree (number of connections); edge thickness indicates coupling strength. Dense networks (liquids) have high viscosity; sparse networks (gases) have low viscosity. (B) Coupling strength vs distance: Van der Waals ($g \sim r^{-6}$), dipole-dipole ($g \sim r^{-3}$), and hydrogen bonds (short-range exponential) show distinct decay profiles. Strong, short-range couplings create rigid networks; weak, long-range couplings create flexible networks. (C) Network density and phase: gas ($\rho_G < 0.2$, sparse network), liquid ($0.2 < \rho_G < 0.6$, percolating network), solid ($\rho_G > 0.6$, dense lattice). Phase transitions correspond to network percolation thresholds. (D) Transport as network navigation: molecular transport (highlighted path) proceeds through the phase-lock network, breaking and reforming connections. The viscosity $\mu = \sum \tau_p \cdot g$ is the total memory cost of all phase-lock reconfigurations along the path.}
\label{fig:coupling_networks}
\end{figure}

\subsection{Types of Molecular Coupling}

\begin{definition}[Van der Waals Coupling]
\label{def:vdw_coupling}
Van der Waals coupling arises from induced dipole-dipole interactions:
\begin{equation}
g_{\text{vdW}} = \frac{3\alpha^2 I}{4r^7}
\label{eq:vdw_coupling}
\end{equation}
where $\alpha$ is polarisability and $I$ is ionisation energy.
\end{definition}

\begin{definition}[Dipole-Dipole Coupling]
\label{def:dipole_coupling}
For permanent dipoles $\mu_1, \mu_2$:
\begin{equation}
g_{\text{dip}} = \frac{2\mu_1^2 \mu_2^2}{3\kB T r^7}
\label{eq:dipole_coupling}
\end{equation}
\end{definition}

\begin{definition}[Hydrogen Bond Coupling]
\label{def:hbond_coupling}
Hydrogen bonding produces strong directional coupling:
\begin{equation}
g_{\text{H-bond}} \approx 20 \text{ kJ/mol/\AA}
\label{eq:hbond_coupling}
\end{equation}
This is $\sim 10 \times$ stronger than typical Van der Waals coupling.
\end{definition}

\subsection{Phase-Lock Network Structure}

\begin{definition}[Molecular Phase-Lock Network]
\label{def:molecular_network}
Molecules in a fluid form a phase-lock network $\phaselockgraph = (V, E)$ where:
\begin{itemize}
\item Vertices $V$: molecular oscillatory modes
\item Edges $E$: phase-lock relationships through intermolecular forces
\end{itemize}
Edge weight $g_{ij}$ quantifies coupling strength.
\end{definition}

\begin{theorem}[Network Density in Liquids]
\label{thm:liquid_network_density}
Liquids have dense phase-lock networks:
\begin{equation}
\rho_{\phaselockgraph} = \frac{|E|}{|V|(|V|-1)/2} \approx 0.4 \text{ to } 0.6
\label{eq:liquid_density}
\end{equation}
Gases have sparse networks ($\rho_{\phaselockgraph} < 0.01$).
\end{theorem}

\begin{proof}
In liquids, each molecule has $\sim 10-12$ nearest neighbours within interaction range. For $N$ molecules:
\begin{equation}
|E| \approx \frac{N \cdot 10}{2} = 5N
\end{equation}

Network density:
\begin{equation}
\rho_{\phaselockgraph} = \frac{5N}{N(N-1)/2} \approx \frac{10}{N-1}
\end{equation}

For local neighbourhood ($N \sim 20$): $\rho_{\phaselockgraph} \approx 0.5$.

In gases, mean free path $\lambda \gg$ molecular diameter $\sigma$. Molecules interact only during rare collisions: $\rho_{\phaselockgraph} < 0.01$. \qed
\end{proof}

\subsection{Total Coupling Strength}

\begin{definition}[Coupling Matrix]
\label{def:coupling_matrix_fluid}
The coupling matrix $\mathbf{G}$ has elements:
\begin{equation}
G_{ij} = g_{ij} \quad \text{if } r_{ij} < r_{\text{cutoff}}
\label{eq:coupling_matrix_fluid}
\end{equation}
and $G_{ij} = 0$ otherwise.
\end{definition}

\begin{theorem}[Total Coupling and Cohesion]
\label{thm:total_coupling_fluid}
The total coupling strength determines cohesive energy:
\begin{equation}
E_{\text{coh}} = -\frac{1}{2} \sum_{i,j} g_{ij} r_{ij}^{\text{eq}}
\label{eq:cohesive_energy}
\end{equation}
\end{theorem}

\begin{proof}
Cohesive energy is the sum of pair interactions:
\begin{equation}
E_{\text{coh}} = \frac{1}{2} \sum_{i \neq j} U_{ij}
\end{equation}

Near equilibrium, $U_{ij} \approx -g_{ij} r_{ij}^{\text{eq}}$ (harmonic approximation). \qed
\end{proof}

\subsection{Coupling and Transport}

\begin{theorem}[Viscosity from Coupling]
\label{thm:viscosity_coupling}
Dynamic viscosity depends on molecular coupling:
\begin{equation}
\mu \propto \sum_{i,j} \tau_{p,ij} \cdot g_{ij}
\label{eq:viscosity_coupling}
\end{equation}
Strong coupling produces high viscosity.
\end{theorem}

\begin{proof}
Momentum transfer between fluid layers requires molecular collisions. Each collision transfers momentum proportional to coupling strength $g_{ij}$. The rate of momentum transfer depends on collision frequency $1/\tau_{p,ij}$.

The shear stress is:
\begin{equation}
\sigma_{xy} \propto \sum_{i,j} g_{ij} \cdot \frac{1}{\tau_{p,ij}} \cdot \delta v
\end{equation}

Viscosity $\mu = \sigma_{xy}/(\partial v/\partial y)$ gives:
\begin{equation}
\mu \propto \sum_{i,j} \frac{g_{ij}}{\tau_{p,ij}}
\end{equation}

For uniform partition lag, this becomes $\mu \propto \sum g_{ij} / \tau_p$. \qed
\end{proof}

\begin{remark}
This explains why:
\begin{itemize}
\item Honey (strong H-bonding, high $g_{ij}$) has high viscosity
\item Helium (weak Van der Waals, low $g_{ij}$) has low viscosity
\item Viscosity typically decreases with temperature (faster collisions, lower $\tau_p$)
\end{itemize}
\end{remark}

\subsection{Molecular Bonds as Categorical Apertures}

The coupling framework acquires additional structure when molecular bonds are recognised as \emph{apertures}—geometric constraints that selectively allow certain molecular configurations to pass while blocking others.

\begin{definition}[Molecular Aperture]
\label{def:molecular_aperture}
A molecular bond creates an \emph{aperture} with selection function:
\begin{equation}
\sigma_{\text{bond}}(\omega) = \begin{cases} 
1 & \text{if } \text{config}(\omega) \in \text{bonding geometry} \\ 
0 & \text{otherwise} 
\end{cases}
\end{equation}
Only molecular configurations compatible with the bond geometry can pass (remain bonded).
\end{definition}

\begin{theorem}[Bond Energy as Aperture Potential]
\label{thm:bond_aperture}
The bond energy is the categorical potential of the aperture:
\begin{equation}
E_{\text{bond}} = \Phi_{\text{aperture}} = -\kB T \ln s
\end{equation}
where $s = \Omega_{\text{bonded}}/\Omega_{\text{total}}$ is the selectivity—the fraction of configurations compatible with bonding.
\end{theorem}

\begin{proof}
A bond restricts the configuration space from $\Omega_{\text{total}}$ (all possible orientations, distances, vibrations) to $\Omega_{\text{bonded}}$ (configurations satisfying bond geometry). 

The entropy reduction upon bonding is:
\begin{equation}
\Delta S_{\text{bond}} = \kB \ln\Omega_{\text{bonded}} - \kB \ln\Omega_{\text{total}} = \kB \ln s < 0
\end{equation}

The free energy cost of this restriction is:
\begin{equation}
\Delta G_{\text{bond}} = -T \Delta S_{\text{bond}} = -\kB T \ln s = \Phi_{\text{aperture}}
\end{equation}

Strong bonds (large $|E_{\text{bond}}|$) correspond to small $s$ (high selectivity). \qed
\end{proof}

\begin{corollary}[Selectivity Ordering]
\label{cor:selectivity_ordering}
Different bond types have characteristic selectivities:
\begin{center}
\begin{tabular}{lcc}
\toprule
\textbf{Bond Type} & \textbf{Selectivity $s$} & \textbf{Potential $\Phi$} \\
\midrule
Covalent & $10^{-6}$ & Very high ($\sim$400 kJ/mol) \\
Ionic & $10^{-5}$ & High ($\sim$300 kJ/mol) \\
Hydrogen bond & $10^{-3}$ & Medium ($\sim$20 kJ/mol) \\
Van der Waals & $10^{-1}$ & Low ($\sim$2 kJ/mol) \\
No bond & 1 & Zero \\
\bottomrule
\end{tabular}
\end{center}
\end{corollary}

\begin{theorem}[Transport as Aperture Navigation]
\label{thm:transport_aperture}
Molecular transport through a fluid is navigation through an aperture network:
\begin{equation}
\text{Transport rate} \propto \prod_{\text{apertures}} s_a
\end{equation}
where the product is over all apertures encountered along the transport path.
\end{theorem}

\begin{proof}
Each aperture along a transport path has selectivity $s_a$. A molecule passes through aperture $a$ with probability proportional to $s_a$. For independent apertures, passage probabilities multiply:
\begin{equation}
P_{\text{transport}} \propto \prod_a s_a
\end{equation}

Low-selectivity paths (high $\prod s_a$) have high transport rates; high-selectivity paths (low $\prod s_a$) are transport barriers. \qed
\end{proof}

\begin{remark}[Viscosity from Aperture Resistance]
Viscosity can now be understood as aperture resistance:
\begin{equation}
\mu \propto \sum_{i,j} \frac{\Phi_{ij}}{\tau_{p,ij}} = -\kB T \sum_{i,j} \frac{\ln s_{ij}}{\tau_{p,ij}}
\end{equation}

High viscosity arises from:
\begin{itemize}
\item Low selectivity $s_{ij}$ (strong bonds, highly constrained configurations)
\item Long partition lag $\tau_{p,ij}$ (slow molecular rearrangement)
\end{itemize}

This explains why breaking bonds (increasing $s$) reduces viscosity, and why heating (reducing $\tau_p$) also reduces viscosity.
\end{remark}

\begin{theorem}[Phase Transitions as Aperture Creation/Destruction]
\label{thm:phase_transition_aperture}
Phase transitions correspond to large-scale aperture changes:
\begin{enumerate}
\item \textbf{Melting}: Destroys lattice apertures (releases $\Delta H_{\text{fusion}}$)
\item \textbf{Vaporisation}: Destroys liquid apertures (releases $\Delta H_{\text{vap}}$)
\item \textbf{Condensation}: Creates liquid apertures (absorbs $\Delta H_{\text{vap}}$)
\item \textbf{Freezing}: Creates lattice apertures (absorbs $\Delta H_{\text{fusion}}$)
\end{enumerate}
\end{theorem}

\begin{proof}
\textbf{Melting}: A crystal lattice has high-selectivity apertures (atoms must occupy specific sites). Melting destroys these apertures, releasing the stored categorical potential as latent heat:
\begin{equation}
\Delta H_{\text{fusion}} = \sum_{\text{lattice}} \Phi_{\text{aperture}} = -\kB T \sum \ln s_{\text{lattice}}
\end{equation}

\textbf{Vaporisation}: Liquid molecules have medium-selectivity apertures (intermolecular bonds). Vaporisation destroys these:
\begin{equation}
\Delta H_{\text{vap}} = \sum_{\text{liquid}} \Phi_{\text{aperture}} = -\kB T \sum \ln s_{\text{liquid}}
\end{equation}

Reverse transitions (condensation, freezing) create apertures and absorb energy. \qed
\end{proof}

\begin{remark}[Connection to Chromatography]
In chromatographic separation, analyte-stationary phase interactions create temporary apertures. Retention occurs because the analyte must wait for aperture ``opening'' (favourable configuration) to continue transport. The Van Deemter $C$-term (mass transfer resistance) is precisely the aperture equilibration time.
\end{remark}

\begin{figure}[htbp]
\centering
\includegraphics[width=\textwidth]{figures/panel_molecular_apertures.pdf}
\caption{\textbf{Molecular Bonds as Categorical Apertures: The Geometry of Selectivity.}
Molecular bonds are not just energetic interactions---they are geometric constraints (apertures) that selectively allow certain configurations while blocking others. (A) Bond as aperture: a covalent bond creates a geometric constraint (selectivity region). Molecular configurations that pass through the aperture (green) are allowed; configurations blocked by the bond (red X) require bond breaking. The aperture potential $\Phi_a = -k_B T \ln s$ quantifies the energetic cost of aperture traversal. (B) Selectivity ordering: covalent bonds (very low $s \approx 0.01$, high barrier), ionic bonds (low $s \approx 0.05$), hydrogen bonds (medium $s \approx 0.1$), Van der Waals (high $s \approx 0.5$, low barrier). This hierarchy determines transport rates and phase behaviour. (C) Transport rate as selectivity product: when a molecule traverses multiple apertures, the rate $\propto \prod_{a \in \text{path}} s_a$ decreases exponentially with path length. This explains why viscosity increases with molecular complexity. (D) Phase transitions as aperture reconfiguration: solid $\to$ liquid destroys lattice apertures; liquid $\to$ gas destroys intermolecular apertures. Latent heat = sum of destroyed aperture potentials. Melting is ``unlocking'' the phase-lock network.}
\label{fig:molecular_apertures}
\end{figure}

%==============================================================================
% TURBULENCE FROM APERTURE STRUCTURE
%==============================================================================

\subsection{Turbulence as Non-Sequential Aperture Access}
\label{subsec:turbulence_apertures}

We now derive turbulence from first principles as the emergence of non-sequential apertures in the categorical state space.

\subsubsection{The Pendulum Analogy}

Consider the cross-sections of a flowing fluid as partitions of a period, analogous to the positions of a pendulum.

\begin{definition}[Sequential Aperture Structure (Laminar)]
\label{def:sequential_apertures}
In \emph{laminar flow}, apertures exist only between adjacent categorical states. For a period divided into $n$ partitions $\{C_1, C_2, \ldots, C_n\}$, the aperture matrix is:
\begin{equation}
A_{ij}^{\text{laminar}} = \begin{cases} 
s_{\text{adj}} & \text{if } j = i \pm 1 \mod n \\
0 & \text{otherwise}
\end{cases}
\label{eq:laminar_apertures}
\end{equation}
Only transitions to adjacent states are permitted: $C_i \to C_{i+1}$ or $C_i \to C_{i-1}$.
\end{definition}

This is analogous to a \emph{single pendulum}: the pendulum bob must pass through intermediate positions; it cannot ``jump'' from one extreme to the other.

\begin{definition}[Non-Sequential Aperture Structure (Turbulent)]
\label{def:nonsequential_apertures}
In \emph{turbulent flow}, additional apertures connect non-adjacent categorical states:
\begin{equation}
A_{ij}^{\text{turbulent}} = \begin{cases} 
s_{\text{adj}} & \text{if } j = i \pm 1 \mod n \\
s_{\text{turb}} > 0 & \text{otherwise}
\end{cases}
\label{eq:turbulent_apertures}
\end{equation}
Transitions can occur between ANY states: $C_i \to C_j$ for all $j$.
\end{definition}

This is analogous to a \emph{double pendulum}: the second pendulum creates additional degrees of freedom that allow the system to access any state from any other state---the hallmark of chaotic dynamics.

\subsubsection{The Double Pendulum as Aperture Generator}

\begin{theorem}[Double Pendulum Aperture Creation]
\label{thm:double_pendulum_apertures}
Adding a second coupled oscillator (double pendulum) creates apertures between previously non-adjacent states. Specifically:

\textbf{Single pendulum}: $n_1$ states $\{A_1, \ldots, A_{n_1}\}$ with sequential apertures only.

\textbf{Double pendulum}: $n_1 \times n_2$ states $\{(A_i, B_j)\}$ where:
\begin{enumerate}
\item Sequential apertures from the first pendulum: $(A_i, B_j) \to (A_{i\pm 1}, B_j)$
\item \textbf{Non-sequential apertures from the second pendulum}: $(A_i, B_j) \to (A_i, B_k)$ for ANY $k$
\item Cross-coupling apertures: $(A_i, B_j) \to (A_k, B_l)$ for ANY $k, l$
\end{enumerate}
\end{theorem}

\begin{proof}
A double pendulum has two coupled oscillatory modes. In the low-energy (laminar) regime, energy remains primarily in the first mode; the second mode oscillates with small amplitude around equilibrium. Transitions are sequential.

In the high-energy (turbulent) regime, energy transfers chaotically between modes. When energy transfers to the second pendulum, it can swing through large amplitudes, accessing states $(A_i, B_k)$ for any $k$. This creates apertures between states that were non-adjacent in the single-pendulum description.

The coupling between pendulums then creates cross-apertures $(A_i, B_j) \to (A_k, B_l)$, enabling access to any state from any other. \qed
\end{proof}

\subsubsection{Reynolds Number as Aperture Control Parameter}

\begin{definition}[Aperture Accessibility]
\label{def:aperture_accessibility}
The \emph{aperture accessibility} $\mathcal{A}$ quantifies how many non-sequential apertures are open:
\begin{equation}
\mathcal{A} = \frac{\sum_{|i-j| > 1} A_{ij}}{\sum_{i,j} A_{ij}} = \frac{\text{non-sequential apertures}}{\text{total apertures}}
\label{eq:aperture_accessibility}
\end{equation}
\end{definition}

\begin{theorem}[Reynolds Number as Aperture Accessibility]
\label{thm:reynolds_apertures}
The Reynolds number is proportional to aperture accessibility:
\begin{equation}
Re = \frac{\rho v L}{\mu} \propto \frac{\mathcal{A}}{1 - \mathcal{A}}
\label{eq:reynolds_accessibility}
\end{equation}
\end{theorem}

\begin{proof}
The Reynolds number measures the ratio of inertial to viscous forces:
\begin{equation}
Re = \frac{\text{inertial forces}}{\text{viscous forces}}
\end{equation}

In the aperture framework:
\begin{itemize}
\item \textbf{Viscous forces} enforce sequential transitions (only adjacent apertures)
\item \textbf{Inertial forces} open non-sequential apertures (allow ``jumping'')
\end{itemize}

When inertia dominates ($Re$ high), molecules have enough energy to pass through the higher-barrier non-sequential apertures. This increases $\mathcal{A}$.

When viscosity dominates ($Re$ low), molecules cannot overcome non-sequential aperture barriers. Only adjacent apertures remain accessible: $\mathcal{A} \to 0$.

The transition occurs at critical $Re_c$ where $\mathcal{A}$ becomes significant:
\begin{equation}
Re_c \approx \frac{\Phi_{\text{non-seq}} - \Phi_{\text{seq}}}{\kB T}
\end{equation}
where $\Phi_{\text{non-seq}}$ is the non-sequential aperture potential and $\Phi_{\text{seq}}$ is the sequential aperture potential. \qed
\end{proof}

\begin{corollary}[Critical Reynolds Number]
\label{cor:critical_reynolds}
The critical Reynolds number for laminar-turbulent transition corresponds to:
\begin{equation}
Re_c \sim \exp\left(\frac{\Delta\Phi}{\kB T}\right)
\label{eq:critical_reynolds}
\end{equation}
where $\Delta\Phi = \Phi_{\text{non-seq}} - \Phi_{\text{seq}}$ is the aperture barrier difference.
\end{corollary}

\subsubsection{Energy Cascade as Aperture Hierarchy}

\begin{definition}[Aperture Hierarchy]
\label{def:aperture_hierarchy}
In turbulent flow, apertures form a hierarchy:
\begin{enumerate}
\item \textbf{Large-scale apertures}: Low barrier, high selectivity $s_L$, large eddies
\item \textbf{Medium-scale apertures}: Medium barrier, medium selectivity $s_M$
\item \textbf{Small-scale apertures}: High barrier, low selectivity $s_S$, small eddies
\end{enumerate}
The hierarchy satisfies $s_L > s_M > s_S$.
\end{definition}



\begin{theorem}[Energy Cascade as Aperture Descent]
\label{thm:energy_cascade}
The turbulent energy cascade from large to small scales corresponds to descent through the aperture hierarchy:
\begin{equation}
\text{Large eddy} \xrightarrow{s_L} \text{Medium eddy} \xrightarrow{s_M} \text{Small eddy} \xrightarrow{s_S} \text{Dissipation}
\label{eq:energy_cascade}
\end{equation}
Energy transfers from low-barrier to high-barrier apertures until it reaches the smallest scales where viscous dissipation occurs.
\end{theorem}

\begin{proof}
In a double pendulum, energy transfers chaotically between the two coupled oscillators. The first pendulum (large, slow) corresponds to large eddies; the second pendulum (small, fast) corresponds to small eddies.

Each energy transfer is a passage through an aperture. Large-to-medium transfers pass through low-barrier apertures ($s_L$ high, easy). Medium-to-small transfers pass through higher-barrier apertures ($s_M$ lower, harder).

At the smallest scales (Kolmogorov scale), apertures become so restrictive ($s_S$ very low) that molecules cannot maintain coherent oscillation. Energy dissipates as heat---random molecular motion that cannot pass through any organised aperture. \qed
\end{proof}

\begin{corollary}[Kolmogorov Scale from Aperture Cutoff]
\label{cor:kolmogorov}
The Kolmogorov length scale $\eta_K$ corresponds to the scale where aperture selectivity approaches unity (no selectivity $\Rightarrow$ no coherent structure):
\begin{equation}
\eta_K \sim L \cdot Re^{-3/4}
\label{eq:kolmogorov_scale}
\end{equation}
where $L$ is the largest length scale.
\end{corollary}

\subsubsection{Vortices as Aperture Structures}

\begin{definition}[Vortex as Rotating Aperture]
\label{def:vortex_aperture}
A vortex is a \emph{rotating aperture} that connects distant cross-sections by folding space:
\begin{equation}
\text{Vortex}: (x_1, y_1, z_1) \leftrightarrow (x_2, y_2, z_2) \quad \text{with } |x_1 - x_2| \gg \ell_{\text{corr}}
\label{eq:vortex_aperture}
\end{equation}
The vortex creates a topological shortcut between points that would otherwise require sequential passage through many intermediate apertures.
\end{definition}

\begin{theorem}[Vortex Creation from Non-Sequential Apertures]
\label{thm:vortex_creation}
Vortices form when non-sequential apertures open sufficiently that distant fluid elements become directly connected:
\begin{equation}
\text{Vortex forms when } s_{\text{non-seq}} > s_c
\label{eq:vortex_condition}
\end{equation}
where $s_c$ is the critical selectivity for sustained non-local coupling.
\end{theorem}

\begin{proof}
When non-sequential apertures open ($Re > Re_c$), fluid elements can ``jump'' to non-adjacent states. This creates correlations between distant elements. 

If the correlation is sustained (aperture remains open), the correlated elements begin to co-rotate---forming a vortex. The vortex is the physical manifestation of an open non-sequential aperture: it literally brings distant cross-sections into contact through rotational folding.

Below $s_c$, non-sequential apertures open only briefly; correlations decay before vortices form. Above $s_c$, apertures remain open long enough for vortex structures to stabilise. \qed
\end{proof}

\subsubsection{Summary: Turbulence from First Principles}

\begin{theorem}[Complete Characterisation of Turbulence]
\label{thm:turbulence_complete}
Turbulence is fully characterised by the aperture structure:
\begin{enumerate}
\item \textbf{Laminar flow}: Only sequential apertures open ($\mathcal{A} = 0$)
\item \textbf{Transitional flow}: Non-sequential apertures begin to open ($0 < \mathcal{A} < 1$)
\item \textbf{Turbulent flow}: All apertures open ($\mathcal{A} \to 1$)
\end{enumerate}

The double pendulum analogy shows that turbulence is the natural consequence of adding coupled degrees of freedom (the second pendulum) that create non-sequential apertures in state space.
\end{theorem}

\begin{remark}[Physical Interpretation]
This derivation explains turbulence without invoking ``instability'' or ``randomness.'' Turbulence is not disorder---it is \emph{expanded accessibility}. A turbulent fluid can access more of its state space because more apertures are open. The apparent ``chaos'' is simply the consequence of navigating a fully-connected (rather than linearly-connected) state space.
\end{remark}

