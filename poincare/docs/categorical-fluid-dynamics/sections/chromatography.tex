%==============================================================================
% SECTION 6: CHROMATOGRAPHY
%==============================================================================

\section{Application: Liquid Chromatography}
\label{sec:chromatography}

\subsection{The Three-Component S-System}

Chromatography involves three interacting S-components:

\begin{definition}[Chromatographic S-System]
\label{def:chromatographic_system}
A chromatographic system consists of:
\begin{enumerate}
\item Analyte: S-coordinates $\Svec_a = (S_{k,a}, S_{t,a}, S_{e,a})$
\item Eluent (mobile phase): S-coordinates $\Svec_m = (S_{k,m}, S_{t,m}, S_{e,m})$
\item Stationary phase: S-coordinates $\Svec_s = (S_{k,s}, S_{t,s}, S_{e,s})$
\end{enumerate}
\end{definition}

\begin{theorem}[S-Coordinate Determination]
\label{thm:s_determination}
The S-coordinates of chromatographic components are determined by:
\begin{align}
S_k &= -\log_2 \prod_i P_i = -\sum_i \log_2 P_i \label{eq:sk_product} \\
S_t &= \log_{10}\left( \frac{\tau_{\text{mol}}}{\tau_0} \right) \label{eq:st_molecular} \\
S_e &= \sum_{\text{modes}} \eta_i \log_2 \eta_i \label{eq:se_modes}
\end{align}
where $P_i$ are functional group probabilities, $\tau_{\text{mol}}$ is the molecular timescale, and $\eta_i$ are mode occupation probabilities.
\end{theorem}

\begin{proof}
For $S_k$: The molecular configuration probability is the product of independent functional group probabilities (by independence). $S_k = -\log_2 P_{\text{config}} = -\log_2 \prod_i P_i = -\sum_i \log_2 P_i$.

For $S_t$: The molecular timescale is determined by the slowest internal mode (rotation, vibration). Taking the logarithm normalises to a comparable scale.

For $S_e$: Mode occupation is distributed across vibrational, rotational, and electronic modes. Shannon entropy over this distribution gives $S_e$.
\end{proof}

\subsection{Retention Time from S-Transformation}

\begin{definition}[Column S-Profile]
\label{def:column_profile}
A chromatographic column of length $L$ has an S-profile $\Svec_s(x)$ for $x \in [0, L]$. For homogeneous columns, $\Svec_s(x) = \Svec_s$ is constant.
\end{definition}

\begin{theorem}[Retention Time Integral]
\label{thm:retention_integral}
The retention time of an analyte is:
\begin{equation}
t_R = \int_0^L \tau(\Toperator_x[\Svec_a(0)]) \, dx
\label{eq:retention_integral}
\end{equation}
where $\tau(\Svec)$ is the residence time per unit length at S-coordinate $\Svec$.
\end{theorem}

\begin{figure}[htbp]
\centering
\includegraphics[width=\textwidth]{figures/topology_categories_panel.png}
\caption{\textbf{Topological Structure of Categorical Spaces: Apertures as Boundaries.}
This figure reveals the deep connection between categorical structure and topology, showing how apertures arise as topological boundaries between categorical regions. (A) Categorical space as topological space: the set of categorical states $\{\mathcal{C}_i\}$ forms a topological space with the ``aperture topology''---open sets are collections of states reachable without crossing apertures. Connected components are phases; boundaries are phase transitions. (B) Apertures as topological boundaries: the aperture between categories $\mathcal{C}_i$ and $\mathcal{C}_j$ is the boundary $\partial(\mathcal{C}_i \cup \mathcal{C}_j) \cap \mathcal{C}_i \cap \mathcal{C}_j$. Selectivity $s$ measures the ``permeability'' of this boundary---how easily the system crosses from one categorical region to another. (C) Homology and transport: the first homology group $H_1$ counts independent loops in categorical space. Each loop corresponds to a cyclic transport pathway; the homology class determines which apertures must be traversed. Non-trivial homology means multiple routes exist (path degeneracy in chromatography). (D) Euler characteristic and phase: the Euler characteristic $\chi = V - E + F$ (vertices - edges + faces in the categorical complex) distinguishes phases. Gases have high $\chi$ (many disconnected components); liquids have low $\chi$ (percolating network); solids have $\chi = 1$ (single connected lattice). Phase transitions are topological transitions in categorical space.}
\label{fig:topology_categories}
\end{figure}

\begin{proof}
By the dimensional reduction theorem (Theorem~\ref{thm:dimensional_reduction}), the analyte S-coordinate at position $x$ is:
\begin{equation}
\Svec_a(x) = \Toperator_{0 \to x}[\Svec_a(0)]
\end{equation}

The time to traverse the infinitesimal segment $dx$ depends on the local S-coordinate:
\begin{equation}
dt = \tau(\Svec_a(x)) \, dx
\end{equation}

Integrating over column length:
\begin{equation}
t_R = \int_0^L \tau(\Svec_a(x)) \, dx = \int_0^L \tau(\Toperator_{0 \to x}[\Svec_a(0)]) \, dx
\end{equation}
\end{proof}

\begin{definition}[Residence Time Function]
\label{def:residence_time}
The residence time per unit length depends on the S-distance from the stationary phase:
\begin{equation}
\tau(\Svec) = \tau_0 \left( 1 + K(\Svec) \right)
\label{eq:residence_time}
\end{equation}
where $\tau_0$ is the dead time per unit length and $K(\Svec)$ is the local retention factor.
\end{definition}

\begin{theorem}[Retention Factor from S-Distance]
\label{thm:retention_factor}
The retention factor is:
\begin{equation}
K(\Svec_a) = K_0 \exp\left( -\frac{d_S(\Svec_a, \Svec_s)}{\sigma_S} \right)
\label{eq:retention_factor}
\end{equation}
where $K_0$ is the maximum retention factor and $\sigma_S$ is the S-selectivity parameter.
\end{theorem}

\begin{proof}
Retention results from analyte-stationary phase interaction. Interaction strength decreases with S-distance (analytes closer in S-space to the stationary phase interact more strongly). The Boltzmann-like form:
\begin{equation}
K \propto \exp(-d_S / \sigma_S)
\end{equation}
arises from the statistical mechanics of partitioning. The normalisation $K_0$ sets the maximum retention at $d_S = 0$.
\end{proof}

\subsection{Separation Efficiency}

\begin{definition}[Resolution]
\label{def:resolution}
The chromatographic resolution between analytes $a$ and $b$ is:
\begin{equation}
R_s = \frac{2(t_{R,b} - t_{R,a})}{w_a + w_b}
\label{eq:resolution}
\end{equation}
where $w$ denotes peak width at base.
\end{definition}

\begin{theorem}[Resolution from S-Distance]
\label{thm:resolution_s}
Resolution between analytes depends on their S-distance:
\begin{equation}
R_s = \frac{d_S(\Svec_a, \Svec_b)}{2\sigma_{\text{eff}}} \cdot \sqrt{N}
\label{eq:resolution_s}
\end{equation}
where $\sigma_{\text{eff}}$ is effective dispersion and $N$ is plate count.
\end{theorem}

\begin{proof}
Retention time difference:
\begin{equation}
\Delta t_R = t_{R,b} - t_{R,a} = \int_0^L \left[ \tau(\Svec_b(x)) - \tau(\Svec_a(x)) \right] dx
\end{equation}

For small S-distance, $\tau(\Svec_b) - \tau(\Svec_a) \approx \frac{\partial \tau}{\partial \Svec} \cdot (\Svec_b - \Svec_a) = \frac{\partial \tau}{\partial \Svec} \cdot d_S$. Thus:
\begin{equation}
\Delta t_R \propto d_S(\Svec_a, \Svec_b) \cdot L
\end{equation}

Peak width $w \propto L / \sqrt{N}$ where $N$ is plate count. Resolution:
\begin{equation}
R_s = \frac{\Delta t_R}{(w_a + w_b)/2} \propto \frac{d_S \cdot L}{L/\sqrt{N}} = d_S \cdot \sqrt{N}
\end{equation}

Including the dispersion coefficient $\sigma_{\text{eff}}$ gives Equation~\ref{eq:resolution_s}.
\end{proof}

\subsection{Computational Algorithm}

\begin{algorithm}[H]
\caption{Retention Time Prediction from S-Coordinates}
\label{alg:retention_prediction}
\begin{algorithmic}[1]
\Require Analyte structure, column parameters $(L, \Svec_s, \sigma_S, K_0)$, flow rate $F$
\Ensure Predicted retention time $t_R$
\State Compute analyte S-coordinates: $\Svec_a \gets \text{ComputeS}(\text{structure})$
\State Compute dead time: $t_0 \gets L / F$
\State Initialise: $t_R \gets 0$, $N_{\text{plates}} \gets L / H$
\For{$i = 1$ to $N_{\text{plates}}$}
    \State $x \gets (i - 0.5) \cdot H$ \Comment{Plate centre position}
    \State $\Svec_a(x) \gets \Toperator_{0 \to x}[\Svec_a]$ \Comment{Transform S-coordinate}
    \State $K_i \gets K_0 \exp(-d_S(\Svec_a(x), \Svec_s) / \sigma_S)$ \Comment{Local retention}
    \State $\Delta t \gets (t_0 / N_{\text{plates}}) \cdot (1 + K_i)$ \Comment{Time for plate $i$}
    \State $t_R \gets t_R + \Delta t$
\EndFor
\State \Return $t_R$
\end{algorithmic}
\end{algorithm}

\begin{theorem}[Algorithm Complexity]
\label{thm:algorithm_complexity}
Algorithm~\ref{alg:retention_prediction} has complexity $\mathcal{O}(N_{\text{plates}})$, independent of molecular count.
\end{theorem}

\begin{proof}
The algorithm iterates $N_{\text{plates}}$ times. Each iteration involves constant-time operations: S-transformation (matrix multiplication), S-distance (Euclidean norm), exponential, and multiplication. Total: $\mathcal{O}(N_{\text{plates}})$.

Molecular count does not appear because the algorithm operates on S-coordinates (three numbers per analyte), not molecular trajectories.
\end{proof}

%==============================================================================
% CHROMATOGRAPHY AS MEMORYLESS TURBULENCE
%==============================================================================

\subsection{Chromatography as Memoryless Turbulence}
\label{subsec:chromatography_turbulence}

A profound connexion exists between chromatography and turbulence: \emph{chromatography is turbulence with a memory reset at each cross-section}.

\subsubsection{The Subtracted Cross-Section}

\begin{definition}[Subtracted Cross-Section]
\label{def:subtracted_cross_section}
In a chromatographic column, the 2D cross-section is not fully available to the mobile phase. The stationary phase particles subtract regions from the cross-section:
\begin{equation}
\Sigma_{\text{eff}}(x) = \Sigma(x) \setminus \Sigma_{\text{stat}}(x)
\label{eq:subtracted_cross_section}
\end{equation}
where $\Sigma_{\text{stat}}(x)$ is the area occupied by the stationary phase.
\end{definition}

\begin{theorem}[Subtracted Regions as Second Pendulum]
\label{thm:subtracted_pendulum}
The subtracted stationary phase regions act as a ``second pendulum,'' creating non-sequential apertures within each cross-section:
\begin{equation}
\text{Stationary phase particles} \longleftrightarrow \text{Second pendulum}
\label{eq:stationary_pendulum}
\end{equation}
\end{theorem}

\begin{proof}
In the pendulum analogy:
\begin{itemize}
\item The mobile phase flow creates the ``first pendulum''—sequential progression along the column axis
\item The stationary phase particles create obstacles that force the flow to deviate
\item These deviations create non-sequential access: a molecule at position $(y_1, z_1)$ can suddenly access $(y_2, z_2)$ by navigating around a particle
\end{itemize}

This is precisely the aperture structure of turbulence: the particles create apertures between non-adjacent states within each cross-section. \qed
\end{proof}

\begin{figure}[htbp]
\centering
\includegraphics[width=\textwidth]{figures/panel_chromatography.pdf}
\caption{\textbf{Chromatography as Experimental Validation: Memoryless Turbulence in Action.}
(A) The subtracted cross-section: stationary phase particles create ``holes'' in the available flow path. Molecules must navigate around these obstacles through non-sequential apertures---this is turbulence. But crucially, memory resets at each theoretical plate, converting chaos into statistics. (B) One plate = one pendulum cycle: each plate represents entry $\to$ turbulent mixing $\to$ equilibration $\to$ memory reset $\to$ exit. The plate height $H$ is the ``pendulum period''---the column length for one complete equilibration cycle. (C) Separation from memory reset: without reset (turbulence), chaotic mixing homogenises analytes, $\Delta t_R \to 0$. With reset (chromatography), each plate contributes independently, $\Delta t_R \propto N$. Memory reset is necessary for separation. (D) Experimental confirmation: predicted retention times from S-transformation match measured values across different column geometries and analyte chemistries, validating the aperture-memory framework. Platform independence confirms that S-dynamics, not hardware details, determine separation.}
\label{fig:chromatography}
\end{figure}

\subsubsection{The Memory Reset Principle}

Here is the crucial distinction between chromatography and turbulence:

\begin{definition}[Memory Reset]
\label{def:memory_reset}
In chromatography, the phase-lock history is \emph{reset} at each cross-section. Each theoretical plate represents a fresh start:
\begin{equation}
\mathcal{M}(x + H) = \emptyset \quad \text{(memory erased)}
\label{eq:memory_reset}
\end{equation}
where $H$ is the plate height and $\mathcal{M}$ is the phase-lock memory.
\end{definition}

\begin{theorem}[Turbulence vs Chromatography]
\label{thm:turbulence_chromatography}
Turbulence and chromatography differ in memory structure:
\begin{center}
\begin{tabular}{lcc}
\toprule
\textbf{Property} & \textbf{Turbulence} & \textbf{Chromatography} \\
\midrule
Aperture structure & Non-sequential & Non-sequential \\
Memory & Persistent & Reset each plate \\
Pendulum cycles & Continuous & Discrete (one per plate) \\
Net effect & Mixing (chaos) & Separation (statistics) \\
\bottomrule
\end{tabular}
\end{center}
\end{theorem}

\begin{proof}
\textbf{Turbulence}: The double pendulum's state at time $t$ depends on its entire history. The second pendulum carries its phase forward, creating persistent correlations. This leads to chaotic mixing---distant parts of the fluid become correlated through accumulated history.

\textbf{Chromatography}: Each theoretical plate is a fresh ``pendulum cycle.'' When a molecule exits plate $i$ and enters plate $i+1$:
\begin{itemize}
\item Its position within the cross-section is randomised
\item Its interaction history is erased
\item Only its S-coordinate (chemical identity) persists
\end{itemize}

The stationary phase creates turbulent-like mixing \emph{within} each plate, but the lack of memory \emph{between} plates prevents global chaos. \qed
\end{proof}

\subsubsection{Why Memory Reset Enables Separation}

\begin{theorem}[Separation from Memory Reset]
\label{thm:separation_memory}
Memory reset is \emph{necessary} for chromatographic separation. Without reset, turbulent mixing would homogenise the analyte distribution.
\end{theorem}

\begin{proof}
Consider two analytes $a$ and $b$ with different S-coordinates ($\Svec_a \neq \Svec_b$).

\textbf{With memory (turbulence)}: Both analytes experience the same chaotic trajectory through state space. Their positions become correlated through the shared turbulent field. Over time, they mix: $\langle x_a(t) - x_b(t) \rangle \to 0$.

\textbf{Without memory (chromatography)}: At each plate, each analyte has an \emph{independent} interaction with the stationary phase. The interaction strength depends only on S-coordinate:
\begin{equation}
K_a = f(\Svec_a), \quad K_b = f(\Svec_b)
\end{equation}

Over $N$ plates, the retention times accumulate:
\begin{align}
t_{R,a} &= \sum_{i=1}^{N} \tau_0 (1 + K_a) = N \tau_0 (1 + K_a) \\
t_{R,b} &= \sum_{i=1}^{N} \tau_0 (1 + K_b) = N \tau_0 (1 + K_b)
\end{align}

Because $K_a \neq K_b$ (different S-coordinates), $t_{R,a} \neq t_{R,b}$. The separation increases with plate count:
\begin{equation}
\Delta t_R = N \tau_0 (K_b - K_a) \propto N
\end{equation}

Memory reset ensures each plate contributes independently to separation, rather than chaotically mixing the analytes. \qed
\end{proof}

\subsubsection{The Cross-Section Geometry}

\begin{definition}[Stationary Phase Geometry]
\label{def:stationary_geometry}
The stationary phase creates ``holes'' in the cross-section with characteristic geometry:
\begin{equation}
\Sigma_{\text{stat}} = \bigcup_{k} D_k(r_k, \mathbf{c}_k)
\label{eq:stationary_geometry}
\end{equation}
where $D_k$ is a disk (or sphere cross-section) of radius $r_k$ centred at $\mathbf{c}_k$.
\end{definition}

\begin{theorem}[Aperture Density from Particle Packing]
\label{thm:aperture_density}
The number of non-sequential apertures per unit area is:
\begin{equation}
\rho_{\text{aperture}} = \frac{n_p \cdot (2\pi r_p)}{\pi R^2}
\label{eq:aperture_density}
\end{equation}
where $n_p$ is the number of particles, $r_p$ is particle radius, and $R$ is column radius.
\end{theorem}

\begin{proof}
Each stationary phase particle of radius $r_p$ creates apertures around its perimeter (where flow must deviate). The perimeter is $2\pi r_p$.

For $n_p$ particles uniformly distributed in column cross-section $\pi R^2$:
\begin{equation}
\rho_{\text{aperture}} = \frac{n_p \cdot 2\pi r_p}{\pi R^2} = \frac{2 n_p r_p}{R^2}
\end{equation}

Higher packing density (more particles) or larger particles increases aperture density, creating more turbulent-like mixing within each plate. \qed
\end{proof}

\subsubsection{The Pendulum Cycle Interpretation}

\begin{theorem}[One Plate = One Pendulum Cycle]
\label{thm:plate_pendulum}
Each theoretical plate represents exactly one complete pendulum cycle:
\begin{enumerate}
\item \textbf{Entry}: Molecule enters plate with initial state
\item \textbf{Turbulent mixing}: Non-sequential apertures (stationary phase) create mixing
\item \textbf{Equilibration}: Molecule partitions between mobile and stationary phases
\item \textbf{Exit}: Molecule exits with reset memory, carrying only S-coordinate
\end{enumerate}
\end{theorem}

\begin{proof}
The theoretical plate model assumes equilibrium is reached within each plate. This is equivalent to the pendulum completing one full cycle:
\begin{itemize}
\item Start of cycle: defined initial conditions
\item During cycle: chaotic exploration of state space (turbulence within plate)
\item End of cycle: return to equilibrium, phase relationships reset
\end{itemize}

The plate height $H$ is the column length required for one equilibration cycle. Smaller $H$ means faster equilibration (shorter pendulum period), more plates, better separation. \qed
\end{proof}

\begin{corollary}[Van Deemter as Pendulum Period]
\label{cor:vandeemter_pendulum}
The Van Deemter equation $H = A + B/u + Cu$ describes how the pendulum period (plate height) depends on flow velocity $u$:
\begin{itemize}
\item $A$: Geometric mixing (eddy diffusion)---inherent aperture structure
\item $B/u$: Axial diffusion---memory leakage between plates at low velocity
\item $Cu$: Mass transfer resistance---time for turbulent mixing to reach equilibrium
\end{itemize}
\end{corollary}

\subsubsection{Implications}

\begin{remark}[Design Principle]
This analysis suggests a design principle for chromatographic columns:
\begin{enumerate}
\item \textbf{Maximise within-plate turbulence}: Use complex particle geometries to create many non-sequential apertures
\item \textbf{Minimise between-plate memory}: Ensure complete equilibration (memory reset) before molecules exit each plate
\item \textbf{Optimise flow rate}: Balance turbulent mixing time (needs slower flow) against memory leakage (worse at slow flow)
\end{enumerate}
\end{remark}

\begin{remark}[Connection to Chaos Theory]
The memory reset in chromatography is analogous to \emph{stroboscopic observation} of a chaotic system. By observing only at regular intervals (once per plate), the chaotic dynamics are converted to a statistical process. This is why chromatographic retention follows simple statistical mechanics (partition equilibria) rather than chaotic dynamics.
\end{remark}

