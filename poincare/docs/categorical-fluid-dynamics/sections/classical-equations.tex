%==============================================================================
% SECTION 5: DERIVING CLASSICAL EQUATIONS
%==============================================================================

\section{Deriving Classical Fluid Equations}
\label{sec:classical}

We now derive the classical equations of fluid dynamics from the categorical-partition-oscillation framework. The key insight is that classical equations emerge as the continuum limit of discrete categorical completions, with viscosity representing time emergence and turbulence representing expanded aperture accessibility.

\subsection{Continuity Equation from Categorical Conservation}

\begin{theorem}[Continuity Equation]
\label{thm:continuity}
The continuity equation
\begin{equation}
\frac{\partial \rho}{\partial t} + \nabla \cdot (\rho \mathbf{v}) = 0
\label{eq:continuity}
\end{equation}
follows from conservation of categorical states through apertures.
\end{theorem}

\begin{proof}
Let $n_{\Cspace}(\mathbf{x}, t)$ be the categorical state density at position $\mathbf{x}$ and time $t$. Categorical states are neither created nor destroyed in bulk flow; they are transported through apertures between adjacent cross-sections.

At each aperture, selectivity $s = 1$ for mass transport (all molecules pass). The conservation law is:
\begin{equation}
\frac{\partial n_{\Cspace}}{\partial t} + \nabla \cdot (n_{\Cspace} \mathbf{v}) = 0
\label{eq:categorical_conservation}
\end{equation}

Mass density $\rho$ is proportional to categorical state density: $\rho = m \cdot n_{\Cspace}$ where $m$ is molecular mass. The aperture selectivity for mass is unity because mass cannot be selectively filtered by molecular apertures. Substituting:
\begin{equation}
\frac{\partial (m \cdot n_{\Cspace})}{\partial t} + \nabla \cdot (m \cdot n_{\Cspace} \mathbf{v}) = 0
\end{equation}

Dividing by constant $m$:
\begin{equation}
\frac{\partial \rho}{\partial t} + \nabla \cdot (\rho \mathbf{v}) = 0
\end{equation}
\end{proof}

\begin{remark}[Aperture Interpretation]
The continuity equation states that mass flux through any aperture equals the rate of change of mass within the bounded region. Since mass apertures have selectivity $s = 1$, there is no ``filtering''---all mass that enters must exit or accumulate.
\end{remark}

\subsection{Momentum Equation from S-Gradient Dynamics}

\begin{theorem}[Navier-Stokes from S-Dynamics]
\label{thm:navier_stokes}
The Navier-Stokes momentum equation
\begin{equation}
\rho \left( \frac{\partial \mathbf{v}}{\partial t} + (\mathbf{v} \cdot \nabla) \mathbf{v} \right) = -\nabla p + \mu \nabla^2 \mathbf{v}
\label{eq:navier_stokes_derived}
\end{equation}
emerges from S-gradient flow with:
\begin{align}
p &= \kB T \cdot \rho_{\Cspace} \cdot R(\Cspace) \label{eq:pressure_categorical} \\
\mu &= \sum_{i,j} \tau_{p,ij} \cdot g_{ij} \label{eq:viscosity_categorical}
\end{align}
where $R(\Cspace)$ is the categorical richness, $\tau_{p,ij}$ are partition lags, and $g_{ij}$ are phase-lock couplings. The viscosity term represents the \emph{cost of time emergence} in a correlated medium.
\end{theorem}

\begin{proof}
\textbf{Step 1: S-gradient flow in physical space.}

From Theorem~\ref{thm:flow_direction}, physical flow follows S-gradient:
\begin{equation}
\rho \frac{D\mathbf{v}}{Dt} = -\nabla_{\mathbf{x}} \Phi_S
\end{equation}
where $\Phi_S$ is the S-potential expressed in physical coordinates.

\textbf{Step 2: Pressure from categorical richness (aperture reconfiguration).}

The S-potential gradient in physical space is:
\begin{equation}
\nabla_{\mathbf{x}} \Phi_S = \frac{\partial \Phi_S}{\partial \Svec} \cdot \frac{\partial \Svec}{\partial \mathbf{x}}
\end{equation}

For an ideal system, $\Phi_S = -\kB T \ln \Omega_{\Cspace}$ where $\Omega_{\Cspace}$ is the accessible categorical state count---the number of apertures available for navigation. The gradient gives:
\begin{equation}
\nabla_{\mathbf{x}} \Phi_S = -\kB T \frac{\nabla \Omega_{\Cspace}}{\Omega_{\Cspace}} = -\kB T \nabla \ln \Omega_{\Cspace}
\end{equation}

For uniform $\Omega_{\Cspace}$ per molecule, $\nabla \ln \Omega_{\Cspace} = \nabla \ln (\rho_{\Cspace} \cdot R)$ where $R$ is per-molecule richness (accessible apertures per molecule). This yields pressure:
\begin{equation}
p = \kB T \cdot \rho_{\Cspace} \cdot R
\end{equation}

\begin{figure}[htbp]
\centering
\includegraphics[width=\textwidth]{figures/panel_classical_equations.pdf}
\caption{\textbf{Derivation of Classical Fluid Dynamics: Navier-Stokes from First Principles.}
(A) Continuity equation: mass conservation emerges from categorical state conservation through apertures. Since mass apertures have selectivity $s = 1$ (all molecules pass), the divergence of mass flux equals zero. This is conservation of categorical states, not an assumption. (B) Navier-Stokes momentum equation: pressure gradient arises from S-potential gradient (aperture accessibility); viscous stress arises from memory accumulation (phase-lock reconfiguration). Pressure is the average aperture potential; viscosity is the rate of memory accumulation per unit strain. (C) Laminar vs turbulent transition: the Reynolds number $\text{Re} = \rho v L / \mu$ measures the ratio of inertial energy (accessing non-sequential apertures) to viscous memory (penalising phase-lock reconfiguration). At high Re, inertia overcomes memory---the ``double pendulum'' dynamics emerge. (D) Energy cascade: at large scales, non-sequential apertures dominate; at the Kolmogorov scale, viscous (sequential) apertures convert kinetic energy to heat through phase-lock breaking. Turbulence is aperture hierarchy, not randomness.}
\label{fig:classical_equations}
\end{figure}

\textbf{Step 3: Viscosity as time emergence.}

Viscous stress arises from the \emph{memory} embedded in phase-lock networks. When adjacent fluid layers have different velocities, moving one layer requires ``pulling'' the next through the phase-lock connections. Each connection must break and reform---a partition operation requiring time $\tau_p$.

This is the \emph{cost of time passing} in a correlated medium. Time cannot emerge without categorical completion, and categorical completion requires breaking/reforming phase-locks:
\begin{equation}
\text{Viscosity} = \text{Memory} = \text{Time emergence cost}
\end{equation}

The stress tensor component is:
\begin{equation}
\sigma_{ij}^{\text{visc}} = \sum_{k,l} \tau_{p,kl} \cdot g_{kl} \cdot \frac{\partial v_i}{\partial x_j}
\end{equation}

where the sum is over phase-lock pairs $(k, l)$ in the velocity gradient direction. For isotropic fluids:
\begin{equation}
\sigma_{ij}^{\text{visc}} = \mu \left( \frac{\partial v_i}{\partial x_j} + \frac{\partial v_j}{\partial x_i} \right)
\end{equation}

with $\mu = \sum_{k,l} \tau_{p,kl} \cdot g_{kl}$.

\textbf{Step 4: Assembly.}

Combining pressure gradient (aperture accessibility) and viscous stress (time emergence):
\begin{equation}
\rho \frac{D\mathbf{v}}{Dt} = -\nabla p + \nabla \cdot \boldsymbol{\sigma}^{\text{visc}} = -\nabla p + \mu \nabla^2 \mathbf{v}
\end{equation}

Expanding the material derivative:
\begin{equation}
\rho \left( \frac{\partial \mathbf{v}}{\partial t} + (\mathbf{v} \cdot \nabla) \mathbf{v} \right) = -\nabla p + \mu \nabla^2 \mathbf{v}
\end{equation}
\end{proof}

\begin{corollary}[Viscosity as Memory]
\label{cor:viscosity_memory}
Viscosity is the accumulated memory of phase-lock history. High viscosity corresponds to:
\begin{itemize}
\item Long partition lags (slow equilibration between layers)
\item Strong phase-lock coupling (dense networks with many connections)
\item Rich phase-lock history (many connections to break/reform)
\end{itemize}
\end{corollary}

\begin{remark}[The Pendulum Interpretation]
In the pendulum analogy:
\begin{itemize}
\item Each molecule is a pendulum oscillating in its potential well
\item Phase-lock networks are the coupling between pendulums
\item Viscosity is the resistance to changing the collective phase relationship
\item Moving one layer ``drags'' adjacent layers through phase-lock connections
\end{itemize}
This explains why viscosity decreases with temperature in liquids (thermal energy disrupts phase-locks) but increases with temperature in gases (more collisions = more phase-lock events).
\end{remark}

\subsection{Energy Equation from Heat-Entropy Decoupling}

\begin{theorem}[Heat-Entropy Decoupling]
\label{thm:heat_entropy}
Heat and entropy are decoupled at the microscopic level:
\begin{align}
\text{Heat:} \quad &Q \lessgtr 0 \quad \text{(can fluctuate)} \\
\text{Entropy:} \quad &\Delta S > 0 \quad \text{(always increases)}
\end{align}
\end{theorem}

\begin{proof}
Heat is energy transfer due to temperature difference. At the molecular level, individual collisions can transfer energy in either direction. Heat flow direction is a statistical property of ensembles.

Entropy measures categorical completion. Each partition operation produces entropy $\Delta S_{\text{part}} = \kB \ln n_{\text{res}} > 0$ (Theorem~\ref{thm:partition_entropy}). Categorical completion is irreversible. Entropy increases regardless of heat direction.
\end{proof}

\begin{theorem}[Energy Equation]
\label{thm:energy_equation}
The energy equation
\begin{equation}
\rho c_p \frac{DT}{Dt} = k \nabla^2 T + \Phi_{\text{diss}}
\label{eq:energy_equation}
\end{equation}
follows from S-coordinate evolution with:
\begin{equation}
\Phi_{\text{diss}} = \mu \left( \nabla \mathbf{v} : \nabla \mathbf{v} \right)
\label{eq:dissipation}
\end{equation}
\end{theorem}

\begin{proof}
Temperature $T$ is related to S-coordinates through:
\begin{equation}
T = \frac{\partial U}{\partial S_e}\bigg|_{S_k, S_t}
\end{equation}

The evolution of $S_e$ follows from the S-transformation (Theorem~\ref{thm:complete_transformation}):
\begin{equation}
\frac{\partial S_e}{\partial t} = D_S \nabla^2 S_e + \dot{S}_{\text{diss}}
\end{equation}

where $\dot{S}_{\text{diss}}$ is entropy production from viscous dissipation. Converting to temperature using $dS_e = c_p dT / T$:
\begin{equation}
\rho c_p \frac{DT}{Dt} = k \nabla^2 T + T \dot{S}_{\text{diss}}
\end{equation}

The dissipation function $\Phi_{\text{diss}} = T \dot{S}_{\text{diss}} = \mu (\nabla \mathbf{v} : \nabla \mathbf{v})$ follows from viscous stress doing work.
\end{proof}

\subsection{Continuum Limit}

\begin{theorem}[Continuum Limit]
\label{thm:continuum_limit}
Classical fluid equations are the continuum limit of discrete S-transformations as:
\begin{enumerate}
\item Spatial discretisation $dx \to 0$
\item Temporal discretisation $dt \to 0$
\item Partition lag $\tau_p \to 0$ while $\sum \tau_p$ remains finite
\item Aperture count $\to \infty$ while selectivity distribution remains finite
\end{enumerate}
\end{theorem}

\begin{proof}
The discrete S-transformation (Equation~\ref{eq:complete_transformation}):
\begin{equation}
\Svec(x + dx) = \Svec(x) - \kappa(\Svec - \Svec_{\text{stat}}) + D_S \nabla_S^2 \Svec \cdot dt - v \nabla_x \Svec \cdot dt
\end{equation}

In the limit $dx \to 0$, $dt \to 0$:
\begin{equation}
\frac{\partial \Svec}{\partial t} + v \frac{\partial \Svec}{\partial x} = D_S \nabla^2 \Svec - \kappa(\Svec - \Svec_{\text{stat}})
\end{equation}

This is the advection-diffusion-reaction equation in S-coordinates. Mapping to physical variables via Theorems~\ref{thm:continuity}, \ref{thm:navier_stokes}, \ref{thm:energy_equation} yields the Navier-Stokes system.

The aperture structure becomes a continuous selectivity field $s(\mathbf{x})$ rather than discrete apertures.
\end{proof}

\subsection{Laminar vs Turbulent Flow: The Aperture Accessibility Transition}

The distinction between laminar and turbulent flow has a natural interpretation in the aperture framework.

\begin{definition}[Sequential Apertures (Laminar)]
\label{def:classical_sequential_apertures}
In laminar flow, apertures are \emph{sequential}---molecules navigate through consecutive apertures in order:
\begin{equation}
\mathcal{C}_1 \xrightarrow{a_{12}} \mathcal{C}_2 \xrightarrow{a_{23}} \mathcal{C}_3 \xrightarrow{a_{34}} \cdots
\label{eq:classical_sequential_apertures}
\end{equation}
where $a_{ij}$ is the aperture connecting states $\mathcal{C}_i$ and $\mathcal{C}_j$ with $|i - j| = 1$.
\end{definition}

\begin{definition}[Non-Sequential Apertures (Turbulent)]
\label{def:classical_nonsequential_apertures}
In turbulent flow, apertures become \emph{non-sequential}---molecules can jump between non-adjacent states:
\begin{equation}
\mathcal{C}_1 \xrightarrow{a_{1k}} \mathcal{C}_k \quad \text{where } |k - 1| > 1
\label{eq:classical_nonsequential_apertures}
\end{equation}
\end{definition}

\begin{theorem}[Reynolds Number as Aperture Accessibility]
\label{thm:classical_reynolds_apertures}
The Reynolds number measures the degree of non-sequential aperture accessibility:
\begin{equation}
\text{Re} = \frac{\rho v L}{\mu} = \frac{\text{Inertial aperture count}}{\text{Viscous aperture resistance}}
\label{eq:reynolds_apertures}
\end{equation}
\end{theorem}

\begin{proof}
Inertia provides kinetic energy to access non-adjacent apertures:
\begin{equation}
E_{\text{inertial}} = \frac{1}{2} \rho v^2 L^3 \propto \rho v L
\end{equation}

Viscosity (memory) resists non-sequential access by penalising phase-lock reconfiguration:
\begin{equation}
E_{\text{viscous}} \propto \mu
\end{equation}

The ratio:
\begin{equation}
\text{Re} = \frac{E_{\text{inertial}}}{E_{\text{viscous}}} = \frac{\rho v L}{\mu}
\end{equation}

At low Re, viscous memory dominates: only sequential apertures are accessible (laminar flow).

At high Re, inertial energy overcomes memory: non-sequential apertures become accessible (turbulent flow). \qed
\end{proof}

\begin{remark}[The Double Pendulum Analogy]
Laminar flow is like a \emph{single} pendulum: deterministic, periodic, one state leads to the next.

Turbulent flow is like a \emph{double} pendulum: the second pendulum introduces non-sequential accessibility between states. At any moment, the system can jump to distant configurations.

The critical Reynolds number marks the transition where the ``second pendulum'' (inertial dynamics) becomes strong enough to override the ``single pendulum'' (viscous, sequential dynamics).
\end{remark}

\begin{theorem}[Energy Cascade as Aperture Hierarchy]
\label{thm:classical_energy_cascade}
The turbulent energy cascade corresponds to traversing apertures at successively smaller scales:
\begin{equation}
\ell_0 \xrightarrow{\text{non-seq}} \ell_1 \xrightarrow{\text{non-seq}} \ell_2 \xrightarrow{\text{non-seq}} \cdots \xrightarrow{\text{seq}} \ell_K
\label{eq:classical_energy_cascade}
\end{equation}
where $\ell_k$ are length scales, non-sequential apertures dominate at large scales, and sequential (viscous) apertures dominate at the Kolmogorov scale $\ell_K$.
\end{theorem}

\begin{proof}
At scale $\ell$, the local Reynolds number is:
\begin{equation}
\text{Re}_\ell = \frac{\rho v_\ell \ell}{\mu}
\end{equation}

Energy cascades from large scales (high $\text{Re}_\ell$, non-sequential apertures) to small scales.

At the Kolmogorov scale $\ell_K$, $\text{Re}_{\ell_K} \sim 1$:
\begin{equation}
\ell_K = \left( \frac{\mu^3}{\rho^3 \epsilon} \right)^{1/4}
\end{equation}
where $\epsilon$ is the energy dissipation rate.

Below $\ell_K$, viscous (sequential) apertures dominate, converting kinetic energy to heat through phase-lock breaking. \qed
\end{proof}

