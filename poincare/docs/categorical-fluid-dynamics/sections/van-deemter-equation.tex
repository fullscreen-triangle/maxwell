%==============================================================================
% SECTION 7: VAN DEEMTER EQUATION
%==============================================================================

\section{Derivation of the Van Deemter Equation}
\label{sec:vandeemter}

The Van Deemter equation describes chromatographic efficiency. In the aperture-memory framework, each term corresponds to a distinct physical mechanism: path degeneracy (aperture multiplicity), memory leakage (incomplete reset), and equilibration time (aperture traversal rate). The plate height is fundamentally the \emph{pendulum period}---the length required for one complete cycle of equilibration with memory reset.

\subsection{Plate Height as Pendulum Period}

\begin{definition}[Plate Height]
\label{def:plate_height}
The plate height $H$ (height equivalent to a theoretical plate) quantifies the column length for one complete pendulum cycle:
\begin{equation}
H = \frac{\sigma_x^2}{L}
\label{eq:plate_height}
\end{equation}
where $\sigma_x^2$ is the spatial variance of the analyte band and $L$ is column length.

Equivalently, $H$ is the distance for one complete equilibration-reset cycle.
\end{definition}

\begin{definition}[Plate Count]
\label{def:plate_count}
The plate count $N$ is the number of pendulum cycles:
\begin{equation}
N = \frac{L}{H} = \frac{L^2}{\sigma_x^2}
\label{eq:plate_count}
\end{equation}
Each plate represents one complete cycle: entry $\to$ turbulent mixing $\to$ equilibration $\to$ memory reset $\to$ exit.
\end{definition}

\begin{theorem}[Van Deemter Equation]
\label{thm:vandeemter}
The plate height depends on linear velocity $u$ according to:
\begin{equation}
H = A + \frac{B}{u} + Cu
\label{eq:vandeemter}
\end{equation}
with coefficients:
\begin{align}
A &= 2\lambda d_p \cdot \mathcal{D}_{\text{path}} \label{eq:A_coefficient} \\
B &= 2\gamma D_m \cdot \tau_{p,\text{res}} \label{eq:B_coefficient} \\
C &= \frac{d_p^2}{D_s} \cdot \frac{\tau_{p,\text{eq}}}{\tau_0} \label{eq:C_coefficient}
\end{align}
where $\lambda$ is packing irregularity, $d_p$ is particle diameter, $\mathcal{D}_{\text{path}}$ is path degeneracy, $\gamma$ is obstruction factor, $D_m$ is molecular diffusion coefficient, $\tau_{p,\text{res}}$ is residue accumulation time, $D_s$ is stationary phase diffusion coefficient, and $\tau_{p,\text{eq}}$ is phase equilibration time.
\end{theorem}

\subsection{The A-Term: Aperture Multiplicity (Non-Sequential Access)}

\begin{definition}[Path Degeneracy as Aperture Count]
\label{def:path_degeneracy}
Path degeneracy $\mathcal{D}_{\text{path}}$ is the number of non-sequential apertures available at each position---the number of categorically equivalent flow paths through the packed bed:
\begin{equation}
\mathcal{D}_{\text{path}} = \exp(S_{\text{path}} / \kB) = \text{Number of non-sequential apertures}
\label{eq:path_degeneracy}
\end{equation}
where $S_{\text{path}}$ is the path entropy.
\end{definition}

\begin{theorem}[A-Term from Aperture Multiplicity]
\label{thm:A_term}
The A-term arises from non-sequential aperture accessibility (turbulent-like mixing within each plate):
\begin{equation}
A = 2\lambda d_p \cdot \mathcal{D}_{\text{path}}
\label{eq:A_term}
\end{equation}
\end{theorem}

\begin{proof}
Consider molecules entering a packed bed at the same position. The stationary phase particles create ``subtracted regions'' from the cross-section (see Section~\ref{subsec:chromatography_turbulence}), forcing molecules to navigate around them through non-sequential apertures.

Each particle of diameter $d_p$ creates apertures around its perimeter. For $n_p$ particles, the number of non-sequential apertures is $\mathcal{D} \propto n_p$.

The path entropy is:
\begin{equation}
S_{\text{path}} = -\kB \sum_i p_i \ln p_i = \kB \ln \mathcal{D}
\end{equation}
for uniform probability over $\mathcal{D}$ paths.

The variance in path length (due to non-sequential access) is:
\begin{equation}
\sigma_\ell^2 = \sum_i p_i (\ell_i - \langle \ell \rangle)^2 \propto d_p^2 \cdot \mathcal{D}
\end{equation}

Plate height contribution:
\begin{equation}
H_A = \frac{\sigma_\ell^2}{L} \propto d_p \cdot \mathcal{D}
\end{equation}

Including the packing parameter $\lambda$:
\begin{equation}
A = 2\lambda d_p \cdot \mathcal{D}_{\text{path}}
\end{equation}
\end{proof}

\begin{remark}[A-Term as Within-Plate Turbulence]
The A-term represents the ``double pendulum'' dynamics \emph{within} each plate---the non-sequential apertures created by stationary phase particles. It is velocity-independent because aperture geometry is fixed by packing, not by flow rate.

This is the turbulent-like mixing that enables equilibration within each plate.
\end{remark}

\subsection{The B-Term: Memory Leakage (Incomplete Reset)}

\begin{definition}[Memory Leakage]
\label{def:memory_leakage}
At low flow velocities, memory is not fully reset at plate boundaries. Molecules have time to diffuse between plates, carrying phase-lock memory across boundaries:
\begin{equation}
\mathcal{M}_{\text{leaked}} = D_m \cdot t_{\text{res}} = D_m \cdot \frac{L}{u}
\label{eq:memory_leakage}
\end{equation}
\end{definition}

\begin{theorem}[B-Term from Memory Leakage]
\label{thm:B_term}
The B-term arises from incomplete memory reset at low velocities:
\begin{equation}
B = 2\gamma D_m \cdot \tau_{p,\text{res}}
\label{eq:B_term}
\end{equation}
where $\gamma$ is the obstruction factor and $\tau_{p,\text{res}}$ is the residue accumulation time.
\end{theorem}

\begin{proof}
Molecular diffusion allows molecules to wander between plates, carrying memory. The diffusion length in time $t$ is:
\begin{equation}
\sigma_{\text{diff}} = \sqrt{2D_m t}
\end{equation}

In a packed bed with obstruction factor $\gamma$:
\begin{equation}
\sigma_{\text{diff}} = \sqrt{2\gamma D_m t}
\end{equation}

The time available for memory leakage is the residence time $t = L/u$:
\begin{equation}
\sigma_x^2 = 2\gamma D_m \cdot \frac{L}{u}
\end{equation}

Plate height:
\begin{equation}
H_B = \frac{\sigma_x^2}{L} = \frac{2\gamma D_m}{u}
\end{equation}

Thus $B = 2\gamma D_m \cdot \tau_{p,\text{res}}$.
\end{proof}

\begin{remark}[B-Term as Memory Violation]
The B-term represents \emph{violation of memory reset}. At low velocities, molecules have time to diffuse across plate boundaries, carrying their phase-lock history. This corrupts the statistical independence of plates, reducing separation efficiency.

Fast flow prevents memory leakage: molecules exit each plate before their memory can diffuse to adjacent plates.
\end{remark}

\subsection{The C-Term: Aperture Traversal Time (Incomplete Equilibration)}

\begin{definition}[Aperture Traversal Time]
\label{def:equilibration_time}
The aperture traversal time $\tau_{p,\text{eq}}$ is the time required for a molecule to navigate through the apertures connecting mobile and stationary phases:
\begin{equation}
\tau_{p,\text{eq}} = \frac{d_p^2}{D_s}
\label{eq:equilibration_time}
\end{equation}
where $d_p$ is the diffusion path length (particle size) and $D_s$ is diffusivity in stationary phase.

This is the time for the pendulum to complete its equilibration swing within each plate.
\end{definition}

\begin{theorem}[C-Term from Incomplete Equilibration]
\label{thm:C_term}
The C-term arises when molecules exit plates before completing their equilibration swing:
\begin{equation}
C = \frac{d_p^2}{D_s} \cdot \frac{\tau_{p,\text{eq}}}{\tau_0}
\label{eq:C_term}
\end{equation}
\end{theorem}

\begin{proof}
Each plate represents one pendulum cycle. For complete equilibration (clean separation), molecules must complete the cycle before exiting.

The contact time is $\tau_{\text{contact}} = d_p / u$.

The non-equilibrium fraction (incomplete pendulum swing) is:
\begin{equation}
f_{\text{neq}} = 1 - \exp\left( -\frac{\tau_{p,\text{eq}}}{\tau_{\text{contact}}} \right)
\end{equation}

For fast flow ($\tau_{\text{contact}} < \tau_{p,\text{eq}}$):
\begin{equation}
f_{\text{neq}} \approx \frac{\tau_{p,\text{eq}}}{\tau_{\text{contact}}} = \frac{\tau_{p,\text{eq}} \cdot u}{d_p}
\end{equation}

Molecules exit before completing equilibration, carrying ``unfinished'' phase-lock states. This corrupts the turbulent mixing within the plate:
\begin{equation}
\sigma_x^2 \propto \tau_{p,\text{eq}} \cdot u \cdot L
\end{equation}

Plate height:
\begin{equation}
H_C = \frac{\sigma_x^2}{L} \propto \tau_{p,\text{eq}} \cdot u
\end{equation}

Thus $C = (d_p^2/D_s) \cdot (\tau_{p,\text{eq}}/\tau_0)$.
\end{proof}

\begin{remark}[C-Term as Truncated Pendulum]
The C-term represents \emph{truncated pendulum cycles}. At high velocities, molecules exit each plate before completing their equilibration with the stationary phase. The pendulum is interrupted mid-swing.

Slow flow allows complete equilibration: each molecule fully explores the aperture landscape before moving to the next plate.
\end{remark}

\subsection{Optimal Velocity}

\begin{theorem}[Optimal Velocity]
\label{thm:optimal_velocity}
The plate height is minimised at:
\begin{equation}
u_{\text{opt}} = \sqrt{\frac{B}{C}}
\label{eq:optimal_velocity}
\end{equation}
with minimum plate height:
\begin{equation}
H_{\text{min}} = A + 2\sqrt{BC}
\label{eq:minimum_H}
\end{equation}
\end{theorem}

\begin{proof}
Taking derivative of Van Deemter equation:
\begin{equation}
\frac{dH}{du} = -\frac{B}{u^2} + C = 0
\end{equation}

Solving:
\begin{equation}
u_{\text{opt}} = \sqrt{\frac{B}{C}}
\end{equation}

Substituting back:
\begin{equation}
H_{\text{min}} = A + \frac{B}{\sqrt{B/C}} + C\sqrt{\frac{B}{C}} = A + \sqrt{BC} + \sqrt{BC} = A + 2\sqrt{BC}
\end{equation}
\end{proof}

\begin{figure}[htbp]
\centering
\includegraphics[width=\textwidth]{figures/panel_vandeemter.pdf}
\caption{\textbf{The Van Deemter Equation: Three Failure Modes of Tamed Turbulence.}
The Van Deemter equation $H = A + B/u + Cu$ describes how plate height (inefficiency) depends on flow velocity. Each term represents a distinct failure mode of the aperture-memory framework. (A) A-term (aperture multiplicity): velocity-independent contribution from non-sequential apertures within each plate. This is the irreducible ``turbulent'' contribution---the within-plate double-pendulum dynamics that enable equilibration. Cannot be eliminated, only minimised by uniform packing. (B) B-term (memory leakage): $\propto 1/u$, dominates at low velocity. Molecules diffuse across plate boundaries, carrying phase-lock memory. This violates the memory reset requirement, corrupting statistical independence. Fast flow prevents leakage. (C) C-term (truncated pendulum): $\propto u$, dominates at high velocity. Molecules exit plates before completing equilibration---the pendulum swing is interrupted. Slow flow allows complete equilibration. (D) Optimal velocity: $u_{\text{opt}} = \sqrt{B/C}$ balances memory leakage against truncated equilibration. At this velocity, each plate achieves complete equilibration with full memory reset---chromatography is ``tamed turbulence'' operating at peak efficiency.}
\label{fig:vandeemter}
\end{figure}

\subsection{Coefficients from S-Coordinates}

\begin{theorem}[Van Deemter Coefficients from Partition Lag Statistics]
\label{thm:coefficients_from_s}
The Van Deemter coefficients can be computed from S-coordinates:
\begin{align}
A &= 2\lambda d_p \cdot \exp(S_{k,\text{path}} / \kB) \\
B &= 2\gamma D_m \cdot \exp(S_{t,\text{res}} / S_0) \\
C &= \frac{d_p^2}{D_s} \cdot \exp(S_{e,\text{eq}} / S_0)
\end{align}
where $S_{k,\text{path}}$, $S_{t,\text{res}}$, $S_{e,\text{eq}}$ are the S-coordinates characterising path degeneracy, residue accumulation, and equilibration respectively.
\end{theorem}

\begin{proof}
Each coefficient relates to a specific aspect of the aperture-memory framework:

\textbf{A-coefficient}: Aperture multiplicity (non-sequential access within plates). $\mathcal{D}_{\text{path}} = \exp(S_k / \kB)$ counts available non-sequential apertures.

\textbf{B-coefficient}: Memory leakage (incomplete reset between plates). $\tau_{p,\text{res}} = \tau_0 \exp(S_t / S_0)$ measures how long memory persists across boundaries.

\textbf{C-coefficient}: Aperture traversal time (incomplete equilibration within plates). $\tau_{p,\text{eq}} = \tau_0 \exp(S_e / S_0)$ measures time to complete the pendulum swing.
\end{proof}

\subsection{Unified Interpretation: The Three Failure Modes}

\begin{theorem}[Van Deemter as Three Failure Modes]
\label{thm:failure_modes}
The Van Deemter equation describes three ways chromatographic separation can fail:
\begin{center}
\begin{tabular}{lccl}
\toprule
\textbf{Term} & \textbf{Velocity} & \textbf{Failure Mode} & \textbf{Physical Meaning} \\
\midrule
$A$ & Independent & Geometric mixing & Too many non-sequential apertures \\
$B/u$ & Decreases & Memory leakage & Too slow: memory crosses plates \\
$Cu$ & Increases & Truncated swing & Too fast: pendulum interrupted \\
\bottomrule
\end{tabular}
\end{center}
\end{theorem}

\begin{remark}[Optimal Velocity Interpretation]
The optimal velocity $u_{\text{opt}} = \sqrt{B/C}$ balances memory leakage against truncated equilibration:
\begin{itemize}
\item Below $u_{\text{opt}}$: Memory leaks between plates, destroying statistical independence
\item Above $u_{\text{opt}}$: Pendulum swings truncated, destroying within-plate equilibration
\item At $u_{\text{opt}}$: Each plate achieves complete equilibration with full memory reset
\end{itemize}
\end{remark}

\begin{remark}[Connection to Turbulence]
Chromatography is ``tamed turbulence''---the non-sequential apertures (A-term) create turbulent-like mixing \emph{within} each plate, but memory reset at plate boundaries prevents global chaos. The Van Deemter equation quantifies the imperfection of this taming: the A-term is the irreducible turbulent contribution, while B and C terms represent memory leakage and incomplete equilibration that corrupt the taming.
\end{remark}

