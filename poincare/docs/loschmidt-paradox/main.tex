\documentclass[12pt,a4paper]{article}

% Packages
\usepackage[utf8]{inputenc}
\usepackage[T1]{fontenc}
\usepackage{amsmath,amssymb,amsthm}
\usepackage{mathtools}
\usepackage{geometry}
\usepackage{graphicx}
\usepackage{hyperref}
\usepackage{cleveref}
\usepackage{enumitem}
\usepackage{booktabs}
\usepackage{array}
\usepackage{natbib}
\usepackage{import}

% Geometry
\geometry{margin=1in}

% Theorem environments
\newtheorem{theorem}{Theorem}[section]
\newtheorem{lemma}[theorem]{Lemma}
\newtheorem{proposition}[theorem]{Proposition}
\newtheorem{corollary}[theorem]{Corollary}
\theoremstyle{definition}
\newtheorem{definition}[theorem]{Definition}
\newtheorem{example}[theorem]{Example}
\theoremstyle{remark}
\newtheorem{remark}[theorem]{Remark}

% Custom commands
\newcommand{\kB}{k_{\mathrm{B}}}
\newcommand{\dcat}{d_{\mathrm{cat}}}
\newcommand{\taulag}{\tau_{\mathrm{lag}}}
\newcommand{\Spart}{S_{\mathrm{part}}}

\title{\textbf{On the Resolution of Loschmidt's Paradox Through Categorical Partition Dynamics: Entropy as Geometric Structure Independent of Temporal Direction}}

\author{
Kundai Farai Sachikonye\\
\texttt{kundai.sachikonye@wzw.tum.de}
}

\date{\today}

\begin{document}

\maketitle

\begin{abstract}

Loschmidt's paradox observes that irreversible macroscopic thermodynamics cannot be derived from time-symmetric microscopic dynamics, since reversing all particle velocities would cause entropy to decrease, contradicting the Second Law. We resolve this paradox by demonstrating that entropy arises from categorical partition structure rather than from temporal dynamics. The partition-oscillation-category equivalence establishes that entropy $S = k_B M \ln n$ is a geometric property of categorical space, independent of the direction of time. Partition operations generate entropy through undetermined residue---categorical states that cannot be assigned to either the pre-partition or post-partition configuration---and this entropy production is invariant under velocity reversal.

The resolution proceeds through four theorems. First, the \textbf{Partition Entropy Theorem} establishes that every partition operation produces entropy $\Delta S = k_B \ln n_{\text{res}} > 0$, where $n_{\text{res}}$ is the count of undetermined residue states. Second, the \textbf{Measurement-Partition Identity} establishes that the velocity reversal required by Loschmidt's thought experiment is itself a partition operation: measuring all particle velocities creates categorical distinctions that generate entropy. Third, the \textbf{Categorical Irreversibility Theorem} proves that partition operations are topologically irreversible---composition cannot recover entropy lost to partition boundaries---regardless of the temporal direction of the underlying dynamics. Fourth, the \textbf{Stosszahlansatz Derivation Theorem} shows that Boltzmann's molecular chaos assumption is not an approximation but a necessary consequence of categorical structure: correlations that would permit entropy decrease reside in thermodynamically inaccessible undetermined residue.

The deepest insight emerges from considering non-actualisations: for any actualised state, infinitely many alternative states were not actualised. When a cup falls and breaks, it has not merely changed physical configuration---it has created infinitely many new non-actualisations (not reassembling, not melting, not teleporting). These non-actualisations are categorical facts that cannot be un-created. Time-reversal would require not only reversing the physical trajectory but also erasing these non-actualisations, which is categorically impossible. The asymmetry between actualisation (finite, specific) and non-actualisation (infinite, accumulating) provides the fundamental explanation for irreversibility.

The framework explains why Boltzmann's H-theorem holds despite time-symmetric dynamics. The H-function measures categorical completion---the fraction of phase space that has been partitioned into distinguishable states. Completion is irreversible because partition boundaries, once created, cannot be erased without generating additional entropy. Time-reversal of particle velocities does not un-partition the system; it merely changes the direction of partition accumulation while preserving the monotonic increase of total partition entropy.

A crucial observation completes the resolution: entropy change is only observable for processes that have terminated. An ongoing process has no definite entropy---it remains in the ``reality stream'' with indeterminate outcome. Once a process terminates, it becomes a categorical fact that cannot be reversed. This leads to the deepest insight: categorical completion and geometric partitioning are the same operation. Both select one outcome from many possibilities, create boundaries between actualised and non-actualised states, and generate entropy. The apparent paradox dissolves: irreversibility is not derived from temporal asymmetry but from the geometric structure of categorical space. The arrow of time is the direction of non-actualisation accumulation---the direction in which partition boundaries accumulate. Reactions should be measured not by clock time but by categorical completion rate, as time itself emerges from the ordering of completed categorical states.


\end{abstract}

\tableofcontents
\newpage

%==============================================================================
% SECTIONS
%==============================================================================

%==============================================================================
%==============================================================================
\section{Introduction}
\label{sec:introduction}
%==============================================================================

\subsection{Statement of the Paradox}

In 1876, Josef Loschmidt raised a fundamental objection to Boltzmann's H-theorem \citep{loschmidt1876}. Boltzmann had derived from kinetic theory that the H-function
\begin{equation}
H = \int f(\mathbf{v}) \ln f(\mathbf{v}) \, d^3v
\label{eq:h_function}
\end{equation}
decreases monotonically toward equilibrium, implying that entropy $S = -k_B H$ increases monotonically. This appeared to derive irreversible macroscopic behavior from the underlying Newtonian dynamics.

Loschmidt's objection was elegant and devastating: Newtonian mechanics is time-symmetric. If there exists a trajectory from state $A$ at time $t_0$ to state $B$ at time $t_1$ with decreasing $H$, then there must exist another trajectory---obtained by reversing all velocities at $t_1$---from state $B$ back to state $A$ with \emph{increasing} $H$. For every entropy-increasing trajectory, a time-reversed entropy-decreasing trajectory exists with equal dynamical validity.

The paradox can be stated precisely:

\begin{quote}
\textbf{Loschmidt's Paradox:} If the microscopic dynamics are time-symmetric, how can the macroscopic Second Law be time-asymmetric? Irreversible processes cannot be logically derived from reversible dynamics.
\end{quote}

The standard resolutions to this paradox invoke special initial conditions, probabilistic arguments, or cosmological boundary conditions \citep{boltzmann1896, lebowitz1993, penrose2004}. We propose a different resolution: the paradox rests on a false premise. Entropy does not arise from temporal dynamics at all. Entropy is a geometric property of categorical space that increases under partition operations, and partition operations generate entropy regardless of the temporal direction of the underlying dynamics.

\subsection{The Partition Framework}

The resolution requires an understanding of three fundamental equivalences that unify apparently distinct descriptions of physical systems.

\subsubsection{Oscillation, Category, and Partition}

Physical systems admit three complementary descriptions:

\paragraph{Oscillatory dynamics.} Any bounded physical system exhibits oscillatory behaviour. A particle in a box oscillates between the walls. An electron in an atom oscillates in its orbital. A molecule vibrates. Each oscillation has a characteristic frequency $\omega$ and period $T = 2\pi/\omega$. The oscillation is bounded: the system returns to configurations that are arbitrarily close to its initial state.

\paragraph{Categorical structure.} The same systems can be described by discrete quantum numbers $(n, \ell, m, s)$ that label energy levels, angular momentum states, magnetic orientations, and spin configurations. These quantum numbers are not continuous parameters; they take discrete values. The discreteness reflects categorical structure: the system occupies one category or another, never a continuous blend.

\paragraph{Partition geometry.} Phase space can be divided into distinct regions separated by boundaries. Each region represents a distinguishable macroscopic state. The boundaries are partition boundaries—they separate states that can be distinguished by macroscopic observation from states that cannot. The number of regions determines the entropy.

\subsubsection{The Fundamental Equivalence}

These three descriptions are mathematically identical. Each yields the same entropy formula:
\begin{equation}
S = k_B M \ln n
\label{eq:unified_entropy}
\end{equation}
where $M$ represents the dimensional depth (number of independent degrees of freedom) and $n$ the branching factor (number of distinguishable states per degree of freedom).

\paragraph{From oscillations:} An oscillator with frequency $\omega$ and energy $E$ has a quantum number $n = E/\hbar\omega$. For $M$ independent oscillators, the number of accessible states is $\Omega = n^M$, giving $S = k_B \ln \Omega = k_B M \ln n$.

\paragraph{From categories:} A system with $M$ categorical degrees of freedom, each taking $n$ possible values, has $\Omega = n^M$ distinguishable states, giving $S = k_B \ln \Omega = k_B M \ln n$.

\paragraph{From partitions:} Dividing phase space into $n^M$ regions (by creating $M$ independent partitions, each with $n$ subregions) gives $\Omega = n^M$ distinguishable macrostates, yielding $S = k_B \ln \Omega = k_B M \ln n$.

The equivalence is not approximate or analogical—it is exact. The three descriptions are different perspectives on the same underlying structure.

\subsubsection{Entropy as Geometric Structure}

The key insight: \textbf{entropy measures categorical structure, not temporal evolution}. 

In the oscillatory description, entropy counts the number of quantum states accessible at a given energy. This count does not depend on whether the system is evolving forward or backward in time—it depends only on the energy, which is conserved.

In the categorical description, entropy counts the number of distinguishable categories. This count is a property of the categorical space itself, independent of any dynamics within that space.

In the partition description, entropy counts the number of partition boundaries. These boundaries are geometric structures—they divide phase space into regions. Once created, they persist as topological features regardless of how particles move through the partitioned space.

This geometric interpretation immediately suggests a resolution to Loschmidt's paradox: if entropy arises from geometric structure rather than temporal dynamics, then reversing the direction of time cannot decrease entropy. The geometric structure persists under time-reversal.

\subsection{Undetermined Residue: The Source of Entropy}

Partition operations do not divide phase space cleanly. Every partition creates \emph{an undetermined residue---states} that cannot be definitively assigned to either the pre-partition or post-partition configuration.

\subsubsection{The Partition Lag}

Consider dividing a gas into "hot" and "cold" regions by inserting a partition. The division is not instantaneous. During the partition lag $\tau_{\text{lag}}$, molecules near the partition boundary have an ambiguous status:

\begin{itemize}
\item A molecule at position $x = L/2 \pm \delta x$ (where $L/2$ is the partition location and $\delta x$ is the thermal de Broglie wavelength) cannot be definitively assigned to the left or right region.

\item A molecule with velocity $v_x \approx 0$ is neither definitively moving left nor definitively moving right.

\item A molecule that crosses the partition during $\tau_{\text{lag}}$ belongs to both regions during the crossing.
\end{itemize}

These ambiguous states constitute the undetermined residue. They represent irreducible uncertainty: no refinement of measurement can eliminate them without creating new partition operations (and thus new residue).

\subsubsection{Residue Count and Entropy}

Let $n_{\text{res}}$ be the number of undetermined residue states created by a partition operation. The entropy generated by the partition is:
\begin{equation}
\Delta S = k_B \ln n_{\text{res}}
\label{eq:residue_entropy}
\end{equation}

This entropy is \emph{geometric}---it counts the number of boundary states that cannot be categorically assigned. It does not depend on the temporal direction of particle motion. Whether particles are moving forward or backward in time, the boundary states remain ambiguous.

\subsubsection{Why Residue Cannot Be Eliminated}

One might object: "Refine the partition to eliminate residue." But refinement is itself a partition operation that creates new residue:

\begin{itemize}
\item Original partition: Divide gas into "left" and "right" at $x = L/2$. Residue: molecules with $|x - L/2| < \delta x$.

\item Refined partition: Divide gas into "far left", "center left", "center right", "far right" at $x = L/4, L/2, 3L/4$. Residue: molecules near three boundaries instead of one.

\item Net effect: Residue count increases. Entropy increases.
\end{itemize}

Residue is not a defect of coarse partitions---it is an intrinsic feature of categorical structure. Any operation that creates distinctions creates boundaries, and boundaries create residue.

\subsection{Outline of Resolution}

We resolve Loschmidt's paradox through four theorems that establish entropy as a geometric property independent of temporal direction:

\begin{enumerate}
\item \textbf{Section~\ref{sec:partition_entropy} (Partition Entropy Theorem)}: Every partition operation produces entropy $\Delta S = k_B \ln n_{\text{res}} > 0$ through undetermined residue. This entropy production is invariant under time-reversal because residue counts depend on geometric structure (partition boundaries) rather than dynamical trajectories.

\item \textbf{Section~\ref{sec:measurement} (Measurement-Partition Identity)}: The velocity reversal required by Loschmidt's thought experiment is itself a partition operation. Measuring all particle velocities creates categorical distinctions (this velocity vs. that velocity) that generate entropy. The measurement entropy exceeds any entropy that could be recovered by reversing the dynamics, ensuring total entropy increases.

\item \textbf{Section~\ref{sec:irreversibility} (Categorical Irreversibility Theorem)}: Partition operations are topologically irreversible. Once a partition boundary is created, it cannot be erased without creating additional boundaries. Composition of partitions cannot recover entropy lost to boundaries. This irreversibility is geometric (topological), not temporal---it holds regardless of whether time flows forward or backward.

\item \textbf{Section~\ref{sec:stosszahlansatz} (Stosszahlansatz Derivation Theorem)}: Boltzmann's molecular chaos assumption (Stosszahlansatz) is not an approximation but a necessary consequence of categorical structure. Correlations between particles that would permit entropy decrease exist in principle, but they reside in the undetermined residue of prior partition operations. Accessing these correlations requires measurements that create new residue, generating more entropy than the correlations could recover. The Stosszahlansatz is exact for thermodynamically accessible states.
\end{enumerate}

These four theorems establish that irreversibility arises from \emph{categorical structure} (the geometry of partition boundaries) rather than \emph{temporal asymmetry} (special initial conditions or time-asymmetric laws). Loschmidt's paradox dissolves: time-reversal cannot decrease entropy because entropy is not a property of temporal evolution---it is a property of geometric structure that persists under time-reversal.

\subsection{The Deepest Insight: Non-Actualisation}

The most profound resolution emerges from considering what does \emph{not} happen. When a cup falls and breaks:

\begin{itemize}
\item \textbf{What actualises}: One specific trajectory. One specific pattern of cracks. One specific final configuration.

\item \textbf{What does not actualise}: Infinitely many alternatives. The cup does not reassemble. It does not melt. It does not teleport. It does not transform into a bird. Each non-actualisation is a categorical fact.
\end{itemize}

The asymmetry is fundamental:
\begin{itemize}
\item \textbf{Actualisation}: Finite. Specific. Singular. One outcome occurs.
\item \textbf{Non-actualisation}: Infinite. General. Accumulating. Infinitely many outcomes do not occur.
\end{itemize}

Time-reversal would require not only reversing the physical trajectory (possible in principle) but also \emph{erasing the non-actualisations} (categorically impossible). The cup's breaking created infinitely many new categorical facts: "did not reassemble", "did not melt", etc. These facts cannot be un-created. They are permanent additions to the structure of categorical space.

This provides the ultimate resolution: \textbf{the arrow of time is the direction of non-actualisation accumulation}. Time flows in the direction in which partition boundaries accumulate, in which categorical facts multiply, in which the structure of what-did-not-happen grows denser. Entropy measures this accumulation. Irreversibility is the impossibility of erasing categorical facts.

\subsection{Implications}

This resolution has profound implications:

\paragraph{Time emerges from completion.} Clock time $t$ is not fundamental. What is fundamental is the \emph{ordering of completed categorical states}. Time is the index that orders actualisations. The "flow" of time is the accumulation of non-actualisations.

\paragraph{Entropy is only observable for terminated processes.} An ongoing process has indeterminate entropy---it remains in the "reality stream" with uncertain outcome. Only when a process terminates (when actualisation selects one outcome and non-actualisation excludes all others) does entropy become definite.

\paragraph{Reactions should be measured by completion rate.} Chemical kinetics traditionally measures reaction rates in time: $d[\text{product}]/dt$. But time is emergent. The fundamental rate is the \emph{categorical completion rate}: how rapidly partition boundaries accumulate, how quickly non-actualisations multiply. This rate determines the thermodynamic arrow.

\paragraph{The Second Law is geometric.} Entropy increases not because of special initial conditions, not because of cosmological boundary conditions, not because of probabilistic fluctuations—but because \emph{partition boundaries cannot be erased}. The Second Law is a theorem of categorical geometry.

The remainder of this paper rigorously establishes these claims through the four theorems outlined above, demonstrates their application to Boltzmann's H-theorem, and derives experimental predictions that distinguish the categorical resolution from standard approaches.


\import{sections/}{partition-entropy.tex}

\import{sections/}{apertures.tex}

\import{sections/}{measurement.tex}

\import{sections/}{irreversibility.tex}

\import{sections/}{stosszahlansatz.tex}

\import{sections/}{h-theorem.tex}

\import{sections/}{non-actualisation.tex}

\import{sections/}{cross-sectional-validation.tex}


%==============================================================================
%==============================================================================
\section{Discussion}
\label{sec:discussion}
%==============================================================================

\subsection{Comparison with Standard Resolutions}

Several resolutions to Loschmidt's paradox exist in the literature. We compare our geometric resolution to the principal alternatives.

\subsubsection{Statistical Mechanics (Boltzmann)}

\textbf{Standard resolution}: Boltzmann argued that entropy-decreasing trajectories exist but are vastly outnumbered by entropy-increasing trajectories \citep{boltzmann1896}. The probability of observing entropy decrease is:
\begin{equation}
P(\Delta S < 0) \sim e^{-|\Delta S|/k_B} \approx e^{-10^{23}}
\end{equation}
for macroscopic systems. This is effectively zero.

\textbf{Limitation}: This statistical argument is correct but incomplete. It does not explain:
\begin{itemize}
\item Why the initial state was low-entropy (requires Past Hypothesis)
\item Why correlations that would permit entropy decrease are absent (assumes molecular chaos)
\item Why measurement is required for velocity reversal (treats measurement as passive observation)
\end{itemize}

\textbf{Our resolution extends Boltzmann's}: Partition structure explains both why entropy typically increases (partition operations produce entropy, Theorem~\ref{thm:partition_entropy}) and why special correlations are inaccessible (they reside in undetermined residue, Theorem~\ref{thm:correlation_inaccessibility}).

The statistical argument is a consequence, not a foundation. Entropy-decreasing trajectories are rare because they require accessing correlations hidden in partition residue, which generates more entropy than could be recovered.

\subsubsection{Cosmological Boundary Conditions (Penrose)}

\textbf{Standard resolution}: Penrose argues that the low entropy of the early universe provides the ultimate explanation for the arrow of time \citep{penrose2004}. The "Past Hypothesis"---that the universe began in a very special low-entropy state---is taken as a fundamental postulate.

\textbf{Limitation}: This resolution:
\begin{itemize}
\item Invokes cosmology to explain laboratory thermodynamics
\item Does not explain why the initial state was special (merely asserts it as boundary condition)
\item Does not explain why entropy continues to increase (requires additional dynamics)
\item Cannot explain irreversibility in subsystems isolated from cosmological effects
\end{itemize}

\textbf{Our resolution provides deeper explanation}: The early universe had low entropy because few partition operations had occurred. The "specialness" of the initial state is not mysterious---it is simply categorical incompleteness:
\begin{equation}
S_{\text{early}} = k_B \sum_{i=1}^{N_{\text{boundaries}}^{\text{early}}} \ln n_i \quad \text{with } N_{\text{boundaries}}^{\text{early}} \ll N_{\text{boundaries}}^{\text{now}}
\end{equation}

Entropy has increased because partitions have accumulated (Corollary~\ref{cor:monotonic_boundaries}), not because of special initial conditions. The Past Hypothesis is unnecessary---entropy increase follows from topological irreversibility (Theorem~\ref{thm:topological_irreversibility}) regardless of initial conditions.

\subsubsection{Information-Theoretic (Landauer-Bennett)}

\textbf{Standard resolution}: Bennett's resolution via Landauer's principle \citep{landauer1961, bennett1982} identifies measurement and information erasure as the key obstacles to entropy reversal. Erasing one bit of information generates entropy:
\begin{equation}
\Delta S_{\text{erasure}} \geq k_B \ln 2
\end{equation}

Velocity reversal requires measuring all velocities, storing them, and eventually erasing the memory. The erasure entropy exceeds any entropy that could be recovered.

\textbf{Relationship to our resolution}: Bennett's resolution is closest to ours. Both identify measurement as the key obstacle. Our partition framework provides the foundation for Landauer's principle:

\begin{theorem}[Landauer's Principle from Partition Entropy]
\label{thm:landauer_from_partition}
Landauer's principle is a consequence of partition entropy. Information erasure is a partition operation that generates entropy:
\begin{equation}
\Delta S_{\text{erasure}} = k_B \ln n_{\text{res}}^{\text{erasure}} \geq k_B \ln 2
\end{equation}
\end{theorem}

\begin{proof}
Erasing one bit of information merges two categories ("0" and "1") into one category ("0"). By Theorem~\ref{thm:topological_irreversibility}, merging categories requires:
\begin{enumerate}
\item Identifying which category to erase (partition operation)
\item Performing the merge (creates residue)
\item Verifying the erasure (partition operation)
\end{enumerate}

The residue count is $n_{\text{res}}^{\text{erasure}} \geq 2$ (at least two outcomes: successful erasure or failed erasure). By Theorem~\ref{thm:partition_entropy}:
\begin{equation}
\Delta S_{\text{erasure}} = k_B \ln n_{\text{res}}^{\text{erasure}} \geq k_B \ln 2
\end{equation}

This is Landauer's principle. \qed
\end{proof}

\textbf{Our resolution is more fundamental}: We show that measurement itself generates entropy (Corollary~\ref{cor:velocity_measurement}), before any erasure. The entropy cost appears at the measurement step, not the erasure step. This is a deeper result because it applies even to measurements that are never erased.

\subsection{Implications for the Resolution of Loschmidt's Paradox}

\subsubsection{Entropy as Geometric Structure}

The partition-theoretic resolution establishes that entropy is a geometric property of categorical space, not a property of temporal evolution:

\begin{equation}
S = k_B \sum_{i=1}^{N_{\text{boundaries}}} \ln n_i
\end{equation}

This formula (Theorem~\ref{thm:entropy_boundary}) counts partition boundaries in configuration space. Entropy measures the fineness of the partition structure---how many categorical distinctions have been created.

\textbf{Implication for Loschmidt's paradox}: Entropy is not a property of trajectories (paths through phase space). It is a property of boundaries (structure in configuration space). Time-reversing trajectories does not reverse entropy because it does not remove boundaries.

\subsubsection{Irreversibility Without Time-Asymmetric Dynamics}

The resolution demonstrates that irreversibility does not require time-asymmetric dynamics:

\begin{theorem}[Compatibility of Reversible Dynamics and Irreversible Entropy]
\label{thm:compatibility}
Time-symmetric microscopic dynamics are fully compatible with macroscopic irreversibility:
\begin{equation}
\text{Time-symmetric dynamics} + \text{Topological boundary persistence} \Rightarrow \text{Irreversible entropy increase}
\end{equation}
\end{theorem}

\begin{proof}
By Theorem~\ref{thm:time_reversal_boundaries}, partition boundaries are time-reversal invariant:
\begin{equation}
\mathcal{T}[\partial] = \partial
\end{equation}

By Theorem~\ref{thm:topological_irreversibility}, boundaries cannot be erased:
\begin{equation}
\Delta N_{\text{boundaries}} \geq 0
\end{equation}

By Theorem~\ref{thm:entropy_boundary}, entropy counts boundaries:
\begin{equation}
S = k_B \sum_i \ln n_i
\end{equation}

Therefore:
\begin{equation}
\frac{dS}{dt} = k_B \sum_{\text{new}} \ln n_i \geq 0
\end{equation}

This holds regardless of whether the dynamics are time-symmetric or time-asymmetric. Irreversibility arises from boundary persistence (topological), not from temporal asymmetry (dynamical). \qed
\end{proof}

\textbf{Implication for Loschmidt's paradox}: The paradox assumed that time-symmetric dynamics should produce time-symmetric entropy evolution. This assumption is false. Entropy evolution is determined by boundary accumulation, not by temporal symmetry of dynamics.

\subsubsection{The Arrow of Time as Boundary Accumulation}

The thermodynamic arrow of time is not fundamental. It is a consequence of observing systems from within categorical space, where partition boundaries accumulate:

\begin{definition}[Arrow of Time]
\label{def:arrow_of_time}
The arrow of time is the direction of monotonic boundary accumulation:
\begin{equation}
\vec{\tau} = \nabla_{N_{\text{boundaries}}} S
\end{equation}
where $\vec{\tau}$ points in the direction of increasing boundary count.
\end{definition}

\textbf{Implication for Loschmidt's paradox}: Time-reversal reverses the dynamical trajectories (velocities) but not the arrow of time (boundary accumulation direction). Therefore, entropy increases along both forward and time-reversed trajectories (Corollary~\ref{cor:both_directions}).

An observer outside categorical space (if such were possible) would see time-symmetric dynamics with no preferred direction. The arrow of time is observer-relative---it depends on being embedded in categorical space where boundaries accumulate.

\subsection{Entropy Is Only Observable in Terminated Processes}

A critical but often overlooked point: entropy change can only be measured for processes that have \emph{terminated}. An ongoing process has no definite entropy---it is still part of the "reality stream" and has not yet become a categorical fact.

\begin{theorem}[Entropy Requires Termination]
\label{thm:entropy_termination}
The entropy change $\Delta S$ of a process is only defined for processes that have terminated. Ongoing processes have indeterminate entropy.
\end{theorem}

\begin{proof}
Consider a process evolving from state $A$ toward state $B$. At any intermediate time $t < t_{\text{final}}$:

\begin{itemize}
\item The process has not completed
\item The final state is not yet determined
\item Multiple outcomes remain possible: $\{B_1, B_2, \ldots, B_n\}$
\item Entropy change depends on which outcome actualizes:
\begin{equation}
\Delta S = S(B_i) - S(A) \quad \text{depends on } i
\end{equation}
\end{itemize}

Only when the process terminates at $t = t_{\text{final}}$ does the final state become definite. Only then can $\Delta S = S(B) - S(A)$ be computed.

Before termination, asking "what is the entropy change?" is asking about a fact that does not yet exist. The question is ill-posed.

\textbf{Analogy}: Asking for the entropy of an ongoing process is like asking for the final score of a game that is still being played. The question has no answer until the game terminates. \qed
\end{proof}

\begin{corollary}[Termination Implies Irreversibility]
\label{cor:termination_irreversibility}
A terminated process cannot be reversed because termination is categorical completion, and categorical states cannot be un-completed.
\end{corollary}

\begin{proof}
When a process terminates:

\begin{enumerate}
\item It becomes a completed categorical state $\mathcal{C}_{\text{final}}$
\item This state is a categorical fact: "the process terminated in state $B$"
\item By Theorem~\ref{thm:non_actualisation_irreversibility}, categorical facts cannot be erased
\item "Reversal" would require returning to the pre-termination configuration
\item But the pre-termination configuration was in the reality stream, not yet categorically complete
\item The only way to reach it is through a \emph{new} process that terminates at a similar spatial configuration
\item This new termination creates a \emph{new} categorical state $\mathcal{C}_{\text{new}}$ distinct from $\mathcal{C}_{\text{final}}$
\end{enumerate}

Therefore, reversal is impossible---one can only create new categorical states that happen to have similar spatial configurations. The categorical history cannot be reversed. \qed
\end{proof}

\textbf{Implication for Loschmidt's paradox}: Loschmidt's velocity reversal assumes we can reverse a completed process. But a completed process has terminated---it is a categorical fact. Reversing it would require un-terminating it, which is categorically impossible.

Even if we restore the spatial configuration (positions and velocities), we cannot restore the categorical history. The system has completed one trajectory (categorical fact). Reversing creates a new trajectory (new categorical fact). Both facts persist. The total boundary count increases.

\subsection{Categorical Completion Is Geometric Partitioning}

The deepest insight connecting our resolution to the broader framework:

\begin{theorem}[Categorical Completion = Partition Operation]
\label{thm:completion_partition}
Categorical completion is identical to geometric partitioning. When a process terminates:

\begin{enumerate}
\item It selects one outcome from many possibilities (partition)
\item It creates a boundary between actualised and non-actualised states
\item It generates undetermined residue (the non-actualisations)
\item It produces entropy $\Delta S = k_B \ln n_{\text{res}}$
\end{enumerate}

\textbf{Categorical completion and partition are the same operation viewed from different perspectives.}
\end{theorem}

\begin{proof}
\paragraph{Step 1: Partition structure.}

Consider a partition operation that divides a set of possibilities $\Omega$ into "selected" (actualised) and "not selected" (non-actualised):
\begin{equation}
\Omega = \Omega_{\text{actual}} \sqcup \Omega_{\text{non-actual}}
\end{equation}

where $\sqcup$ denotes disjoint union: $\Omega_{\text{actual}} \cap \Omega_{\text{non-actual}} = \emptyset$.

\paragraph{Step 2: Categorical completion structure.}

Consider a process that terminates by selecting one outcome $\omega^* \in \Omega$ from many possibilities:

\begin{itemize}
\item \textbf{Before termination}: All of $\Omega$ is possible (in the reality stream)
\item \textbf{At termination}: One state $\omega^*$ is selected (actualised)
\item \textbf{After termination}: $\omega^*$ becomes categorical fact; $\Omega \setminus \{\omega^*\}$ becomes non-actualisations
\end{itemize}

\paragraph{Step 3: Identity.}

The partition structure and categorical completion structure are identical:
\begin{align}
\Omega_{\text{actual}} &= \{\omega^*\} \quad \text{(the selected outcome)} \\
\Omega_{\text{non-actual}} &= \Omega \setminus \{\omega^*\} \quad \text{(all other possibilities)}
\end{align}

The partition boundary is the termination event itself. The non-actualised possibilities $\Omega \setminus \{\omega^*\}$ are the undetermined residue generating entropy:
\begin{equation}
\Delta S = k_B \ln |\Omega \setminus \{\omega^*\}| = k_B \ln(|\Omega| - 1) \approx k_B \ln |\Omega|
\end{equation}

for $|\Omega| \gg 1$.

\paragraph{Conclusion.}

Categorical completion and geometric partition are the same operation:
\begin{equation}
\boxed{\text{Categorical completion} \equiv \text{Geometric partition}}
\end{equation}

They are identical, viewed from different perspectives:
\begin{itemize}
\item \textbf{Categorical perspective}: A process terminates, selecting one outcome
\item \textbf{Geometric perspective}: Configuration space is partitioned, creating a boundary
\end{itemize}

Both perspectives describe the same physical reality. \qed
\end{proof}

\begin{corollary}[Reactions Should Be Measured by Completion Rate]
\label{cor:completion_rate}
Chemical and physical reactions should be characterized by their categorical completion rate $\dot{C} = dC/dt$, not by clock time alone.
\end{corollary}

\begin{proof}
Time itself emerges from categorical completion---it is the ordering of completed states, not an external parameter. Therefore:

\begin{itemize}
\item Clock time $t$ is a derived quantity, emergent from completion dynamics
\item The fundamental measure is the number of categorical states completed: $C(t)$
\item Reaction "rate" in the deepest sense is $\dot{C} = dC/dt$, not $d[\text{Product}]/dt$
\end{itemize}

Two reactions completing the same number of categorical states have the same fundamental "progress," even if they differ in clock time.

\textbf{Example}: A fast reaction at high temperature and a slow reaction at low temperature may complete the same number of categorical states (same number of partition operations) in different clock times. Their fundamental progress is the same. \qed
\end{proof}

\begin{remark}[Universal Irreversibility]
This explains why irreversibility appears universal: every physical process is a sequence of categorical completions (partitions). Each completion:

\begin{enumerate}
\item Terminates a portion of the reality stream
\item Creates a geometric boundary (partition)
\item Generates non-actualisations that cannot be erased
\item Produces entropy $\Delta S = k_B \ln n_{\text{res}} > 0$
\end{enumerate}

Loschmidt's paradox asked: "Why can't we reverse entropy increase?"

The answer is now complete:

\begin{center}
\fbox{\begin{minipage}{0.9\textwidth}
\textbf{Entropy increase IS categorical completion IS geometric partitioning.}

Reversal would require un-partitioning---erasing the boundary between actual and non-actual. But boundaries, once created, are permanent features of categorical geometry (Theorem~\ref{thm:topological_irreversibility}).

They define the structure of what has happened versus what has not happened. Erasing them would be erasing the distinction between being and non-being.

This is not merely difficult---it is categorically impossible.
\end{minipage}}
\end{center}
\end{remark}

%==============================================================================
\section{Conclusion}
\label{sec:conclusion}
%==============================================================================

Loschmidt's paradox dissolves when entropy is recognized as a geometric property of categorical space rather than a temporal property of dynamical evolution. The resolution rests on eight key results:

\begin{enumerate}
\item \textbf{Partition Entropy Theorem (Theorem~\ref{thm:partition_entropy})}: Every partition operation produces entropy $\Delta S = k_B \ln n_{\text{res}} > 0$ through undetermined residue. This entropy is unavoidable---it arises from the geometric structure of partition boundaries, not from statistical averaging or coarse-graining.

\item \textbf{Temporal Independence (Theorem~\ref{thm:temporal_independence})}: Partition entropy is invariant under time-reversal. Entropy increases regardless of the direction of temporal evolution:
\begin{equation}
\Delta S[\gamma(t)] = \Delta S[\gamma(-t)]
\end{equation}
for any trajectory $\gamma$.

\item \textbf{Measurement-Partition Identity (Theorem~\ref{thm:measurement_partition})}: The velocity reversal required by Loschmidt's thought experiment is a partition operation. Measuring $N$ particle velocities generates entropy:
\begin{equation}
\Delta S_{\text{measure}} = N k_B \ln n_{\text{res}}^{\text{total}} \approx 16 N k_B
\end{equation}
This exceeds any entropy that could be recovered by reversed evolution (Theorem~\ref{thm:measurement_barrier}).

\item \textbf{Topological Irreversibility (Theorem~\ref{thm:topological_irreversibility})}: Partition boundaries cannot be erased without creating additional boundaries:
\begin{equation}
\Delta N_{\text{boundaries}}[\text{any operation}] \geq 0
\end{equation}
Irreversibility is geometric (boundary persistence), not temporal (time-asymmetric dynamics).

\item \textbf{Stosszahlansatz as Theorem (Corollary~\ref{cor:stosszahlansatz})}: Molecular chaos is a necessary consequence of partition structure, not an assumption. Correlations permitting entropy decrease exist in principle but are thermodynamically inaccessible---they reside in partition residue, and accessing them generates more entropy than they could recover (Theorem~\ref{thm:correlation_inaccessibility}).

\item \textbf{Non-Actualisation Asymmetry (Theorem~\ref{thm:non_actualisation_accumulation})}: Every actualisation creates infinitely many non-actualisations. At radius $r$, the ratio is:
\begin{equation}
\frac{N_{\text{non-act}}(r)}{\Omega_{\text{act}}(r)} = \frac{e^{-\alpha t}}{1 - e^{-\alpha t}} \gg 1
\end{equation}
Time-reversal would require un-creating these non-actualisations, which is categorically impossible (Theorem~\ref{thm:non_actualisation_irreversibility}). The arrow of time is the direction of non-actualisation accumulation.

\item \textbf{Entropy Requires Termination (Theorem~\ref{thm:entropy_termination})}: Entropy change is only defined for processes that have terminated. Ongoing processes have indeterminate entropy---they are still in the "reality stream." Once terminated, a process cannot be reversed because termination is categorical completion (Corollary~\ref{cor:termination_irreversibility}).

\item \textbf{Categorical Completion = Geometric Partitioning (Theorem~\ref{thm:completion_partition})}: Categorical completion and partition operations are identical. Both select one outcome from many, create boundaries between actualised and non-actualised states, and generate entropy. Irreversibility is the impossibility of erasing partition boundaries.
\end{enumerate}

\subsection{The Deepest Insight: Non-Actualisation Asymmetry}

The non-actualisation asymmetry provides the deepest insight into irreversibility. When a cup falls and breaks, it does not merely change physical configuration---it generates infinitely many new categorical facts about what it is \emph{not} doing:

\begin{itemize}
\item Not reassembling
\item Not melting
\item Not teleporting
\item Not transforming into a bird
\item Not remaining intact
\item ... (infinitely many non-actualisations)
\end{itemize}

These non-actualisations are categorical facts. They record what did not happen. They cannot be erased even if the physical configuration is restored.

\textbf{Reversing the physical trajectory of particles is conceivable; reversing the categorical history of non-actualisations is not.}

Loschmidt's thought experiment focused on the former (reversing particle trajectories) while ignoring the latter (reversing non-actualisation accumulation). The paradox dissolves when we recognize that:

\begin{equation}
\text{Physical reversal} \neq \text{Categorical reversal}
\end{equation}

Physical reversal (negating velocities) is possible in principle. Categorical reversal (erasing non-actualisations) is impossible in principle.

Entropy measures categorical structure (non-actualisations), not physical configuration (particle positions). Therefore, entropy cannot be reversed even if physical configuration is reversed.

\subsection{Resolution of the Apparent Conflict}

The resolution reveals that the apparent conflict between time-symmetric dynamics and irreversible thermodynamics was based on a false premise: that irreversibility must derive from temporal asymmetry.

\textbf{The premise is false.}

Irreversibility derives from categorical structure---the geometry of partition space---which is independent of temporal direction:

\begin{center}
\begin{tabular}{lcc}
\toprule
\textbf{Aspect} & \textbf{Dynamics} & \textbf{Thermodynamics} \\
\midrule
Describes & Motion in phase space & Structure in categorical space \\
Governed by & Hamilton's equations & Partition accumulation \\
Time symmetry & Symmetric ($\mathcal{T}$-invariant) & Asymmetric (boundaries accumulate) \\
Reversibility & Reversible (trajectories) & Irreversible (boundaries) \\
\midrule
\textbf{Compatibility} & \multicolumn{2}{c}{Fully compatible---describe different aspects} \\
\bottomrule
\end{tabular}
\end{center}

Time-symmetric microscopic dynamics and irreversible macroscopic thermodynamics are fully compatible because they describe different aspects of physical reality:

\begin{itemize}
\item \textbf{Dynamics} describes motion: how particles move through phase space
\item \textbf{Thermodynamics} describes structure: how partition boundaries accumulate in categorical space
\end{itemize}

There is no conflict. The appearance of conflict arose from conflating motion (which is reversible) with structure (which is irreversible).

\subsection{Final Statement}

Loschmidt's paradox, far from revealing a flaw in thermodynamics, confirms the geometric nature of entropy and the independence of irreversibility from temporal direction.

The key insights:

\begin{enumerate}
\item \textbf{Entropy is geometric}: $S = k_B \sum_i \ln n_i$ counts partition boundaries, not temporal evolution

\item \textbf{Irreversibility is topological}: Boundaries cannot be erased (Theorem~\ref{thm:topological_irreversibility}), not because of time-asymmetric laws, but because of geometric persistence

\item \textbf{Time-reversal preserves boundaries}: $\mathcal{T}[\partial] = \partial$ (Theorem~\ref{thm:time_reversal_boundaries}), so entropy increases in both temporal directions

\item \textbf{Measurement generates entropy}: $\Delta S_{\text{measure}} \geq |\Delta S_{\text{reverse}}|$ (Theorem~\ref{thm:measurement_barrier}), making Loschmidt's reversal impossible

\item \textbf{Non-actualisations dominate}: $N_{\text{non-act}} \gg \Omega_{\text{act}}$ (Corollary~\ref{cor:non_act_dominate}), creating an asymmetric gradient that defines the arrow of time

\item \textbf{Completion is partition}: Categorical completion $\equiv$ geometric partitioning (Theorem~\ref{thm:completion_partition}), unifying the categorical and geometric perspectives
\end{enumerate}

The arrow of time is not imposed from outside physics---it emerges from the logical asymmetry between finite actualisation and infinite non-actualisation. Every process actualizes one outcome while creating infinitely many non-actualisations. This asymmetry is not statistical (99.9\% vs. 0.1\%) but absolute (1 vs. $\infty$).

Loschmidt asked: "If dynamics are time-symmetric, why can't we reverse entropy increase?"

The answer: Because entropy measures categorical structure (partition boundaries and non-actualisations), not dynamical trajectories. Time-reversal reverses trajectories but not structure. Structure accumulates monotonically regardless of temporal direction.

\textbf{The paradox is resolved.}




%==============================================================================
% Bibliography
%==============================================================================

\bibliographystyle{plainnat}
\bibliography{references}

\end{document}
