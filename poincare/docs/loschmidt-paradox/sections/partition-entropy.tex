%==============================================================================
\section{Partition Entropy: Independent of Time}
\label{sec:partition_entropy}
%==============================================================================

\subsection{The Partition Operation}

\begin{definition}[Partition Operation]
\label{def:partition}
A partition operation $\Pi: \mathcal{C} \to \mathcal{C}_1 \sqcup \mathcal{C}_2$ divides a categorical state $\mathcal{C}$ into distinguishable sub-states $\mathcal{C}_1$ and $\mathcal{C}_2$. The partition creates a categorical boundary separating states that were previously indistinguishable.
\end{definition}

\begin{definition}[Partition Lag]
\label{def:partition_lag}
The partition lag $\taulag$ is the irreducible temporal interval between initiating a partition and establishing the partitioned result:
\begin{equation}
\taulag = t_{\text{partitioned}} - t_{\text{initiate}} > 0
\label{eq:partition_lag}
\end{equation}
\end{definition}

\begin{theorem}[Positive Partition Lag]
\label{thm:positive_lag}
Partition operations require positive time: $\taulag > 0$ for all partitions.
\end{theorem}

\begin{proof}
Partitioning distinguishes between categorical states. Distinguishing requires information acquisition about which side of the partition boundary a state occupies. Information acquisition in physical systems requires finite time by causality constraints. Hence $\taulag > 0$. \qed
\end{proof}

\subsection{Undetermined Residue}

\begin{definition}[Undetermined Residue]
\label{def:residue}
During partition lag $\taulag$, the system exists in undetermined superposition across the partition boundary. The undetermined residue $n_{\text{res}}$ counts states that cannot be assigned to either $\mathcal{C}_1$ or $\mathcal{C}_2$ during the partition interval.
\end{definition}

\begin{theorem}[Partition Entropy Production]
\label{thm:partition_entropy}
Every partition operation produces entropy:
\begin{equation}
\Delta S = \kB \ln n_{\text{res}} > 0
\label{eq:partition_entropy}
\end{equation}
\end{theorem}

\begin{proof}
Undetermined residue represents states that contribute to $\ln n_{\text{res}}$ entropy during partition lag. By Theorem~\ref{thm:positive_lag}, $\taulag > 0$, hence undetermined residue exists during every partition. Since $n_{\text{res}} \geq 2$ (at least one state on each side of the boundary is temporarily undetermined), $\Delta S = \kB \ln n_{\text{res}} \geq \kB \ln 2 > 0$. \qed
\end{proof}

\subsection{Temporal Independence}

\begin{theorem}[Temporal Independence of Partition Entropy]
\label{thm:temporal_independence}
Partition entropy production is independent of the temporal direction of the underlying dynamics:
\begin{equation}
\Delta S(\Pi, t \to t') = \Delta S(\Pi, t' \to t) = \kB \ln n_{\text{res}}
\label{eq:temporal_independence}
\end{equation}
\end{theorem}

\begin{proof}
The entropy $\Delta S = \kB \ln n_{\text{res}}$ depends only on:
\begin{enumerate}
\item The categorical structure being partitioned
\item The number of states in undetermined residue
\end{enumerate}

Neither of these quantities depends on whether time flows forward ($t \to t'$) or backward ($t' \to t$). The partition creates categorical boundaries regardless of temporal direction. The undetermined residue exists during partition lag regardless of which direction the lag is measured.

Formally, let $\mathcal{T}: t \to -t$ denote time reversal. The partition operation $\Pi$ acts on categorical structure $\mathcal{C}$, which is a configuration space property, not a phase space property. Under time reversal, velocities $\mathbf{v} \to -\mathbf{v}$ but configurations $\mathbf{x} \to \mathbf{x}$ are unchanged. Since $\mathcal{C}$ depends only on configurations:
\begin{equation}
\mathcal{T}[\Pi(\mathcal{C})] = \Pi(\mathcal{T}[\mathcal{C}]) = \Pi(\mathcal{C})
\end{equation}

The partition operation commutes with time reversal. Therefore, partition entropy is time-reversal invariant. \qed
\end{proof}

\begin{corollary}[Entropy Increases in Both Temporal Directions]
\label{cor:both_directions}
Under time-symmetric dynamics, entropy increases regardless of the direction of temporal evolution.
\end{corollary}

\begin{proof}
Let $\gamma$ be a trajectory from $A$ to $B$ and $\gamma^R$ be the time-reversed trajectory from $B$ to $A$. Along $\gamma$, partition operations produce entropy $\Delta S_\gamma > 0$. Along $\gamma^R$, partition operations produce entropy $\Delta S_{\gamma^R} > 0$.

Both trajectories increase entropy. The time-reversal symmetry of the underlying dynamics does not imply time-reversal of entropy production, because partition entropy is temporal-direction-independent. \qed
\end{proof}

