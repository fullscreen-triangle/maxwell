%==============================================================================
\section{Measurement as Partition}
\label{sec:measurement}
%==============================================================================

\subsection{Loschmidt's Velocity Reversal}

Loschmidt's paradox requires reversing all particle velocities at some instant $t_1$. This operation, while conceptually simple, has specific physical requirements:

\begin{enumerate}
\item Measure all particle velocities $\{\mathbf{v}_i\}$
\item Negate each velocity: $\mathbf{v}_i \to -\mathbf{v}_i$
\item Allow the system to evolve under time-symmetric dynamics
\end{enumerate}

The paradox claims that after reversal, entropy should decrease along the reversed trajectory. We now show that the measurement step (1) generates entropy that exceeds any subsequent decrease.

\subsection{Measurement-Partition Identity}

\begin{theorem}[Measurement-Partition Identity]
\label{thm:measurement_partition}
Every measurement is a partition operation. Measuring a property $O$ partitions the system's categorical space into eigenstates of $O$.
\end{theorem}

\begin{proof}
Before measurement, the system may occupy any state compatible with its prior constraints. After measurement, the system is known to occupy a specific eigenstate (or eigenspace) of the measured observable. The measurement creates a categorical distinction: ``measured value $= o_i$'' versus ``measured value $\neq o_i$''.

This distinction is precisely a partition. The measurement partitions the accessible categorical space into regions corresponding to different measurement outcomes. \qed
\end{proof}

\begin{corollary}[Velocity Measurement Entropy]
\label{cor:velocity_measurement}
Measuring the velocity of $N$ particles produces entropy:
\begin{equation}
\Delta S_{\text{measure}} = N \cdot \kB \ln n_v
\label{eq:velocity_measurement_entropy}
\end{equation}
where $n_v$ is the number of distinguishable velocity states per particle.
\end{corollary}

\begin{proof}
Each particle's velocity measurement is an independent partition operation. With $n_v$ possible outcomes per measurement, each measurement produces entropy $\kB \ln n_v$. For $N$ particles:
\begin{equation}
\Delta S_{\text{measure}} = N \cdot \kB \ln n_v
\end{equation}
\qed
\end{proof}

\subsection{The Measurement Barrier}

\begin{theorem}[Measurement Barrier to Entropy Reversal]
\label{thm:measurement_barrier}
The entropy generated by velocity measurement exceeds any entropy that could be recovered by subsequent reversed evolution:
\begin{equation}
\Delta S_{\text{measure}} > |\Delta S_{\text{reverse}}|
\label{eq:measurement_barrier}
\end{equation}
\end{theorem}

\begin{proof}
Let $\Delta S_{\text{forward}}$ be the entropy produced along the forward trajectory from $A$ to $B$. By the H-theorem, this entropy arises from molecular collisions that partition velocity space.

The reversed trajectory, if it could be executed perfectly, would ``un-collide'' the molecules, potentially recovering entropy. However, this recovery is bounded:
\begin{equation}
|\Delta S_{\text{reverse}}| \leq \Delta S_{\text{forward}}
\end{equation}

The measurement required to implement the reversal produces:
\begin{equation}
\Delta S_{\text{measure}} = N \cdot \kB \ln n_v
\end{equation}

For a macroscopic system with $N \sim 10^{23}$ particles, even modest velocity resolution ($n_v \sim 10^3$) gives:
\begin{equation}
\Delta S_{\text{measure}} \sim 10^{23} \kB \cdot 7 \approx 10^{24} \kB
\end{equation}

This vastly exceeds any entropy change along the trajectory, which scales as:
\begin{equation}
\Delta S_{\text{forward}} \sim N \kB \cdot \mathcal{O}(1)
\end{equation}

Therefore, $\Delta S_{\text{measure}} \gg |\Delta S_{\text{reverse}}|$. \qed
\end{proof}

\begin{remark}
This resolves Loschmidt's paradox directly: the velocity reversal cannot be implemented without generating more entropy than it could possibly recover. The thought experiment is self-defeating.
\end{remark}

\subsection{Comparison with Information-Theoretic Resolution}

The information-theoretic resolution of Loschmidt's paradox, developed through the work of Szilard, Landauer, and Bennett \citep{szilard1929, landauer1961, bennett1982}, reaches a similar conclusion through different reasoning: velocity reversal requires measurement, measurement acquires information, and information erasure (required to reset the measuring device) generates entropy via Landauer's principle.

Our partition-theoretic resolution is more fundamental:

\begin{enumerate}
\item \textbf{No information concept required:} Partition entropy arises from categorical structure, not from Shannon information.

\item \textbf{No erasure required:} Entropy is produced during measurement itself (partition), not during subsequent erasure.

\item \textbf{Geometric rather than computational:} Irreversibility is a property of categorical space, not of computation or information processing.
\end{enumerate}

The partition framework explains \emph{why} Landauer's principle holds: information erasure is a partition operation (distinguishing ``erased'' from ``not erased''), and all partition operations produce entropy.

