%==============================================================================
\subsection{Partition Boundaries as Categorical Apertures}
\label{subsec:apertures}
%==============================================================================

The partition framework acquires additional structure when partition boundaries are recognised as \emph{apertures}—geometric constraints that selectively allow certain configurations to pass while blocking others.

\begin{definition}[Categorical Aperture]
\label{def:aperture}
An \emph{aperture} $a$ is a partition boundary with a selection function:
\begin{equation}
\sigma_a(\omega) = \begin{cases}
1 & \text{if } \text{config}(\omega) \in \text{shape}(a) \\
0 & \text{otherwise}
\end{cases}
\end{equation}
where $\text{config}(\omega)$ is the configuration of state $\omega$ and $\text{shape}(a)$ is the set of configurations compatible with passage through aperture $a$.
\end{definition}

\begin{theorem}[Aperture Potential]
\label{thm:aperture_potential}
Each aperture carries a categorical potential:
\begin{equation}
\Phi_a = -\kB T \ln\left(\frac{\Omega_{\text{pass}}}{\Omega_{\text{total}}}\right) = -\kB T \ln s
\end{equation}
where $s = \Omega_{\text{pass}}/\Omega_{\text{total}}$ is the \emph{selectivity} of the aperture—the fraction of configurations that can pass.
\end{theorem}

\begin{proof}
The aperture restricts the accessible configuration space from $\Omega_{\text{total}}$ to $\Omega_{\text{pass}}$. By the Boltzmann relation, this restriction corresponds to an entropy reduction:
\begin{equation}
\Delta S = \kB \ln \Omega_{\text{pass}} - \kB \ln \Omega_{\text{total}} = \kB \ln s
\end{equation}

Since $s \leq 1$, we have $\ln s \leq 0$, so $\Delta S \leq 0$. The energy required to enforce this restriction is:
\begin{equation}
\Phi_a = -T \Delta S = -\kB T \ln s
\end{equation}
\end{proof}

\begin{corollary}[Selectivity Extremes]
\label{cor:selectivity}
The aperture potential has limiting cases:
\begin{enumerate}
\item $s = 1$ (no selectivity, all configurations pass): $\Phi_a = 0$
\item $s \to 0$ (high selectivity, almost nothing passes): $\Phi_a \to +\infty$
\item $0 < s < 1$ (partial selectivity): $0 < \Phi_a < \infty$
\end{enumerate}
\end{corollary}

\begin{theorem}[Non-Actualisations as Aperture-Blocked States]
\label{thm:non_actualisation_aperture}
Non-actualisations are precisely the configurations blocked by partition apertures:
\begin{equation}
\mathcal{N}(\mathcal{A}) = \{ \omega : \sigma_a(\omega) = 0 \text{ for the aperture } a \text{ that produced } \mathcal{A} \}
\end{equation}
\end{theorem}

\begin{proof}
When a partition operation occurs with aperture $a$:
\begin{itemize}
\item Configurations with $\sigma_a(\omega) = 1$ can pass—these become candidates for actualisation
\item Configurations with $\sigma_a(\omega) = 0$ are blocked—these become non-actualisations
\end{itemize}

The actualisation $\mathcal{A}$ is selected from the passing configurations. All blocked configurations become non-actualisations. Since $|\{\omega : \sigma_a(\omega) = 0\}| = \Omega_{\text{total}} - \Omega_{\text{pass}} = (1 - s)\Omega_{\text{total}}$, the number of non-actualisations is typically vast when $s \ll 1$ (high selectivity).
\end{proof}

\begin{remark}[Connection to Entropy]
The partition entropy $\Delta S = \kB \ln n_{\text{res}}$ can now be expressed in aperture terms:
\begin{equation}
\Delta S = \kB \ln\left(\frac{1}{s}\right) = \frac{\Phi_a}{T}
\end{equation}

Higher selectivity (smaller $s$) produces more entropy. This makes physical sense: a highly selective partition creates more non-actualisations, hence more categorical facts, hence more entropy.
\end{remark}

\begin{theorem}[Catalysis as Aperture Cycling]
\label{thm:catalysis_aperture}
A catalyst operates by creating and then destroying apertures, with net aperture change zero:
\begin{equation}
\Delta\Phi_{\text{catalyst}} = \Phi_{\text{created}} - \Phi_{\text{destroyed}} = 0
\end{equation}
\end{theorem}

\begin{proof}
A catalyst:
\begin{enumerate}
\item Binds the substrate, creating an active site aperture with potential $\Phi_{\text{active}}$
\item Facilitates the reaction within the aperture
\item Releases the product, destroying the aperture
\end{enumerate}

The aperture created equals the aperture destroyed: $\Phi_{\text{created}} = \Phi_{\text{destroyed}} = \Phi_{\text{active}}$. Therefore:
\begin{equation}
\Delta\Phi_{\text{catalyst}} = \Phi_{\text{active}} - \Phi_{\text{active}} = 0
\end{equation}

This explains why catalysts are not consumed: they create and destroy apertures in a balanced cycle.
\end{proof}

\begin{corollary}[Catalysis Does Not Affect Entropy Production]
\label{cor:catalysis_entropy}
A catalyst changes the \emph{path} through aperture space but not the \emph{total entropy production}:
\begin{equation}
\Delta S_{\text{catalysed}} = \Delta S_{\text{uncatalysed}}
\end{equation}
\end{corollary}

\begin{proof}
Entropy is produced by the reaction itself (the partition from reactants to products), not by the catalyst. The catalyst provides a different sequence of apertures—a lower-barrier pathway—but the initial and final states are identical. Since entropy is a state function, $\Delta S$ depends only on these endpoints.
\end{proof}

\begin{remark}[Irreversibility Through Aperture Creation]
This aperture framework deepens the resolution of Loschmidt's paradox. Every physical process creates apertures:
\begin{itemize}
\item Chemical bonds are apertures (geometric constraints on molecular approach)
\item Phase transitions destroy or create apertures (lattice structure)
\item Collisions are transient apertures (scattering cross-sections)
\end{itemize}

These apertures generate non-actualisations. Reversal would require un-creating the apertures and their associated non-actualisations. But apertures, once created, have determined what passed and what did not. This determination is categorical and cannot be erased.

The impossibility of reversal is not merely thermodynamic (statistically improbable) but geometric (categorically impossible). Apertures are partition boundaries, and partition boundaries are permanent features of categorical space.
\end{remark}

