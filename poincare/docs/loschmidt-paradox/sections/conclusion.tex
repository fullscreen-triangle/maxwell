%==============================================================================
\section{Conclusion}
\label{sec:conclusion}
%==============================================================================

Loschmidt's paradox dissolves when entropy is recognised as a geometric property of categorical space rather than a temporal property of dynamical evolution. The key results are:

\begin{enumerate}
\item \textbf{Partition Entropy Theorem:} Every partition operation produces entropy $\Delta S = \kB \ln n_{\text{res}} > 0$ through undetermined residue.

\item \textbf{Temporal Independence:} Partition entropy is invariant under time-reversal. Entropy increases regardless of the direction of temporal evolution.

\item \textbf{Measurement-Partition Identity:} The velocity reversal required by Loschmidt's thought experiment is a partition operation that generates entropy exceeding any possible recovery.

\item \textbf{Topological Irreversibility:} Partition boundaries cannot be erased without creating additional boundaries. Irreversibility is geometric, not temporal.

\item \textbf{Stosszahlansatz as Theorem:} Molecular chaos is a necessary consequence of partition structure, not an assumption. Correlations permitting entropy decrease exist but are thermodynamically inaccessible.

\item \textbf{Non-Actualisation Asymmetry:} Every actualisation creates infinitely many non-actualisations. Time-reversal would require un-creating these non-actualisations, which is categorically impossible. The arrow of time is the direction of non-actualisation accumulation.

\item \textbf{Entropy Requires Termination:} Entropy change is only defined for processes that have terminated. Ongoing processes have indeterminate entropy—they are still in the ``reality stream.'' Once terminated, a process cannot be reversed because termination is categorical completion.

\item \textbf{Categorical Completion = Geometric Partitioning:} Categorical completion and partition operations are identical. Both select one outcome from many, create boundaries between actualised and non-actualised states, and generate entropy. Irreversibility is the impossibility of erasing partition boundaries.
\end{enumerate}

The non-actualisation asymmetry provides the deepest insight into irreversibility. When a cup falls and breaks, it does not merely change physical configuration—it generates infinitely many new categorical facts about what it is \emph{not} doing (not reassembling, not melting, not teleporting). These non-actualisations cannot be erased even if the physical configuration is restored. Reversing the physical trajectory of particles is conceivable; reversing the categorical history of non-actualisations is not. Loschmidt's thought experiment focused on the former while ignoring the latter.

The resolution reveals that the apparent conflict between time-symmetric dynamics and irreversible thermodynamics was based on a false premise: that irreversibility must derive from temporal asymmetry. Irreversibility derives from categorical structure—the geometry of partition space—which is independent of temporal direction. Time-symmetric microscopic dynamics and irreversible macroscopic thermodynamics are fully compatible because they describe different aspects of physical reality: dynamics describes motion in phase space, while thermodynamics describes structure in categorical space.

The partition-oscillation-category equivalence provides the mathematical foundation: oscillatory dynamics, categorical structure, and partition operations yield identical entropy formulations. This equivalence is not coincidental but necessary—oscillation, category, and partition are three perspectives on a single underlying structure. Loschmidt's paradox, far from revealing a flaw in thermodynamics, confirms the geometric nature of entropy and the independence of irreversibility from temporal direction. The arrow of time is not imposed from outside physics—it emerges from the logical asymmetry between finite actualisation and infinite non-actualisation.

