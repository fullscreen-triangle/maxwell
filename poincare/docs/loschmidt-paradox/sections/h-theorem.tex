%==============================================================================
\section{The H-Theorem Reinterpreted}
\label{sec:h_theorem}
%==============================================================================

\subsection{H as Categorical Incompletion}

\begin{definition}[Categorical Completion]
\label{def:completion}
A categorical state $\mathcal{C}$ is \emph{complete} if all partition operations that could distinguish substates have been performed. The \emph{completion fraction} $\phi \in [0, 1]$ measures the fraction of potential partitions that have been actualised.
\end{definition}

\begin{theorem}[H-Function as Incompletion]
\label{thm:h_incompletion}
Boltzmann's H-function measures categorical incompletion:
\begin{equation}
H = -\phi \ln \phi - (1-\phi) \ln(1-\phi)
\label{eq:h_incompletion}
\end{equation}
where $\phi$ is the completion fraction. H decreases as $\phi \to 1$.
\end{theorem}

\begin{proof}
The H-function $H = \int f \ln f \, d^3v$ measures the ``peakedness'' of the velocity distribution. A highly peaked distribution (far from equilibrium) has large $|H|$. A uniform distribution (equilibrium) has $H \to 0$.

In categorical terms:
\begin{itemize}
\item Peaked distribution: few velocity states occupied, many potential partitions not yet actualised, low $\phi$
\item Uniform distribution: all velocity states equally occupied, all possible partitions actualised, high $\phi$
\end{itemize}

As the system evolves, collisions actualise partitions, increasing $\phi$ and decreasing $H$. \qed
\end{proof}

\subsection{Why H Decreases}

\begin{theorem}[Monotonic Completion]
\label{thm:monotonic_completion}
Categorical completion increases monotonically:
\begin{equation}
\frac{d\phi}{dt} \geq 0
\label{eq:monotonic_completion}
\end{equation}
\end{theorem}

\begin{proof}
Completion increases when partitions are actualised. Partitions are actualised by physical interactions (collisions). Physical interactions occur forward in time.

Completion does not decrease because erasing a partition requires a new partition (Theorem~\ref{thm:topological_irreversibility}), which increases completion further. \qed
\end{proof}

\begin{corollary}[H-Theorem from Completion]
\label{cor:h_theorem}
The H-theorem $dH/dt \leq 0$ follows from monotonic completion:
\begin{equation}
\frac{dH}{dt} = \frac{\partial H}{\partial \phi} \frac{d\phi}{dt} \leq 0
\label{eq:h_theorem}
\end{equation}
since $\partial H/\partial \phi < 0$ (H decreases with completion) and $d\phi/dt \geq 0$ (completion increases).
\end{corollary}

\subsection{Time-Reversal and Completion}

\begin{theorem}[Time-Reversal Does Not Reverse Completion]
\label{thm:completion_irreversible}
Time-reversal of particle velocities does not decrease categorical completion.
\end{theorem}

\begin{proof}
Completion $\phi$ counts actualised partitions. Actualised partitions are categorical boundaries in configuration space. Time-reversal acts on phase space, not configuration space.

After time-reversal:
\begin{enumerate}
\item All prior partitions remain actualised (boundaries unchanged)
\item The measurement required for reversal actualises additional partitions
\item Therefore, $\phi_{\text{after}} > \phi_{\text{before}}$
\end{enumerate}

Completion increases under time-reversal, not decreases. \qed
\end{proof}

