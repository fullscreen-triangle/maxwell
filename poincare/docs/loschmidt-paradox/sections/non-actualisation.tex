%==============================================================================
\section{Non-Actualisation Asymmetry}
\label{sec:non_actualisation}
%==============================================================================

The deepest insight into irreversibility emerges from considering what does NOT happen when something happens.

\subsection{The Structure of Non-Actualisation}

\begin{definition}[Non-Actualisation Set]
\label{def:non_actualisation}
For any actualised state $\mathcal{A}$, the non-actualisation set $\mathcal{N}(\mathcal{A})$ comprises all states that could have been actualised but were not:
\begin{equation}
\mathcal{N}(\mathcal{A}) = \{ X : X \text{ was possible but } X \neq \mathcal{A} \}
\label{eq:non_actualisation_set}
\end{equation}
This set is typically infinite.
\end{definition}

\begin{example}[The Cup on the Table]
\label{ex:cup}
Consider a cup resting on a table. The actualised state is ``cup on table.'' The non-actualisation set includes:
\begin{itemize}
\item Cup not turning into gold
\item Cup not becoming sentient
\item Cup not teleporting to Mars
\item Cup not spontaneously shattering
\item Cup not falling
\item ... (infinitely many)
\end{itemize}

Each of these is something the cup is \emph{not doing} at this moment. They are categorical facts about the cup's state—facts defined by negation.
\end{example}

\subsection{Actualisation Creates Non-Actualisations}

\begin{theorem}[Non-Actualisation Creation]
\label{thm:non_actualisation_creation}
Every actualisation creates new non-actualisations. If state $\mathcal{A}_1$ transitions to state $\mathcal{A}_2$, the non-actualisation sets satisfy:
\begin{equation}
|\mathcal{N}(\mathcal{A}_2)| > |\mathcal{N}(\mathcal{A}_1)|
\label{eq:non_actualisation_growth}
\end{equation}
Non-actualisations accumulate monotonically.
\end{theorem}

\begin{proof}
When the cup falls and breaks:

\textbf{Before (cup on table):}
\begin{itemize}
\item Non-actualisation: ``Cup not having fallen''
\item Non-actualisation: ``Cup not being broken''
\end{itemize}

\textbf{After (broken cup on floor):}
\begin{itemize}
\item ``Cup not having fallen'' is no longer a non-actualisation—it became actualised
\item But NEW non-actualisations are created:
\begin{itemize}
\item ``Broken cup not reassembling''
\item ``Broken cup not turning into sand''
\item ``Shards not rearranging into original cup''
\item ``Pieces not spontaneously fusing''
\item ... (infinitely many new non-actualisations)
\end{itemize}
\end{itemize}

The transition from one actualisation to another creates more non-actualisations than it resolves because:
\begin{enumerate}
\item One non-actualisation (``not falling'') became an actualisation
\item But the new state (broken cup) has its own infinite set of non-actualisations
\item These include non-actualisations that \emph{could not have existed} for the unbroken cup
\end{enumerate}

Therefore, $|\mathcal{N}(\mathcal{A}_2)| > |\mathcal{N}(\mathcal{A}_1)|$. \qed
\end{proof}

\subsection{Why Reversal is Impossible}

\begin{theorem}[Non-Actualisation Irreversibility]
\label{thm:non_actualisation_irreversibility}
Time-reversal would require un-creating non-actualisations, which is categorically impossible.
\end{theorem}

\begin{proof}
Suppose we attempt to reverse the cup's fall, restoring ``cup on table'' from ``broken cup on floor.''

This would require:
\begin{enumerate}
\item Reassembling the broken pieces (physical reversal)
\item \textbf{AND} un-creating the non-actualisations that were generated
\end{enumerate}

The second requirement is impossible because:

\textbf{Non-actualisations are categorical facts.} Once the statement ``the broken cup did not reassemble at time $t_1$'' becomes true, it remains true eternally. We cannot make it false—we can only make a \emph{different} statement true at a later time (``the cup reassembled at time $t_2$'').

\textbf{Non-actualisations accumulate in categorical history.} The fact that the cup was broken at $t_1$ and all the non-actualisations of that state (``not reassembling,'' ``not melting,'' etc.) are now permanent features of categorical history. Even if we reassemble the cup, we have not erased the history—we have added to it.

Therefore, reversal restores the physical configuration but cannot restore the categorical state. The non-actualisations remain. \qed
\end{proof}

\begin{corollary}[Asymmetry of Time]
\label{cor:time_asymmetry}
The arrow of time is the direction of non-actualisation accumulation.
\end{corollary}

\begin{proof}
Forward in time: actualisations occur, non-actualisations accumulate.

Backward in time would require: actualisations un-occur, non-actualisations diminish.

But non-actualisations cannot diminish (Theorem~\ref{thm:non_actualisation_irreversibility}).

Therefore, only one temporal direction is tenable. \qed
\end{proof}

\subsection{Connection to Partition Entropy}

\begin{theorem}[Non-Actualisations as Undetermined Residue]
\label{thm:non_actualisation_residue}
Non-actualisations are the categorical manifestation of undetermined residue:
\begin{equation}
\mathcal{N}(\mathcal{A}) \cong \text{Undetermined Residue of Partition}(\mathcal{A})
\label{eq:non_actualisation_residue}
\end{equation}
\end{theorem}

\begin{proof}
When a partition operation selects actualisation $\mathcal{A}$:
\begin{itemize}
\item The selected state becomes $\mathcal{A}$ (determined)
\item The non-selected states become $\mathcal{N}(\mathcal{A})$ (undetermined—they are defined only by what they are NOT)
\end{itemize}

The undetermined residue—states that cannot be assigned to either side of the partition during partition lag—corresponds precisely to the non-actualisations: states that are defined by exclusion rather than by positive determination.

Entropy $S = \kB \ln |\mathcal{N}|$ counts non-actualisations. \qed
\end{proof}

\subsection{The Fundamental Asymmetry}

\begin{theorem}[Fundamental Asymmetry of Existence]
\label{thm:fundamental_asymmetry}
Actualisation and non-actualisation are fundamentally asymmetric:
\begin{center}
\begin{tabular}{ll}
\toprule
\textbf{Actualisation} & \textbf{Non-Actualisation} \\
\midrule
One state selected & Infinitely many excluded \\
Positive determination & Negative determination \\
Creates new non-actualisations & Cannot be un-created \\
Finite information content & Infinite information content \\
\bottomrule
\end{tabular}
\end{center}
\end{theorem}

\begin{proof}
The asymmetry is logical: to say ``$X$ happened'' is to simultaneously say ``all other possibilities did not happen.'' One actualisation generates infinitely many non-actualisations.

But to reverse this would require: one un-actualisation generating infinitely many un-non-actualisations. This is incoherent—un-non-actualisation would be actualisation, but we are trying to undo, not redo.

The asymmetry is irreducible. \qed
\end{proof}

\begin{remark}
This explains why Loschmidt's paradox seemed compelling: it focused only on the actualisation (particle positions and velocities) while ignoring the non-actualisations. Time-reversing the actualisations appears possible; time-reversing the non-actualisations is categorically impossible. The paradox dissolves when both are considered.
\end{remark}

