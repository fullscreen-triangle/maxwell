\section{DNA/RNA as Evolved Charge Capacitors}
\label{sec:charge_capacitor_evolution}

\subsection{DNA as Charge Storage System}

We establish that DNA's primary physical function is charge capacitance, with information storage emerging as an evolutionary bonus enabled by the sequence-independence of charge function.

\begin{theorem}[DNA Charge Content]
\label{thm:dna_charge}
The human genome carries a total charge of:
\begin{equation}
    Q_{\text{DNA}} = N_{\text{bp}} \times 2 \times q_{\text{phosphate}} = 3 \times 10^9 \times 2 \times (-e) = -6 \times 10^9 e
\end{equation}
where $N_{\text{bp}}$ is the number of base pairs and each nucleotide contributes one phosphate group with charge $-e$.
\end{theorem}

\begin{theorem}[DNA Electrostatic Energy Storage]
\label{thm:dna_energy}
The electrostatic energy stored in the DNA charge distribution is:
\begin{equation}
    U_{\text{DNA}} = \frac{1}{2}\sum_{i,j} \frac{q_i q_j}{4\pi\epsilon_0 r_{ij}} \approx 2 \times 10^{-12} \text{ J}
\end{equation}
This exceeds the cellular ATP pool ($\sim 2 \times 10^{-17}$ J) by a factor of $10^5$.
\end{theorem}

\begin{proof}
For a linear charge distribution of length $L$ with linear charge density $\lambda$:
\begin{equation}
    U = \frac{\lambda^2}{4\pi\epsilon_0} \int_0^L \int_0^L \frac{dx \, dy}{|x-y| + a}
\end{equation}
where $a$ is a regularization parameter (approximately the DNA radius).

For the human genome with $L \approx 2$ m when fully extended and $\lambda \approx 3 \times 10^9 e / 2$ m:
\begin{equation}
    U_{\text{DNA}} \approx \frac{(3 \times 10^9 \times 1.6 \times 10^{-19})^2}{4\pi \times 8.85 \times 10^{-12}} \times \ln\left(\frac{L}{a}\right) \approx 2 \times 10^{-12} \text{ J}
\end{equation}
\end{proof}

\subsection{Charge Oscillations at Metabolic Frequencies}

\begin{theorem}[Metabolic Coupling of DNA Charge]
\label{thm:metabolic_coupling}
DNA charge dynamics couple to metabolic processes through ion concentration oscillations that modulate Debye screening:
\begin{equation}
    \lambda_D(t) = \sqrt{\frac{\epsilon_0 \epsilon_r k_B T}{2 N_A e^2 I(t)}}
\end{equation}
where $I(t)$ is the time-varying ionic strength.
\end{theorem}

Metabolic processes (glycolysis, TCA cycle, oxidative phosphorylation) generate ion concentration oscillations with periods of 0.1-100 s \citep{goldbeter1996biochemical}. These oscillations modulate DNA-protein interactions through screening length changes.

\begin{corollary}[DNA Surface Potential Oscillations]
\label{cor:surface_potential}
The DNA surface potential oscillates with metabolic activity:
\begin{equation}
    \Phi_{\text{surface}}(t) = \frac{\sigma}{\epsilon_0 \epsilon_r \kappa(t)}
\end{equation}
where $\kappa(t) = 1/\lambda_D(t)$ is the time-varying inverse screening length.
\end{corollary}

\subsection{Sequence Independence of Charge Function}

\begin{theorem}[Sequence-Independent Charge Storage]
\label{thm:seq_independence}
The charge storage function of DNA is independent of nucleotide sequence. All four nucleotides contribute identical phosphate charges:
\begin{equation}
    q_A = q_T = q_G = q_C = -e \text{ (from phosphate group)}
\end{equation}
\end{theorem}

\begin{corollary}[Information as Evolutionary Bonus]
\label{cor:info_bonus}
Because charge function is sequence-independent, nucleotide sequence can be repurposed for information storage without compromising charge function. Information storage is thus an ``evolutionary bonus'' that emerged after charge capacitance was established.
\end{corollary}

This resolves a puzzle in evolutionary biology: why does the genetic code exist? Not because information storage was primordial, but because the charge-storage molecule (polynucleotide) happened to have sequence variability that could be co-opted for information encoding.

\subsection{Histone Charge Neutralization}

\begin{theorem}[Histone-DNA Charge Complementarity]
\label{thm:histone_charge}
Histone proteins partially neutralize DNA charge:
\begin{equation}
    Q_{\text{histones}} = +4.08 \times 10^9 e
\end{equation}
Creating a net nucleosome charge:
\begin{equation}
    Q_{\text{net}} = Q_{\text{DNA}} + Q_{\text{histones}} = -1.92 \times 10^9 e
\end{equation}
\end{theorem}

This partial neutralization creates a charge capacitor with:
\begin{itemize}
    \item DNA phosphates as negative plate
    \item Histone lysines/arginines as positive plate
    \item Intervening space as dielectric
\end{itemize}

\begin{theorem}[Nucleosome Capacitance]
\label{thm:nucleosome_cap}
Each nucleosome functions as a capacitor with capacitance:
\begin{equation}
    C_{\text{nucleosome}} = \frac{Q_{\text{wrapped}}}{V_{\text{nucleosome}}} \approx 10^{-18} \text{ F}
\end{equation}
where $Q_{\text{wrapped}} \approx 300e$ (147 bp wrapped around histone octamer) and $V_{\text{nucleosome}} \approx 50$ mV.
\end{theorem}

\subsection{Genomic Processes as Charge Dynamics}

\begin{theorem}[Replication as Charge Wave]
\label{thm:replication_charge}
DNA replication timing correlates with chromatin charge state:
\begin{equation}
    t_{\text{replication}} \propto \frac{1}{\sigma_{\text{local}}}
\end{equation}
where $\sigma_{\text{local}}$ is the local charge density. Early-replicating regions have lower charge density (more histone neutralization); late-replicating regions have higher charge density.
\end{theorem}

\begin{theorem}[Transcription as Charge Oscillation]
\label{thm:transcription_charge}
Transcriptional bursting follows charge fluctuation dynamics:
\begin{equation}
    P(\text{burst}) = f(\Delta \Phi_{\text{promoter}}, [ATP], [Mg^{2+}])
\end{equation}
where $\Delta \Phi_{\text{promoter}}$ is the promoter region potential fluctuation.
\end{theorem}

Experimental observations support these predictions:
\begin{itemize}
    \item Okazaki fragment length oscillates 110-190 nt with ATP synthesis periodicity \citep{smith2015okazaki}
    \item Transcriptional bursting frequency modulates with metabolic state \citep{larsson2019genomic}
    \item Nucleosome breathing exhibits 50\% amplitude oscillation at metabolic timescales \citep{li2005nucleosome}
\end{itemize}

\subsection{``Junk DNA'' as Charge Scaffolding}

\begin{theorem}[Non-Coding DNA as Capacitor Scaffolding]
\label{thm:junk_dna}
The 98\% of the human genome that does not encode proteins functions as charge scaffolding:
\begin{equation}
    Q_{\text{scaffolding}} = 0.98 \times Q_{\text{DNA}} = -5.88 \times 10^9 e
\end{equation}
This scaffolding provides:
\begin{enumerate}
    \item Charge storage capacity
    \item Spatial separation between functional elements
    \item Chromatin organization scaffolding
    \item Buffer against charge fluctuations
\end{enumerate}
\end{theorem}

\begin{proof}
If DNA's primary function were information storage, non-coding DNA would be evolutionarily costly (replication energy, mutation load) and would be eliminated by selection. The persistence of large genomes in organisms with small gene numbers (C-value paradox \citep{gregory2001coincidence}) is explained by charge scaffolding function:
\begin{equation}
    \text{Genome size} \propto \text{Capacitance requirement} + \text{Information content}
\end{equation}
rather than:
\begin{equation}
    \text{Genome size} \propto \text{Information content alone}
\end{equation}
\end{proof}

\subsection{Evolutionary Trajectory}

The evolutionary sequence was:
\begin{enumerate}
    \item \textbf{Electron transport partitioning}: Primordial charge separation
    \item \textbf{Membrane scaffolding}: Stabilization of electron transport
    \item \textbf{Charge storage polymers}: Polynucleotides as capacitors
    \item \textbf{Information encoding}: Co-option of sequence variability
    \item \textbf{Genetic code}: Systematic sequence-function mapping
\end{enumerate}

This trajectory explains why:
\begin{itemize}
    \item DNA has phosphate backbone (charge function) with variable bases (information function)
    \item RNA preceded DNA (less stable charge storage, more versatile)
    \item The genetic code is universal (established once charge storage was optimized)
\end{itemize}

