\section{The Spare Tire Principle: Dual Function and Observational Bias}
\label{sec:spare_tire}

The preceding sections establish that the genome functions primarily as a charge modulator with information storage as a secondary function. This section introduces an analogy that captures the essence of this dual function and explains why the primary function remained hidden for seventy years of molecular biology: the spare tire principle.

\subsection{Formulation of the Spare Tire Principle}

Consider a driver who uses the spare tire not only for emergency replacement but also for weight balancing of the vehicle. The spare tire contributes to the car's mass distribution continuously, affecting handling, stability, and fuel efficiency. This function operates without interruption from the moment the tire is placed in the trunk until the moment it is removed. Yet the driver remains unaware of this primary function because it produces no discrete, observable events. The driver only becomes aware of the spare tire's existence when a flat tire occurs---a rare emergency that activates the tire's secondary function.

\begin{definition}[The Spare Tire Principle]
\label{def:spare_tire}
An object exhibits the spare tire principle when it possesses a primary function that operates continuously and invisibly, and a secondary function that activates conditionally and visibly. The observational bias inherent in studying systems through perturbations causes the secondary function to be mistaken for the primary function.
\end{definition}

The genome operates according to this principle. The primary function of genomic DNA is charge balancing, which operates continuously through the mere presence of the negatively charged phosphate backbone. The secondary function is information storage, which activates conditionally when genes are expressed. Because molecular biology has historically studied genes during expression events, the secondary function was mistaken for the primary function.

\begin{theorem}[Functional Time Allocation]
\label{thm:time_allocation}
The ratio of time spent performing primary versus secondary function is given by:
\begin{equation}
    \frac{t_{\text{primary}}}{t_{\text{secondary}}} = \frac{t_{\text{total}}}{t_{\text{expression}}} \approx \frac{1}{f_{\text{expr}}}
\end{equation}
where $f_{\text{expr}}$ is the fraction of time a gene is expressed.
\end{theorem}

\begin{proof}
The primary function of charge balancing operates continuously: $t_{\text{primary}} = t_{\text{total}}$. The secondary function of information storage operates only during expression: $t_{\text{secondary}} = f_{\text{expr}} \times t_{\text{total}}$. For typical genes with $f_{\text{expr}} < 0.01$, the ratio exceeds 100. For rarely expressed genes with $f_{\text{expr}} < 0.0001$, the ratio exceeds 10,000.
\end{proof}

For the human genome over a typical lifespan:
\begin{align}
    t_{\text{charge balancing}} &= 80 \text{ years} \times 365 \text{ days} \times 24 \text{ hours} = 700{,}800 \text{ hours} \\
    t_{\text{expression}} &\approx 0.02 \times 700{,}800 = 14{,}016 \text{ hours}
\end{align}

The genome spends approximately fifty times more time performing charge balancing than encoding proteins. For genes expressed in fewer than one percent of cells or cell cycles, this ratio exceeds 10,000.

\subsection{The Flat Tire Fallacy}

The observational methodology of molecular biology systematically favors detection of secondary functions over primary functions. This methodological bias constitutes what we term the flat tire fallacy.

\begin{definition}[The Flat Tire Fallacy]
\label{def:flat_tire_fallacy}
The flat tire fallacy is the epistemological error of confusing a rare, visible function with a continuous, invisible function due to studying systems only during perturbation events.
\end{definition}

Consider a scientist who studies automobiles exclusively during flat tire events. The scientist observes the spare tire being removed from the trunk, mounted on the wheel hub, and the car resuming motion. The scientist concludes that the spare tire's function is emergency replacement. What the scientist misses is that the spare tire was performing weight balancing for 99.99\% of its existence. The emergency replacement function, though dramatic and observable, is statistically negligible compared to the continuous weight-balancing function.

The parallel in genomics is precise. A molecular biologist studies genes during expression events. The biologist observes mRNA being transcribed, protein being translated, and cellular function changing. The biologist concludes that the gene's function is protein encoding. What the biologist misses is that the gene was performing charge balancing for 99.99\% of its existence. The protein encoding function, though dramatic and measurable, is statistically negligible compared to the continuous charge-balancing function.

\begin{theorem}[Observational Bias in Perturbation Studies]
\label{thm:perturbation_bias}
Studies that discover function through perturbation are inherently biased toward detecting discrete, visible functions and against detecting continuous, invisible functions.
\end{theorem}

\begin{proof}
The standard methodology of molecular biology proceeds as follows: remove gene, observe phenotype, infer function; mutate sequence, observe change, infer mechanism; inhibit expression, observe effect, infer role. Each step requires a discrete, observable change. Continuous functions that operate through mere presence produce no discrete changes when present and produce only subtle, distributed effects when removed. The methodology is therefore systematically blind to continuous functions.

Formally, let $\mathcal{F}_d$ denote the set of discrete functions detectable by perturbation and $\mathcal{F}_c$ denote the set of continuous functions. The detection probability satisfies:
\begin{equation}
    P(\text{detect} | f \in \mathcal{F}_d) \gg P(\text{detect} | f \in \mathcal{F}_c)
\end{equation}

This bias explains why charge balancing remained hidden while information storage was discovered.
\end{proof}

\begin{figure*}[htbp]
\centering
\includegraphics[width=0.90\textwidth]{figures/spare_tire_panel.png}
\caption{\textbf{The Spare Tire Principle: Dual Function and Observational Bias in Genomic Interpretation.} 
Analogy demonstrating why observing rare secondary function leads to systematic misidentification of primary function. 
\textbf{(A)} Dual function analogy: spare tire's primary function is weight balancing (continuous, invisible, 100\% of time), while secondary function is emergency replacement (rare, visible, $<$0.01\% of time). Observing only tire replacement leads to incorrect conclusion that "tire storage is the primary function." Genome parallel: charge balancing (100\% of time) vs. information expression ($<$2\% of time). 
\textbf{(B)} Functional time allocation: spare tire used 43,800× more time for weight balancing than replacement; typical genome used 50× more time for charge balancing than expression; even rarely expressed genes spend $>$95\% of time in charge balancing function. Blue bars (primary function) dominate; red bars (secondary function) barely visible at this scale. 
\textbf{(C)} The flat tire fallacy: scientist observing gene expression ($\sim$2\% of time) concludes "information storage is the primary function," missing 98\% of time spent in charge balancing (invisible without electromagnetic measurements). This observational bias has dominated molecular biology for 70 years. 
\textbf{(D)} Replacement cycle charge variance: during gene expression (pink "Crisis expression!" region), charge variance increases 4-fold (blue oscillations, baseline dashed line at 0.2), creating selection pressure for rapid return to charge-balanced state (green "Normal" and "Restoration" regions). Expression is perturbation, not primary function. 
\textbf{(E)} Evolutionary selection: fitness (purple solid line) tracks charge stability (blue dashed line), not information content (red dashed line, which remains near zero). Optimum expression frequency is $\sim$0\% (maximum fitness at zero expression), with fitness declining monotonically as expression increases—evolution selects against expression, confirming it is costly perturbation. 
\textbf{(F)} Puzzles resolved: charge-first perspective resolves C-value paradox (genome size $\propto$ charge needs, not complexity), non-coding DNA (charge balancing, not "junk"), conservation without expression (conserved for charge, not information), and 70 years of missed function (perturbation bias from observing only rare visible events).}
\label{fig:spare_tire}
\end{figure*}

\subsection{Why the Primary Function Is Invisible}

The invisibility of the primary function is not accidental but follows necessarily from its continuous nature. Four properties render the charge-balancing function undetectable by standard molecular biology methods.

First, the function operates continuously. DNA carries negative charge at every moment of its existence. There are no discrete events marking the beginning or end of charge balancing. Without discrete events, there is nothing to observe.

Second, the function requires no transcription. Charge balancing operates through mere physical presence of the phosphate backbone. The DNA need not be read, transcribed, or processed. Since molecular biology focuses on gene expression, a function that operates without expression is invisible to standard assays.

Third, the effect is distributed. Charge balancing affects the entire cytoplasm through electric field modulation. The effect is not localized to a specific organelle, pathway, or phenotype. Distributed effects are difficult to attribute to specific sequences.

Fourth, removal causes subtle degradation rather than catastrophic failure. Deleting charge-balancing sequences increases charge variance and membrane potential instability, but these effects are subtle compared to losing a protein-coding gene. Subtle phenotypes are easily attributed to secondary effects or experimental noise.

\begin{corollary}[Detection of Primary Function]
\label{cor:detection}
To detect the primary function, one must measure continuous variables such as charge distribution and membrane potential variance, rather than discrete events such as gene expression and protein levels.
\end{corollary}

\subsection{The Replacement Cycle and Observational Masking}

When the secondary function activates, the system undergoes a replacement cycle that temporarily disrupts the primary function. This disruption is masked by the crisis that triggered the secondary function.

\begin{definition}[The Replacement Cycle]
\label{def:replacement_cycle}
The replacement cycle consists of four states through which the system transitions during activation of the secondary function:
\begin{enumerate}
    \item \textbf{Normal state}: Charge balanced, genes silent, proteins functional
    \item \textbf{Crisis state}: Charge imbalanced due to protein depletion or metabolic stress
    \item \textbf{Response state}: Gene expressed, charge further imbalanced during transcription
    \item \textbf{Restoration state}: Gene silenced, charge rebalanced, system returns to normal
\end{enumerate}
\end{definition}

\begin{theorem}[Charge Dynamics During the Replacement Cycle]
\label{thm:replacement_dynamics}
The charge variance follows a characteristic trajectory during the replacement cycle:
\begin{equation}
    \text{Var}[\sigma]_{\text{response}} > \text{Var}[\sigma]_{\text{crisis}} > \text{Var}[\sigma]_{\text{normal}} = \text{Var}[\sigma]_{\text{restoration}}
\end{equation}
\end{theorem}

\begin{proof}
In the normal state, total charge is balanced:
\begin{equation}
    \sigma_{\text{membrane}} + \sigma_{\text{DNA}} + \sigma_{\text{cytoplasm}} \approx 0
\end{equation}
with variance $\text{Var}[\sigma] = \sigma_0^2$ representing baseline fluctuations.

In the crisis state, protein depletion or metabolic stress alters cytoplasmic charge:
\begin{equation}
    \Delta\sigma_{\text{cytoplasm}} = \sum_i q_i \Delta n_i
\end{equation}
where $q_i$ is the charge of species $i$ and $\Delta n_i$ is the concentration change. This increases variance to $\text{Var}[\sigma] = \sigma_0^2 + (\Delta\sigma_{\text{crisis}})^2$.

In the response state, gene expression transfers negative charge from DNA to mRNA in the cytoplasm:
\begin{equation}
    \Delta\sigma_{\text{mRNA}} = -2 N_{\text{nt}} \times e
\end{equation}
where $N_{\text{nt}}$ is the number of nucleotides transcribed. This further increases variance to $\text{Var}[\sigma] = \sigma_0^2 + (\Delta\sigma_{\text{crisis}})^2 + (\Delta\sigma_{\text{mRNA}})^2$.

In the restoration state, mRNA is degraded, protein is replenished, and variance returns to baseline $\sigma_0^2$.
\end{proof}

The critical insight is that the response state exhibits maximum charge variance, but this is precisely when the cell is responding to a crisis. The dramatic phenotypes of stress response, metabolic adaptation, and protein synthesis mask the subtle charge imbalance. The observer is focused on the crisis and misses the transient charge disruption.

\begin{theorem}[Observational Masking]
\label{thm:masking}
The conditions that activate gene expression produce phenotypes that mask the charge imbalance caused by expression.
\end{theorem}

\begin{proof}
Gene expression is activated by metabolic stress, protein depletion, environmental change, or signaling events. Each of these produces observable phenotypes: altered metabolism, stress response activation, morphological changes, or pathway modulation. These phenotypes dominate experimental observation.

The charge imbalance during expression is transient, lasting minutes to hours compared to the hours to days of the crisis response. The imbalance is small, affecting less than $10^{-6}$ of total cellular charge for a single gene. The imbalance is compensated by other genes adjusting their chromatin conformation. By the time the crisis resolves, charge balance is restored and no evidence remains of the transient imbalance.

The none-the-wiser principle applies: the system returns to baseline, erasing evidence of the disruption.
\end{proof}

\subsection{Evolutionary Origin of the Spare Tire Architecture}

The spare tire principle is not merely an analogy but describes the evolutionary pressure that shaped genome architecture. For early RNA and DNA to function as charge capacitors while remaining evolutionarily relevant, they had to encode something useful. However, if they encoded frequently-used proteins, they would be constantly transcribed and thus poor capacitors. The solution was to encode proteins that are rarely needed but critical when needed.

\begin{theorem}[Optimal Encoding Strategy]
\label{thm:optimal_encoding}
For RNA or DNA functioning as a charge capacitor, the evolutionarily optimal strategy is to encode proteins that are rarely needed but critical when needed.
\end{theorem}

\begin{proof}
Consider an RNA sequence that must satisfy two constraints. First, to function as a charge capacitor, it must remain mostly untranscribed, keeping charge localized. Second, to be evolutionarily relevant, it must encode something useful, providing a fitness advantage beyond mere charge balancing.

If the RNA encodes a frequently-used protein:
\begin{equation}
    f_{\text{transcribed}} \approx 1 \implies f_{\text{charge balancing}} \approx 0
\end{equation}
The RNA is a poor capacitor, constantly depleted of its charge contribution.

If the RNA encodes a rarely-used protein:
\begin{equation}
    f_{\text{transcribed}} \approx 0.01 \implies f_{\text{charge balancing}} \approx 0.99
\end{equation}
The RNA is an excellent capacitor, maintaining charge contribution 99\% of the time while retaining the emergency information function.

The fitness function is:
\begin{equation}
    W = w_{\text{charge}} \cdot f_{\text{silent}} + w_{\text{info}} \cdot f_{\text{expressed}} \cdot V_{\text{protein}}
\end{equation}
where $w_{\text{charge}}$ is the weight of charge-balancing function, $w_{\text{info}}$ is the weight of information function, and $V_{\text{protein}}$ is the value of the encoded protein when needed.

For $w_{\text{charge}} \gg w_{\text{info}}$ in primordial conditions, fitness is maximized by maximizing $f_{\text{silent}}$ while maintaining $V_{\text{protein}} > 0$. This is achieved by encoding rarely-needed but critical proteins.
\end{proof}

This evolutionary logic explains the observed distribution of gene expression frequencies. Housekeeping genes that are constantly expressed represent a small fraction of the genome because they sacrifice charge-balancing function. Tissue-specific genes expressed only in particular cell types are excellent capacitors in all other tissues. Stress-response genes expressed only during crises are optimal spare tires, providing critical function when needed while maintaining charge balance the rest of the time. Non-coding sequences that are never transcribed are perfect capacitors, contributing purely to charge balancing.

\begin{corollary}[Selection Pressure on Non-Coding DNA]
\label{cor:noncoding_selection}
Non-coding DNA is under strong selection pressure to be retained even though it is never transcribed, because its primary function is charge balancing.
\end{corollary}

\begin{proof}
Deletion of non-coding DNA reduces total negative charge:
\begin{equation}
    \Delta \sigma_{\text{DNA}} = -2 \times N_{\text{deleted}} \times e
\end{equation}

This increases charge variance:
\begin{equation}
    \Delta \text{Var}[\sigma_{\text{total}}] \propto (\Delta \sigma_{\text{DNA}})^2
\end{equation}

Increased charge variance destabilizes electron transport and membrane potential, reducing fitness. Therefore, sequences are maintained for charge balancing even if never expressed. The selection coefficient for retention is:
\begin{equation}
    s_{\text{total}} = s_{\text{charge}} + s_{\text{info}} \times P_{\text{expr}}
\end{equation}

For non-expressed sequences with $P_{\text{expr}} = 0$:
\begin{equation}
    s_{\text{total}} = s_{\text{charge}} > 0
\end{equation}

Selection acts on charge function alone.
\end{proof}

\subsection{Resolution of Genomic Puzzles}

The spare tire principle resolves multiple long-standing puzzles in genomics that information-first theories fail to explain.

\subsubsection{The C-Value Paradox}

The C-value paradox is the observation that genome size varies 200,000-fold across eukaryotes with similar complexity, with no correlation between genome size and organismal complexity or gene number \citep{gregory2001coincidence}. The onion genome is five times larger than the human genome, yet onions are not five times more complex than humans.

\begin{theorem}[Resolution of the C-Value Paradox]
\label{thm:cvalue_resolution}
Genome size correlates with charge requirements rather than information content:
\begin{equation}
    \text{Genome size} \propto \sigma_{\text{cytoplasm}} \times V_{\text{cell}} + I_{\text{genetic}}
\end{equation}
where $\sigma_{\text{cytoplasm}}$ is cytoplasmic charge density requiring neutralization, $V_{\text{cell}}$ is cell volume, and $I_{\text{genetic}}$ is genetic information content.
\end{theorem}

\begin{proof}
For charge balance:
\begin{equation}
    \sigma_{\text{DNA}} \approx -\sigma_{\text{cytoplasm}} \times V_{\text{cell}}
\end{equation}

Since each nucleotide contributes $-2e$:
\begin{equation}
    N_{\text{bp}} \propto \frac{\sigma_{\text{cytoplasm}} \times V_{\text{cell}}}{2e}
\end{equation}

Organisms with larger cells or higher metabolic rates (more charged metabolites) require larger genomes for charge balancing, independent of information content.

Onion storage parenchyma cells have volume approximately 30,000 $\mu$m$^3$, while average human cells have volume approximately 2,000 $\mu$m$^3$. The volume ratio of 15 accounts for much of the genome size difference. The C-value paradox is not a paradox but a confirmation of charge-balancing theory.
\end{proof}

\subsubsection{Persistence of Non-Coding DNA}

Information-first theories predict that non-functional DNA should be eliminated by selection due to replication costs. Yet 98\% of the human genome is non-coding, and organisms with streamlined genomes (bacteria, some eukaryotes) are the exception rather than the rule.

The spare tire principle explains persistence: non-coding DNA is not non-functional but performs the primary function of charge balancing. The replication cost is offset by the fitness benefit of stable charge distribution. Organisms that eliminated non-coding DNA would experience increased charge variance, membrane instability, and reduced electron transport efficiency.

\subsubsection{Sequence Conservation Without Expression}

Many genomic sequences show conservation across species despite never being transcribed in any cell type or developmental stage. Information-first theories struggle to explain conservation without function.

The spare tire principle predicts conservation based on charge distribution rather than expression. Sequences are conserved because they contribute to charge balancing, and mutations that alter local charge density are selected against. Conservation correlates with charge contribution, not with transcriptional activity.

\begin{theorem}[Charge-Based Conservation]
\label{thm:charge_conservation}
Sequence conservation should correlate with charge distribution requirements:
\begin{equation}
    C_{\text{sequence}} = f(\sigma_{\text{local}}, \Delta\sigma_{\text{tolerance}})
\end{equation}
where $C_{\text{sequence}}$ is conservation score, $\sigma_{\text{local}}$ is local charge density, and $\Delta\sigma_{\text{tolerance}}$ is the tolerance for charge variation.
\end{theorem}

\subsection{Experimental Predictions}

The spare tire principle generates falsifiable predictions that distinguish it from information-first theories.

\subsubsection{The Weight Distribution Test}

\begin{theorem}[Charge Distribution Test]
\label{thm:charge_test}
The experimental design for testing charge-balancing function proceeds as follows:
\begin{enumerate}
    \item Measure cytoplasmic charge distribution with full genome
    \item Delete non-expressed genes (complete removal)
    \item Measure charge distribution again
    \item \textbf{Prediction}: Charge variance increases
\end{enumerate}

\textbf{Critical control}: Replace deleted genes with charge-equivalent DNA (scrambled sequence, same length).
\begin{enumerate}
    \item Complete deletion: charge distribution changes (confirms charge function)
    \item Charge-equivalent replacement: charge distribution unchanged (confirms charge sufficiency)
\end{enumerate}
\end{theorem}

\subsubsection{Charge Variance During Expression}

\begin{theorem}[Transient Variance Test]
\label{thm:variance_test}
During gene expression, charge variance should transiently increase:
\begin{equation}
    \text{Var}[\Delta\Phi]_{\text{expressing}} > \text{Var}[\Delta\Phi]_{\text{baseline}}
\end{equation}
followed by return to baseline after expression completes:
\begin{equation}
    \text{Var}[\Delta\Phi]_{\text{final}} \approx \text{Var}[\Delta\Phi]_{\text{baseline}}
\end{equation}

Information-first theory predicts no change in charge variance during expression.
\end{theorem}

\subsubsection{Charge-Neutral Genome Editing}

\begin{theorem}[Charge-Neutral Editing Prediction]
\label{thm:charge_neutral_editing}
Genome edits that preserve total charge should have minimal phenotypic effects, even if they alter sequence:
\begin{itemize}
    \item \textbf{Control}: Delete 1 Mb of non-coding DNA ($\Delta\sigma = -2 \times 10^6 e$). Expected: reduced fitness, increased membrane potential variance.
    \item \textbf{Experimental}: Replace 1 Mb with different sequence of same length ($\Delta\sigma = 0$). Expected: no fitness change, no variance change.
\end{itemize}

This prediction extends to large genomic regions. Charge-neutral replacement of up to 10\% of the genome should be phenotypically neutral. Information-first theory predicts significant phenotypic effects from any large-scale sequence change due to regulatory elements or chromatin structure disruption.
\end{theorem}

\subsection{Implications for Molecular Biology}

The spare tire principle necessitates a fundamental revision of how we study and interpret genomes.

\subsubsection{Methodological Revision}

Traditional molecular biology measures discrete events: gene expression, protein levels, phenotypic changes. The spare tire principle requires measuring continuous variables: charge distribution, membrane potential variance, electric field geometry. This methodological shift would reveal the primary function that discrete measurements miss.

\subsubsection{Genome Engineering}

Synthetic biology currently designs genomes for information density, encoding essential genes and minimizing non-coding sequence. The spare tire principle suggests designing genomes for charge distribution first and encoding genes second. The algorithm proceeds as follows: calculate required charge from cytoplasmic volume and metabolic rate; design sequences satisfying charge requirements; within the charge constraint, encode essential genes; fill remaining sequence with charge-balancing non-coding DNA.

Synthetic genomes designed by charge-balancing principles should be more stable than those designed by information-first principles, as demonstrated by the instability of minimal genomes such as \textit{Mycoplasma mycoides} JCVI-syn3.0.

\subsubsection{Therapeutic Implications}

If gene expression is modulated by charge state, then charge-modulating interventions could regulate gene expression without targeting specific genes. Ionophores that alter cytoplasmic charge distribution could shift expression thresholds. Membrane potential modulators could alter gene accessibility. This opens therapeutic strategies orthogonal to traditional gene-targeting approaches.

\subsection{Why Information-First Theories Failed}

Information-first theories failed because they committed the flat tire fallacy systematically. The historical development of molecular biology followed a pattern that guaranteed this error.

First, molecular biology emerged from the study of rare, dramatic events: mutations, gene knockouts, expression changes. These perturbation studies revealed the secondary function (information storage) while rendering the primary function (charge balancing) invisible.

Second, the technological tools of molecular biology measure discrete events: PCR amplifies specific sequences, Northern blots detect transcripts, Western blots detect proteins. No standard technique measures continuous charge distribution.

Third, the conceptual framework of molecular biology assumed that function requires activity. A sequence that is never transcribed was assumed to be functionless. The possibility of function through mere presence was not considered.

Fourth, the success of the information-first framework in explaining some phenomena (genetic code, protein synthesis, heredity) created confidence that the framework was complete. Anomalies (C-value paradox, non-coding DNA persistence, sequence conservation without expression) were treated as puzzles to be solved within the framework rather than evidence against it.

The spare tire principle explains why seventy years of molecular biology missed the primary function of the genome. It was not incompetence or oversight but a systematic methodological bias toward detecting secondary functions. The genome's primary function was invisible precisely because it was continuous.

\subsection{Conclusion}

The genome operates according to the spare tire principle. Its primary function is charge balancing, which operates continuously through the mere presence of negatively charged DNA. Its secondary function is information storage, which activates conditionally when genes are expressed. Molecular biology discovered the secondary function first because perturbation studies are biased toward detecting rare, visible functions. The primary function remained hidden because it is continuous and invisible.

The spare tire principle resolves the C-value paradox, explains the persistence of non-coding DNA, accounts for sequence conservation without expression, and predicts that charge-neutral genome edits should be phenotypically neutral. These predictions distinguish charge-balancing theory from information-first theories and provide a falsifiable research program for testing the framework.

The genome has been performing charge balancing for 3.5 billion years. We only noticed it encoding proteins because that is when we looked. The primary function is invisible precisely because it is continuous.

\begin{figure}[H]
    \centering
    \includegraphics[width=\textwidth]{figures/spare_tire_panel.png}
    \caption{The Spare Tire Principle. (A) The dual function analogy: spare tire performs weight balancing continuously and emergency replacement rarely. (B) Functional time allocation showing charge balancing dominates expression time by orders of magnitude. (C) The flat tire fallacy: observational bias toward discrete events causes secondary function to be mistaken for primary. (D) The replacement cycle: charge variance peaks during expression and returns to baseline. (E) Evolutionary selection for rarely-used genes optimizes charge capacitance. (F) Resolution of genomic puzzles through the spare tire framework.}
    \label{fig:spare_tire_panel}
\end{figure}


