\section{Electron Transport as Charge Partitioning}
\label{sec:electron_transport_partitioning}

\subsection{Charge Separation as Fundamental Partition}

We establish that electron transport constitutes the most fundamental form of partitioning—a categorical operation that divides phase space into distinct regions.

\begin{definition}[Charge Partition]
\label{def:charge_partition}
A \emph{charge partition} $\Pi_q$ is a spatial separation of positive and negative charge densities:
\begin{equation}
    \Pi_q: \mathbb{R}^3 \rightarrow \{+, -, 0\}
\end{equation}
such that for regions $\Omega_+$ and $\Omega_-$:
\begin{equation}
    \int_{\Omega_+} \rho(\mathbf{r}) \, d^3r > 0, \quad \int_{\Omega_-} \rho(\mathbf{r}) \, d^3r < 0
\end{equation}
where $\rho(\mathbf{r})$ is the charge density.
\end{definition}

\begin{theorem}[Electron Transport Creates Partition]
\label{thm:et_creates_partition}
Any electron transport event across a spatial boundary creates a charge partition. This partition is:
\begin{enumerate}
    \item \textbf{Instantaneous}: Created at quantum mechanical timescales ($\tau \sim 10^{-15}$ s)
    \item \textbf{Self-sustaining}: The electric field resists charge recombination
    \item \textbf{Information-free}: Requires no external instructions or templates
\end{enumerate}
\end{theorem}

\begin{proof}
Consider an electron moving from position $\mathbf{r}_1$ to $\mathbf{r}_2$ across a boundary $\Sigma$. Before transport:
\begin{equation}
    \rho_{\text{before}}(\mathbf{r}) = \rho_0(\mathbf{r})
\end{equation}

After transport:
\begin{equation}
    \rho_{\text{after}}(\mathbf{r}) = \rho_0(\mathbf{r}) - e\delta^3(\mathbf{r} - \mathbf{r}_1) + e\delta^3(\mathbf{r} - \mathbf{r}_2)
\end{equation}

This creates an electric field:
\begin{equation}
    \mathbf{E}(\mathbf{r}) = \frac{e}{4\pi\epsilon_0} \left( \frac{\mathbf{r} - \mathbf{r}_2}{|\mathbf{r} - \mathbf{r}_2|^3} - \frac{\mathbf{r} - \mathbf{r}_1}{|\mathbf{r} - \mathbf{r}_1|^3} \right)
\end{equation}

This field constitutes a partition: regions near $\mathbf{r}_1$ are positive, and regions near $\mathbf{r}_2$ are negative. The field itself resists recombination, as returning the electron requires work against the field.
\end{proof}

\begin{figure*}[htbp]
\centering
\includegraphics[width=0.90\textwidth]{figures/aperture_as_field_panel.png}
\caption{\textbf{Apertures as External Charge Fields: Electromagnetic Molecular Selection.} 
Categorical apertures function as external charge field configurations rather than mechanical filters. 
\textbf{(A)} Monopole aperture (simple ion channel) creates radial field selecting ions by charge sign. 
\textbf{(B)} Dipole aperture (K$^+$ selectivity filter) creates asymmetric field matching K$^+$ charge distribution while excluding Na$^+$. 
\textbf{(C)} Quadrupole aperture (ribosome tRNA selection) creates complex field geometry matching correct tRNA charge configuration while rejecting incorrect tRNAs. 
\textbf{(D)} Membrane as charge field barrier: $-70$ mV potential creates selection gradient. 
\textbf{(E)} Ion channel as localized field aperture embedded in membrane barrier. 
\textbf{(F)} Unifying principle: all molecular selection is electromagnetic (charge configuration matching), not mechanical (size filtering), explaining temperature-independent prebiotic chemistry and universal biological selectivity. Field lines show force direction; color indicates potential magnitude.}
\label{fig:aperture_fields}
\end{figure*}

\subsection{Quantum Mechanical Foundation}

Electron transport occurs through quantum tunnelling, which is temperature-independent at the fundamental level.

\begin{definition}[Electron Tunneling Probability]
\label{def:tunneling_prob}
For a barrier of height $V_0$, width $d$, and electron energy $E < V_0$, the tunnelling probability is:
\begin{equation}
    P_{\text{tunnel}} = \frac{16E(V_0 - E)}{V_0^2} \exp\left(-\frac{2d}{\hbar}\sqrt{2m_e(V_0 - E)}\right)
\end{equation}
where $m_e$ is the electron mass.
\end{definition}

\begin{theorem}[Temperature Independence of Quantum Tunneling]
\label{thm:temp_independence}
The tunnelling probability $P_{\text{tunnel}}$ depends on barrier geometry $(V_0, d)$ and electron energy $E$, not on temperature $T$. Temperature affects only the population of electrons at energy $E$.
\end{theorem}

\begin{proof}
The tunnelling probability is determined by the Schrödinger equation:
\begin{equation}
    -\frac{\hbar^2}{2m_e}\frac{d^2\psi}{dx^2} + V(x)\psi = E\psi
\end{equation}

This equation contains no temperature dependence. Temperature enters only through the Fermi-Dirac distribution determining electron populations:
\begin{equation}
    f(E) = \frac{1}{e^{(E-E_F)/k_BT} + 1}
\end{equation}

At low temperatures, $f(E)$ approaches a step function, but tunnelling at energy $E$ remains possible with a probability of $P_{\text{tunnel}}(E)$.
\end{proof}

\subsection{Electron Transport as Categorical Operation}

\begin{definition}[Categorical Aperture from Charge Field]
\label{def:aperture_from_charge}
An electron transport event creates a categorical aperture $\mathcal{A}$ defined by the resulting electric field geometry:
\begin{equation}
    \mathcal{A} = \{ \mathbf{r} : \Phi(\mathbf{r}) \in [\Phi_{\min}, \Phi_{\max}] \}
\end{equation}
where $\Phi(\mathbf{r})$ is the electrostatic potential created by the charge partition.
\end{definition}

\begin{theorem}[Charge Fields as Molecular Filters]
\label{thm:charge_filter}
The electric field geometry created by electron transport functions as a molecular filter, selecting molecules based on charge configuration:
\begin{equation}
    P(\text{passage}|\text{molecule } M) = \begin{cases}
        \sim 1 & \text{if } \int_M \rho_M(\mathbf{r}) \cdot \mathbf{E}(\mathbf{r}) \, d^3r < 0 \\
        \sim 0 & \text{otherwise}
    \end{cases}
\end{equation}
\end{theorem}

\begin{proof}
A molecule $M$ with charge distribution $\rho_M$ in an electric field $\mathbf{E}$ experiences a force:
\begin{equation}
    \mathbf{F} = \int_M \rho_M(\mathbf{r}) \mathbf{E}(\mathbf{r}) \, d^3r
\end{equation}

Molecules with favourable charge distributions are attracted through the aperture; unfavourable distributions are repelled. This selection is geometric, depending on the spatial relationship between $\rho_M$ and $\mathbf{E}$, not on molecular velocity or temperature.
\end{proof}

\subsection{Energy Landscape of Charge Partitioning}

\begin{theorem}[Partitioning Free Energy]
\label{thm:partition_energy}
The creation and maintenance of a charge partition has associated free energy:
\begin{equation}
    \Delta G_{\text{partition}} = \Delta G_{\text{electrostatic}} + \Delta G_{\text{entropy}} + \Delta G_{\text{solvation}}
\end{equation}
where:
\begin{align}
    \Delta G_{\text{electrostatic}} &= \frac{1}{2}\epsilon_0 \int |\mathbf{E}|^2 \, d^3r > 0 \\
    \Delta G_{\text{entropy}} &= -T\Delta S_{\text{ion distribution}} \\
    \Delta G_{\text{solvation}} &\text{ depends on local dielectric environment}
\end{align}
\end{theorem}

In biological systems, this energy is offset by electron transport from high-energy donors to low-energy acceptors, making the process thermodynamically spontaneous:
\begin{equation}
    \Delta G_{\text{total}} = \Delta G_{\text{partition}} - \Delta G_{\text{redox}} < 0
\end{equation}

where $\Delta G_{\text{redox}}$ is the free energy of the redox reaction driving electron transport.

