\section{The Logical Priority of Actualisation}
\label{sec:priority_existence}

We now establish that actualisation (existence) is logically prior to non-actualisation (non-existence). Every negation presupposes what it negates—the statement "not-$X$" requires $X$ to exist as its referent. This logical priority has profound consequences: non-actualisations cannot exist without actualisations to anchor them, explaining why dark matter (accumulated non-actualisations) requires ordinary matter (actualisations) to exist, and resolving the ancient question "Why is there something rather than nothing?" The answer: "nothing" presupposes "something" to be meaningful, so pure non-existence is self-contradictory.

\subsection{The Presupposition Principle}

Every meaningful negation anchors to an existing referent. This is not merely a linguistic convention but a fundamental logical constraint.

\begin{axiom}[Negation Presupposes Affirmation]
\label{axiom:presupposition}
Every negation $\neg X$ presupposes the existence of $X$ as a meaningful referent:
\begin{equation}
    \boxed{\neg X \text{ is meaningful} \quad \Rightarrow \quad X \text{ exists as referent}}
\end{equation}
Without a referent $X$, the negation $\neg X$ is not false but undefined—it lacks semantic content.
\end{axiom}

Examples:
\begin{itemize}
    \item "The cup is not red" presupposes "the cup" exists as a referent
    \item "The electron is not at position $\mathbf{x}$" presupposes "the electron" and "position $\mathbf{x}$" exist as referents
    \item "This event did not happen" presupposes "this event" is a meaningful possibility that could have happened
\end{itemize}

\begin{theorem}[Negation Cannot Float Freely]
\label{thm:no_free_negation}
A negation without a referent is not a negation but a null expression:
\begin{equation}
    \boxed{\neg(\text{nothing}) = \text{undefined}}
\end{equation}
\end{theorem}

\begin{proof}
Consider the expression "$\neg X$" where $X$ has no referent—$X$ does not exist as a concept, object, or possibility.

The negation operator $\neg$ is a function that takes an input (the thing being negated) and produces an output (the negation of that thing). Formally:
\begin{equation}
    \neg: \text{Referents} \to \text{Negations}
\end{equation}

Without a referent $X$, the operator $\neg$ has no input, and the expression "$\neg X$" is ill-formed—it is not a well-defined function application.

Concretely:
\begin{itemize}
    \item "Not the cup" requires "the cup" to exist as a concept being negated
    \item "Not [undefined]" is not a statement at all—it has no semantic content
    \item Attempting to negate nothing yields nothing, not a negation
\end{itemize}

Therefore, every meaningful negation must anchor to an existing referent. Negations cannot "float freely" without referents.

Figure~\ref{fig:priority_experiments}(A) visualizes this: a blue circle labeled "CUP" (the referent) is surrounded by four boxes containing negations: "not red," "not book," "not there," "not car." All arrows point from the negations to the central referent. The annotation states: "REFERENT. Every $\neg X$ requires $X$ to exist. 'not-cup' is meaningless without 'cup.'"
\end{proof}

\begin{remark}[Philosophical Significance]
This result has profound implications for metaphysics and logic:

\textbf{1. Negation is not primitive:} Classical logic treats negation as a primitive operator alongside conjunction, disjunction, etc. But our analysis shows that negation is derivative—it depends on the prior existence of referents. Affirmation (existence) is logically prior to negation (non-existence).

\textbf{2. Negative facts are grounded:} The philosophical problem of "negative facts" (what makes "the cup is not red" true?) is resolved: negative facts are grounded in positive actualisations. "The cup is not red" is true because (a) the cup exists (actualisation), and (b) the cup has some other color (e.g., yellow), which excludes red.

\textbf{3. Absence requires presence:} Absence is not the mere lack of presence but a determined relation to presence. "The cup is absent from the table" presupposes both "the cup" and "the table" exist as referents. Pure absence (absence of everything) is incoherent.
\end{remark}

\subsection{The Intersection Argument for Existence}

A powerful argument for the necessity of existence comes from the convergence of negations.

\begin{theorem}[Existence from Negation Intersection]
\label{thm:intersection}
If infinitely many distinct negations $\{\neg P_1, \neg P_2, \ldots\}$ are meaningful, then their common referent must exist:
\begin{equation}
    \boxed{X = \bigcap_{i} \{\text{what } P_i \text{ negates}\} \neq \emptyset}
\end{equation}
The intersection of all negations' referents is non-empty—something must exist to anchor all these negations.
\end{theorem}

\begin{proof}
Let $\{P_i\}$ be a collection of properties (e.g., "red," "book," "on floor," "in Paris"), and let $\{\neg P_i\}$ be their negations applied to some putative entity $X$:
\begin{itemize}
    \item $\neg P_1$: "$X$ is not red"
    \item $\neg P_2$: "$X$ is not a book"
    \item $\neg P_3$: "$X$ is not on the floor"
    \item $\vdots$
\end{itemize}

Each negation $\neg P_i$ asserts "$X$ does not have property $P_i$." For this assertion to be meaningful:
\begin{enumerate}
    \item \textbf{$X$ must exist as a referent:} By Axiom~\ref{axiom:presupposition}, every negation presupposes its referent. If $X$ does not exist, then "$X$ is not $P_i$" is undefined, not meaningful.
    
    \item \textbf{Property $P_i$ must be applicable to $X$:} The negation "$X$ is not $P_i$" presupposes that $P_i$ is a property that $X$ could have (even if it doesn't). For example, "$X$ is not red" presupposes that $X$ is the kind of thing that can have color. Otherwise, the negation is a category error (e.g., "the number 7 is not red" is meaningless because numbers don't have colors).
\end{enumerate}

If ALL the negations $\{\neg P_i\}$ are meaningful, then:
\begin{itemize}
    \item $X$ must be the common entity to which all these negations apply
    \item $X$ must be the kind of thing to which all properties $\{P_i\}$ are applicable
\end{itemize}

The intersection of all referents is:
\begin{equation}
    X = \bigcap_{i} \{\text{entities to which } \neg P_i \text{ applies}\}
\end{equation}

This intersection is non-empty—it contains at least $X$ itself. If the intersection were empty, then at least one negation would lack a common referent with the others, making the entire collection of negations incoherent (they would be negating different things, not the same thing).

Conversely, if the intersection is non-empty, then something exists—namely, the entity $X$ that all the negations reference.

Figure~\ref{fig:priority_experiments}(B) visualizes this: a Venn diagram with three overlapping circles labeled "$\neg$red" (pink), "$\neg$book" (yellow), and "$\neg$car" (green). The intersection of all three circles (center) contains a small blue circle labeled "$n = \text{Cup}$." The annotation states: "Intersection of all negations = THE THING."
\end{proof}

\begin{example}[The Cup Defined by Negations]
\label{ex:cup_negations}
Consider a yellow cup on a table. This cup satisfies infinitely many negations:
\begin{itemize}
    \item \textbf{Color negations:} Not red, not blue, not green, not purple, $\ldots$ (infinitely many colors it's not)
    \item \textbf{Object negations:} Not a book, not a lamp, not a car, not a tree, not a star, $\ldots$ (infinitely many objects it's not)
    \item \textbf{Location negations:} Not on the floor, not in Paris, not on the moon, not in Andromeda, $\ldots$ (infinitely many places it's not)
    \item \textbf{Time negations:} Not at $t = 0$, not at $t = 1$, not at $t = 2$, $\ldots$ (infinitely many times it's not at)
\end{itemize}

Each of these negations presupposes the cup exists. The cup IS the entity that all these negations reference—it exists as the intersection of what all these "not-$X$" statements are negating.

The cup's existence is not merely asserted but forced by the meaningfulness of all these negations. If the cup did not exist, then none of these negations would be meaningful—they would all be undefined expressions.

This is the thermodynamic version of Descartes' cogito: "I think, therefore I am" becomes "I am negated, therefore I am." The vast field of negations surrounding an entity forces that entity into existence as their common referent.
\end{example}

\begin{remark}[Connection to Quantum Measurement]
This argument provides a new perspective on quantum measurement collapse. Before measurement, the system is in a superposition—no determinate value exists. Measurement creates a vast field of negations: "not spin-down" (if spin-up is measured), "not position $\mathbf{x}_1$" (if position $\mathbf{x}_2$ is measured), etc.

The accumulation of these negations forces a determinate value into existence as their common referent. The "collapse" is not a physical process but the logical consequence of negations requiring referents. The measured value emerges as the intersection of all the negations created by the measurement process.
\end{remark}

\subsection{Non-Actualisation Depends on Actualisation}

The logical priority of actualisation over non-actualisation is asymmetric: non-actualizations depend on actualizations, but actualizations do not depend on non-actualizations.

\begin{theorem}[Ontological Dependence]
\label{thm:dependence}
Non-actualisations depend ontologically on actualisations, but the converse does not hold:
\begin{equation}
    \boxed{\text{Non-actualisation } \neg A \text{ exists} \quad \Rightarrow \quad \text{Actualisation } A \text{ exists}}
\end{equation}
\begin{equation}
    \boxed{\text{Actualisation } A \text{ exists} \quad \not\Rightarrow \quad \text{Non-actualisation } \neg A \text{ exists}}
\end{equation}
\end{theorem}

\begin{proof}
\textbf{Forward direction ($\neg A$ depends on $A$):}

A non-actualisation $\neg A$ is the determination "$A$ did not happen." This determination presupposes:
\begin{enumerate}
    \item \textbf{$A$ is a coherent possibility:} For "$A$ did not happen" to be meaningful, $A$ must be something that could have happened. If $A$ is incoherent (e.g., "square circle"), then "$A$ did not happen" is not a meaningful determination but a category error.
    
    \item \textbf{Some actualisation occurred that resolved $A$ into "did not happen":} By Theorem~\ref{thm:resolution}, non-actualisations are created by actualisations. Before any actualisation, $A$ is undetermined—neither actual nor non-actual, just an unresolved possibility. Only when some actualisation $B \neq A$ occurs does $A$ become determined as "did not happen."
\end{enumerate}

Without the actualisation that created the determination, $\neg A$ would be undetermined—it would not exist as a determined fact. Therefore, $\neg A$ depends on some actualisation to exist.

\textbf{Reverse direction fails ($A$ does not depend on $\neg A$):}

An actualisation $A$ does not require non-actualisations to exist. Logically, $A$ could be the only entity in existence—a universe containing only $A$ and nothing else.

In such a universe:
\begin{itemize}
    \item $A$ exists (actualised)
    \item No non-actualisations exist (because there are no other possibilities to resolve into "did not happen")
    \item $A$'s existence is not conditioned on the existence of non-actualisations
\end{itemize}

Non-actualisations arise BECAUSE $A$ exists (everything else becomes "not $A$"), but $A$'s existence is not conditioned on them. The dependence is one-way: non-actualisations depend on actualisations, not vice versa.

Figure~\ref{fig:priority_experiments}(C) visualizes this asymmetry: a blue box labeled "Actualisation (exists)" with a red arrow pointing to a purple box labeled "Non-Actualisation (depends)." The annotation states: "requires" (on the arrow) and "$\times$ No reverse dependence. Actualisation does NOT require non-actualisation."
\end{proof}

\begin{figure*}[htbp]
\centering
\includegraphics[width=0.95\textwidth]{figures/priority_existence_panel.png}
\caption{\textbf{The Logical Priority of Actualisation: Negation Presupposes Affirmation—Something Is Necessary.} 
\textbf{(A)} Negation requires a referent: blue circle labeled "CUP" (referent) surrounded by four boxes: "not red," "not book," "not there," "not car." All arrows point to central referent. Annotation: "REFERENT. Every $\neg X$ requires $X$ to exist. 'not-cup' is meaningless without 'cup.'" Demonstrates that negations anchor to referents. 
\textbf{(B)} Existence from negation intersection: Venn diagram with three overlapping circles labeled "$\neg$red" (pink), "$\neg$book" (yellow), "$\neg$car" (green). Intersection (center) contains small blue circle "$n = \text{Cup}$." Annotation: "Intersection of all negations = THE THING." Demonstrates that common referent must exist. 
\textbf{(C)} Non-actualisation depends on actualisation: blue box "Actualisation (exists)" with red arrow labeled "requires" pointing to purple box "Non-Actualisation (depends)." Annotation: "$\times$ No reverse dependence. Actualisation does NOT require non-actualisation. Dark matter requires ordinary matter." Demonstrates asymmetric dependence. 
\textbf{(D)} Pure nothing is self-contradictory: logical chain with downward arrows. Gray box: "Suppose 'nothing exists'." Yellow box: "'Nothing exists' is a determination." Orange box: "Determination = Non-actualisation." Red box: "Non-actualisation requires actualisation." Red box with $\times$: "CONTRADICTION $\times$." Demonstrates impossibility of pure nothing. 
\textbf{(E)} Existence is logically necessary: yellow hexagon labeled "$\exists$" (something exists). Annotation: "$\neg (\exists A : A \text{ is actualised})$. In every possible world, something exists. Because: 'Empty world' = determination. Determination requires referent. $\therefore$ Something must exist for 'nothing' to mean anything." Demonstrates necessity of existence. 
\textbf{(F)} Structure of reality: three concentric circles. Innermost (blue): "$A$" (actualisation, center, primary). Middle ring (gray): "Paired $\neg A$ (ordinary matter)." Outermost ring (purple): "Unpaired $\neg A$ (dark matter)." Annotation: "Actualisation (center, primary). Paired $\neg A$ (ordinary matter). Unpaired $\neg A$ (dark matter)." Demonstrates three-level ontological hierarchy.}
\label{fig:priority_experiments}
\end{figure*}

\begin{corollary}[Dark Matter Requires Ordinary Matter]
\label{cor:dark_requires_ordinary}
Dark matter (accumulated non-actualisations) cannot exist without ordinary matter (actualisations) to anchor it. Formally:
\begin{equation}
    \boxed{M_{\text{dark}} > 0 \quad \Rightarrow \quad M_{\text{ordinary}} > 0}
\end{equation}
\end{corollary}

\begin{proof}
Dark matter is the accumulated "what didn't happen"—the mass-energy of resolved non-actualisations (Section~\ref{sec:dark_matter}).

Each "didn't happen" presupposes a "did happen" that resolved it (Theorem~\ref{thm:dependence}). Without actualisations, there would be no determinations, hence no determined non-actualisations, hence no dark matter.

Therefore, dark matter cannot exist without ordinary matter to anchor it. The observed dark matter in the universe is evidence that ordinary matter exists—the vast halo of non-actualisations requires a core of actualisations to reference.

The annotation in Figure~\ref{fig:priority_experiments}(C) states: "Dark matter requires ordinary matter."
\end{proof}

\begin{remark}[Cosmological Implications]
This result has profound cosmological implications:

\textbf{1. Dark matter follows ordinary matter:} The distribution of dark matter in the universe should correlate with the distribution of ordinary matter, because dark matter (non-actualisations) requires ordinary matter (actualisations) to anchor it. This is consistent with observations: dark matter halos surround galaxies, not empty voids.

\textbf{2. Dark matter density decreases with distance:} By the shell structure (Section~\ref{sec:geometry_non_actualisation}), non-actualisations at larger categorical distance from actualisations have lower "pairing strength." This predicts that dark matter density should decrease with distance from ordinary matter, consistent with observed halo profiles.

\textbf{3. No "dark matter only" regions:} There cannot be regions of space containing only dark matter with no ordinary matter, because dark matter requires ordinary matter to exist. Any observed "dark matter dominated" region must contain at least some ordinary matter (even if below detection threshold) to anchor the dark matter.
\end{remark}

\subsection{The Impossibility of Pure Nothing}

The logical priority of actualisation leads to a surprising result: pure non-existence is self-contradictory.

\begin{theorem}[Impossibility of Pure Nothing]
\label{thm:no_nothing}
Pure nothing—the absence of all actualisation—is self-contradictory. Formally:
\begin{equation}
    \boxed{\text{"Nothing exists"} \quad \Rightarrow \quad \text{CONTRADICTION}}
\end{equation}
\end{theorem}

\begin{proof}
Suppose there is "nothing"—no actualisation whatsoever. We will show this supposition leads to contradiction.

\textbf{Step 1: "Nothing" is a determination.}

The statement "nothing exists" is itself a determination—a definite claim about the state of reality. It asserts that the property "existence" is not instantiated by anything.

\textbf{Step 2: Determinations are non-actualisations.}

A determination is a resolved fact. "Nothing exists" is the determination that "existence did not happen"—it is a non-actualisation of existence.

Formally:
\begin{equation}
    \text{"Nothing exists"} \equiv \neg(\exists X)
\end{equation}

This is a negation—the negation of existence.

\textbf{Step 3: Non-actualisations require actualisations.}

By Theorem~\ref{thm:dependence}, every non-actualisation depends on some actualisation to anchor it. The non-actualisation "existence did not happen" presupposes that "existence" is a meaningful referent—something that could have happened but didn't.

For "existence" to be a meaningful referent, some actualisation must exist to ground the concept of existence. Without any actualisation, "existence" is not a meaningful concept, so "non-existence" (the negation of existence) is undefined.

\textbf{Step 4: Contradiction.}

The supposition "nothing exists" (no actualisation) requires some actualisation to exist (to ground the determination "nothing exists"). This is a contradiction:
\begin{equation}
    \text{No actualisation exists} \quad \land \quad \text{Some actualisation exists}
\end{equation}

Therefore, the supposition "nothing exists" is self-contradictory. Pure nothing is impossible.

Figure~\ref{fig:priority_experiments}(D) visualizes this argument as a logical chain:
\begin{enumerate}
    \item Suppose "nothing exists"
    \item $\downarrow$
    \item "Nothing exists" is a determination
    \item $\downarrow$
    \item Determination = Non-actualisation
    \item $\downarrow$
    \item Non-actualisation requires actualisation
    \item $\downarrow$
    \item CONTRADICTION $\times$
\end{enumerate}

The boxes are color-coded: gray (supposition), yellow (determination), orange (non-actualisation), red (contradiction).
\end{proof}

\begin{remark}[Why "Nothing" Is Self-Undermining]
The argument above shows that "nothing" is self-undermining: asserting "nothing exists" presupposes the meaningfulness of "existence," which requires something to exist to ground the concept.

This is not a linguistic trick but a deep logical constraint. The determination "nothing" cannot be made without something to anchor the determination. Pure absence requires presence to be meaningful.

This resolves Heidegger's question "Why is there something rather than nothing?" The question presupposes that "nothing" is a coherent alternative to "something," but our analysis shows that "nothing" presupposes "something" to be meaningful. The question is therefore malformed—"nothing" cannot be the alternative to "something" because "nothing" depends on "something."
\end{remark}

\begin{theorem}[Existence Is Logically Necessary]
\label{thm:something_necessary}
The existence of something (some actualisation) is logically necessary. In every possible world, something is actualised:
\begin{equation}
    \boxed{\Box \, (\exists A : A \text{ is actualised})}
\end{equation}
where $\Box$ denotes necessity (true in all possible worlds).
\end{theorem}

\begin{proof}
\textbf{Proof by contraposition:} By Theorem~\ref{thm:no_nothing}, pure nothing (no actualisation) is self-contradictory. If pure nothing is impossible, then its negation—something exists—is necessary.

Formally:
\begin{equation}
    \neg \Box \neg (\exists A) \quad \Leftrightarrow \quad \Box (\exists A)
\end{equation}

"It is not necessary that nothing exists" is equivalent to "It is necessary that something exists."

\textbf{Alternative proof:} Consider any possible world $W$. We will show that $W$ contains at least one actualisation.

\textbf{Case 1: $W$ contains some entity.} Then that entity is an actualisation, so $\exists A$ in $W$.

\textbf{Case 2: $W$ contains no entities.} Then $W$ is "empty." But "empty" is itself a determination—the determination that "$W$ contains no entities." This determination is an actualisation (the actualisation of the state "empty $W$"). Therefore, $\exists A$ in $W$ (namely, the actualisation of emptiness).

\textbf{Case 3: $W$ is not a world at all.} If $W$ contains no entities and no determinations, then $W$ is not a possible world but the absence of any state. This is not a coherent possibility—it is not a "world" but nothing. By Theorem~\ref{thm:no_nothing}, this is impossible.

Therefore, in every possible world, something is actualised. Existence is necessary.

Figure~\ref{fig:priority_experiments}(E) visualizes this: a yellow hexagon labeled "$\exists$" (something exists) with the annotation: "$\neg (\exists A : A \text{ is actualised})$. In every possible world, something exists. Because: 'Empty world' = determination. Determination requires referent. $\therefore$ Something must exist for 'nothing' to mean anything."
\end{proof}

\begin{remark}[Resolution of Leibniz's Question]
Leibniz famously asked: "Why is there something rather than nothing?" This question has puzzled philosophers for centuries. Our framework provides a definitive answer:

\textbf{The question is malformed.} "Nothing" presupposes "something" to be meaningful (Axiom~\ref{axiom:presupposition}). Pure non-existence requires existence to be a coherent concept (Theorem~\ref{thm:no_nothing}). Therefore, "nothing" cannot be the alternative to "something"—"nothing" depends on "something."

The correct question is not "Why something rather than nothing?" but "Why this particular something rather than some other something?" The existence of something is necessary; what is contingent is which particular things exist.

This resolves the puzzle: there is no mystery about why something exists (it must exist), only about why these particular things exist (contingent facts about our universe).
\end{remark}

\subsection{Mutual Constitution of Actual and Non-Actual}

Although non-actualisations depend on actualisations (Theorem~\ref{thm:dependence}), the relationship is not entirely one-way. Actualisations are partly constituted by their non-actualisations.

\begin{theorem}[Mutual Constitution]
\label{thm:mutual_constitution}
The identity of an actualisation $A$ is partly constituted by its non-actualisations—by what $A$ is not. Formally:
\begin{equation}
    \boxed{\text{Identity}(A) = \text{Intrinsic}(A) \cup \{\neg B : B \neq A, \, d(A, B) \leq r_{\text{pair}}\}}
\end{equation}
where $\text{Intrinsic}(A)$ are the intrinsic properties of $A$, and $\{\neg B\}$ are the negations (non-actualisations) within the pairing radius.
\end{theorem}

\begin{proof}
The identity of an actualisation $A$ includes two components:

\textbf{1. Intrinsic properties:} What $A$ is in itself—its positive determinations. For example:
\begin{itemize}
    \item The cup is yellow (color)
    \item The cup is ceramic (material)
    \item The cup is cylindrical (shape)
\end{itemize}

\textbf{2. Negative properties:} What $A$ is not—its negative determinations. For example:
\begin{itemize}
    \item The cup is not red (color negation)
    \item The cup is not plastic (material negation)
    \item The cup is not cubic (shape negation)
\end{itemize}

These negative determinations are not merely epistemic (things we know about the cup) but constitutive (what makes the cup what it is). The cup's "not-red-ness" is as much a part of its identity as its "yellow-ness."

By Section~\ref{sec:geometry_non_actualisation}, these negative properties form the pairing structure with nearby actualisations. The cup's "not a mug"-ness pairs with the mug's "not a cup"-ness, creating a closed reference loop (Theorem~\ref{thm:pairing_structure}).

Therefore, the identity of $A$ is:
\begin{equation}
    \text{Identity}(A) = \text{Intrinsic}(A) \cup \{\neg B : d(A, B) \leq r_{\text{pair}}\}
\end{equation}

The intrinsic properties plus the paired negations within the pairing radius.

\textbf{Dependence asymmetry:} Although $A$'s identity includes negations $\{\neg B\}$, the existence of $A$ does not depend on these negations (Theorem~\ref{thm:dependence}). The negations depend on $A$ to exist, not vice versa. But once both $A$ and the negations exist, they mutually constitute each other's identities.
\end{proof}

\begin{corollary}[No Actualisation Is Fully Isolated]
\label{cor:no_isolation}
Every actualisation is relationally connected to every other actualisation through mutual non-actualisation:
\begin{equation}
    \boxed{\forall A, B: \quad A \xleftrightarrow{\neg} B}
\end{equation}
where $\xleftrightarrow{\neg}$ denotes mutual non-actualisation (each is in the other's non-actualisation space).
\end{corollary}

\begin{proof}
By Theorem~\ref{thm:universal_mutual}, all distinct actualisations are mutually non-actualising: $A \in \neg B$ and $B \in \neg A$ for all $A \neq B$.

This creates a universal web of relations: every actualisation is defined partly by what it is not, which includes all other actualisations. No actualisation is fully isolated—each is embedded in a network of mutual exclusions.

For actualisations within the pairing radius ($d(A, B) \leq r_{\text{pair}}$), these mutual exclusions form stable reference loops (Theorem~\ref{thm:pairing_structure}), constituting the structure of ordinary matter.

For actualisations beyond the pairing radius ($d(A, B) > r_{\text{pair}}$), the mutual exclusions remain but do not form stable structures—they contribute to the unpaired non-actualisations (dark matter).
\end{proof}

\begin{remark}[Structuralism Vindicated]
This result provides a thermodynamic foundation for structuralism in metaphysics—the view that objects are not independent substances but nodes in a network of relations.

Our framework shows why structuralism is true: isolated objects (with no mutual references) are thermodynamically unstable. They lack the entropic support of the reference network. Only structured networks (with mutual references within the pairing radius) are stable.

The identity of each object is partly constituted by its relations to other objects. An electron is defined partly by not being a proton, a proton by not being a neutron, etc. The entire structure of the Standard Model is a network of mutual categorical exclusions—particles are what they are by virtue of what they are not.
\end{remark}

\subsection{The Structure of Reality}

Combining the results above, we obtain a complete picture of reality's structure.

\begin{theorem}[Reality as Actualisation-Anchored Non-Actualisation Web]
\label{thm:reality_structure}
The structure of reality consists of three layers:
\begin{enumerate}[(i)]
    \item \textbf{Actualisations (primary):} The logically prior entities that anchor all determinations. These are the "things that exist"—the positive facts about reality.
    
    \item \textbf{Paired non-actualisations (ordinary matter):} The mutual exclusions between nearby actualisations (within pairing radius $r \leq r_{\text{pair}}$), forming the structured, observable, partitionable substance we call ordinary matter.
    
    \item \textbf{Unpaired non-actualisations (dark matter):} The distant non-actualisations (beyond pairing radius $r > r_{\text{pair}}$) without local anchors, forming the non-partitionable, non-observable, gravitationally present substance we call dark matter.
\end{enumerate}
The ratio between (ii) and (iii) is determined by the geometry of non-actualisation space: $M_{\text{dark}} / M_{\text{ordinary}} \approx 5.4$ (Theorem~\ref{thm:ratio_shells}).
\end{theorem}

\begin{proof}
Combines results from Sections~\ref{sec:dark_matter} and \ref{sec:geometry_non_actualisation}:

\textbf{(i) Actualisations exist necessarily:} By Theorem~\ref{thm:something_necessary}, something must be actualised in every possible world. Actualisations are logically prior—they anchor all non-actualisations (Theorem~\ref{thm:dependence}).

\textbf{(ii) Each actualisation generates non-actualisations:} By Theorem~\ref{thm:resolution}, each actualisation resolves infinitely many alternatives into "did not happen." These non-actualisations organize into exponentially growing shells (Theorem~\ref{thm:shell_growth}).

\textbf{(iii) Close non-actualisations pair:} By Theorem~\ref{thm:pairing_structure}, non-actualisations within pairing radius form stable reference loops, constituting the structure of ordinary matter.

\textbf{(iv) Distant non-actualisations remain unpaired:} By Definition~\ref{def:unpaired}, non-actualisations beyond pairing radius lack local anchors and remain structureless, constituting dark matter.

\textbf{(v) The ratio is $\approx 5:1$:} By Theorem~\ref{thm:ratio_shells}, the geometric structure of shells determines $M_{\text{dark}} / M_{\text{ordinary}} \approx k/(k-1) \approx 5.4$ for $k = 3$ (ternary categorical branching).

Figure~\ref{fig:priority_experiments}(F) visualizes this structure: three concentric circles. The innermost circle (blue) is labeled "$A$" (actualisation, center, primary). The middle ring (gray) is labeled "Paired $\neg A$ (ordinary matter)." The outermost ring (purple) is labeled "Unpaired $\neg A$ (dark matter)." The annotation states: "Actualisation (center, primary). Paired $\neg A$ (ordinary matter). Unpaired $\neg A$ (dark matter)."
\end{proof}

\begin{remark}[The Ontological Hierarchy]
Reality has a three-level ontological hierarchy:

\textbf{Level 1 (Primary):} Actualisations—the things that exist. These are logically necessary (Theorem~\ref{thm:something_necessary}) and ontologically independent (they do not depend on non-actualisations to exist).

\textbf{Level 2 (Derivative):} Paired non-actualisations—the structured network of mutual exclusions. These depend on actualisations to exist (Theorem~\ref{thm:dependence}) but partly constitute the identities of actualisations (Theorem~\ref{thm:mutual_constitution}). This is ordinary matter—observable, partitionable, structured.

\textbf{Level 3 (Derivative):} Unpaired non-actualisations—the unstructured residue of distant alternatives. These also depend on actualisations to exist but do not form stable structures. This is dark matter—non-observable, non-partitionable, gravitationally present.

The hierarchy is asymmetric: Level 1 is primary, Levels 2 and 3 are derivative. But Levels 2 and 3 vastly outnumber Level 1 in mass-energy (by a factor of $\sim 5:1$), so the derivative dominates the primary in quantity, even though the primary is logically prior.
\end{remark}

\subsection{Summary: Existence Precedes Non-Existence}

The main results of this section:

\begin{enumerate}
    \item \textbf{Negation presupposes affirmation:} Every "not-$X$" requires $X$ to exist as a referent (Axiom~\ref{axiom:presupposition})
    
    \item \textbf{Negations cannot float freely:} A negation without a referent is undefined, not false (Theorem~\ref{thm:no_free_negation})
    
    \item \textbf{Existence from negation intersection:} If infinitely many negations are meaningful, their common referent must exist (Theorem~\ref{thm:intersection})
    
    \item \textbf{Non-actualisations depend on actualisations:} Non-existence depends on existence, not vice versa (Theorem~\ref{thm:dependence})
    
    \item \textbf{Dark matter requires ordinary matter:} Accumulated non-actualisations require actualisations to anchor them (Corollary~\ref{cor:dark_requires_ordinary})
    
    \item \textbf{Pure nothing is impossible:} "Nothing exists" is self-contradictory (Theorem~\ref{thm:no_nothing})
    
    \item \textbf{Something is necessary:} In every possible world, something is actualised (Theorem~\ref{thm:something_necessary})
    
    \item \textbf{Mutual constitution:} Actualisations are partly defined by what they're not (Theorem~\ref{thm:mutual_constitution})
    
    \item \textbf{Reality structure:} Actualisations (primary) + paired non-actualisations (ordinary matter) + unpaired non-actualisations (dark matter) (Theorem~\ref{thm:reality_structure})
\end{enumerate}

\begin{remark}[Resolution of Classical Paradoxes]
This analysis resolves multiple classical puzzles in metaphysics and logic:

\textbf{1. Parmenides' puzzle:} "Non-being cannot be" (Fragment 2). Parmenides was correct: non-being (non-actualisation) depends on being (actualisation) to exist. Non-being is not independent but derivative.

\textbf{2. Leibniz's question:} "Why is there something rather than nothing?" (Principles of Nature and Grace, 1714). The question is malformed: "nothing" presupposes "something" to be meaningful, so "nothing" cannot be the alternative to "something."

\textbf{3. Aristotle's place paradox:} Place must exist because "not this place" requires "place" as its referent (Physics, Book IV). Every "not here" presupposes a "here."

\textbf{4. The problem of negative facts:} What makes "the cup is not red" true? Answer: the positive actualisation (the cup is yellow) that excludes red. Negative facts are grounded in positive actualisations.

\textbf{5. Heidegger's question:} "Why are there beings rather than nothing?" (Introduction to Metaphysics, 1935). Same answer as Leibniz: "nothing" depends on "beings" to be meaningful.

The logical priority of actualisation provides a unified resolution: existence is primary, non-existence is derivative and dependent. This is not merely a linguistic convention but a fundamental constraint on the structure of reality.
\end{remark}


