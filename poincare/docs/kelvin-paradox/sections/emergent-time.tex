% Section: Time as Emergent Categorical Completion Rate

We establish that time is not fundamental but emergent from categorical completion. The rate of categorical completion is constant due to self-similar structure, explaining the uniform ``flow'' of time experienced by observers.

\subsection{Time from Categorical Completion}

\begin{definition}[Categorical Completion Rate]
The categorical completion rate at cosmic state $\gamma(t)$ is:
\begin{equation}
\rho_C(t) = \frac{d|\gamma(t)|}{dt}
\end{equation}
where $|\gamma(t)|$ is the count of completed categories.
\end{definition}

\begin{definition}[Emergent Time]
Time $\tau$ as experienced by observers is defined as:
\begin{equation}
\tau = \int_0^t \rho_C(t') \, dt' = |\gamma(t)| - |\gamma(0)|
\end{equation}
Experienced time equals accumulated categorical completions.
\end{definition}

\begin{theorem}[Time is Emergent]
\label{thm:time_emergent}
Time does not exist independently; it emerges from the process of categorical completion observed by finite entities.
\end{theorem}

\begin{proof}
For time to ``exist'' independently requires:
\begin{enumerate}
    \item A substrate that ``flows'' regardless of events
    \item A measure independent of any process
    \item Existence even in absence of categorical change
\end{enumerate}

But:
\begin{enumerate}
    \item All ``time measurement'' is categorical completion (clock ticks, oscillations)
    \item No measure exists independent of terminated processes
    \item At singularity (zero categories), time is undefined, not zero
\end{enumerate}

Therefore, time is identical with observed categorical completion rate, not an independent substrate.
\end{proof}

\subsection{Constant Rate from Self-Similarity}

\begin{theorem}[Constant Completion Rate]
\label{thm:constant_rate}
The rate of categorical completion is constant despite exponential growth in category count.
\end{theorem}

\begin{proof}
From the $3^k$ branching theorem, at level $k$:
\begin{equation}
|\mathcal{C}^{(k)}| = 3^k \cdot |\mathcal{C}^{(0)}|
\end{equation}

Each category at level $k$ spawns 3 categories at level $k+1$:
\begin{equation}
\frac{|\mathcal{C}^{(k+1)}|}{|\mathcal{C}^{(k)}|} = 3 = \text{constant}
\end{equation}

The completion rate is:
\begin{equation}
\rho_C = \frac{\text{new categories}}{\text{existing categories}} = \frac{3^{k+1} - 3^k}{3^k} = 3 - 1 = 2 = \text{constant}
\end{equation}

The RATE of branching is constant even as the NUMBER grows exponentially. This is the self-similar structure of categorical space.
\end{proof}

\begin{corollary}[Uniform Time Flow]
Observers experience uniform time flow because:
\begin{enumerate}
    \item Time = categorical completion rate
    \item Completion rate is constant (self-similar structure)
    \item Therefore time ``flows'' uniformly
\end{enumerate}
This is not because time is fundamental, but because the categorical branching ratio is.
\end{corollary}

\subsection{Time at Singularity}

\begin{theorem}[No Time at Singularity]
\label{thm:no_time_singularity}
At the singularity, time does not exist (not merely $t = 0$).
\end{theorem}

\begin{proof}
At singularity:
\begin{enumerate}
    \item Categories: $|\mathcal{C}| = 1$ (or 0, equivalently---no distinctions)
    \item Oscillations: None (nothing to oscillate around)
    \item Completion rate: $\rho_C = 0$ (nothing to complete)
\end{enumerate}

Time $\tau = \int \rho_C \, dt$ is undefined when $\rho_C = 0$ everywhere. This is not ``frozen time'' or ``$t = 0$'' but the non-existence of time as a concept.

The question ``what happened before the Big Bang?'' is malformed. Before categorical completion began, there was no ``before''---time itself is a product of completion.
\end{proof}

\subsection{Categories Beget Categories}

\begin{theorem}[Generative Categories]
\label{thm:generative}
Every completed category generates new potential categories:
\begin{equation}
C_{completed} \to \{C_1^{new}, C_2^{new}, \ldots, C_n^{new}\}
\end{equation}
with $n \geq 2$ (at minimum, the tri-dimensional S-space decomposition gives $n = 3$).
\end{theorem}

\begin{proof}
When category $C$ is completed:
\begin{enumerate}
    \item All ``cannot happen'' states relative to $C$ are resolved
    \item These resolutions create new categorical facts
    \item Each new fact opens new possibility spaces
    \item By tri-dimensional decomposition, at least 3 new directions emerge
\end{enumerate}

This self-generating property ensures categorical completion never halts (until singularity forces restart).
\end{proof}

\begin{corollary}[Time Cannot Stop]
Since categories beget categories and time = completion rate, time cannot ``stop'' while categories remain. Only at singularity (one category) does the generative process halt, and with it, time.
\end{corollary}

\subsection{The Arrow of Time}

\begin{theorem}[Categorical Arrow]
\label{thm:categorical_arrow}
The arrow of time is identical with the direction of categorical completion.
\end{theorem}

\begin{proof}
The arrow of time requires:
\begin{enumerate}
    \item A distinction between ``past'' and ``future''
    \item An asymmetry favoring one direction
    \item A universal, objective direction
\end{enumerate}

Categorical completion provides all three:
\begin{enumerate}
    \item Past = completed categories; Future = potential categories
    \item Asymmetry from irreversibility (categories cannot uncomplete)
    \item Universal (all observers embedded in same categorical space)
\end{enumerate}

Therefore, the arrow of time IS the arrow of categorical completion.
\end{proof}

\begin{remark}
This resolves the puzzle of why time has a direction. Standard physics has time-symmetric laws, yet time flows one way. The asymmetry comes not from physical laws but from categorical structure---completion is irreversible, giving time its arrow.
\end{remark}

\begin{figure}[H]
\centering
\includegraphics[width=\textwidth]{figures/emergent_time_panel.png}
\caption{Time as emergent categorical completion rate. (A) Time emerging from category counting, not as fundamental substrate. (B) Constant branching ratio giving uniform time flow. (C) Self-similar structure: each level looks the same. (D) Singularity: no categories, no time (undefined, not zero). (E) Categories begetting categories (generative process). (F) Arrow of time = direction of completion.}
\label{fig:emergent_time}
\end{figure}

