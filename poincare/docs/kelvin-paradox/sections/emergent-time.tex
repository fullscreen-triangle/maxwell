\section{Time as Emergent Categorical Completion}
\label{sec:time}

One of the deepest questions in physics is the nature of time. Is time a fundamental feature of reality, an independent dimension in which events unfold? Or is time emergent, arising from more fundamental processes? We demonstrate that time is not fundamental but emergent from the process of categorical completion. The "flow" of time—the subjective experience of duration and succession—is identical to the rate at which categorical distinctions are completed. The uniformity of this flow, the fact that time seems to pass at a constant rate, arises from the self-similar structure of categorical space: each level of the recursive hierarchy has the same branching ratio, producing a constant completion rate despite exponential growth in the number of categories. This framework resolves longstanding puzzles about the arrow of time, the nature of the present moment, and the meaning of "before the Big Bang."

\subsection{Time from Categorical Completion}

We begin by defining time in terms of categorical completion rather than as an independent substrate.

\begin{definition}[Categorical Completion Rate]
\label{def:completion_rate}
The \emph{categorical completion rate} at cosmic state $\gamma(t)$ is the rate at which new categorical distinctions are completed:
\begin{equation}
\rho_C(t) = \frac{d|\gamma(t)|}{dt}
\end{equation}
where $|\gamma(t)|$ is the number of categories that have been completed by time $t$, and $t$ is the parameter (coordinate time) used to track the evolution of the system.
\end{definition}

This definition treats $t$ as a parameter—a label for different states of the system—rather than as a fundamental entity. The physically meaningful quantity is not $t$ itself but the rate of change of categorical completion with respect to $t$.

\begin{definition}[Emergent Time]
\label{def:emergent_time}
The \emph{emergent time} $\tau$ as experienced by observers is defined as the accumulated number of categorical completions:
\begin{equation}
\tau = \int_0^t \rho_C(t') \, dt' = |\gamma(t)| - |\gamma(0)|
\end{equation}
Experienced time equals the total number of categorical distinctions that have been completed since some reference state (typically the Big Bang, where $|\gamma(0)| = 0$).
\end{definition}

This definition makes time a \emph{derived} quantity rather than a fundamental one. Time is not the independent variable in which events occur; rather, time is the \emph{count} of events (categorical completions) that have occurred. The distinction is subtle but profound: in the standard view, time exists independently and events happen "in" time. In the emergent view, events (categorical completions) are primary, and time is the measure of how many events have occurred.

\begin{theorem}[Time is Emergent, Not Fundamental]
\label{thm:time_emergent}
Time does not exist as an independent substrate. It emerges from the process of categorical completion as observed by finite entities embedded within the system.
\end{theorem}

\begin{proof}
For time to exist independently—as a fundamental feature of reality—it would need to satisfy the following criteria:
\begin{enumerate}
    \item \emph{Independent substrate}: Time would be a "container" or "stage" in which events occur, existing whether or not any events actually happen.
    \item \emph{Measurement independence}: Time could be measured independently of any physical process, without reference to clocks, oscillations, or state changes.
    \item \emph{Existence in absence of change}: Time would continue to exist even in a completely static universe where no categorical changes occur.
\end{enumerate}

We now show that all three criteria fail.

\textbf{Criterion 1: Independent substrate.}

All actual measurements of time are measurements of categorical completion. A clock "measures time" by counting oscillations—each tick is a completed categorical state. An atomic clock counts the oscillations of cesium atoms. A pendulum clock counts the swings of the pendulum. A biological clock counts metabolic cycles. In every case, what we call "time measurement" is actually \emph{categorical completion counting}.

There is no way to measure time without measuring some process, and every process is a sequence of categorical completions. Therefore, time is not an independent substrate but the accumulated count of completions.

\textbf{Criterion 2: Measurement independence.}

By Theorem~\ref{thm:termination}, observation requires termination—a completed outcome. An observer cannot measure time "as it flows" but can only count completed events. The "present moment" is not observable; only the past (completed events) can be observed.

Therefore, time measurement is necessarily dependent on categorical completion. There is no measurement of time that does not reduce to counting completed categories.

\textbf{Criterion 3: Existence in absence of change.}

Consider a hypothetical universe in which no categorical changes occur—no oscillations, no state transitions, no distinctions. In such a universe, $\rho_C(t) = 0$ for all $t$. By Definition~\ref{def:emergent_time}, the emergent time is:
\begin{equation}
\tau = \int_0^t 0 \, dt' = 0
\end{equation}

The emergent time does not advance. From the perspective of any observer in such a universe (if observers could exist), no time passes. The universe is "frozen" not because time exists but is stopped, but because time does not exist at all—there are no categorical completions to count.

Therefore, time does not exist in the absence of categorical change. Time is not an independent entity but an emergent property of categorical completion.
\end{proof}

This theorem establishes that time is a derived concept, not a fundamental one. The implications are profound: questions like "what happened before the Big Bang?" or "what is time made of?" are revealed to be malformed. Time is not a substance or a dimension—it is a counting process.

\subsection{Constant Rate from Self-Similarity}

A potential objection to the emergent view of time is that it seems to predict non-uniform time flow. If time is the count of categorical completions, and the number of categories grows exponentially (as established in Theorem~\ref{thm:recursive_enumeration}), shouldn't time "speed up" as more categories are completed? We now prove that the rate of categorical completion is constant despite exponential growth in category count, due to the self-similar structure of categorical space.

\begin{theorem}[Constant Completion Rate]
\label{thm:constant_rate}
The rate of categorical completion $\rho_C$ is constant throughout cosmic evolution (except at the singularity), despite the exponential growth in the total number of categories. This constancy arises from the self-similar structure of categorical space.
\end{theorem}

\begin{proof}
By the $3^k$ branching theorem (Theorem~\ref{thm:3k_branching}), the number of categories at hierarchical level $k$ is:
\begin{equation}
|\mathcal{C}^{(k)}| = 3^k \cdot |\mathcal{C}^{(0)}|
\end{equation}
where $|\mathcal{C}^{(0)}|$ is the number of categories at the base level (typically $|\mathcal{C}^{(0)}| = 1$, the primordial inside-outside distinction).

Each category at level $k$ spawns exactly 3 categories at level $k+1$, corresponding to the three dimensions of S-space: $(\Sentropy_k, \Sentropy_t, \Sentropy_e)$. Therefore, the ratio of categories at successive levels is:
\begin{equation}
\frac{|\mathcal{C}^{(k+1)}|}{|\mathcal{C}^{(k)}|} = \frac{3^{k+1}}{3^k} = 3 = \text{constant}
\end{equation}

The number of \emph{new} categories created at level $k+1$ is:
\begin{equation}
\Delta |\mathcal{C}^{(k+1)}| = |\mathcal{C}^{(k+1)}| - |\mathcal{C}^{(k)}| = 3^{k+1} - 3^k = 3^k(3 - 1) = 2 \cdot 3^k
\end{equation}

The \emph{rate} of new category creation relative to existing categories is:
\begin{equation}
\frac{\Delta |\mathcal{C}^{(k+1)}|}{|\mathcal{C}^{(k)}|} = \frac{2 \cdot 3^k}{3^k} = 2 = \text{constant}
\end{equation}

This is the key result: the rate at which new categories are created, \emph{relative to the number of existing categories}, is constant. Even though the absolute number of categories grows exponentially ($3^k$), the relative growth rate remains fixed at 2 (meaning each level doubles the number of categories from the previous level, after accounting for the base).

Since observers are embedded within the categorical structure—they are themselves composed of categorical distinctions—they experience time as the relative rate of completion, not the absolute count. From the observer's perspective, the rate is constant:
\begin{equation}
\rho_C = \frac{d|\gamma|/dt}{|\gamma|} = \text{constant}
\end{equation}

This is analogous to exponential growth in economics: if a population grows at a constant percentage rate (e.g., 2\% per year), the absolute number grows exponentially, but the relative rate (the percentage) remains constant. Observers embedded in the population experience the growth as uniform, not accelerating.
\end{proof}

\begin{corollary}[Uniform Time Flow]
\label{cor:uniform_time}
Observers experience uniform time flow—the subjective sense that time passes at a constant rate—because:
\begin{enumerate}[(i)]
    \item Time is identical to the categorical completion rate (Definition~\ref{def:emergent_time}),
    \item The completion rate is constant due to self-similar structure (Theorem~\ref{thm:constant_rate}),
    \item Therefore, time "flows" uniformly from the observer's perspective.
\end{enumerate}
This uniformity is not because time is a fundamental substrate with an intrinsic "flow rate," but because the categorical branching ratio is constant.
\end{corollary}

This corollary explains one of the most basic features of temporal experience: the sense that time passes at a steady, uniform rate (barring relativistic effects, which are not considered here). The uniformity is not a property of time itself but a consequence of the self-similar structure of categorical space.

\begin{figure}[htbp]
\centering
\includegraphics[width=\textwidth]{figures/emergent_time_panel.png}
\caption{\textbf{Time as emergent categorical completion.} (A) Time emerging from category counting: the "flow" of time is the accumulation of completed categorical distinctions. Time is not a substrate but a derived measure. (B) Constant branching ratio giving uniform time flow: each level of the categorical hierarchy has the same branching ratio (3 sub-categories per category), producing a constant relative completion rate despite exponential absolute growth. (C) Self-similar structure: each level of the hierarchy "looks the same" in terms of branching structure. This self-similarity ensures constant completion rate. (D) Singularity: no categories, no time. At the singularity, where $|\mathcal{C}| = 1$ (or 0), the completion rate $\rho_C = 0$, and time does not exist—not as $t = 0$ but as undefined. (E) Categories beget categories: each completed category generates new potential categories through the tri-dimensional decomposition, ensuring the generative process continues. (F) Arrow of time = direction of completion: the past consists of completed categories, the future consists of potential categories. The arrow of time is the direction in which categories are completed, from potential to actual.}
\label{fig:emergent_time}
\end{figure}

\subsection{Time at the Singularity}

A critical test of the emergent time framework is its behavior at the singularity—the state with zero internal distinctions.

\begin{theorem}[No Time at Singularity]
\label{thm:no_time_singularity}
At the singularity, time does not exist. This is not merely $t = 0$ (time exists but has a particular value) but the non-existence of time as a concept.
\end{theorem}

\begin{proof}
At the singularity, by Theorem~\ref{thm:categorical_equiv}, there are zero internal categorical distinctions:
\begin{equation}
|\mathcal{C}_{\text{internal}}(\text{singularity})| = 0
\end{equation}

Alternatively, we can say there is exactly one category: the singularity itself, with no subdivisions. Either way, there are no categorical distinctions to complete.

Without categorical distinctions, there are no oscillations. By Theorem~\ref{thm:oscillation_topology}, oscillation requires a center to oscillate around. At the singularity, there is only the center—there is nothing to oscillate. Therefore:
\begin{equation}
\text{Number of oscillations at singularity} = 0
\end{equation}

The categorical completion rate is:
\begin{equation}
\rho_C(\text{singularity}) = \frac{d|\gamma|}{dt}\bigg|_{\text{singularity}} = 0
\end{equation}

By Definition~\ref{def:emergent_time}, emergent time is:
\begin{equation}
\tau = \int \rho_C \, dt
\end{equation}

When $\rho_C = 0$ everywhere (as at the singularity), the integral is either zero or undefined, depending on the interpretation. But more fundamentally, the concept of "integrating over $t$" presupposes that $t$ is a meaningful parameter. At the singularity, where no categorical changes occur, $t$ has no physical meaning—it is not a parameter that tracks any observable quantity.

Therefore, time does not exist at the singularity. This is not "frozen time" (time exists but does not advance) or "$t = 0$" (time exists and has the value zero). It is the \emph{non-existence} of time as a concept. Time is not defined at the singularity because there are no categorical completions to count.
\end{proof}

\begin{corollary}["Before the Big Bang" is Meaningless]
\label{cor:before_big_bang}
The question "what happened before the Big Bang?" is malformed. If the Big Bang is the transition from the singularity (where time does not exist) to the first categorical distinctions (where time begins to exist), then there is no "before" the Big Bang. Time itself is created at the Big Bang, so there is no temporal framework in which to ask about prior events.
\end{corollary}

This corollary resolves one of the most common questions about cosmology. The answer is not "nothing happened before the Big Bang" (which would imply that time existed but nothing occurred) but rather "the concept of 'before' does not apply." Time is a product of categorical completion, and categorical completion begins at the Big Bang.

\subsection{Categories Beget Categories}

A key feature of categorical space is its self-generating nature: each completed category creates new potential categories.

\begin{theorem}[Generative Categories]
\label{thm:generative}
Every completed category generates new potential categories. Specifically, each completed category spawns at least $n \geq 2$ new potential categories. For tri-dimensional S-space, $n = 3$.
\end{theorem}

\begin{proof}
When a category $C$ is completed, it represents a resolved distinction—a definite state that has been actualized. This completion has several consequences:

\textbf{Step 1: Resolution of "cannot happen" states.}

Before $C$ is completed, there are multiple potential outcomes: $C$ could be completed, or alternative categories $C'$, $C''$, etc., could be completed instead. Once $C$ is completed, all alternatives are resolved—they are now known not to have happened. This resolution is itself a new categorical fact: "category $C$ was completed, not $C'$ or $C''$."

\textbf{Step 2: Creation of new categorical facts.}

The fact that $C$ was completed creates new information about the system. This information opens new possibility spaces: given that $C$ is completed, what can happen next? The answer depends on the structure of categorical space.

\textbf{Step 3: Tri-dimensional decomposition.}

By the structure of S-space (Section~\ref{sec:oscillatory}), each category can be decomposed into three sub-categories corresponding to the three dimensions: $(\Sentropy_k, \Sentropy_t, \Sentropy_e)$. When category $C$ is completed, it spawns three new potential categories:
\begin{equation}
C_{\text{completed}} \to \{C_1^{\text{new}}, C_2^{\text{new}}, C_3^{\text{new}}\}
\end{equation}
where each $C_i^{\text{new}}$ corresponds to a refinement of $C$ in one of the three dimensions.

Therefore, each completed category generates at least $n = 3$ new potential categories. This self-generating property ensures that categorical completion never halts (until the singularity is reached, at which point the generative process restarts).
\end{proof}

\begin{corollary}[Time Cannot Stop]
\label{cor:time_cannot_stop}
Since categories beget categories (Theorem~\ref{thm:generative}) and time is the rate of categorical completion (Definition~\ref{def:emergent_time}), time cannot "stop" as long as categories remain to be completed. Only at the singularity, where $|\mathcal{C}| = 1$ and no further subdivisions are possible, does the generative process halt—and with it, time ceases to exist.
\end{corollary}

This corollary explains why time is unidirectional and inexorable. Time does not "stop" because categorical completion is a self-sustaining process: each completion generates new potential completions, ensuring that the process continues. The only way for time to stop is for the system to reach a state where no further categorical distinctions can be made—the singularity.

\subsection{The Arrow of Time}

The emergent view of time provides a natural explanation for the arrow of time—the asymmetry between past and future.

\begin{theorem}[Categorical Arrow of Time]
\label{thm:categorical_arrow}
The arrow of time is identical to the direction of categorical completion. Past is the set of completed categories; future is the set of potential categories yet to be completed.
\end{theorem}

\begin{proof}
The arrow of time requires three properties:
\begin{enumerate}
    \item \emph{Distinction between past and future}: There must be a clear, objective difference between what we call "past" and what we call "future."
    \item \emph{Asymmetry}: The distinction must be asymmetric—past and future must be fundamentally different, not merely labeled differently.
    \item \emph{Universality}: The arrow must be universal and objective, applying to all observers and all systems.
\end{enumerate}

Categorical completion provides all three properties:

\textbf{Property 1: Distinction.}

The past is the set of completed categories:
\begin{equation}
\text{Past} = \gamma(t) = \{C \in \mathcal{C} : C \text{ has been completed by time } t\}
\end{equation}

The future is the set of potential categories yet to be completed:
\begin{equation}
\text{Future} = \mathcal{C} \setminus \gamma(t) = \{C \in \mathcal{C} : C \text{ has not yet been completed}\}
\end{equation}

This provides a clear, operational distinction: past categories are actual (they have been completed), future categories are potential (they have not yet been completed).

\textbf{Property 2: Asymmetry.}

By Axiom~\ref{axiom:cat_irreversibility}, categorical completion is irreversible. Once a category is completed, it cannot be uncompleted. Therefore:
\begin{equation}
C \in \gamma(t) \implies C \in \gamma(t') \text{ for all } t' > t
\end{equation}

This irreversibility creates an asymmetry: the set of completed categories can only grow (or remain constant), never shrink. The past is fixed and immutable; the future is open and contingent. This is the arrow of time.

\textbf{Property 3: Universality.}

All observers are embedded in the same categorical space $\mathcal{C}$. While different observers may have access to different subsets of completed categories (due to their limited perspectives), the underlying structure is universal. The direction of categorical completion—from potential to actual—is the same for all observers.

Therefore, the arrow of time is identical to the arrow of categorical completion: the direction in which categories transition from potential (future) to actual (past).
\end{proof}

\begin{remark}[Resolution of Time Asymmetry Puzzle]
\label{rem:time_asymmetry}
The categorical arrow resolves a longstanding puzzle in physics: why does time have a direction when the fundamental laws of physics are time-symmetric? The laws of classical mechanics, quantum mechanics, and even general relativity (with some caveats) are invariant under time reversal: if you reverse the direction of time, the laws remain the same.

The resolution is that the asymmetry of time does not come from the laws of physics but from the structure of categorical space. The laws of physics describe how systems evolve in the parameter $t$, but they do not explain why $t$ has a direction. The direction comes from categorical irreversibility: categories can be completed but not uncompleted. This irreversibility is not a law of physics but a structural feature of categorical space, analogous to how the axioms of set theory are not "laws" but foundational structures.
\end{remark}

The emergent view of time developed in this section establishes several key results: (1) time is not fundamental but emerges from the process of categorical completion; (2) the rate of categorical completion is constant due to self-similar structure, explaining uniform time flow; (3) at the singularity, time does not exist—not as $t = 0$ but as undefined; (4) categories beget categories, ensuring time cannot stop except at the singularity; (5) the arrow of time is identical to the direction of categorical completion, from potential to actual. These results provide a unified framework for understanding the nature of time, resolving puzzles about its flow, its arrow, and its relationship to the Big Bang.

