% Section: Irreversibility from Asymmetric Categorical Branching

The standard explanation for categorical irreversibility---that categories cannot be re-occupied---is correct but incomplete. We establish the deeper mechanism: asymmetric branching, where every actualisation resolves infinitely many non-actualisations.

\subsection{The Resolution of Non-Actualisations}

\begin{definition}[Categorical Potential]
For any system in state $S$, define:
\begin{align}
\mathcal{P}_{can}(S) &= \{E : E \text{ can happen from } S\} \\
\mathcal{P}_{cannot}(S) &= \{E : E \text{ cannot happen from } S\}
\end{align}
where $\mathcal{P}_{can}$ is finite and $\mathcal{P}_{cannot}$ is infinite.
\end{definition}

\begin{example}[Cup on Table Edge]
For a cup on a table edge:
\begin{align}
\mathcal{P}_{can} &= \{\text{fall}, \text{not fall}, \text{be pushed}, \ldots\} \quad (\text{finite}) \\
\mathcal{P}_{cannot} &= \{\text{turn to gold}, \text{become sentient}, \text{fly upward}, \ldots\} \quad (\text{infinite})
\end{align}
\end{example}

\begin{theorem}[Resolution Theorem]
\label{thm:resolution}
When event $E \in \mathcal{P}_{can}(S)$ actualises, every element of $\mathcal{P}_{cannot}(S)$ is simultaneously resolved as ``did not happen.''
\end{theorem}

\begin{proof}
Let $E$ actualise at time $t$. For any $E' \in \mathcal{P}_{cannot}(S)$:
\begin{enumerate}
    \item Before $t$: $E'$ is in state ``cannot happen'' (undetermined non-possibility)
    \item After $t$: $E'$ is in state ``did not happen'' (determined non-actualisation)
\end{enumerate}

The transition from ``cannot happen'' (potential) to ``did not happen'' (actual absence) constitutes categorical resolution. Since $|\mathcal{P}_{cannot}(S)| = \infty$, every actualisation resolves infinitely many non-actualisations.
\end{proof}

\begin{corollary}[Things That Cannot Happen, Happen]
When an event occurs, things that cannot happen also ``happen''---they happen as not-happening. Their resolution into determined absence is itself a categorical event.
\end{corollary}

\subsection{Asymmetric Branching}

\begin{definition}[Forward and Backward Branching]
For transition $S \to S'$:
\begin{align}
B_{forward}(S \to S') &= |\mathcal{P}_{can}(S')| + |\text{resolved } \mathcal{P}_{cannot}(S)| \\
B_{backward}(S' \to S) &= |\{\text{paths from } S' \text{ back to } S\}|
\end{align}
\end{definition}

\begin{theorem}[Asymmetric Branching Theorem]
\label{thm:asymmetric_branching}
For any non-trivial transition:
\begin{equation}
\frac{B_{forward}}{B_{backward}} \to \infty
\end{equation}
The ratio of forward to backward categorical paths is unbounded.
\end{theorem}

\begin{proof}
Forward branching includes:
\begin{enumerate}
    \item New possibilities from $S'$: $|\mathcal{P}_{can}(S')| \sim O(n)$
    \item Resolved impossibilities: $|\mathcal{P}_{cannot}(S)| = \infty$
\end{enumerate}

Backward ``branching'' requires:
\begin{enumerate}
    \item Returning $S'$ to $S$ spatially
    \item Un-resolving all ``did not happen'' back to ``cannot happen''
\end{enumerate}

But resolved non-actualisations are now categorical facts---they cannot be undetermined. There exists at most one precise backward path (and typically zero). Therefore:
\begin{equation}
\frac{B_{forward}}{B_{backward}} = \frac{\infty + O(n)}{O(1)} = \infty
\end{equation}
\end{proof}

\subsection{The Category Self-Division Problem}

\begin{definition}[Category Self-Division]
A category $C$ satisfies normal self-division if:
\begin{equation}
\frac{C}{C} = 1
\end{equation}
meaning traversing $C$ and returning yields unity (identity).
\end{definition}

\begin{theorem}[Non-Unity of Categorical Return]
\label{thm:non_unity}
For categorical states:
\begin{equation}
\frac{C}{C} \neq 1
\end{equation}
Traversing a category and ``returning'' does not yield the original state.
\end{theorem}

\begin{proof}
Let $C_0$ be the initial state. After occupation:
\begin{enumerate}
    \item $C_0$ is marked as completed (irreversibility axiom)
    \item New categories $\{C_1, C_2, \ldots\}$ are created from resolved non-actualisations
    \item Any ``return'' must occupy a new state $C_0'$ with $C_0' \neq C_0$
\end{enumerate}

The apparent return creates:
\begin{equation}
\frac{C_0}{C_0} = C_0' \neq C_0
\end{equation}

The residue $C_0' - C_0$ represents the categorical history accumulated during the traversal---the record of all non-actualisations that were resolved.
\end{proof}

\begin{corollary}[Categorical Residue]
Every attempt to ``undo'' a categorical transition leaves residue: the accumulated record of what didn't happen. This residue is irreducible and constitutes new categorical information.
\end{corollary}

\subsection{Information Creation Through Non-Actualisation}

\begin{theorem}[Information from Absence]
\label{thm:information_absence}
The broken cup contains more categorical information than the intact cup on the table.
\end{theorem}

\begin{proof}
The broken cup's categorical state includes:
\begin{enumerate}
    \item Its current configuration (shard positions, etc.)
    \item The resolved fact: ``did not turn to gold while falling''
    \item The resolved fact: ``did not become sentient while falling''
    \item The resolved fact: ``did not fly upward while falling''
    \item ... (infinitely many resolved non-actualisations)
\end{enumerate}

The intact cup's state includes:
\begin{enumerate}
    \item Its current configuration (position on table)
    \item Undetermined non-possibilities (not yet resolved)
\end{enumerate}

Determined facts constitute more information than undetermined possibilities. Therefore:
\begin{equation}
I(\text{broken cup}) > I(\text{intact cup})
\end{equation}
\end{proof}

\begin{corollary}[Entropy as Accumulated Absence]
Entropy increase measures the accumulation of resolved non-actualisations---the growing record of everything that didn't happen.
\end{corollary}

\begin{figure}[H]
\centering
\includegraphics[width=\textwidth]{figures/asymmetric_branching_panel.png}
\caption{Asymmetric categorical branching. (A) Event actualisation resolving infinite non-possibilities. (B) Forward vs backward branching ratio. (C) Category self-division yielding residue, not unity. (D) Information content: broken cup $>$ intact cup. (E) Accumulation of ``didn't happen'' as categorical record. (F) Why reversal is impossible: un-resolving determined facts.}
\label{fig:asymmetric_branching}
\end{figure}

