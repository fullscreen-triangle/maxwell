% Section: Topology of Categorical Spaces

We develop the topological structure of categorical spaces, establishing the partial order, completion dynamics, and the characteristic $3^k$ branching that emerges from tri-dimensional decomposition.

\subsection{Categorical Space Structure}

\begin{definition}[Categorical Space]
\label{def:categorical_space}
A categorical space is a quadruple $(\mathcal{C}, \prec, \mu, \tau)$ where:
\begin{enumerate}[(i)]
    \item $\mathcal{C}$ is a set of categorical states
    \item $\prec$ is a partial order on $\mathcal{C}$ (the completion order)
    \item $\mu: \mathcal{C} \times \mathbb{R}_{\geq 0} \to \{0, 1\}$ is the completion operator
    \item $\tau$ is the specialisation topology induced by $\prec$
\end{enumerate}
\end{definition}

\begin{axiom}[Categorical Irreversibility]
\label{axiom:cat_irreversibility}
For all $C \in \mathcal{C}$ and all $t_1 \leq t_2$:
\begin{equation}
\mu(C, t_1) = 1 \implies \mu(C, t_2) = 1
\end{equation}
Once a categorical state is completed, it remains completed and cannot be re-occupied.
\end{axiom}

\begin{theorem}[Non-Negative Completion Rate]
\label{thm:nonneg_rate}
For any completion trajectory $\gamma(t) = \{C \in \mathcal{C} : \mu(C, t) = 1\}$:
\begin{equation}
\dot{C}(t) = \frac{d|\gamma(t)|}{dt} \geq 0 \quad \forall t \geq 0
\end{equation}
\end{theorem}

\begin{proof}
By Axiom~\ref{axiom:cat_irreversibility}, $\gamma(t_1) \subseteq \gamma(t_2)$ for $t_1 \leq t_2$. Therefore $|\gamma(t)|$ is monotonically non-decreasing, and its derivative is non-negative.
\end{proof}

\subsection{Tri-Dimensional S-Space Decomposition}

\begin{definition}[S-Entropy Space]
\label{def:s_space}
The S-entropy coordinate system decomposes categorical space into three orthogonal dimensions:
\begin{equation}
\Sentropy = \Sentropy_k \times \Sentropy_t \times \Sentropy_e
\end{equation}
where:
\begin{itemize}
    \item $\Sentropy_k$: knowledge/information dimension
    \item $\Sentropy_t$: temporal/ordering dimension
    \item $\Sentropy_e$: entropy/constraint dimension
\end{itemize}
\end{definition}

\begin{axiom}[Recursive Decomposition]
\label{axiom:recursive}
Every categorical space admits canonical decomposition:
\begin{equation}
\mathcal{C} \cong \mathcal{C}_k \times \mathcal{C}_t \times \mathcal{C}_e
\end{equation}
where each factor is itself a categorical space admitting the same decomposition.
\end{axiom}

\begin{theorem}[Recursive Self-Similarity]
\label{thm:self_similar}
Under Axiom~\ref{axiom:recursive}, each factor decomposes recursively:
\begin{align}
\mathcal{C}_k &\cong \mathcal{C}_{k,k} \times \mathcal{C}_{k,t} \times \mathcal{C}_{k,e} \\
\mathcal{C}_t &\cong \mathcal{C}_{t,k} \times \mathcal{C}_{t,t} \times \mathcal{C}_{t,e} \\
\mathcal{C}_e &\cong \mathcal{C}_{e,k} \times \mathcal{C}_{e,t} \times \mathcal{C}_{e,e}
\end{align}
This continues to arbitrary depth: $\mathcal{C} \cong \prod_{i_1, i_2, \ldots \in \{k,t,e\}^{\mathbb{N}}} \mathcal{C}_{i_1, i_2, \ldots}$.
\end{theorem}

\subsection{The $3^k$ Branching Structure}

\begin{theorem}[$3^k$ Branching]
\label{thm:3k_branching}
Under tri-dimensional decomposition, a cascade of depth $k$ generates:
\begin{equation}
|\mathcal{C}^{(k)}| = 3^k \times |\mathcal{C}^{(0)}|
\end{equation}
categorical states at level $k$.
\end{theorem}

\begin{proof}
At each level, the tri-dimensional decomposition creates 3 sub-spaces:
\begin{align}
|\mathcal{C}^{(1)}| &= 3 \times |\mathcal{C}^{(0)}| \\
|\mathcal{C}^{(2)}| &= 3 \times |\mathcal{C}^{(1)}| = 3^2 \times |\mathcal{C}^{(0)}| \\
&\vdots \\
|\mathcal{C}^{(k)}| &= 3^k \times |\mathcal{C}^{(0)}|
\end{align}
\end{proof}

\begin{corollary}[Exponential Category Growth]
The total number of categories after $k$ levels of decomposition grows exponentially with base 3:
\begin{equation}
\sum_{i=0}^{k} |\mathcal{C}^{(i)}| = |\mathcal{C}^{(0)}| \cdot \frac{3^{k+1} - 1}{3 - 1} = |\mathcal{C}^{(0)}| \cdot \frac{3^{k+1} - 1}{2}
\end{equation}
\end{corollary}

\subsection{Scale Ambiguity}

\begin{theorem}[Scale Ambiguity]
\label{thm:scale_ambiguity}
Given a categorical state $C$ at level $n$, there exists an isometry:
\begin{equation}
\Psi_n: \mathcal{C}^{(n)} \to \mathcal{C}^{(n+1)}
\end{equation}
preserving all topological and metric structure. It is impossible to determine hierarchical level from local structure alone.
\end{theorem}

\begin{proof}
The recursive decomposition (Theorem~\ref{thm:self_similar}) shows that structure at level $n$ is isomorphic to structure at level $n+1$. The tri-dimensional factorisation is identical at every scale.

Define $\Psi_n$ by:
\begin{equation}
C^{(n)} = (c_k, c_t, c_e) \mapsto C^{(n+1)} = (c_{k,k}, c_{k,t}, c_{k,e})
\end{equation}
This is an isometry because S-distance structure is scale-invariant by construction.
\end{proof}

\begin{corollary}[Local-Global Indistinguishability]
It is impossible to determine from local examination whether a categorical state represents a global system-level configuration, a subsystem at intermediate level, or a component at fine-grained level. All levels are mathematically equivalent.
\end{corollary}

\subsection{Categorical Completion Dynamics}

\begin{definition}[Categorical Completion]
\label{def:completion}
A categorical space $\mathcal{C}$ achieves completion at time $T$ if:
\begin{equation}
\gamma(T) = \mathcal{C}
\end{equation}
meaning all categorical states have been occupied.
\end{definition}

\begin{theorem}[Finite Completion]
\label{thm:finite_completion}
For finite categorical space $|\mathcal{C}| < \infty$ with $\dot{C}(t) > \epsilon > 0$:
\begin{equation}
\exists T < \infty: \gamma(T) = \mathcal{C}
\end{equation}
\end{theorem}

\begin{proof}
With $|\mathcal{C}| = N < \infty$ and completion rate bounded below by $\epsilon$:
\begin{equation}
|\gamma(t)| = \int_0^t \dot{C}(s) \, ds > \epsilon t
\end{equation}
Setting $\epsilon t = N$ gives $t = N/\epsilon$. Thus $T \leq N/\epsilon < \infty$.
\end{proof}

\begin{theorem}[Asymptotic Slowing]
\label{thm:asymptotic}
As categorical space approaches completion:
\begin{equation}
\lim_{t \to T^-} \dot{C}(t) = 0
\end{equation}
\end{theorem}

\begin{proof}
Let $\mathcal{C}_{\text{rem}}(t) = \mathcal{C} \setminus \gamma(t)$ be remaining unoccupied states. The completion rate scales with available states:
\begin{equation}
\dot{C}(t) \propto |\mathcal{C}_{\text{rem}}(t)|
\end{equation}
As $t \to T$: $|\mathcal{C}_{\text{rem}}(t)| \to 0 \implies \dot{C}(t) \to 0$.
\end{proof}

\begin{figure}[H]
\centering
\includegraphics[width=\textwidth]{figures/topology_categories_panel.png}
\caption{Topology of categorical spaces. (A) Partial order structure showing completion precedence. (B) Tri-dimensional S-space decomposition into $\Sentropy_k$, $\Sentropy_t$, $\Sentropy_e$. (C) $3^k$ branching tree through recursive decomposition. (D) Scale ambiguity: identical structure at levels $n$ and $n+1$. (E) Completion trajectory $\gamma(t)$ as monotonically expanding set. (F) Asymptotic slowing of completion rate near $T$.}
\label{fig:topology_categories}
\end{figure}

