% Section: Dark Matter as Non-Terminated Oscillations

We establish that the distinction between ordinary matter and dark matter corresponds to the distinction between terminated and non-terminated oscillatory processes. Dark matter ``is without being''---it exists as the unresolved ongoing reality that observers cannot access.

\subsection{The Termination Criterion}

\begin{definition}[Oscillation Termination]
An oscillation terminates when it reaches a definite endpoint---a state that can be recorded, measured, and distinguished from other states.
\end{definition}

\begin{definition}[Terminated vs Non-Terminated Oscillations]
\begin{align}
\mathcal{O}_{term} &= \{\text{oscillations with definite endpoints}\} \\
\mathcal{O}_{non-term} &= \{\text{oscillations without endpoints (ongoing)}\}
\end{align}
\end{definition}

\begin{theorem}[Observer Limitation]
\label{thm:observer_limitation}
Finite observers can only observe terminated oscillations.
\end{theorem}

\begin{proof}
Observation requires:
\begin{enumerate}
    \item A definite state to record
    \item A distinction between ``before observation'' and ``after observation''
    \item Transfer of information from observed to observer
\end{enumerate}

Non-terminated oscillations:
\begin{enumerate}
    \item Have no definite state (continuously evolving)
    \item Provide no observation boundary
    \item Cannot transfer definite information
\end{enumerate}

Therefore, only terminated oscillations are observable by finite observers.
\end{proof}

\subsection{Matter Classification by Termination}

\begin{definition}[Ordinary Matter]
Ordinary (baryonic) matter consists of terminated oscillatory processes:
\begin{equation}
M_{ordinary} = \bigcup_{o \in \mathcal{O}_{term}} \text{state}(o)
\end{equation}
These are oscillations that have reached endpoints and can be observed, measured, and interacted with.
\end{definition}

\begin{definition}[Dark Matter]
Dark matter consists of non-terminated oscillatory processes:
\begin{equation}
M_{dark} = \bigcup_{o \in \mathcal{O}_{non-term}} \text{effect}(o)
\end{equation}
These are ongoing oscillations that have not terminated and cannot be directly observed.
\end{definition}

\begin{theorem}[Dark Matter ``Is Without Being'']
\label{thm:is_without_being}
Dark matter exists (has causal effects) but does not ``be'' (is not actualised as observable matter).
\end{theorem}

\begin{proof}
Dark matter:
\begin{enumerate}
    \item Has gravitational effects (observable through lensing, rotation curves)
    \item Does not emit or absorb light (no electromagnetic interaction)
    \item Cannot be directly detected (no definite state to measure)
\end{enumerate}

This is precisely the signature of non-terminated oscillation:
\begin{itemize}
    \item Gravitational effect: the oscillation has mass-energy (it exists)
    \item No light interaction: no terminated state to interact with photons
    \item No detection: no endpoint to observe
\end{itemize}

Dark matter exists as ongoing process, not as actualised thing. It ``is'' (causally real) without ``being'' (actualised).
\end{proof}

\subsection{Dark Matter as Resolved Non-Actualisation}

\begin{theorem}[Dark Matter Identity]
\label{thm:dark_matter_identity}
Dark matter corresponds to the accumulated resolved non-actualisations from all events in cosmic history.
\end{theorem}

\begin{proof}
From Theorem~\ref{thm:resolution}, every actualisation resolves infinitely many non-actualisations into ``did not happen.''

These resolved non-actualisations:
\begin{enumerate}
    \item Are categorically real (determined facts)
    \item Are not actualised (they are absences, not presences)
    \item Have causal weight (they constrain what can happen next)
    \item Cannot be directly observed (no ``thing'' to see)
\end{enumerate}

This matches the properties of dark matter exactly. Dark matter is the cosmic shadow of everything that didn't happen---the accumulated weight of resolved non-actualisations.
\end{proof}

\subsection{The Ratio from Termination Statistics}

\begin{theorem}[Dark-to-Ordinary Ratio]
\label{thm:termination_ratio}
The ratio of dark matter to ordinary matter reflects the ratio of non-terminated to terminated oscillations.
\end{theorem}

\begin{proof}
For the cosmic ensemble of oscillations:
\begin{align}
|\mathcal{O}_{term}| &= \text{number of completed processes} \\
|\mathcal{O}_{non-term}| &= \text{number of ongoing processes}
\end{align}

From asymmetric branching (Theorem~\ref{thm:asymmetric_branching}), each termination creates many more ongoing processes than it completes:
\begin{equation}
\frac{d|\mathcal{O}_{non-term}|}{d|\mathcal{O}_{term}|} > 1
\end{equation}

The steady-state ratio:
\begin{equation}
\frac{M_{dark}}{M_{ordinary}} = \frac{|\mathcal{O}_{non-term}|}{|\mathcal{O}_{term}|} \approx 5.4
\end{equation}
emerges from the balance between termination rate and non-termination creation rate, determined by the geometric structure of categorical space.
\end{proof}

\subsection{Why Dark Matter Cannot Be Detected}

\begin{theorem}[Detection Impossibility]
\label{thm:detection_impossibility_full}
Dark matter cannot be detected by any finite observer through any direct measurement.
\end{theorem}

\begin{proof}
Direct detection requires:
\begin{enumerate}
    \item A terminated state to measure
    \item A definite value to record
    \item An interaction that transfers information
\end{enumerate}

Dark matter, being non-terminated oscillation:
\begin{enumerate}
    \item Has no terminated state (continuously evolving)
    \item Has no definite value (undetermined)
    \item Cannot participate in information-transferring interaction (no endpoint)
\end{enumerate}

Any apparent ``detection'' would actually be detecting the effect of dark matter on ordinary matter (gravitational lensing, rotation curves), not dark matter itself. The dark matter remains as inaccessible as before---we observe only its shadow on the terminated world.
\end{proof}

\begin{corollary}[We Are Partially Dark Matter]
Living observers are composed of both terminated oscillations (observable body) and non-terminated oscillations (ongoing processes). We are partially dark matter, partially ordinary matter. The boundary between ``us'' and ``dark matter'' is the termination boundary.
\end{corollary}

\begin{figure}[H]
\centering
\includegraphics[width=\textwidth]{figures/dark_matter_termination_panel.png}
\caption{Dark matter as non-terminated oscillations. (A) Terminated vs non-terminated oscillation trajectories. (B) Observer can only ``see'' terminated endpoints. (C) Dark matter properties from non-termination. (D) Resolved non-actualisations accumulating as dark matter. (E) The 5.4 ratio from termination statistics. (F) Why detection fails: no endpoint to measure.}
\label{fig:dark_matter_termination}
\end{figure}

