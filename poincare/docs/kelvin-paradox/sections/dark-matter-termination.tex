% Section: Dark Matter as Non-Terminated Oscillations

We establish that the distinction between ordinary matter and dark matter corresponds to the distinction between terminated and non-terminated oscillatory processes. Dark matter ``is without being''---it exists as the unresolved ongoing reality that observers cannot access.

\subsection{The Termination Criterion}

\begin{definition}[Oscillation Termination]
An oscillation terminates when it reaches a definite endpoint—a state that can be recorded, measured, and distinguished from other states.
\end{definition}

\begin{definition}[Terminated vs Non-Terminated Oscillations]
\begin{align}
\mathcal{O}_{term} &= \{\text{oscillations with definite endpoints}\} \\
\mathcal{O}_{non-term} &= \{\text{oscillations without endpoints (ongoing)}\}
\end{align}
\end{definition}

\begin{theorem}[Observer Limitation]
\label{thm:observer_limitation}
Finite observers can only observe terminated oscillations.
\end{theorem}

\begin{proof}
Observation requires:
\begin{enumerate}
    \item A definite state to record
    \item A distinction between ``before observation'' and ``after observation''
    \item Transfer of information from the observed to the observer
\end{enumerate}

Non-terminated oscillations:
\begin{enumerate}
    \item Have no definite state (continuously evolving)
    \item Provide no observation boundary
    \item Cannot transfer definite information
\end{enumerate}

Therefore, only terminated oscillations are observable by finite observers.
\end{proof}

\subsection{Matter Classification by Termination}

\begin{definition}[Ordinary Matter]
Ordinary (baryonic) matter consists of terminated oscillatory processes:
\begin{equation}
M_{ordinary} = \bigcup_{o \in \mathcal{O}_{term}} \text{state}(o)
\end{equation}
These are oscillations that have reached endpoints and can be observed, measured, and interacted with.
\end{definition}

\begin{definition}[Dark Matter]
Dark matter consists of non-terminating oscillatory processes:
\begin{equation}
M_{dark} = \bigcup_{o \in \mathcal{O}_{non-term}} \text{effect}(o)
\end{equation}
These are ongoing oscillations that have not terminated and cannot be directly observed.
\end{definition}

\begin{theorem}[Dark Matter ``Is Without Being'']
\label{thm:is_without_being}
Dark matter exists (has causal effects) but does not ``be'' (is not actualised as observable matter).
\end{theorem}

\begin{proof}
Dark matter:
\begin{enumerate}
    \item Has gravitational effects (observable through lensing, rotation curves)
    \item Does not emit or absorb light (no electromagnetic interaction)
    \item Cannot be directly detected (no definite state to measure)
\end{enumerate}

This is precisely the signature of non-terminated oscillation:
\begin{itemize}
    \item Gravitational effect: the oscillation has mass-energy (it exists)
    \item No light interaction: no terminated state to interact with photons
    \item No detection: no endpoint to observe
\end{itemize}

Dark matter exists as an ongoing process, not as an actualised thing. It ``is'' (causally real) without ``being'' (actualised).
\end{proof}

\begin{figure*}[htbp]
\centering
\includegraphics[width=0.90\textwidth]{figures/dark_matter_termination_panel.png}
\caption{\textbf{Dark Matter as Non-Terminated Processes: Termination Statistics Predict 5.4:1 Ratio.} \textbf{(A)} Terminated vs. non-terminated oscillations: terminated oscillation (blue curve) crosses zero at endpoints (marked with stars), making it observable and countable; non-terminated oscillation (purple curve) remains in continuous evolution with no definite endpoints—amplitude never reaches zero, so process never completes. \textbf{(B)} Observers see only terminated states: observer (beige circle) can only detect processes that reach termination boundary (gray dashed line)—terminated states (blue stars) are visible, while non-terminated states (purple dots) remain invisible because they lack definite endpoints to measure. \textbf{(C)} Dark matter = non-terminated processes: dark matter has three properties—(1) has gravity (green, mass-energy exists), (2) no light (red, no terminated state to emit/absorb photons), (3) not detected (red, no endpoint to measure)—approximately 5.4× ordinary matter (termination ratio). Dark matter IS (has mass-energy) without BEING (no terminated state). \textbf{(D)} Dark matter = ``what didn't happen'': pie chart shows 5.4:1 ratio where dark matter (purple, 84\%) represents non-terminated possibilities and ordinary matter (blue, 16\%) represents actualized presence—dark matter is the accumulated ``didn't happen'' that still carries gravitational mass. \textbf{(E)} Ratio from termination statistics: each termination event (blue bars) creates multiple non-terminations (purple bars) at recursive levels; Level 3 shows ratio of 18.0 non-terminated to 1 terminated—termination statistics naturally generate $\sim$5:1 ratio through branching structure. \textbf{(F)} Why dark matter cannot be detected: detection requires (1) terminated state [X continuously evolving], (2) definite value [X undetermined], (3) info transfer [X no endpoint]—all three requirements fail for non-terminated processes. We see dark matter's shadow (gravitational effects on ordinary matter) not dark matter itself. Dark matter is not exotic particles but non-terminated categorical processes that carry mass-energy without having definite states.}
\label{fig:dark_matter_termination}
\end{figure*}

\subsection{Dark Matter as Resolved Non-Actualisation}

\begin{theorem}[Dark Matter Identity]
\label{thm:dark_matter_identity}
Dark matter corresponds to the accumulated resolved non-actualizations from all events in cosmic history.
\end{theorem}

\begin{proof}
From Theorem~\ref{thm:resolution}, every actualisation resolves infinitely many non-actualizations into ``did not happen.''

These resolved non-actualizations:
\begin{enumerate}
    \item Are categorically real (determined facts)
    \item Are not actualised (they are absences, not presences)
    \item Have causal weight (they constrain what can happen next)
    \item Cannot be directly observed (no ``thing'' to see)
\end{enumerate}

This matches the properties of dark matter exactly. Dark matter is the cosmic shadow of everything that didn't happen—the accumulated weight of resolved non-actualizations.
\end{proof}

\subsection{The Ratio from Termination Statistics}

\begin{theorem}[Dark-to-Ordinary Ratio]
\label{thm:termination_ratio}
The ratio of dark matter to ordinary matter reflects the ratio of non-terminated to terminated oscillations.
\end{theorem}

\begin{proof}
For the cosmic ensemble of oscillations:
\begin{align}
|\mathcal{O}_{term}| &= \text{number of completed processes} \\
|\mathcal{O}_{non-term}| &= \text{number of ongoing processes}
\end{align}

From asymmetric branching (Theorem~\ref{thm:asymmetric_branching}), each termination creates many more ongoing processes than it completes:
\begin{equation}
\frac{d|\mathcal{O}_{non-term}|}{d|\mathcal{O}_{term}|} > 1
\end{equation}

The steady-state ratio:
\begin{equation}
\frac{M_{dark}}{M_{ordinary}} = \frac{|\mathcal{O}_{non-term}|}{|\mathcal{O}_{term}|} \approx 5.4
\end{equation}
emerges from the balance between the termination rate and non-termination creation rate, determined by the geometric structure of categorical space.
\end{proof}

\subsection{Why Dark Matter Cannot Be Detected}

\begin{theorem}[Detection Impossibility]
\label{thm:detection_impossibility_full}
Dark matter cannot be detected by any finite observer through any direct measurement.
\end{theorem}

\begin{proof}
Direct detection requires:
\begin{enumerate}
    \item A terminated state to measure
    \item A definite value to record
    \item An interaction that transfers information
\end{enumerate}

Dark matter, being a non-terminated oscillation:
\begin{enumerate}
    \item Has no terminated state (continuously evolving)
    \item Has no definite value (undetermined)
    \item Cannot participate in information-transferring interactions (no endpoint)
\end{enumerate}

Any apparent ``detection'' would actually be detecting the effect of dark matter on ordinary matter (gravitational lensing, rotation curves), not dark matter itself. The dark matter remains as inaccessible as before—we observe only its shadow on the terminated world.
\end{proof}

\begin{corollary}[Termination Boundary]
Living observers are composed of both terminated oscillations (observable body) and non-terminated oscillations (ongoing processes). Observers are composed partly of dark matter and partly of ordinary matter. The boundary between ``us'' and ``dark matter'' is the termination boundary.
\end{corollary}



