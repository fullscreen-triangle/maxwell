\section{The Exclusion Principle}
\label{sec:exclusion_principle}

We prove that no two categorical states can occupy the same partition coordinate. This \emph{exclusion principle} emerges as a fundamental consequence of categorical distinguishability in bounded phase space.

\subsection{Coordinate Uniqueness}

\begin{axiom}[Categorical Distinguishability]
\label{ax:categorical_distinguishability}
Two categorical states are distinguishable if and only if they differ in at least one partition coordinate:
\begin{equation}
    S_1 \neq S_2 \iff (n_1, l_1, m_1, s_1) \neq (n_2, l_2, m_2, s_2)
\end{equation}
\end{axiom}

This axiom asserts that partition coordinates provide complete information for distinguishing categorical states. No additional "hidden" properties are needed.

\begin{theorem}[Coordinate-State Bijection]
\label{thm:coordinate_bijection}
There exists a one-to-one correspondence between valid partition coordinates and categorical states:
\begin{equation}
    \text{States} \leftrightarrow \{(n, l, m, s) : n \geq 1, \, 0 \leq l < n, \, -l \leq m \leq l, \, s = \pm\tfrac{1}{2}\}
\end{equation}
\end{theorem}

\begin{proof}
\textbf{Surjectivity}: By Theorem~\ref{thm:completeness}, every categorical state in bounded phase space has a unique partition coordinate.

\textbf{Injectivity}: Suppose two states $S_1$ and $S_2$ have the same partition coordinate $(n, l, m, s)$. Then:
\begin{itemize}
    \item They have the same partition depth $n$ (same radial structure)
    \item They have the same complexity $l$ (same angular structure)
    \item They have the same orientation $m$ (same spatial alignment)
    \item They have the same chirality $s$ (same handedness)
\end{itemize}

By Axiom~\ref{ax:categorical_distinguishability}, states with identical coordinates are indistinguishable; hence, they are identical: $S_1 = S_2$.

Therefore, the mapping from coordinates to states is both surjective and injective, establishing a bijection.
\end{proof}

\begin{figure}[htbp]
\centering
\includegraphics[width=\textwidth]{figures/topology_categories_panel.png}
\caption{\textbf{Topology of Categorical Spaces: Partial Orders, Branching, and Completion Dynamics.}
\textbf{(A)} Partial order (completion precedence). Diagram shows seven nodes (cyan circles) arranged in diamond lattice. Top node: most complete state. Bottom node: least complete state (ground state). Edges (blue lines) indicate precedence relations: lower states must be completed before higher states. This is the Hasse diagram of the partition poset (partially ordered set). The structure reflects the Aufbau principle: electrons fill lower energy states before higher states. The partial order is not total (not all states are comparable)—for example, the three middle nodes are incomparable (no precedence relation), corresponding to degenerate states with same energy but different quantum numbers (e.g., $2p_x$, $2p_y$, $2p_z$).

\textbf{(B)} Tri-dimensional S-space. Three-dimensional coordinate system showing three orthogonal axes: $S_c$ (red, center chirality), $S_t$ (green, temporal state), $S_s$ (blue, spatial state). Yellow dot: a point in S-space representing a complete partition coordinate $(n,l,m,s,s_c)$. The state space is not Euclidean (not $\mathbb{R}^3$) but categorical (discrete points on a lattice). The three dimensions correspond to three independent degrees of freedom: spatial structure ($n,l,m$), temporal evolution ($s$), and nuclear coupling ($s_c$). 

\textbf{(C)} $3^k$ branching structure. Tree diagram showing hierarchical branching. Top node (cyan): root state. Three branches (blue, green, red) lead to three second-level nodes. Each second-level node branches into three third-level nodes (9 total). Each third-level node branches into three fourth-level nodes (27 total, shown at bottom). The branching factor is 3 at each level, giving $3^k$ nodes at level $k$. This structure represents the partition coordinate tree: each level corresponds to a quantum number, and each branch corresponds to a possible value. For example, level 1 might be $n$ (principal quantum number), level 2 might be $l$ (angular momentum), level 3 might be $m$ (magnetic quantum number). The exponential growth ($3^k$) explains the rapid increase in complexity with increasing $n$: the number of possible states grows exponentially.

\textbf{(D)} Scale ambiguity: identical structure. Two triangular structures (left: Level $n$, right: Level $n+1$) with identical topology but different scales. Both have three nodes (cyan circles) connected by three edges (blue lines). Red symbol $\Psi_n$ between them indicates structural isomorphism. This demonstrates scale invariance: the partition structure repeats at different energy scales. For example, the $2s$ subshell has the same internal structure as the $3s$ subshell, just at different energy. This self-similarity is a key property of categorical spaces, enabling recursive construction of complex systems from simple templates.

\textbf{(E)} Completion trajectory $\gamma(t)$ expanding. Plot shows fraction completed (0-1) vs. time (0-10). Green curve: $|\gamma(t)|/|c|$ (ratio of completed to total states), starting at 0 and asymptotically approaching 1 (red dashed line). Green shading: completed region. The trajectory is sublinear (concave down), indicating that completion slows as the system approaches the final state. This is the signature of Poincaré computing: the system must explore increasingly fine-grained regions of state space, requiring exponentially more time to complete each additional fraction. 

\textbf{(F)} Asymptotic slowing: $\dot{C}(t) \to 0$. Plot shows completion rate $\dot{C}(t)$ (fraction per unit time) vs. time (0-10). Red curve: instantaneous completion rate, starting at $\sim 0.3$ and decaying to $\sim 0.02$ by time 10. Red shading: rate distribution. Black dashed line: completion time $T$ (when rate reaches zero, extrapolated to $t \to \infty$). The rate decays approximately as $\dot{C}(t) \propto 1/t$ (hyperbolic), indicating logarithmic completion: $C(t) \propto \log(t)$.}
\label{fig:topology_categories}
\end{figure}


\subsection{The Exclusion Principle}

\begin{theorem}[Partition Coordinate Exclusion]
\label{thm:exclusion_principle}
No two distinct categorical states can occupy the same partition coordinate:
\begin{equation}
    S_1 \neq S_2 \implies (n_1, l_1, m_1, s_1) \neq (n_2, l_2, m_2, s_2)
\end{equation}
Equivalently: each partition coordinate can be occupied by \emph{at most one} categorical state.
\end{theorem}

\begin{proof}
This is the contrapositive of the injectivity statement in Theorem~\ref{thm:coordinate_bijection}:
\begin{align}
    \text{Injectivity:} \quad &(n_1, l_1, m_1, s_1) = (n_2, l_2, m_2, s_2) \implies S_1 = S_2 \\
    \text{Contrapositive:} \quad &S_1 \neq S_2 \implies (n_1, l_1, m_1, s_1) \neq (n_2, l_2, m_2, s_2)
\end{align}

Therefore, distinct states must have distinct coordinates. Each coordinate can accommodate at most one state.
\end{proof}

\begin{remark}[Geometric Origin]
The exclusion principle is not an additional postulate—it follows necessarily from the bijection between states and coordinates. It reflects the fact that partition coordinates provide a complete labeling of categorical states in bounded phase space.
\end{remark}

\subsection{Occupation Numbers}

\begin{definition}[Occupation Number]
\label{def:occupation_number}
For each partition coordinate $(n, l, m, s)$, the \emph{occupation number} $N_{n,l,m,s}$ is:
\begin{equation}
    N_{n,l,m,s} = \begin{cases}
        1 & \text{if coordinate $(n,l,m,s)$ is occupied} \\
        0 & \text{if coordinate $(n,l,m,s)$ is unoccupied}
    \end{cases}
\end{equation}
\end{definition}

\begin{theorem}[Occupation Number Constraint]
\label{thm:occupation_constraint}
The exclusion principle is equivalent to the constraint:
\begin{equation}
    N_{n,l,m,s} \in \{0, 1\} \quad \text{for all } (n, l, m, s)
\end{equation}
\end{theorem}

\begin{proof}
By Theorem~\ref{thm:exclusion_principle}, each coordinate can be occupied by at most one state. Therefore $N \leq 1$. Since $N$ counts the number of states (a non-negative integer), $N \geq 0$. Hence $N \in \{0, 1\}$.
\end{proof}

\begin{corollary}[Idempotency Condition]
\label{cor:idempotency}
The occupation numbers satisfy:
\begin{equation}
    N_{n,l,m,s}^2 = N_{n,l,m,s}
\end{equation}
for all coordinates.
\end{corollary}

\begin{proof}
If $N = 0$, then $N^2 = 0 = N$. If $N = 1$, then $N^2 = 1 = N$. Since $N \in \{0, 1\}$ by Theorem~\ref{thm:occupation_constraint}, the idempotency condition holds.
\end{proof}

\begin{theorem}[Total Occupation]
\label{thm:total_occupation}
For a system with $Z$ categorical states:
\begin{equation}
    \sum_{n,l,m,s} N_{n,l,m,s} = Z
\end{equation}
and the idempotency condition implies:
\begin{equation}
    \sum_{n,l,m,s} N_{n,l,m,s}^2 = Z
\end{equation}
\end{theorem}

\subsection{Consequences of Exclusion}

\begin{corollary}[Shell Capacity Enforcement]
\label{cor:capacity_enforcement}
The exclusion principle enforces the shell capacity formula $C(n) = 2n^2$:
\begin{itemize}
    \item At depth $n$, there are exactly $2n^2$ distinct coordinates
    \item Each coordinate can hold at most one state
    \item Therefore, at most $2n^2$ states can occupy depth $n$
\end{itemize}
\end{corollary}

\begin{corollary}[Forced Filling Order]
\label{cor:forced_filling}
When adding states to a system, the exclusion principle forces the filling sequence:
\begin{enumerate}
    \item The first state occupies the lowest-energy coordinate: $(1, 0, 0, +\tfrac{1}{2})$
    \item The second state occupies the next available coordinate: $(1, 0, 0, -\tfrac{1}{2})$
    \item Subsequent states fill in order of increasing energy
    \item No state can occupy an already-filled coordinate
\end{enumerate}

This produces the filling sequence derived in Section~\ref{sec:energy_ordering}.
\end{corollary}

\begin{figure}[htbp]
\centering
\includegraphics[width=\textwidth]{figures/ship_theseus_panel.png}
\caption{\textbf{Identity Persistence Under Sequential Component Exchange: The Ship of Theseus Paradox Resolved.}
\textbf{(A)} Component state matrix over time. Heatmap showing component index (0-20, vertical axis) vs. exchange number (0-30, horizontal axis). Color indicates component state: green = original component (value 1), red = replaced component (value 0). Initially (exchange 0), all components are green (original ship). As exchanges proceed, green patches are progressively replaced by red patches, moving from top to bottom. By exchange 30, the matrix is entirely red (no original components remain). 

\textbf{(B)} Identity decay: multiple experimental trials. Plot shows identity remaining (fraction, 0-1) vs. number of exchanges (0-50). Four colored curves show different trials (Trial 1-4), all following similar exponential decay. Red dashed line: 50\% threshold (half of original identity lost). Pink shaded region: below threshold (less than half original identity). All trials cross the threshold around 20-25 exchanges, despite different replacement orders. 

\textbf{(C)} Entropy sources: partition + composition. The plot shows cumulative entropy $\Delta S$ (arbitrary units) vs. exchange number (0-40). Two contributions: cyan line (partition $\Delta S$, entropy from changing partition structure) and green line (composition $\Delta S$, entropy from changing material composition). Black line: cumulative total (sum of both). 

\textbf{(D)} Identity distribution: modified vs. reassembled. Polar plot comparing three ships: original (gray dashed outline), modified (blue filled region), and reassembled (red outline). Five axes: original material, original history, functional continuity, temporal continuity, structural continuity. The original ship scores 1.0 on all axes (perfect pentagon). The modified ship (blue) retains high temporal continuity (1.0, same ship continuously modified) and functional continuity (0.8), but low original material (0.2). The reassembled ship (red) retains high original material (1.0, all original components) but low temporal continuity (0.2, assembled from scattered parts).

\textbf{(E)} Identity-entropy phase diagram. Plot shows identity remaining (fraction, 0-1) vs. cumulative entropy $S$ (arbitrary units). Black dashed curve: $I = e^{-\alpha S}$ (exponential decay). Color scale: entropy in units of $k_B$ (Boltzmann constant). The curve shows that identity and entropy are conjugate variables: as entropy increases, identity decreases. The relationship is exponential, not linear, because entropy is extensive (additive) while identity is intensive (multiplicative). High entropy (orange/red, $S > 100$) corresponds to low identity ($I < 0.2$), while low entropy (purple/blue, $S < 20$) corresponds to high identity ($I > 0.8$). "

\textbf{(F)} Identity-entropy conservation. Sankey diagram showing identity flow during ship transformation. Left: original identity (green bar, 100\%). Right: final state after all exchanges. Three outflows: (1) Modified ship (blue, $\sim 30\%$ identity retained), (2) Reassembled ship (red, $\sim 50\%$ identity retained), (3) Entropy (gray, $\sim 20\%$ identity dissipated as entropy). Yellow box: conservation law $I_0 = I_{\text{mod}} + I_{\text{reass}} + \Delta S$. }
\label{fig:ship_theseus}
\end{figure}

\begin{corollary}[Degeneracy Pressure]
\label{cor:degeneracy_pressure}
In a system with many categorical states confined to a bounded region, the exclusion principle creates an effective \emph{degeneracy pressure}:
\begin{itemize}
    \item States cannot be compressed into already-occupied coordinates
    \item Adding more states requires occupying higher-energy coordinates
    \item This resists further compression of the system
\end{itemize}

The degeneracy pressure scales as:
\begin{equation}
    P_{\text{deg}} \propto \frac{Z^{5/3}}{V}
\end{equation}
where $Z$ is the number of states and $V$ is the volume of the bounded region.
\end{corollary}

\subsection{Antisymmetric State Functions}

\begin{definition}[Multi-State Function]
\label{def:multistate_function}
A system of $Z$ categorical states is described by a function:
\begin{equation}
    \Psi(\xi_1, \xi_2, \ldots, \xi_Z)
\end{equation}
where $\xi_i = (n_i, l_i, m_i, s_i)$ represents the partition coordinates of the $i$-th state.
\end{definition}

\begin{theorem}[Antisymmetry Requirement]
\label{thm:antisymmetry}
To enforce the exclusion principle, the multi-state function must be antisymmetric under the exchange of any two coordinates:
\begin{equation}
    \Psi(\ldots, \xi_i, \ldots, \xi_j, \ldots) = -\Psi(\ldots, \xi_j, \ldots, \xi_i, \ldots)
\end{equation}
for all $i \neq j$.
\end{theorem}

\begin{proof}
Suppose $\Psi$ is antisymmetric. If two coordinates are identical, $\xi_i = \xi_j$, then:
\begin{equation}
    \Psi(\ldots, \xi_i, \ldots, \xi_i, \ldots) = -\Psi(\ldots, \xi_i, \ldots, \xi_i, \ldots)
\end{equation}

This implies $\Psi = -\Psi$, hence $\Psi = 0$. Therefore, the state function vanishes whenever two coordinates are identical, enforcing the exclusion principle.

Conversely, if the exclusion principle holds, the state function must vanish for identical coordinates, which requires antisymmetry.
\end{proof}

\begin{corollary}[Slater Determinant Form]
\label{cor:slater_determinant}
An antisymmetric multi-state function can be written as a determinant:
\begin{equation}
    \Psi(\xi_1, \ldots, \xi_Z) = \frac{1}{\sqrt{Z!}} \begin{vmatrix}
        \psi_1(\xi_1) & \psi_1(\xi_2) & \cdots & \psi_1(\xi_Z) \\
        \psi_2(\xi_1) & \psi_2(\xi_2) & \cdots & \psi_2(\xi_Z) \\
        \vdots & \vdots & \ddots & \vdots \\
        \psi_Z(\xi_1) & \psi_Z(\xi_2) & \cdots & \psi_Z(\xi_Z)
    \end{vmatrix}
\end{equation}
where $\psi_i(\xi)$ is the single-state function for coordinate $\xi_i$.
\end{corollary}

\begin{proof}
The determinant is antisymmetric by construction: exchanging any two columns (corresponding to exchanging two coordinates) changes the sign of the determinant. The normalisation factor $1/\sqrt{Z!}$ ensures proper normalisation.
\end{proof}

\subsection{Chirality and Statistics}

\begin{theorem}[Chirality-Statistics Connection]
\label{thm:chirality_statistics}
The connection between chirality and exclusion is encoded in the exchange phase:
\begin{equation}
    \Psi(\ldots, \xi_i, \ldots, \xi_j, \ldots) = e^{i\pi(2s_i)(2s_j)} \Psi(\ldots, \xi_j, \ldots, \xi_i, \ldots)
\end{equation}

For half-integer chirality ($s = \pm\tfrac{1}{2}$):
\begin{equation}
    e^{i\pi(2s_i)(2s_j)} = e^{i\pi(\pm 1)(\pm 1)} = e^{i\pi} = -1
\end{equation}
producing antisymmetry and enforcing exclusion.

For integer chirality ($s = 0, \pm 1, \ldots$):
\begin{equation}
    e^{i\pi(2s_i)(2s_j)} = e^{i 2\pi k} = +1
\end{equation}
producing symmetry and allowing multiple occupation.
\end{theorem}

\begin{proof}
Under a full rotation by $2\pi$, a state with chirality $s$ acquires a phase $e^{i 2\pi s}$. Exchanging two states is equivalent to a rotation by $\pi$ in the space of state labels, giving a phase $e^{i\pi(2s_i)(2s_j)}$.

For $s = \pm\tfrac{1}{2}$, this phase is $-1$, enforcing antisymmetry. For integer $s$, this phase is $+1$, allowing symmetry.
\end{proof}

\begin{corollary}[Fermionic vs. Bosonic Statistics]
\label{cor:fermion_boson}
Categorical states with half-integer chirality obey \emph{fermionic statistics} (exclusion, antisymmetry). Categorical states with integer chirality obey \emph{bosonic statistics} (multiple occupation, symmetry).
\end{corollary}

\subsection{Comparison to Quantum Mechanics}

\begin{remark}[Correspondence to Pauli Exclusion Principle]
\label{rem:pauli_correspondence}
The exclusion principle derived here is mathematically identical to the Pauli exclusion principle of quantum mechanics:

\begin{center}
\begin{tabular}{ll}
\toprule
\textbf{Partition Coordinates} & \textbf{Quantum Mechanics} \\
\midrule
No two states with same $(n, l, m, s)$ & No two fermions with same $(n, l, m_l, m_s)$ \\
Occupation number $N \in \{0, 1\}$ & Fermionic occupation $\{0, 1\}$ \\
Antisymmetric state function & Antisymmetric wave function \\
Half-integer chirality $\Rightarrow$ exclusion & Half-integer spin $\Rightarrow$ Pauli principle \\
Integer chirality $\Rightarrow$ no exclusion & Integer spin $\Rightarrow$ Bose statistics \\
\bottomrule
\end{tabular}
\end{center}

The partition coordinate framework provides a geometric origin for the Pauli principle: it emerges from the bijection between states and coordinates in bounded phase space, combined with the half-integer nature of boundary chirality.
\end{remark}

\begin{remark}[Spin-Statistics Theorem]
The connection between chirality and statistics (Theorem~\ref{thm:chirality_statistics}) mirrors the spin-statistics theorem of quantum field theory:
\begin{itemize}
    \item Half-integer spin $\Rightarrow$ fermions (antisymmetric, exclusion)
    \item Integer spin $\Rightarrow$ bosons (symmetric, multiple occupation)
\end{itemize}

In the partition coordinate framework, this connection arises from the phase acquired under coordinate exchange, which depends on the chirality quantum number $s$. This suggests that spin may be the physical manifestation of partition boundary chirality.
\end{remark}

\begin{remark}[Degeneracy Pressure]
The degeneracy pressure (Corollary~\ref{cor:degeneracy_pressure}) is the same as the electron degeneracy pressure that stabilizes white dwarf stars and the neutron degeneracy pressure that stabilises neutron stars. In both cases, the pressure arises from the Pauli exclusion principle, which prevents further compression.

The partition coordinate framework provides a geometric interpretation: degeneracy pressure is the resistance to compressing categorical states into already-occupied partition coordinates.
\end{remark}

\subsection{Summary}

We have derived:

\begin{enumerate}
    \item Coordinate-state bijection: one-to-one correspondence between coordinates and states (Theorem~\ref{thm:coordinate_bijection})
    \item Exclusion principle: no two states can occupy the same coordinate (Theorem~\ref{thm:exclusion_principle})
    \item Occupation number constraint: $N \in \{0, 1\}$ (Theorem~\ref{thm:occupation_constraint})
    \item Antisymmetric state functions (Theorem~\ref{thm:antisymmetry})
    \item Slater determinant form (Corollary~\ref{cor:slater_determinant})
    \item Chirality-statistics connection: half-integer $\Rightarrow$ exclusion, integer $\Rightarrow$ multiple occupation (Theorem~\ref{thm:chirality_statistics})
    \item Degeneracy pressure from exclusion (Corollary~\ref{cor:degeneracy_pressure})
\end{enumerate}

All results follow from the categorical distinguishability axiom and the geometry of partition coordinates. The correspondence to the Pauli exclusion principle and spin-statistics theorem is exact.

In the next section, we develop the mathematical framework for partition boundary functions and show how they satisfy differential equations analogous to the Schrödinger equation.
