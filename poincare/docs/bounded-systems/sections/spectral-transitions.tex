\section{Spectral Transitions and Selection Rules}
\label{sec:spectral_transitions}

We derive the rules governing transitions between partition coordinates and show that these transitions produce discrete spectral signatures. The selection rules follow from geometric continuity; the spectral structure follows from the energy ordering derived in Section~\ref{sec:energy_ordering}.

\subsection{Transition Energies}

\begin{definition}[Partition Coordinate Transition]
\label{def:coordinate_transition}
A \emph{transition} is a change from an initial partition coordinate $(n_i, l_i, m_i, s_i)$ to a final coordinate $(n_f, l_f, m_f, s_f)$, accompanied by energy exchange:
\begin{equation}
    \Delta E = E(n_f, l_f) - E(n_i, l_i)
\end{equation}
where $E(n, l)$ is given by Theorem~\ref{thm:complexity_energy}.
\end{definition}

For emission processes, $E_f < E_i$ (more stable final state) and $\Delta E < 0$. For absorption processes, $E_f > E_i$ and $\Delta E > 0$.

\begin{theorem}[Transition Energy with Complexity]
\label{thm:transition_energy_full}
The energy exchanged in a transition $(n_i, l_i) \to (n_f, l_f)$ is:
\begin{equation}
    \Delta E = E_0 \left[ \frac{1}{(n_i + \alpha l_i)^2} - \frac{1}{(n_f + \alpha l_f)^2} \right]
\end{equation}
where $E_0$ is the characteristic energy scale and $\alpha$ is the penetration parameter.
\end{theorem}

\begin{proof}
From Theorem~\ref{thm:complexity_energy}:
\begin{align}
    E(n_i, l_i) &= -\frac{E_0}{(n_i + \alpha l_i)^2} \\
    E(n_f, l_f) &= -\frac{E_0}{(n_f + \alpha l_f)^2}
\end{align}

The transition energy is:
\begin{align}
    \Delta E &= E(n_f, l_f) - E(n_i, l_i) \\
             &= -\frac{E_0}{(n_f + \alpha l_f)^2} + \frac{E_0}{(n_i + \alpha l_i)^2} \\
             &= E_0 \left[ \frac{1}{(n_i + \alpha l_i)^2} - \frac{1}{(n_f + \alpha l_f)^2} \right] \qedhere
\end{align}
\end{proof}

For transitions between states with the same complexity ($l_i = l_f = l$), this simplifies to:
\begin{equation}
    \Delta E = E_0 \left[ \frac{1}{(n_i + \alpha l)^2} - \frac{1}{(n_f + \alpha l)^2} \right]
\end{equation}

\subsection{Geometric Selection Rules}

Not all transitions are geometrically allowed. Boundary continuity imposes strict constraints.

\begin{axiom}[Continuous Boundary Deformation]
\label{ax:continuous_deformation}
A transition between partition coordinates must proceed through continuous deformation of partition boundaries. Discontinuous changes in boundary topology are forbidden.
\end{axiom}

\begin{theorem}[Complexity Selection Rule]
\label{thm:complexity_selection}
Allowed transitions must satisfy:
\begin{equation}
    \Delta l = l_f - l_i = \pm 1
\end{equation}
Transitions with $\Delta l = 0$ or $|\Delta l| \geq 2$ are forbidden.
\end{theorem}

\begin{proof}
The complexity parameter $l$ counts the number of nodal surfaces in the partition boundary (Definition~\ref{def:angular_complexity}). During a transition, the boundary must continuously deform from the initial to the final configuration.

\textbf{Case $\Delta l = 0$:} The boundary retains the same nodal structure. No energy is exchanged with the angular degrees of freedom. While geometrically allowed, such transitions have zero amplitude because there is no mechanism to couple the initial and final states.

\textbf{Case $\Delta l = +1$:} A new nodal surface emerges continuously from the boundary. This is geometrically allowed and corresponds to increasing angular complexity.

\textbf{Case $\Delta l = -1$:} An existing nodal surface merges continuously into the boundary. This is geometrically allowed and corresponds to decreasing angular complexity.

\textbf{Case $|\Delta l| \geq 2$:} Multiple nodal surfaces would need to appear or disappear simultaneously. This requires a discontinuous change in boundary topology, violating Axiom~\ref{ax:continuous_deformation}.

Therefore, only $\Delta l = \pm 1$ transitions have non-zero amplitude.
\end{proof}

\begin{theorem}[Orientation Selection Rule]
\label{thm:orientation_selection}
Allowed transitions must satisfy:
\begin{equation}
    \Delta m = m_f - m_i \in \{-1, 0, +1\}
\end{equation}
\end{theorem}

\begin{proof}
The orientation parameter $m$ specifies the spatial alignment of the boundary's nodal structure (Definition~\ref{def:spatial_orientation}). The orientation states form a $(2l+1)$-dimensional representation of the rotation group $\text{SO}(3)$.

A transition involves coupling between the boundary and an external field or oscillation. This coupling can transfer angular momentum to or from the boundary. The angular momentum transfer is quantized in units of one.

Therefore, the boundary orientation can change by at most one unit: $\Delta m \in \{-1, 0, +1\}$. Larger changes would require simultaneous transfer of multiple angular momentum quanta, which has zero amplitude in the dipole approximation.
\end{proof}

\begin{theorem}[Chirality Conservation]
\label{thm:chirality_conservation}
For electric dipole transitions:
\begin{equation}
    \Delta s = s_f - s_i = 0
\end{equation}
Chirality is conserved.
\end{theorem}

\begin{proof}
Chirality is a topological invariant of the boundary surface (Theorem~\ref{thm:binary_chirality}). It specifies the handedness of the boundary orientation.

Electric dipole coupling preserves parity and therefore cannot change chirality. A chirality-changing transition would require the boundary to undergo a parity-violating deformation, which is forbidden for electric dipole interactions.

Chirality-changing transitions ($\Delta s = \pm 1$) can occur through magnetic dipole or higher-order multipole interactions, but these have much smaller amplitudes than electric dipole transitions.
\end{proof}

\begin{corollary}[Forbidden Transitions]
\label{cor:forbidden_transitions}
The following transitions are geometrically or dynamically forbidden:
\begin{itemize}
    \item $\Delta l = 0$: no angular coupling (zero amplitude)
    \item $|\Delta l| \geq 2$: discontinuous boundary change (forbidden)
    \item $|\Delta m| \geq 2$: multiple angular momentum transfer (zero amplitude in dipole approximation)
    \item $\Delta s \neq 0$: chirality change (forbidden for electric dipole)
\end{itemize}
\end{corollary}

\subsection{Spectral Series}

Transitions terminating at a common final state produce characteristic spectral series.

\begin{definition}[Spectral Series]
\label{def:spectral_series}
A \emph{spectral series} is the set of all transitions from initial states $(n_i, l_i)$ to a fixed final state $(n_f, l_f)$:
\begin{equation}
    \mathcal{S}_{n_f, l_f} = \left\{ \Delta E(n_i, l_i \to n_f, l_f) : n_i > n_f, \, l_i = l_f \pm 1 \right\}
\end{equation}
\end{definition}

For simplicity, consider transitions between states with the same complexity ($l_i = l_f = l$). The selection rule $\Delta l = \pm 1$ is satisfied by transitions where complexity changes during the process.

\begin{theorem}[Series Limit]
\label{thm:series_limit}
For a spectral series terminating at $(n_f, l)$, the transition energies converge to a series limit as $n_i \to \infty$:
\begin{equation}
    \lim_{n_i \to \infty} \Delta E(n_i, l \to n_f, l) = \frac{E_0}{(n_f + \alpha l)^2}
\end{equation}
\end{theorem}

\begin{proof}
From Theorem~\ref{thm:transition_energy_full}:
\begin{equation}
    \Delta E = E_0 \left[ \frac{1}{(n_i + \alpha l)^2} - \frac{1}{(n_f + \alpha l)^2} \right]
\end{equation}

As $n_i \to \infty$:
\begin{equation}
    \lim_{n_i \to \infty} \frac{1}{(n_i + \alpha l)^2} = 0
\end{equation}

Therefore:
\begin{equation}
    \lim_{n_i \to \infty} \Delta E = E_0 \cdot \frac{1}{(n_f + \alpha l)^2} \qedhere
\end{equation}
\end{proof}

The series limit represents the energy required to completely remove an entity from the partition coordinate $(n_f, l)$ to infinite depth ($n_i \to \infty$).

\begin{theorem}[Series Convergence]
\label{thm:series_convergence}
The spectral lines in a series converge toward the series limit from below. The spacing between consecutive lines decreases as $n_i$ increases:
\begin{equation}
    \Delta E(n_i+1, l \to n_f, l) - \Delta E(n_i, l \to n_f, l) \propto \frac{1}{n_i^3}
\end{equation}
\end{theorem}

\begin{proof}
The difference between consecutive transition energies is:
\begin{align}
    \delta(\Delta E) &= E_0 \left[ \frac{1}{(n_i + \alpha l)^2} - \frac{1}{(n_i + 1 + \alpha l)^2} \right] \\
                     &\approx E_0 \cdot \frac{2}{(n_i + \alpha l)^3} \quad \text{(for large $n_i$)}
\end{align}

Thus the spacing decreases as $1/n_i^3$, causing the lines to converge rapidly toward the series limit.
\end{proof}

\begin{figure}[htbp]
\centering
\includegraphics[width=\textwidth]{figures/spectral_analysis_panel.png}
\caption{\textbf{Hydrogen Spectral Lines: The Fingerprint of Partition Transitions.}
\textbf{(Top)} Complete hydrogen emission spectrum from ultraviolet to infrared, showing the three major series: Lyman ($n \to 1$, UV), Balmer ($n \to 2$, visible), and Paschen ($n \to 3$, IR). Each vertical line represents a transition between partition coordinates $(n_i, l_i) \to (n_f, l_f)$ with $\Delta l = \pm 1$. Line heights indicate relative intensities. The series converge to their respective limits as $n_i \to \infty$, corresponding to the ionization threshold for each final state.
\textbf{(Bottom)} Energy level diagram showing partition coordinate assignments. Horizontal lines represent bound states with quantum numbers $(n, l)$. Vertical arrows show observed transitions with wavelengths: Lyman-$\alpha$ (121.6 nm, $2p \to 1s$), Balmer-$\alpha$ (H$\alpha$, 656.3 nm, $3p \to 2s$), and Balmer-$\beta$ (486.1 nm, $4p \to 2s$). Energy scale shows binding energies: ground state at $-13.60$ eV, first excited state at $-3.40$ eV, second excited state at $-1.51$ eV. The $1/n^2$ energy scaling is evident from the level spacing.
Each spectral line is a direct measurement of the energy difference between two partition coordinates: $h\nu = E_{n_i} - E_{n_f} = R_\infty(1/n_f^2 - 1/n_i^2)$. The complete spectrum provides overdetermined measurements of all partition energies.
Wavelengths from NIST Atomic Spectra Database, accurate to $\pm 0.001$ nm.}
\label{fig:spectral_analysis}
\end{figure}


\subsection{Principal Series}

\begin{definition}[Principal Series]
\label{def:principal_series}
The \emph{principal series} consists of transitions to the ground state $(n_f = 1, l_f = 0)$ from excited states $(n_i, l_i = 1)$:
\begin{equation}
    \mathcal{S}_{\text{principal}} = \{ \Delta E(n_i, 1 \to 1, 0) : n_i \geq 2 \}
\end{equation}
\end{definition}

\begin{theorem}[Principal Series Formula]
\label{thm:principal_series}
The transition energies in the principal series are:
\begin{equation}
    \Delta E_n = E_0 \left[ \frac{1}{1^2} - \frac{1}{(n + \alpha)^2} \right] = E_0 \left[ 1 - \frac{1}{(n + \alpha)^2} \right]
\end{equation}
for $n = 2, 3, 4, \ldots$
\end{theorem}

The principal series has series limit $\Delta E_\infty = E_0$ and first line at:
\begin{equation}
    \Delta E_2 = E_0 \left[ 1 - \frac{1}{(2 + \alpha)^2} \right]
\end{equation}

For $\alpha = 0$, this gives $\Delta E_2 = 3E_0/4 = 0.75 E_0$.

\subsection{Additional Series}

\begin{table}[h]
\centering
\caption{Spectral series for transitions to low-lying states}
\label{tab:spectral_series}
\begin{tabular}{ccccc}
\toprule
Series name & Final state $(n_f, l_f)$ & Initial states & Series limit & First line \\
\midrule
Principal & $(1, 0)$ & $(n, 1)$, $n \geq 2$ & $E_0$ & $n=2 \to 1$ \\
Sharp & $(2, 0)$ & $(n, 1)$, $n \geq 3$ & $E_0/4$ & $n=3 \to 2$ \\
Diffuse & $(2, 1)$ & $(n, 2)$, $n \geq 3$ & $E_0/(2+\alpha)^2$ & $n=3 \to 2$ \\
Fundamental & $(3, 0)$ & $(n, 1)$, $n \geq 4$ & $E_0/9$ & $n=4 \to 3$ \\
\bottomrule
\end{tabular}
\end{table}

Each series is characterized by its final state and produces a characteristic pattern of spectral lines converging to a series limit.

\subsection{Wavelength Representation}

When transition energy is carried by electromagnetic radiation, it is often expressed as wavelength.

\begin{definition}[Transition Wavelength]
\label{def:transition_wavelength}
For a transition with energy $\Delta E$, the corresponding wavelength is:
\begin{equation}
    \lambda = \frac{hc}{|\Delta E|}
\end{equation}
where $h$ is Planck's constant and $c$ is the speed of light.
\end{definition}

\begin{theorem}[Wavelength Formula]
\label{thm:wavelength_formula}
The transition wavelength can be written as:
\begin{equation}
    \frac{1}{\lambda} = \frac{E_0}{hc} \left[ \frac{1}{(n_f + \alpha l_f)^2} - \frac{1}{(n_i + \alpha l_i)^2} \right]
\end{equation}
\end{theorem}

Defining the Rydberg constant $R_\infty = E_0/(hc)$, this becomes:
\begin{equation}
    \frac{1}{\lambda} = R_\infty \left[ \frac{1}{(n_f + \alpha l_f)^2} - \frac{1}{(n_i + \alpha l_i)^2} \right]
\end{equation}

This is the generalized Rydberg formula with quantum defect $\alpha l$.

\subsection{Transition Intensities}

\begin{theorem}[Transition Amplitude]
\label{thm:transition_amplitude}
The amplitude for a transition $(n_i, l_i, m_i) \to (n_f, l_f, m_f)$ is proportional to the dipole matrix element:
\begin{equation}
    A_{if} \propto \langle n_f, l_f, m_f | \hat{\mathbf{r}} | n_i, l_i, m_i \rangle
\end{equation}
where $\hat{\mathbf{r}}$ is the position operator in partition space.
\end{theorem}

\begin{theorem}[Selection Rule Enforcement]
\label{thm:selection_enforcement}
The dipole matrix element vanishes unless the selection rules are satisfied:
\begin{equation}
    \langle n_f, l_f, m_f | \hat{\mathbf{r}} | n_i, l_i, m_i \rangle = 0 \quad \text{unless} \quad \Delta l = \pm 1, \, \Delta m \in \{0, \pm 1\}
\end{equation}
\end{theorem}

\begin{proof}[Sketch]
The position operator $\hat{\mathbf{r}}$ transforms as a vector under rotations. By the Wigner-Eckart theorem, its matrix elements between states with angular quantum numbers $(l, m)$ vanish unless the selection rules $\Delta l = \pm 1$ and $\Delta m \in \{0, \pm 1\}$ are satisfied.

The detailed proof requires the explicit form of partition boundary functions, which we develop in Section~\ref{sec:boundary_functions}.
\end{proof}

\begin{corollary}[Forbidden Transition Intensity]
\label{cor:forbidden_intensity}
Transitions violating the selection rules have exactly zero intensity in the electric dipole approximation. They are said to be \emph{forbidden}.
\end{corollary}

\subsection{Comparison to Atomic Spectroscopy}

\begin{remark}[Correspondence to Rydberg Formula]
\label{rem:rydberg_correspondence}
The transition energy formula:
\begin{equation}
    \Delta E = E_0 \left[ \frac{1}{(n_i + \alpha l_i)^2} - \frac{1}{(n_f + \alpha l_f)^2} \right]
\end{equation}
is identical in form to the Rydberg formula for atomic spectral lines:
\begin{equation}
    \Delta E = R_\infty hc \left[ \frac{1}{(n_f - \delta_f)^2} - \frac{1}{(n_i - \delta_i)^2} \right]
\end{equation}
where $\delta$ is the quantum defect. Our parameter $\alpha l$ plays the role of the quantum defect.

For hydrogen (where quantum defects are negligible), the formula simplifies to:
\begin{equation}
    \Delta E = 13.6 \text{ eV} \left[ \frac{1}{n_f^2} - \frac{1}{n_i^2} \right]
\end{equation}
which is the classic Rydberg formula with $E_0 = 13.6$ eV.
\end{remark}

\begin{remark}[Selection Rule Correspondence]
The selection rules derived here:
\begin{itemize}
    \item $\Delta l = \pm 1$ (complexity selection rule)
    \item $\Delta m \in \{0, \pm 1\}$ (orientation selection rule)
    \item $\Delta s = 0$ (chirality conservation)
\end{itemize}
are identical to the electric dipole selection rules in atomic spectroscopy:
\begin{itemize}
    \item $\Delta l = \pm 1$ (orbital angular momentum)
    \item $\Delta m_l \in \{0, \pm 1\}$ (magnetic quantum number)
    \item $\Delta m_s = 0$ (spin conservation)
\end{itemize}

This correspondence is exact, with no adjustable parameters.
\end{remark}

\begin{remark}[Spectral Series Correspondence]
The spectral series structure (principal, sharp, diffuse, fundamental) matches the historical classification of atomic spectral lines. The series limits, convergence behavior, and line spacings all follow the same mathematical form.

This suggests that atomic spectra are direct manifestations of partition coordinate transitions in bounded phase space.
\end{remark}

\subsection{Summary}

We have derived:

\begin{enumerate}
    \item Transition energies: $\Delta E = E_0[(n_i + \alpha l_i)^{-2} - (n_f + \alpha l_f)^{-2}]$ (Theorem~\ref{thm:transition_energy_full})
    \item Selection rules: $\Delta l = \pm 1$, $\Delta m \in \{0, \pm 1\}$, $\Delta s = 0$ (Theorems~\ref{thm:complexity_selection}--\ref{thm:chirality_conservation})
    \item Spectral series with characteristic limits (Theorem~\ref{thm:series_limit})
    \item Wavelength formula (generalized Rydberg) (Theorem~\ref{thm:wavelength_formula})
    \item Intensity rules from dipole matrix elements (Theorem~\ref{thm:transition_amplitude})
\end{enumerate}

All results follow from geometric continuity of partition boundaries and energy ordering. The correspondence to atomic spectroscopy is exact and parameter-free.

In the next section, we investigate hyperfine structure arising from chirality coupling.
