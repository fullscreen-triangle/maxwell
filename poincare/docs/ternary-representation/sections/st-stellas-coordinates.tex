\section{S-Entropy Coordinate Space}
\label{sec:coordinates}

\subsection{The Three-Dimensional Structure}

The S-entropy coordinate space is a bounded three-dimensional manifold encoding categorical state.

\begin{definition}[S-Entropy Space]
The \textbf{S-entropy space} is the compact metric space:
\begin{equation}
\Sspace = [0,1]^3 = \{(\Sk, \St, \Se) : \Sk, \St, \Se \in [0,1]\}
\end{equation}
equipped with the Euclidean metric $d(\Scoord_1, \Scoord_2) = \|\Scoord_1 - \Scoord_2\|_2$.
\end{definition}

\begin{definition}[S-Entropy Coordinates]
The three coordinates of S-space are:
\begin{enumerate}
    \item \textbf{Knowledge entropy} $\Sk \in [0,1]$: Quantifies uncertainty in categorical state identification. $\Sk = 0$ indicates complete knowledge; $\Sk = 1$ indicates maximum uncertainty.

    \item \textbf{Temporal entropy} $\St \in [0,1]$: Quantifies uncertainty in timing relationships. $\St = 0$ indicates precise temporal location; $\St = 1$ indicates complete temporal uncertainty.

    \item \textbf{Evolution entropy} $\Se \in [0,1]$: Quantifies uncertainty in trajectory progression. $\Se = 0$ indicates deterministic evolution; $\Se = 1$ indicates maximum trajectory uncertainty.
\end{enumerate}
\end{definition}

\subsection{Geometric Properties}

\begin{theorem}[Compactness]
The S-entropy space $\Sspace = [0,1]^3$ is compact.
\end{theorem}

\begin{proof}
$[0,1]$ is compact in $\mathbb{R}$ (Heine-Borel theorem). The product of compact spaces is compact (Tychonoff theorem). Therefore, $[0,1]^3$ is compact. \qed
\end{proof}

\begin{theorem}[Path-Connectedness]
The S-entropy space $\Sspace$ is path-connected: for any $\Scoord_1, \Scoord_2 \in \Sspace$, there exists a continuous path $\gamma: [0,1] \to \Sspace$ with $\gamma(0) = \Scoord_1$ and $\gamma(1) = \Scoord_2$.
\end{theorem}

\begin{proof}
The straight-line path:
\begin{equation}
\gamma(t) = (1-t)\Scoord_1 + t\Scoord_2
\end{equation}
is continuous and connects $\Scoord_1$ to $\Scoord_2$. Convexity of $[0,1]^3$ ensures $\gamma(t) \in \Sspace$ for all $t \in [0,1]$. \qed
\end{proof}

\subsection{The $3^k$ Hierarchical Partition}

\begin{definition}[Hierarchical Partition]
For depth $k \in \mathbb{Z}_{\geq 0}$, the \textbf{level-$k$ partition} of $\Sspace$ divides each dimension into $3^{k/3}$ equal intervals (for $k$ divisible by 3) or the appropriate fractional refinement, producing $3^k$ congruent cells:
\begin{equation}
\mathcal{C}_k = \{C_{i_1, i_2, \ldots, i_k} : i_j \in \{0, 1, 2\}\}
\end{equation}
where each cell $C_{i_1, \ldots, i_k}$ is a rectangular box with side length $3^{-\lceil k/3 \rceil}$ along each axis.
\end{definition}

More precisely, we define the partition through sequential refinement:

\begin{definition}[Sequential Refinement]
At each level, one dimension is refined by factor 3:
\begin{align}
k \equiv 0 \pmod{3} &: \text{refine } \Sk \text{ dimension} \\
k \equiv 1 \pmod{3} &: \text{refine } \St \text{ dimension} \\
k \equiv 2 \pmod{3} &: \text{refine } \Se \text{ dimension}
\end{align}
After $3m$ refinements, each dimension has been refined $m$ times, producing cells of size $3^{-m} \times 3^{-m} \times 3^{-m}$.
\end{definition}

\begin{proposition}[Cell Count]
The level-$k$ partition contains exactly $3^k$ cells.
\end{proposition}

\begin{proof}
Each refinement triples the cell count. Starting from 1 cell at $k=0$:
\begin{equation}
|\mathcal{C}_k| = 3^k
\end{equation}
\qed
\end{proof}

\subsection{Cell Addressing}

\begin{definition}[Cell Address]
A cell $C \in \mathcal{C}_k$ is addressed by a string $(i_1, i_2, \ldots, i_k)$ where $i_j \in \{0, 1, 2\}$ specifies the choice at refinement level $j$.
\end{definition}

\begin{proposition}[Address Uniqueness]
Each cell has a unique address, and each valid address specifies a unique cell.
\end{proposition}

\begin{proof}
The mapping from addresses to cells is constructed inductively:
\begin{itemize}
    \item Level 0: One cell (empty address)
    \item Level $j \to j+1$: Each cell splits into 3 subcells, indexed by $i_j \in \{0, 1, 2\}$
\end{itemize}
This produces a bijection between $\{0,1,2\}^k$ and $\mathcal{C}_k$. \qed
\end{proof}

\subsection{Nesting Structure}

\begin{theorem}[Hierarchical Nesting]
For $k' > k$, every cell $C' \in \mathcal{C}_{k'}$ is contained in exactly one cell $C \in \mathcal{C}_k$:
\begin{equation}
C' \subseteq C \iff \text{address}(C') \text{ extends } \text{address}(C)
\end{equation}
\end{theorem}

\begin{proof}
If address$(C) = (i_1, \ldots, i_k)$ and address$(C') = (i_1, \ldots, i_k, i_{k+1}, \ldots, i_{k'})$, then $C'$ is obtained from $C$ by $(k' - k)$ additional refinements, each of which selects a subset. Hence $C' \subseteq C$.

Conversely, if $C' \subseteq C$, the refinement sequence producing $C'$ must pass through $C$, so address$(C')$ extends address$(C)$. \qed
\end{proof}

\begin{corollary}[Tree Structure]
The hierarchy $\{\mathcal{C}_k\}_{k=0}^\infty$ forms a tree with branching factor 3. The root is $\Sspace$ itself; each node has exactly 3 children.
\end{corollary}

\subsection{Coordinate Recovery}

\begin{theorem}[Coordinate from Address]
Given an address $(i_1, i_2, \ldots, i_k)$, the cell centre has coordinates:
\begin{align}
\Sk &= \sum_{j : j \equiv 0 \pmod 3} \frac{2i_j + 1}{2 \cdot 3^{\lceil j/3 \rceil}} \\
\St &= \sum_{j : j \equiv 1 \pmod 3} \frac{2i_j + 1}{2 \cdot 3^{\lceil j/3 \rceil}} \\
\Se &= \sum_{j : j \equiv 2 \pmod 3} \frac{2i_j + 1}{2 \cdot 3^{\lceil j/3 \rceil}}
\end{align}
\end{theorem}

\begin{proof}
At refinement level $j$ affecting dimension $d$, the interval is subdivided into thirds. Choosing $i_j \in \{0, 1, 2\}$ selects the subinterval $[i_j/3, (i_j+1)/3]$ relative to the current interval. The centre of this subinterval is at $(2i_j + 1)/6$ relative to the current interval, which is $(2i_j + 1)/(2 \cdot 3^m)$ relative to $[0,1]$ where $m$ is the number of prior refinements of that dimension.

Summing over all refinements of each dimension gives the coordinates. \qed
\end{proof}

\begin{remark}
This theorem establishes that a ternary address encodes continuous coordinates through a well-defined limiting process. The discrete address converges to exact coordinates as $k \to \infty$.
\end{remark}

