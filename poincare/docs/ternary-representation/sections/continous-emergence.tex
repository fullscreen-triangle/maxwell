\section{Continuous Emergence}
\label{sec:continuous}

\subsection{The Limit Construction}

We now establish that infinite ternary strings correspond exactly to points in the continuous space $[0,1]^3$.

\begin{theorem}[Continuous Emergence]\label{thm:continuous-emergence}
The mapping $\phi : \mathbb{T}^\infty \to [0,1]^3$ defined by:
\begin{equation}
\phi(t_1, t_2, t_3, \ldots) = \lim_{k \to \infty} \phi_k(t_1, \ldots, t_k)
\end{equation}
is well-defined and surjective.
\end{theorem}

\begin{proof}
\textbf{Well-definedness:} For fixed infinite string $\mathbf{t} = (t_1, t_2, \ldots)$, consider the sequence of cells $C_k = \phi_k(t_1, \ldots, t_k)$.

By the nesting theorem, $C_{k+1} \subseteq C_k$ for all $k$. The cells are closed sets with:
\begin{equation}
\text{diam}(C_k) = \sqrt{3} \cdot 3^{-\lfloor k/3 \rfloor} \to 0 \text{ as } k \to \infty
\end{equation}

By the nested closed set theorem (Cantor's intersection theorem), since $[0,1]^3$ is complete:
\begin{equation}
\bigcap_{k=0}^\infty C_k = \{\Scoord\}
\end{equation}
is a singleton. Define $\phi(\mathbf{t}) = \Scoord$.

\textbf{Surjectivity:} For any $\Scoord \in [0,1]^3$, Theorem~\ref{thm:address-from-coords} constructs an infinite ternary string $\mathbf{t}$ with $\phi(\mathbf{t}) = \Scoord$. \qed
\end{proof}

\begin{theorem}[Continuity of Limit Map]\label{thm:continuity}
The map $\phi : \mathbb{T}^\infty \to [0,1]^3$ is continuous, where $\mathbb{T}^\infty$ carries the product topology.
\end{theorem}

\begin{proof}
The product topology on $\mathbb{T}^\infty$ has basis sets:
\begin{equation}
U_{t_1, \ldots, t_k} = \{(s_1, s_2, \ldots) \in \mathbb{T}^\infty : s_i = t_i \text{ for } i = 1, \ldots, k\}
\end{equation}

For any open ball $B_\epsilon(\Scoord) \subset [0,1]^3$, we need $\phi^{-1}(B_\epsilon(\Scoord))$ to be open in $\mathbb{T}^\infty$.

Choose $k$ such that $\sqrt{3} \cdot 3^{-\lfloor k/3 \rfloor} < \epsilon$. Let $\mathbf{t} \in \phi^{-1}(B_\epsilon(\Scoord))$ with $\phi(\mathbf{t}) = \Scoord' \in B_\epsilon(\Scoord)$.

The basis set $U_{t_1, \ldots, t_k}$ contains $\mathbf{t}$. For any $\mathbf{t}' \in U_{t_1, \ldots, t_k}$:
\begin{equation}
d(\phi(\mathbf{t}'), \phi(\mathbf{t})) \leq \sqrt{3} \cdot 3^{-\lfloor k/3 \rfloor} < \epsilon
\end{equation}

by Theorem~\ref{thm:distance}. Therefore, $\phi(\mathbf{t}') \in B_{2\epsilon}(\Scoord)$. Since $\epsilon$ was arbitrary, continuity follows. \qed
\end{proof}

\subsection{The Ternary-Continuous Bridge}

\begin{definition}[Finite Precision Approximation]
For $\Scoord \in [0,1]^3$ and precision $\delta > 0$, the \textbf{$\delta$-approximation} is the shortest ternary string $\mathbf{t}$ such that:
\begin{equation}
d(\phi(\mathbf{t}), \Scoord) < \delta
\end{equation}
\end{definition}

\begin{theorem}[Approximation Depth]\label{thm:approx-depth}
A $\delta$-approximation requires at most:
\begin{equation}
k = 3 \left\lceil \log_3 \frac{\sqrt{3}}{\delta} \right\rceil
\end{equation}
trits.
\end{theorem}

\begin{proof}
A $k$-trit string addresses a cell of diameter at most $\sqrt{3} \cdot 3^{-\lfloor k/3 \rfloor}$.

For $d(\phi(\mathbf{t}), \Scoord) < \delta$, we need:
\begin{equation}
\sqrt{3} \cdot 3^{-\lfloor k/3 \rfloor} < \delta
\end{equation}

Solving:
\begin{equation}
\lfloor k/3 \rfloor > \log_3 \frac{\sqrt{3}}{\delta} \implies k \geq 3 \left\lceil \log_3 \frac{\sqrt{3}}{\delta} \right\rceil
\end{equation}
\qed
\end{proof}

\begin{example}
For $\delta = 0.01$ (1\% precision):
\begin{equation}
k = 3 \left\lceil \log_3 \frac{\sqrt{3}}{0.01} \right\rceil = 3 \left\lceil \log_3 173.2 \right\rceil = 3 \cdot 5 = 15 \text{ trits}
\end{equation}

A 15-trit string (2.5 trytes) specifies S-coordinates to 1\% precision.
\end{example}

\subsection{Cantor Set Connection}

\begin{definition}[Ternary Cantor Set]
The \textbf{Cantor set} $\mathcal{K}$ is the set of points in $[0,1]$ with ternary expansions using only digits $\{0, 2\}$:
\begin{equation}
\mathcal{K} = \left\{ \sum_{i=1}^\infty t_i \cdot 3^{-i} : t_i \in \{0, 2\} \right\}
\end{equation}
\end{definition}

\begin{remark}
The Cantor set is the prototypical example of a set that is ``large'' (uncountable, same cardinality as $\mathbb{R}$) yet ``small'' (measure zero, totally disconnected). It arises naturally from ternary representation when the middle digit is excluded.

In our framework, the full ternary set $\{0, 1, 2\}$ is used, so we obtain the full interval $[0,1]$ rather than the Cantor set. The Cantor set can be viewed as the ``skeleton'' of the ternary structure.
\end{remark}

\subsection{Space-Filling Properties}

\begin{theorem}[Ternary Space-Filling]\label{thm:space-filling}
The map $\phi : \mathbb{T}^\infty \to [0,1]^3$ covers the entire cube: for every $\Scoord \in [0,1]^3$, there exists $\mathbf{t} \in \mathbb{T}^\infty$ with $\phi(\mathbf{t}) = \Scoord$.
\end{theorem}

\begin{proof}
This is the surjectivity part of Theorem~\ref{thm:continuous-emergence}. \qed
\end{proof}

\begin{remark}
This theorem establishes that the ternary representation is \textit{complete}: every point in S-space has a ternary address. There are no ``gaps'' or ``holes'' in the coverage.

The connection to space-filling curves \citep{peano1890courbe, sagan1994space} is deep. While Peano's original curve uses a continuous surjection $[0,1] \to [0,1]^2$, our construction uses the product structure of $\mathbb{T}^\infty$ to achieve a similar covering of $[0,1]^3$.
\end{remark}

\subsection{Measure-Theoretic Properties}

\begin{theorem}[Measure Correspondence]\label{thm:measure}
The Lebesgue measure on $[0,1]^3$ corresponds to the product measure on $\mathbb{T}^\infty$ (with uniform distribution on each $\{0, 1, 2\}$).
\end{theorem}

\begin{proof}
A basic open set in $\mathbb{T}^\infty$ is:
\begin{equation}
U_{t_1, \ldots, t_k} = \{(s_1, s_2, \ldots) : s_i = t_i \text{ for } i \leq k\}
\end{equation}
with product measure $3^{-k}$.

This set maps to cell $\phi_k(t_1, \ldots, t_k)$, which has Lebesgue measure:
\begin{equation}
\lambda(\phi_k(t_1, \ldots, t_k)) = (3^{-\lfloor k/3 \rfloor})^3 = 3^{-3\lfloor k/3 \rfloor}
\end{equation}

For $k = 3m$, this equals $3^{-3m} = 3^{-k}$, matching the product measure.

For general $k$, the cell has one or two dimensions at finer resolution, but the limiting measure correspondence holds. \qed
\end{proof}

\subsection{Fractal Structure}

\begin{theorem}[Self-Similarity]\label{thm:self-similarity}
The ternary addressing scheme exhibits $3^k$ self-similarity: the structure at any scale is identical to the structure at any other scale.
\end{theorem}

\begin{proof}
Consider a cell $C = \phi_k(t_1, \ldots, t_k)$. The subcells of $C$ are $\{\phi_{k+m}(t_1, \ldots, t_k, s_1, \ldots, s_m) : (s_1, \ldots, s_m) \in \mathbb{T}^m\}$.

The structure of these subcells within $C$ is identical to the structure of level-$m$ cells within $[0,1]^3$, up to scaling by factor $3^{-\lfloor k/3 \rfloor}$.

This self-similarity at all scales is characteristic of fractal structures \citep{falconer2003fractal}. \qed
\end{proof}

\begin{corollary}[Scale Ambiguity]
Given only the local structure of a ternary-addressed region, it is impossible to determine the absolute scale (the value of $k$).
\end{corollary}

\begin{remark}
This scale ambiguity is the mathematical basis for the ``scale ambiguity theorem'' in categorical computing: local and global problems have identical structure, differing only in the number of prefix trits.
\end{remark}

