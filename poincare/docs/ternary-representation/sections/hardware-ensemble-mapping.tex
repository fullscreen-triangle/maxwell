\section{Hardware Ensemble Mapping}
\label{sec:hardware}

\subsection{Three-Phase Oscillator Systems}

Ternary representation finds natural physical instantiation in three-phase oscillatory systems.

\begin{definition}[Three-Phase Oscillator]
A \textbf{three-phase oscillator ensemble} consists of three oscillators with phase relationship:
\begin{equation}
\phi_i = \phi_0 + \frac{2\pi i}{3} \quad \text{for } i \in \{0, 1, 2\}
\end{equation}
where $\phi_0$ is the reference phase.
\end{definition}

\begin{theorem}[Phase-Trit Correspondence]\label{thm:phase-trit}
Oscillator dominance encodes trit value:
\begin{equation}
t = \arg\max_{i \in \{0,1,2\}} A_i(t)
\end{equation}
where $A_i(t)$ is the amplitude of oscillator $i$ at time $t$.
\end{theorem}

\begin{proof}
At any instant, one of the three oscillators has maximum amplitude (except at crossing points, which form a set of measure zero). This dominant oscillator determines the trit value.

The phase separation of $2\pi/3$ ensures each oscillator dominates for exactly one-third of the cycle, providing uniform trit distribution. \qed
\end{proof}

\subsection{Physical Implementations}

\begin{example}[Three-Phase AC Power]
Industrial AC power systems operate with three phases separated by $2\pi/3$:
\begin{align}
V_0(t) &= V_m \sin(\omega t) \\
V_1(t) &= V_m \sin(\omega t - 2\pi/3) \\
V_2(t) &= V_m \sin(\omega t - 4\pi/3)
\end{align}

This existing infrastructure provides a physical substrate for ternary computation at power-line frequencies ($\sim$50--60 Hz).
\end{example}

\begin{example}[Three-Phase Clock]
A ternary clock circuit generates three phase-shifted square waves:
\begin{align}
C_0(t) &= \text{sgn}(\sin(\omega t)) \\
C_1(t) &= \text{sgn}(\sin(\omega t - 2\pi/3)) \\
C_2(t) &= \text{sgn}(\sin(\omega t - 4\pi/3))
\end{align}

At any instant, exactly two clocks are high (or low), encoding trit value through the pattern.
\end{example}

\begin{example}[Coupled Oscillator Network]
Three coupled oscillators with symmetric coupling:
\begin{align}
\ddot{x}_0 + \omega^2 x_0 &= \kappa(x_1 - x_0) + \kappa(x_2 - x_0) \\
\ddot{x}_1 + \omega^2 x_1 &= \kappa(x_0 - x_1) + \kappa(x_2 - x_1) \\
\ddot{x}_2 + \omega^2 x_2 &= \kappa(x_0 - x_2) + \kappa(x_1 - x_2)
\end{align}

The normal modes include a rotating mode with $2\pi/3$ phase separation, providing stable ternary encoding.
\end{example}

\subsection{Trit Extraction from Oscillators}

\begin{definition}[Phase Detector]
A \textbf{ternary phase detector} compares three oscillator phases and outputs the index of the leading phase:
\begin{equation}
\text{Detect}(\phi_0, \phi_1, \phi_2) = \arg\max_i \cos(\phi_i)
\end{equation}
\end{definition}

\begin{theorem}[Continuous Trit Stream]
A three-phase oscillator at frequency $f$ generates a trit stream at rate:
\begin{equation}
R_{\text{trit}} = 3f
\end{equation}
\end{theorem}

\begin{proof}
Each full cycle ($2\pi$ radians) traverses all three phases. At frequency $f$ (cycles per second), the oscillator completes $f$ cycles per second, generating $3f$ trit transitions. \qed
\end{proof}

\begin{example}
A 1 GHz three-phase oscillator generates 3 billion trits per second, equivalent to:
\begin{equation}
3 \times 10^9 \times \log_2 3 \approx 4.75 \times 10^9 \text{ bits/second}
\end{equation}
information throughput.
\end{example}

\subsection{S-Coordinate Extraction}

\begin{definition}[S-Coordinate Extractor]
An \textbf{S-coordinate extractor} converts a trit stream to S-entropy coordinates by interleaved accumulation:
\begin{align}
\Sk &= \sum_{j=1}^{n} \frac{t_{3j} + 0.5}{3^j} \\
\St &= \sum_{j=0}^{n-1} \frac{t_{3j+1} + 0.5}{3^{j+1}} \\
\Se &= \sum_{j=0}^{n-1} \frac{t_{3j+2} + 0.5}{3^{j+1}}
\end{align}
\end{definition}

\begin{proposition}[Hardware Implementation]
S-coordinate extraction requires:
\begin{itemize}
    \item 3 accumulators (one per dimension)
    \item 1 mod-3 counter (for dimension selection)
    \item 3 multipliers (for $3^{-j}$ scaling)
    \item 3 adders (for running sum)
\end{itemize}
Total: $O(1)$ hardware complexity, independent of precision $n$.
\end{proposition}

\subsection{Oscillator Ensemble Architecture}

\begin{definition}[Oscillator Ensemble]
An \textbf{oscillator ensemble} for ternary S-entropy computation comprises:
\begin{enumerate}
    \item \textbf{Core oscillators}: Three phase-locked oscillators at frequency $f_0$
    \item \textbf{Trit extractor}: Phase detector outputting dominant oscillator index
    \item \textbf{Coordinate accumulator}: Three-channel accumulator for $(\Sk, \St, \Se)$
    \item \textbf{Cell address register}: Current ternary string $\mathbf{t}$
\end{enumerate}
\end{definition}

\begin{theorem}[Ensemble Processing Rate]
An oscillator ensemble at frequency $f_0$ processes:
\begin{equation}
R_{\text{cell}} = f_0 \text{ cells per second}
\end{equation}
where one ``cell'' is a complete refinement of all three dimensions (3 trits).
\end{theorem}

\begin{proof}
Three trits (one per dimension) require one full cycle of the three-phase oscillator. At frequency $f_0$, this gives $f_0$ complete refinements per second. \qed
\end{proof}

\subsection{Multi-Scale Oscillator Hierarchy}

\begin{definition}[Hierarchical Oscillator Network]
A \textbf{hierarchical oscillator network} comprises multiple three-phase oscillators at different frequencies:
\begin{equation}
f_k = f_0 \cdot 3^{-k} \quad \text{for } k = 0, 1, 2, \ldots
\end{equation}
\end{definition}

\begin{theorem}[Scale-Frequency Correspondence]
Level $k$ in the $3^k$ hierarchy corresponds to oscillator frequency $f_k = f_0 \cdot 3^{-k}$.
\end{theorem}

\begin{proof}
Each level requires 3 trits to refine. The time to generate 3 trits at frequency $f$ is $1/f$ seconds. For level $k$, the cumulative refinement time is:
\begin{equation}
T_k = \sum_{j=0}^{k-1} \frac{1}{f_j} = \sum_{j=0}^{k-1} \frac{3^j}{f_0} = \frac{3^k - 1}{2f_0}
\end{equation}

The frequency at level $k$ determines the refinement rate for that level. \qed
\end{proof}

\begin{remark}
This hierarchical structure mirrors the timescale separation in physical systems: fast oscillations (high frequency) correspond to fine-grained structure; slow oscillations (low frequency) correspond to coarse-grained structure. The ternary representation naturally couples to this hierarchy.
\end{remark}

\subsection{Quantum Considerations}

\begin{remark}
Three-level quantum systems (qutrits) provide another physical instantiation of ternary representation. A qutrit has basis states $|0\rangle$, $|1\rangle$, $|2\rangle$ with general state:
\begin{equation}
|\psi\rangle = \alpha_0 |0\rangle + \alpha_1 |1\rangle + \alpha_2 |2\rangle
\end{equation}
where $|\alpha_0|^2 + |\alpha_1|^2 + |\alpha_2|^2 = 1$.

Measurement collapses to one of the three basis states, yielding a classical trit. Qutrit-based quantum computing may offer advantages over qubit-based systems for problems naturally suited to three-dimensional S-entropy representation.
\end{remark}

