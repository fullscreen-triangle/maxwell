%==============================================================================
\section{Autocatalytic Apertures: The Ball Game Derivation}
\label{sec:autocatalysis}
%==============================================================================

We present a thought experiment that derives autocatalysis from first principles using the aperture model, demonstrating that catalysis is inherently autocatalytic at the categorical level.

\subsection{The Ball Game Thought Experiment}

Consider two teams separated by a partition containing apertures (holes). The rules:

\begin{enumerate}
    \item Each team has balls that must be shot through apertures to the opposing side
    \item Players cannot hold balls---upon receiving, they must immediately shoot
    \item The goal is to prevent opposing balls from entering your side
    \item Balls colliding at an aperture deflect (mutual blocking)
\end{enumerate}

This setup maps precisely onto the two-container reaction system:

\begin{center}
\begin{tabular}{ll}
\toprule
\textbf{Ball Game} & \textbf{Chemical System} \\
\midrule
Ball & Molecule \\
Aperture (hole) & Catalyst active site \\
Opposing ball blocking & Transition state occupation \\
Shot timing & Categorical availability \\
Ball through aperture & Reaction completion \\
Score & Product formation \\
\bottomrule
\end{tabular}
\end{center}

\subsection{Velocity Independence}

\begin{theorem}[Velocity Independence]
\label{thm:velocity-independence}
In the ball game, success depends solely on aperture availability, not ball velocity.
\end{theorem}

\begin{proof}
A ball scores if and only if:
\begin{equation}
\text{Score}(b) = \begin{cases}
1 & \text{if aperture unblocked when ball arrives} \\
0 & \text{if aperture blocked when ball arrives}
\end{cases}
\end{equation}

The condition ``aperture unblocked'' is a categorical state---either blocked or not. This state is independent of:
\begin{itemize}
    \item Ball velocity $v$
    \item Distance from aperture $d$
    \item Travel time $t = d/v$
\end{itemize}

A slow ball through an open aperture scores. A fast ball hitting a blocked aperture fails. Velocity is categorically irrelevant.
\end{proof}

\begin{corollary}[Maxwell's Demon Inapplicability]
Sorting balls by velocity cannot improve scoring. The demon's velocity-based selection is useless in this system.
\end{corollary}

\subsection{Equilibrium as Mutual Blocking}

\begin{proposition}[Equilibrium State]
At equilibrium, every ball from Team A is blocked by a ball from Team B, and vice versa.
\end{proposition}

This is the \emph{penultimate state} of Section~\ref{sec:penultimate}: both sides are one categorical step from completion (getting a ball through), but mutual blocking prevents either from advancing.

\subsection{The Autocatalytic Cascade}

We now derive the central result: catalysis is inherently autocatalytic.

\begin{theorem}[Autocatalytic Apertures]
\label{thm:autocatalysis}
Each successful transit through an aperture reduces resistance to subsequent transits. Catalysis is inherently autocatalytic.
\end{theorem}

\begin{proof}
Let Team A and Team B each start with $n$ balls and $k$ apertures, where initially $n = k$ (full coverage).

\textbf{Initial state:}
\begin{align}
\text{Balls}_A &= n \\
\text{Balls}_B &= n \\
\text{Coverage}_B &= \frac{\min(n, k)}{k} = 1 \quad \text{(all holes blocked)}
\end{align}

\textbf{After Team A scores once:}
\begin{align}
\text{Balls}_A &= n - 1 \\
\text{Balls}_B &= n + 1 \\
\text{Coverage}_B &= \frac{\min(n+1, k)}{k} = 1 \quad \text{(still full, but...)}
\end{align}

Here is the critical insight: Team B now has $n + 1$ balls but only $k = n$ apertures. Since players cannot hold balls, at least one player must handle two balls simultaneously.

\textbf{The overflow problem:}
\begin{equation}
\text{Excess balls}_B = (n + 1) - k = 1
\end{equation}

This excess ball creates chaos:
\begin{itemize}
    \item One player is juggling two balls
    \item That player cannot properly aim at apertures
    \item Effective blocking capacity decreases
\end{itemize}

\textbf{After $m$ scores by Team A:}
\begin{align}
\text{Balls}_B &= n + m \\
\text{Excess balls}_B &= m \\
\text{Effective coverage}_B &< 1 \quad \text{(decreasing)}
\end{align}

Define \emph{resistance} as the probability that an aperture is blocked:
\begin{equation}
R(m) = \frac{k}{n + m} = \frac{n}{n + m}
\end{equation}

This is a decreasing function of $m$:
\begin{equation}
\frac{dR}{dm} = -\frac{n}{(n + m)^2} < 0
\end{equation}

Each score reduces resistance to the next score. This is positive feedback---\emph{autocatalysis}.
\end{proof}

\begin{corollary}[Product Accelerates Reaction]
The formation of product B on the receiving side increases the probability of subsequent A $\to$ B transitions.
\end{corollary}

\begin{corollary}[Lag-Exponential-Saturation Kinetics]
The autocatalytic cascade produces characteristic kinetics:
\begin{enumerate}
    \item \textbf{Lag phase:} Initial scores are difficult (full blocking, $R \approx 1$)
    \item \textbf{Exponential phase:} Once scoring begins, resistance drops rapidly
    \item \textbf{Saturation:} Eventually limited by ball availability on Team A
\end{enumerate}
\end{corollary}

\subsection{Time Independence of the Cascade}

\begin{proposition}[Temporal Irrelevance]
\label{prop:time-irrelevant}
The autocatalytic cascade is independent of temporal parameters.
\end{proposition}

\begin{proof}
A player may shoot from:
\begin{itemize}
    \item Far from the aperture (long travel time)
    \item Adjacent to the aperture (short travel time)
\end{itemize}

In neither case does travel time affect whether the aperture is blocked. The blocking state is determined by:
\begin{itemize}
    \item Number of balls on receiving side
    \item Number of apertures
    \item Categorical configuration at arrival
\end{itemize}

None of these depend on ball velocity or travel time.

Teams ``cannot have time-based hole allocation'' because:
\begin{equation}
\text{Aperture state} \neq f(t)
\end{equation}

The aperture is blocked or not based on categorical occupancy, not temporal scheduling.
\end{proof}

\subsection{``Seeing Behind the Wall''}

\begin{proposition}[Categorical Information Transfer]
When a ball successfully transits an aperture, it creates categorical structure on the receiving side.
\end{proposition}

This is described colloquially as ``seeing behind the wall'': the scoring team has established a categorical presence on the opposing side. This presence:

\begin{enumerate}
    \item Increases the categorical burden on the receiving team
    \item Creates ``demand'' for that ball to be processed
    \item Opens pathways for additional transits
\end{enumerate}

In chemical terms: product B on the product side creates categorical demand for more B, which drives the forward reaction.

\subsection{Implications for Enzyme Catalysis}

\begin{theorem}[Enzymes as Autocatalytic Apertures]
\label{thm:enzyme-autocatalysis}
Enzymes do not merely ``speed up'' reactions. They create apertures whose successful traversal reduces categorical resistance, making subsequent traversals easier.
\end{theorem}

This explains:

\begin{enumerate}
    \item \textbf{Cooperativity:} First substrate binding reduces resistance for second binding
    \item \textbf{Allosteric effects:} Effector binding changes aperture configuration, affecting resistance
    \item \textbf{Product inhibition:} Excess product overwhelms the system, increasing resistance in the reverse direction
    \item \textbf{Substrate inhibition:} Excess substrate at high concentration creates blocking at the aperture
\end{enumerate}

\subsection{The Resistance Equation}

\begin{definition}[Categorical Resistance]
For a system with $k$ apertures and $n_B$ balls on the receiving side:
\begin{equation}
R = \frac{k}{n_B} \quad \text{for } n_B \geq k
\end{equation}

When $n_B < k$, some apertures are permanently open: $R = n_B / k < 1$.
\end{definition}

\begin{proposition}[Resistance Dynamics]
As reaction proceeds (products accumulate on receiving side):
\begin{equation}
\frac{dR}{dt} = -\frac{k}{n_B^2} \cdot \frac{dn_B}{dt} < 0
\end{equation}

Resistance decreases as products accumulate---positive feedback.
\end{proposition}

\subsection{Connection to Le Chatelier}

The autocatalytic cascade connects to Le Chatelier's principle:

\begin{itemize}
    \item \textbf{Add reactants:} More balls on Team A side, more pressure on apertures, more likely to score
    \item \textbf{Add products:} More balls on Team B side, Team B more overwhelmed, resistance decreases, reaction shifts forward
    \item \textbf{Remove products:} Team B has fewer balls, better coverage, resistance increases, reaction slows
\end{itemize}

\begin{theorem}[Le Chatelier from Categorical Resistance]
Le Chatelier's principle is the system's response to changes in categorical resistance:
\begin{equation}
\text{Perturbation} \to \Delta R \to \text{Shift to restore } R_{\text{eq}}
\end{equation}
\end{theorem}

\subsection{Conservation and the ``Meaningless Victory''}

A deeper insight emerges from the ball game: the total number of balls is conserved.

\begin{definition}[Ball Conservation]
Let $N$ be the total number of balls in the system. At all times:
\begin{equation}
n_A(t) + n_B(t) = N = \text{constant}
\end{equation}
where $n_A$ and $n_B$ are the balls on Team A and Team B sides respectively.
\end{definition}

\begin{theorem}[The Meaningless Victory]
\label{thm:meaningless-victory}
If Team A scores all available balls, they cannot continue playing. ``Victory'' is self-defeating.
\end{theorem}

\begin{proof}
Suppose Team A scores $m = n_A(0)$ balls (all their balls). Then:
\begin{align}
n_A(t) &= 0 \\
n_B(t) &= N
\end{align}

Team A's scoring rate depends on having balls to shoot:
\begin{equation}
\text{Rate}_A = f(n_A) = 0 \quad \text{when } n_A = 0
\end{equation}

Team A cannot continue. The ``victory'' of scoring all balls results in inability to play.

For the game to continue, Team B must score balls back to Team A:
\begin{equation}
n_A > 0 \implies \text{Team B must score}
\end{equation}

This forces reversal of the dominant direction.
\end{proof}

\begin{corollary}[Reactions Cannot Go to Completion]
A chemical reaction A $\to$ B cannot proceed to 100\% completion because:
\begin{enumerate}
    \item As [A] $\to$ 0, forward rate $\to$ 0
    \item As [B] increases, reverse rate increases
    \item Complete conversion requires [A] = 0, which halts the forward reaction
\end{enumerate}
\end{corollary}

\begin{theorem}[Equilibrium from Conservation]
\label{thm:equilibrium-conservation}
Equilibrium is the configuration where both sides have sufficient balls to continue playing indefinitely.
\end{theorem}

\begin{proof}
For sustained play, both teams need balls:
\begin{align}
n_A &> 0 \quad \text{(Team A can shoot)} \\
n_B &> 0 \quad \text{(Team B can shoot)}
\end{align}

The game reaches steady state when:
\begin{equation}
\text{Rate}_{A \to B} = \text{Rate}_{B \to A}
\end{equation}

At this point:
\begin{itemize}
    \item Neither side ``wins''
    \item Both sides continue playing
    \item Ball distribution oscillates around equilibrium values
\end{itemize}

This is \emph{dynamic equilibrium}: not a static stop, but a dynamic balance where both processes continue at equal rates.
\end{proof}

\begin{proposition}[Equilibrium Distribution]
At equilibrium, the ball distribution satisfies:
\begin{equation}
\frac{n_B^{\text{eq}}}{n_A^{\text{eq}}} = K_{\text{eq}}
\end{equation}
where $K_{\text{eq}}$ depends on the relative aperture geometries and categorical resistances of each side.
\end{proposition}

\begin{remark}[Dynamic vs Static Equilibrium]
The ball game makes vivid what ``dynamic equilibrium'' means:
\begin{itemize}
    \item \textbf{Static view:} Reaction stops, nothing happens
    \item \textbf{Dynamic view:} Both reactions continue, rates are equal
\end{itemize}

In the ball game, equilibrium is not ``nobody plays''---it is ``everybody plays, nobody wins.'' Balls continuously cross in both directions at equal rates.
\end{remark}

\begin{corollary}[Le Chatelier Revisited]
Adding balls to one side disturbs equilibrium:
\begin{enumerate}
    \item Add balls to Team A $\to$ Team A scores more $\to$ equilibrium shifts toward B
    \item Remove balls from Team B $\to$ Team B scores less $\to$ equilibrium shifts toward B
    \item But total $N$ is conserved: new equilibrium has same $N$, different distribution
\end{enumerate}
\end{corollary}

\subsection{Summary}

The ball game thought experiment reveals that:

\begin{enumerate}
    \item Catalysis success depends on categorical availability, not velocity
    \item Each successful transit reduces resistance to subsequent transits
    \item Catalysis is inherently autocatalytic at the categorical level
    \item Time and velocity are categorically irrelevant
    \item Products create categorical structure that facilitates more products
    \item Conservation prevents ``complete victory''---reactions cannot go to 100\% completion
    \item Equilibrium is not cessation but \emph{sustained mutual play}
    \item Dynamic equilibrium emerges from the requirement that both sides must have balls to continue
\end{enumerate}

This completes the categorical theory of catalysis: catalysts are not time accelerators but \emph{autocatalytic apertures} whose successful traversal progressively reduces categorical resistance. The ``meaningless victory'' theorem explains why chemical equilibrium exists: complete conversion would halt the reaction entirely, so the system finds a distribution where both forward and reverse processes can continue indefinitely.

