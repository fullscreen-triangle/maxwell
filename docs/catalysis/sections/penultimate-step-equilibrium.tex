%==============================================================================
\section{Equilibrium and the Penultimate State}
\label{sec:penultimate}
%==============================================================================

We analyze chemical equilibrium through the lens of categorical completion.

\subsection{The Penultimate State}

\begin{definition}[Penultimate State]
\label{def:penultimate}
A categorical state $C_p$ is \emph{penultimate} with respect to target state $C_t$ if:
\begin{equation}
\dcat(C_p, C_t) = 1
\end{equation}

The penultimate state is one categorical transition away from completion.
\end{definition}

In transition state theory, the transition state corresponds to the penultimate categorical state. It is the configuration that requires only one more elementary transition (e.g., bond breaking or formation) to reach the product.

\subsection{Equilibrium as Mutual Penultimate Blocking}

\begin{theorem}[Equilibrium as Mutual Blocking]
\label{thm:equilibrium}
At chemical equilibrium, forward and reverse reactions mutually occupy their penultimate states, blocking each other's completion.
\end{theorem}

\begin{proof}
Consider a reversible reaction:
\begin{equation}
A \rightleftharpoons B
\end{equation}

At equilibrium, both forward (A $\to$ B) and reverse (B $\to$ A) processes occur continuously with equal rates.

In categorical terms:
\begin{itemize}
    \item Forward process: $C_A \to C_{AB}^* \to C_B$
    \item Reverse process: $C_B \to C_{BA}^* \to C_A$
\end{itemize}

where $C_{AB}^*$ and $C_{BA}^*$ are transition states (penultimate states).

At equilibrium:
\begin{align}
\dcat(C_{\mathrm{eq}}, C_B) &= 1 \quad \text{(one step from products)} \\
\dcat(C_{\mathrm{eq}}, C_A) &= 1 \quad \text{(one step from reactants)}
\end{align}

The system occupies a state equidistant from both endpoints. Neither the forward nor reverse reaction can complete because each blocks the other's final transition.
\end{proof}

\subsection{Preservation of Equilibrium Constant}

\begin{theorem}[$K_{\mathrm{eq}}$ Invariance]
\label{thm:keq}
Catalysts preserve equilibrium constants because they create symmetric categorical pathways.
\end{theorem}

\begin{proof}
The equilibrium constant is:
\begin{equation}
K_{\mathrm{eq}} = \frac{k_f}{k_r}
\end{equation}

In the categorical framework, rate constants relate to categorical distance:
\begin{equation}
k \propto \frac{1}{\dcat \cdot \tau_{\mathrm{step}}}
\end{equation}

For uncatalyzed reaction:
\begin{align}
k_f^{\mathrm{uncat}} &\propto \frac{1}{\dcat^{\mathrm{uncat}}(A \to B) \cdot \tau_{\mathrm{step}}} \\
k_r^{\mathrm{uncat}} &\propto \frac{1}{\dcat^{\mathrm{uncat}}(B \to A) \cdot \tau_{\mathrm{step}}}
\end{align}

For catalyzed reaction, the catalyst creates a new pathway with distance $\dcat^{\cat}$:
\begin{align}
k_f^{\cat} &\propto \frac{1}{\dcat^{\cat}(A \to B) \cdot \tau'_{\mathrm{step}}} \\
k_r^{\cat} &\propto \frac{1}{\dcat^{\cat}(B \to A) \cdot \tau'_{\mathrm{step}}}
\end{align}

The catalyst creates a \emph{single} bidirectional pathway. Both forward and reverse reactions traverse the same categorical space:
\begin{equation}
\dcat^{\cat}(A \to B) = \dcat^{\cat}(B \to A)
\end{equation}

Therefore:
\begin{equation}
\frac{k_f^{\cat}}{k_r^{\cat}} = \frac{\dcat^{\cat}(B \to A)}{\dcat^{\cat}(A \to B)} = 1 \cdot \frac{k_f^{\mathrm{uncat}}}{k_r^{\mathrm{uncat}}} = K_{\mathrm{eq}}
\end{equation}

The equilibrium constant is preserved because both directions are equally affected by the new pathway.
\end{proof}

\subsection{Why Catalysts Cannot Change Equilibrium}

\begin{corollary}[Equilibrium Immutability]
\label{cor:immutable}
No catalyst can alter the equilibrium constant of a reaction.
\end{corollary}

\begin{proof}
A catalyst that changed $K_{\mathrm{eq}}$ would need to create an asymmetric pathway:
\begin{equation}
\dcat^{\cat}(A \to B) \neq \dcat^{\cat}(B \to A)
\end{equation}

But any physical pathway connecting A and B is traversable in both directions. A molecule following the catalyzed path from A to B can return from B to A via the same intermediate states.

Therefore, forward and reverse categorical distances are necessarily equal, and $K_{\mathrm{eq}}$ is preserved.
\end{proof}

\subsection{The Transition State as Aperture}

In the categorical framework, the transition state is the narrowest aperture in the catalytic pathway:

\begin{definition}[Transition State Aperture]
The transition state $C^*$ is the categorical state with the smallest acceptance region $G_{C^*}$:
\begin{equation}
|G_{C^*}| = \min_{i} |G_{C_i}|
\end{equation}
where the minimum is over all states in the pathway.
\end{definition}

The traditional activation energy $E_a$ corresponds to the ``height'' of this aperture---the energetic cost of achieving the precise configuration required for passage.

\begin{proposition}[Transition State Stabilization]
Catalysts ``stabilize'' transition states by widening the acceptance region of the transition state aperture:
\begin{equation}
|G_{C^*}^{\cat}| > |G_{C^*}^{\mathrm{uncat}}|
\end{equation}

A wider aperture accepts more configurations, increasing the probability of topological completion.
\end{proposition}

This is the categorical interpretation of Pauling's hypothesis that enzymes bind transition states more tightly than substrates \citep{pauling1946}.

