%==============================================================================
\section{Phase-Lock Networks and Categorical Topology}
\label{sec:topology}
%==============================================================================

We formalize the representation of categorical states using phase-lock networks.

\subsection{Phase-Lock Networks}

\begin{definition}[Phase-Lock Network]
\label{def:phase-lock}
A \emph{phase-lock network} is a graph $G = (V, E)$ where:
\begin{itemize}
    \item $V$ is the set of entities (atoms, molecules, functional groups)
    \item $E \subseteq V \times V$ is the set of phase-lock edges representing geometric constraints
\end{itemize}

An edge $e = (v_i, v_j) \in E$ indicates that entities $v_i$ and $v_j$ are geometrically constrained to maintain a specific spatial relationship (distance, angle, orientation).
\end{definition}

\begin{definition}[Categorical State]
A \emph{categorical state} $C$ is an equivalence class of molecular configurations that share the same phase-lock network topology:
\begin{equation}
C = \{m \in \mathcal{M} : G(m) \cong G_C\}
\end{equation}
where $G(m)$ is the phase-lock network of configuration $m$ and $\cong$ denotes graph isomorphism.
\end{definition}

\begin{definition}[Categorical Transition]
A \emph{categorical transition} $C_i \to C_j$ occurs when the phase-lock network changes topology:
\begin{equation}
G_i \not\cong G_j
\end{equation}

Elementary transitions involve adding or removing a single edge:
\begin{align}
\text{Edge addition:} \quad &G_j = (V, E_i \cup \{e\}) \\
\text{Edge removal:} \quad &G_j = (V, E_i \setminus \{e\})
\end{align}
\end{definition}

\subsection{Categorical Distance}

\begin{definition}[Categorical Distance]
The \emph{categorical distance} $\dcat(C_i, C_j)$ between states $C_i$ and $C_j$ is the minimum number of elementary transitions required:
\begin{equation}
\dcat(C_i, C_j) = \min\{n : \exists \text{ path } C_i = C^{(0)} \to C^{(1)} \to \cdots \to C^{(n)} = C_j\}
\end{equation}
\end{definition}

\begin{proposition}[Metric Properties]
Categorical distance satisfies metric axioms:
\begin{enumerate}
    \item Non-negativity: $\dcat(C_i, C_j) \geq 0$
    \item Identity: $\dcat(C_i, C_i) = 0$
    \item Symmetry: $\dcat(C_i, C_j) = \dcat(C_j, C_i)$
    \item Triangle inequality: $\dcat(C_i, C_k) \leq \dcat(C_i, C_j) + \dcat(C_j, C_k)$
\end{enumerate}
\end{proposition}

\begin{proof}
Properties 1--3 follow directly from the definition. Property 4 holds because any path from $C_i$ to $C_k$ through $C_j$ is at least as long as the minimum path from $C_i$ to $C_k$.
\end{proof}

\subsection{Example: Chemical Bond Formation}

Consider the formation of H$_2$ from two hydrogen atoms:
\begin{equation}
\text{H} + \text{H} \to \text{H}_2
\end{equation}

\textbf{Initial state $C_1$:}
\begin{itemize}
    \item Entities: $V = \{\text{H}_1, \text{H}_2\}$
    \item Edges: $E_1 = \emptyset$ (no bond)
    \item Network: $G_1 = (\{\text{H}_1, \text{H}_2\}, \emptyset)$
\end{itemize}

\textbf{Final state $C_2$:}
\begin{itemize}
    \item Entities: $V = \{\text{H}_1, \text{H}_2\}$
    \item Edges: $E_2 = \{(\text{H}_1, \text{H}_2)\}$ (covalent bond)
    \item Network: $G_2 = (\{\text{H}_1, \text{H}_2\}, \{(\text{H}_1, \text{H}_2)\})$
\end{itemize}

\textbf{Categorical distance:}
\begin{equation}
\dcat(C_1, C_2) = 1
\end{equation}

One edge added, one categorical step.

\subsection{Example: Serine Protease Catalytic Triad}

The catalytic triad of chymotrypsin (Ser195-His57-Asp102) forms a phase-lock network \citep{blow1969, hedstrom2002}:

\textbf{Entities:}
\begin{equation}
V = \{\text{Ser195-OH}, \text{His57-N}, \text{Asp102-COO}^-, \text{Peptide-C=O}\}
\end{equation}

\textbf{Phase-lock edges with distances} \citep{polgar2005}:
\begin{align}
e_1 &= (\text{Ser195-OH}, \text{Peptide-C=O}), \quad d_1 \approx 2.8 \text{ \AA} \\
e_2 &= (\text{Ser195-OH}, \text{His57-N}), \quad d_2 \approx 3.0 \text{ \AA} \\
e_3 &= (\text{His57-N}, \text{Asp102-COO}^-), \quad d_3 \approx 2.8 \text{ \AA}
\end{align}

\textbf{Phase-lock network:}
\begin{equation}
\text{Peptide} \xleftrightarrow{2.8 \text{ \AA}} \text{Ser195} \xleftrightarrow{3.0 \text{ \AA}} \text{His57} \xleftrightarrow{2.8 \text{ \AA}} \text{Asp102}
\end{equation}

This network enables electron flow through hydrogen-bonded pathway. The precise distances are critical: perturbations beyond $\pm 0.5$ \AA{} disrupt the network and abolish catalytic activity.

\subsection{Geometric Precision in Phase-Lock Networks}

\begin{proposition}[Distance Sensitivity]
\label{prop:distance}
Catalytic activity depends critically on phase-lock edge distances. For hydrogen-bonded networks:
\begin{center}
\begin{tabular}{cc}
\toprule
Distance & Relative Activity \\
\midrule
2.8 \AA{} (optimal) & 100\% \\
3.5 \AA{} & $\sim$45\% \\
4.0 \AA{} & $\sim$12\% \\
5.0 \AA{} & $\sim$2\% \\
\bottomrule
\end{tabular}
\end{center}
\end{proposition}

This distance sensitivity confirms that catalysis operates through geometric phase-lock networks, not temporal acceleration. If catalysis were temporal, distance would affect only binding affinity, not catalytic mechanism.

\subsection{Temporal Independence}

\begin{theorem}[Temporal-Categorical Independence]
\label{thm:independence}
Categorical distance $\dcat$ is independent of temporal duration $\Delta t$. Two processes with identical $\dcat$ may have different $\Delta t$, and vice versa.
\end{theorem}

\begin{proof}
Categorical distance counts phase-lock network topology changes:
\begin{equation}
\dcat = |\{i : G^{(i)} \not\cong G^{(i-1)}\}|
\end{equation}

Temporal duration sums transition times:
\begin{equation}
\Delta t = \sum_{i=1}^{n} \tau_i
\end{equation}
where $\tau_i$ is the time for transition $i$.

Since $\tau_i$ depends on local conditions (temperature, pressure, concentration) while $\dcat$ depends only on topology, these quantities are independent.
\end{proof}

\begin{corollary}
Catalysts reduce categorical distance $\dcat$ by creating new pathways, not by reducing transition times $\tau_i$.
\end{corollary}

