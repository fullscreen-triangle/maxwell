
\subsection{Optimization Problems}

Optimization problems seek variable values that minimize or maximize an objective function subject to constraints.

\begin{definition}[Optimization Problem Encoding]
An optimization problem is encoded as:
\begin{itemize}
    \item Entities: Variables $\{x_1, \ldots, x_n\}$ with bounds $[\text{lo}_i, \text{hi}_i]$
    \item Relations: Dependencies between variables (if any)
    \item Constraints: Feasibility conditions
    \item Objective: Function $f(x_1, \ldots, x_n)$ to minimize/maximize
\end{itemize}
\end{definition}

\begin{example}[Quadratic Optimization]
The problem $\min_{\mathbf{x}} \mathbf{x}^T A \mathbf{x} + \mathbf{b}^T \mathbf{x}$ subject to $\mathbf{x} \in [l, u]^n$ encodes as:
\begin{itemize}
    \item Entities: Variables $x_1, \ldots, x_n$ with bounds $[l, u]$
    \item Objective: ``$\sum_{i,j} A_{ij} x_i x_j + \sum_i b_i x_i$''
\end{itemize}
Navigation minimizes the objective while respecting bounds.
\end{example}

\subsection{Search Problems}

Search problems seek an element from a collection that satisfies a target condition.

\begin{definition}[Search Problem Encoding]
A search problem is encoded as:
\begin{itemize}
    \item Entities: Elements of the search space $\{e_1, \ldots, e_n\}$
    \item Relations: Similarity or ordering between elements
    \item Constraints: Target condition as a structural constraint
\end{itemize}
\end{definition}

\begin{example}[Binary Search]
Finding value $v$ in sorted list $[a_1, \ldots, a_n]$ encodes as:
\begin{itemize}
    \item Entities: List elements with properties $\{\text{value}: a_i\}$
    \item Constraint: ``$|e.\text{value} - v| < \epsilon$'' (structure type)
\end{itemize}
Navigation converges to elements satisfying the target condition.
\end{example}

\subsection{Constraint Satisfaction Problems}

Constraint satisfaction problems seek variable assignments that satisfy all constraints, without optimization.

\begin{definition}[CSP Encoding]
A constraint satisfaction problem is encoded as:
\begin{itemize}
    \item Entities: Variables with domains
    \item Relations: Constraint dependencies
    \item Constraints: Conditions that must all be satisfied
\end{itemize}
\end{definition}

\begin{example}[N-Queens]
Placing $N$ queens on an $N \times N$ board such that none attack each other:
\begin{itemize}
    \item Entities: Variables $q_0, \ldots, q_{N-1}$ with domain $[0, N-1]$, where $q_i$ is the row of the queen in column $i$
    \item Constraints:
    \begin{itemize}
        \item No two queens in same row: $|q_i - q_j| > 0$ for $i \neq j$
        \item No two queens on same diagonal: $|q_i - q_j| \neq |i - j|$ for $i \neq j$
    \end{itemize}
\end{itemize}
Navigation finds an assignment where all constraints evaluate to satisfied.
\end{example}

\subsection{Pattern Matching Problems}

Pattern matching problems seek elements that match a given pattern.

\begin{definition}[Pattern Match Encoding]
A pattern matching problem is encoded as:
\begin{itemize}
    \item Entities: Pattern entity plus candidate entities
    \item Relations: Similarity relations between pattern and each candidate
    \item Constraints: Minimum similarity threshold
\end{itemize}
\end{definition}

\begin{proposition}[Match Score]
The match score between pattern $p$ and candidate $c$ is:
\begin{equation}
\text{score}(p, c) = \frac{\sum_{k \in \text{keys}(p)} \mathbf{1}[p[k] = c[k]]}{|\text{keys}(p)|}
\end{equation}
\end{proposition}

Navigation finds candidates with high match scores, ranked by similarity to the pattern.

\subsection{Biological System Problems}

Biological problems model molecular systems, interactions, and pathways.

\begin{definition}[Biological Problem Encoding]
A biological problem is encoded as:
\begin{itemize}
    \item Entities: Molecules with properties (mass, charge, binding sites, etc.)
    \item Relations: Interactions (binding, catalysis, inhibition)
    \item Constraints: Biochemical feasibility conditions
    \item Objective: Target property (binding affinity, pathway flux, etc.)
\end{itemize}
\end{definition}

\begin{example}[Ligand-Receptor Binding]
Modeling drug binding to a receptor:
\begin{itemize}
    \item Entities: Ligand molecule and receptor molecule with physical properties
    \item Relations: Binding interaction between ligand and receptor
    \item Constraints: Favorable binding energy (negative)
    \item Objective: Minimize binding energy (maximize affinity)
\end{itemize}
\end{example}

\begin{example}[Metabolic Pathway]
Modeling a metabolic pathway:
\begin{itemize}
    \item Entities: Enzymes and substrates/metabolites
    \item Relations: Catalytic reactions (enzyme + substrate $\to$ product)
    \item Constraints: Conservation of mass, thermodynamic feasibility
    \item Objective: Maximize pathway flux
\end{itemize}
\end{example}

\subsection{Problem Complexity}

Different problem types have different navigation characteristics.

\begin{proposition}[Navigation Complexity by Type]
\begin{center}
\begin{tabular}{lcc}
\toprule
Problem Type & Constraint Landscape & Typical Convergence \\
\midrule
Optimization (convex) & Single minimum & Fast ($O(\log \epsilon^{-1})$) \\
Optimization (non-convex) & Multiple minima & Variable \\
Search & Sparse solutions & Fast ($O(\log n)$) \\
Constraint satisfaction & Discrete feasible set & Variable \\
Pattern matching & Smooth similarity & Fast \\
Biological & Complex interactions & Variable \\
\bottomrule
\end{tabular}
\end{center}
\end{proposition}

The categorical approach is most effective for problems with smooth constraint landscapes, where gradient-based navigation can efficiently find completion points. Problems with highly discrete or discontinuous constraints may require stochastic strategies.

