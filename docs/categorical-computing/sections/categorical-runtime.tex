
\subsection{Runtime Architecture}

The categorical runtime executes problems by navigating through S-entropy space rather than by executing instructions. Navigation continues until a categorical completion point is reached---a position where all constraints are satisfied.

\begin{definition}[Categorical Runtime]
A categorical runtime $\mathcal{R}$ consists of:
\begin{enumerate}
    \item An oscillator capture module for precision-by-difference values
    \item A precision calculator for S-coordinate computation
    \item A categorical hierarchy for position representation
    \item A memory controller for state management
    \item A navigation engine for completion search
\end{enumerate}
\end{definition}

\subsection{Execution Context}

Each problem execution maintains a context that tracks the current state.

\begin{definition}[Execution Context]
An execution context $\xi$ consists of:
\begin{align}
\xi &= (\mathcal{P}, \Scoord_{\text{current}}, \sigma, \tau, V, t_0)
\end{align}
where:
\begin{itemize}
    \item $\mathcal{P}$ is the problem being executed
    \item $\Scoord_{\text{current}}$ is the current S-entropy position
    \item $\sigma$ is the current variable/entity state
    \item $\tau$ is the trajectory of positions visited
    \item $V$ is the map of constraint violations
    \item $t_0$ is the execution start time
\end{itemize}
\end{definition}

\subsection{Navigation Strategies}

The runtime supports multiple navigation strategies, each suited to different problem types.

\subsubsection{Gradient Descent}

Gradient navigation follows the constraint gradient toward satisfaction.

\begin{definition}[Constraint Gradient]
The constraint gradient at state $\sigma$ is:
\begin{equation}
\nabla_\sigma V = \sum_{c \in \mathcal{C}} \nabla_\sigma \text{violation}(c, \sigma)
\end{equation}
\end{definition}

\begin{algorithmic}[1]
\Function{GradientStep}{context $\xi$, step size $\alpha$}
    \State Compute gradient: $g \gets \nabla_\sigma V$
    \For{each variable $x$ in $\sigma$}
        \State $\sigma[x] \gets \sigma[x] - \alpha \cdot g[x]$
        \State Enforce bounds on $\sigma[x]$ if applicable
    \EndFor
    \State Update S-coordinate from new state
    \State \Return updated context
\EndFunction
\end{algorithmic}

\subsubsection{Categorical Completion}

Categorical completion navigates toward the predicted trajectory endpoint.

\begin{definition}[Completion Navigation]
Given current trajectory $\tau$ with predicted completion $\Scoord^*$, the navigation direction is:
\begin{equation}
\mathbf{d} = \frac{\Scoord^* - \Scoord_{\text{current}}}{\|\Scoord^* - \Scoord_{\text{current}}\|}
\end{equation}
\end{definition}

\begin{algorithmic}[1]
\Function{CompletionStep}{context $\xi$, step size $\alpha$}
    \State Predict completion: $\Scoord^* \gets$ predict($\tau$)
    \State Compute direction: $\mathbf{d} \gets (\Scoord^* - \Scoord_{\text{current}}) / \|\cdot\|$
    \State Update position: $\Scoord_{\text{new}} \gets \Scoord_{\text{current}} + \alpha \cdot \mathbf{d}$
    \State Map back to state space: $\sigma \gets$ coordinate\_to\_state($\Scoord_{\text{new}}$)
    \State \Return updated context
\EndFunction
\end{algorithmic}

\subsubsection{Simulated Annealing}

Simulated annealing provides stochastic exploration with temperature-controlled acceptance.

\begin{definition}[Annealing Temperature]
The temperature at step $k$ is:
\begin{equation}
T(k) = \frac{T_0}{1 + \beta k}
\end{equation}
where $T_0$ is the initial temperature and $\beta$ controls cooling rate.
\end{definition}

\begin{algorithmic}[1]
\Function{AnnealingStep}{context $\xi$, step $k$}
    \State Compute temperature: $T \gets T_0 / (1 + \beta k)$
    \State Generate perturbation: $\delta \gets \mathcal{N}(0, T)$
    \State Perturb state: $\sigma' \gets \sigma + \delta$
    \State Evaluate: $V' \gets$ total\_violation($\sigma'$)
    \If{$V' < V$ or $\text{random}() < \exp(-(V' - V)/T)$}
        \State Accept: $\sigma \gets \sigma'$
    \EndIf
    \State \Return updated context
\EndFunction
\end{algorithmic}

\subsubsection{Harmony Search}

Harmony search uses harmonic coincidences from hardware oscillations to guide navigation.

\begin{algorithmic}[1]
\Function{HarmonyStep}{context $\xi$}
    \State Capture precision signature: $\boldsymbol{\delta} \gets$ capture\_signature()
    \For{each variable $x$ in $\sigma$}
        \State Get harmonic factor: $h \gets 1 + \boldsymbol{\delta}[x \mod |\boldsymbol{\delta}|] \cdot 10$
        \State Compute harmonic position: $p \gets (\sin(h \cdot \pi) + 1) / 2$
        \State Map to variable bounds: $\sigma[x] \gets \text{lo}[x] + p \cdot (\text{hi}[x] - \text{lo}[x])$
    \EndFor
    \State \Return updated context
\EndFunction
\end{algorithmic}

\subsection{Convergence Criteria}

Navigation terminates when convergence criteria are met.

\begin{definition}[Convergence]
A context $\xi$ has converged when:
\begin{equation}
\sum_{c \in \mathcal{C}} \text{violation}(c, \sigma) < \epsilon
\end{equation}
where $\epsilon$ is the convergence threshold.
\end{definition}

Additional termination conditions:
\begin{itemize}
    \item Maximum steps reached: $|\tau| > K_{\max}$
    \item Timeout exceeded: $t_{\text{current}} - t_0 > T_{\max}$
    \item Position stagnation: $\|\Scoord_k - \Scoord_{k-1}\| < \epsilon_{\text{move}}$ for multiple steps
\end{itemize}

\subsection{Solution Construction}

Upon convergence, the solution is extracted from the final context.

\begin{definition}[Categorical Solution]
A solution $\mathcal{S}$ consists of:
\begin{align}
\mathcal{S} &= (\text{solved}, \text{result}, \tau, V, \Scoord^*, t_{\text{solve}})
\end{align}
where:
\begin{itemize}
    \item solved indicates whether convergence was achieved
    \item result contains the solution values
    \item $\tau$ is the complete navigation trajectory
    \item $V$ is the final constraint violation map
    \item $\Scoord^*$ is the final S-coordinate (completion point)
    \item $t_{\text{solve}}$ is the total solution time
\end{itemize}
\end{definition}

The result format depends on problem type:
\begin{itemize}
    \item Optimization: Variable values achieving minimum/maximum
    \item Search: Entity satisfying target condition
    \item Constraint: Variable assignment satisfying all constraints
    \item Pattern Match: Matching candidates ranked by similarity
\end{itemize}

\subsection{State-Coordinate Mapping}

The runtime maintains bidirectional mapping between variable states and S-coordinates.

\begin{definition}[State to Coordinate]
Given state $\sigma = \{x_1, \ldots, x_n\}$ with numeric values:
\begin{align}
\Sk &= \text{std}(\{x_1, \ldots, x_n\}) \\
\St &= \text{mean}(\{x_1, \ldots, x_n\}) \\
\Se &= H(\{x_1, \ldots, x_n\})
\end{align}
where $H$ is histogram entropy.
\end{definition}

\begin{definition}[Coordinate to State]
Given S-coordinate $\Scoord = (\Sk, \St, \Se)$ and variable bounds:
\begin{equation}
x_i = \St + \left(i - \frac{n}{2}\right) \cdot \frac{2\Sk}{n}
\end{equation}
with clamping to bounds $[\text{lo}_i, \text{hi}_i]$.
\end{definition}

This mapping enables navigation in S-space to translate to variable updates, and variable changes to translate to S-space movement.

