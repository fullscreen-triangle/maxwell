%==============================================================================
\section{Heat Transfer versus Entropy: The Fundamental Decoupling}
\label{sec:heat_transfer}
%==============================================================================

A critical insight emerges from analysing what happens when the demon's door operation results in a collision near the aperture. The analysis reveals that \textit{heat transfer and entropy change are fundamentally decoupled}---heat can flow in either direction during a collision, while entropy increases regardless.

\subsection{The Collision Scenario}

Consider the demon opening the door to allow a fast molecule from the hot container (A) to pass to the cold container (B). Near the aperture, this molecule collides with a molecule already in container B.

\begin{definition}[Door Collision Event]
A \textbf{door collision event} occurs when a molecule transiting through the demon's aperture collides with a molecule in the receiving container before fully entering. Let:
\begin{itemize}
    \item $m_A$: transiting molecule from container A with initial velocity $\mathbf{v}_A$
    \item $m_B$: molecule in container B with initial velocity $\mathbf{v}_B$
    \item $\mathbf{v}_A'$, $\mathbf{v}_B'$: post-collision velocities
\end{itemize}
\end{definition}

We analyse three cases based on collision outcomes.

\subsection{Case 1: Elastic Collision with Bounce-Back}

\begin{proposition}[Bounce-Back Heat Transfer]
\label{prop:bounce_back}
If molecule $m_A$ bounces back to container A after an elastic collision:
\begin{enumerate}
    \item Energy transfers from $m_A$ to $m_B$
    \item The ``counted'' molecule returns to its origin
    \item Heat has transferred hot $\to$ cold
    \item Entropy increases in both containers
\end{enumerate}
\end{proposition}

\begin{proof}
In an elastic collision, kinetic energy is conserved:
\begin{equation}
\frac{1}{2}m_A v_A^2 + \frac{1}{2}m_B v_B^2 = \frac{1}{2}m_A v_A'^2 + \frac{1}{2}m_B v_B'^2
\end{equation}

If $m_A$ bounces back (returns to A), it must have lost momentum and energy to $m_B$. Therefore:
\begin{align}
E_A' &< E_A \quad \text{(energy lost by } m_A\text{)} \\
E_B' &> E_B \quad \text{(energy gained by } m_B\text{)}
\end{align}

Energy has flowed from the hot container to the cold container, even though the ``sorted'' molecule returned.

For entropy: the collision creates new phase-lock correlations between $m_A$ and $m_B$. Their trajectories are now correlated through the collision event:
\begin{equation}
\Delta S_{\text{correlation}} = k_B \ln \Omega_{\text{correlated}} > 0
\end{equation}

Both containers experience categorical completion through new correlations.
\end{proof}

\subsection{Case 2: Inelastic Collision---Cold Molecule Accelerates}

\begin{proposition}[Standard Heat Transfer]
\label{prop:standard_heat}
If the collision is inelastic and $m_B$ gains significant energy:
\begin{enumerate}
    \item Heat transfers hot $\to$ cold (standard direction)
    \item Entropy increases in both containers
    \item This is the ``expected'' thermodynamic outcome
\end{enumerate}
\end{proposition}

\begin{proof}
Energy conservation with dissipation:
\begin{equation}
E_A + E_B = E_A' + E_B' + Q_{\text{dissipated}}
\end{equation}

If $E_B' > E_B$ (cold molecule accelerates), energy has flowed from hot to cold. This is conventional heat transfer.

Entropy increases through:
\begin{enumerate}
    \item Dissipation: $\Delta S_{\text{dissipation}} = Q_{\text{dissipated}}/T > 0$
    \item Phase-lock correlation: new correlations form from collision
    \item Categorical completion: both containers advance categorically
\end{enumerate}
\end{proof}

\subsection{Case 3: Inelastic Collision---Cold Molecule Decelerates}

This case is the most revealing.

\begin{theorem}[Reverse Heat Transfer with Entropy Increase]
\label{thm:reverse_heat}
If the collision is inelastic and $m_B$ loses significant energy (decelerates):
\begin{enumerate}
    \item Heat transfers cold $\to$ hot (reverse direction)
    \item The fast molecule $m_A$ may return to A with \textbf{more} energy than it started
    \item Entropy \textbf{still increases} in both containers
\end{enumerate}
\end{theorem}

\begin{proof}
Consider an inelastic collision where:
\begin{align}
E_B' &< E_B \quad \text{(cold molecule lost energy)} \\
E_A' &> E_A \quad \text{(hot molecule gained energy)}
\end{align}

This is possible if the collision extracts energy from $m_B$'s existing motion and transfers it to $m_A$. The hot molecule returns to container A with \textit{more} energy than it left with.

\textbf{Heat direction:} Heat has flowed from cold to hot---apparently violating thermodynamics.

\textbf{Entropy direction:} The collision creates phase-lock correlations regardless of energy flow direction:
\begin{equation}
\Delta S_{\text{total}} = \Delta S_A + \Delta S_B + \Delta S_{\text{correlation}} > 0
\end{equation}

Container A: The returning molecule with higher energy creates new phase-lock relationships with its neighbours. Categorical advancement: $C_A \prec C_A'$.

Container B: The molecule that lost energy now has different phase relationships with its neighbours. Network reconfiguration constitutes categorical completion: $C_B \prec C_B'$.

The collision itself: Creates irreducible correlations between $m_A$ and $m_B$ that increase the accessible categorical state space.

\textbf{Entropy increases despite reverse heat flow.}
\end{proof}

\subsection{The Fundamental Decoupling}

\begin{theorem}[Heat-Entropy Decoupling]
\label{thm:heat_entropy_decoupling}
Heat transfer and entropy change are fundamentally decoupled:
\begin{align}
\text{Heat direction} &\in \{\text{hot} \to \text{cold}, \text{cold} \to \text{hot}, \text{zero}\} \\
\text{Entropy change} &> 0 \quad \text{(always)}
\end{align}
for any collision event at the demon's door.
\end{theorem}

\begin{proof}
Heat transfer is a \textit{kinetic} property: which way did energy flow?

Entropy change is a \textit{categorical} property: how did phase-lock correlations change?

Every collision---regardless of energy flow direction---creates new phase-lock correlations. The collision event itself is a categorical completion that increases accessible states.

The three cases demonstrate:
\begin{itemize}
    \item Case 1: Heat $\to$ cold, entropy $\uparrow$
    \item Case 2: Heat $\to$ cold, entropy $\uparrow$
    \item Case 3: Heat $\to$ hot, entropy $\uparrow$
\end{itemize}

Entropy increases in all cases. Heat direction is variable.
\end{proof}

\subsection{Why Maxwell Conflated Heat and Entropy}

\begin{proposition}[Maxwell's Conflation]
Maxwell implicitly assumed:
\begin{equation}
\Delta Q > 0 \iff \Delta S > 0
\end{equation}
That is, heat flow and entropy change are equivalent. This assumption fails at the microscopic level.
\end{proposition}

\begin{proof}
The macroscopic second law states:
\begin{equation}
dS \geq \frac{\delta Q}{T}
\end{equation}

This relates heat flow to entropy change \textit{on average, in the thermodynamic limit}. At the single-molecule level:
\begin{itemize}
    \item Individual collisions can transfer energy in either direction
    \item The inequality becomes an equality only statistically
    \item Fluctuations can produce local ``violations'' that average out
\end{itemize}

Maxwell, reasoning at the single-molecule level, assumed the demon could exploit these fluctuations. But he measured the wrong quantity: heat instead of entropy.
\end{proof}

\subsection{Heat is Statistical, Entropy is Categorical}

\begin{definition}[Heat as Statistical Average]
Heat flow $Q$ is the \textbf{statistical average} of energy transfer over many collisions:
\begin{equation}
Q = \lim_{N \to \infty} \frac{1}{N} \sum_{i=1}^{N} \Delta E_i
\end{equation}
Individual $\Delta E_i$ can be positive or negative; only the average has thermodynamic significance.
\end{definition}

\begin{definition}[Entropy as Categorical Completion]
Entropy change $\Delta S$ is the \textbf{categorical advancement} through phase-lock network densification:
\begin{equation}
\Delta S = k_B \ln \frac{\Omega_{\text{final}}}{\Omega_{\text{initial}}}
\end{equation}
where $\Omega$ counts accessible categorical states. This is \textit{always} non-negative for spontaneous processes.
\end{definition}

\begin{theorem}[Heat Statistical, Entropy Fundamental]
\label{thm:heat_statistical}
\begin{enumerate}
    \item Heat is an emergent statistical property (sum over fluctuating microscopic transfers)
    \item Entropy is a fundamental categorical property (monotonic increase through completion)
    \item The Second Law constrains entropy, not heat
    \item Heat obeys the Second Law only on average
\end{enumerate}
\end{theorem}

\begin{proof}
The Second Law states $\Delta S \geq 0$ for isolated systems. It does not state $\Delta Q \geq 0$ or fix the direction of energy flow.

Energy conservation (First Law) is separate from entropy increase (Second Law). A process can:
\begin{itemize}
    \item Conserve energy while increasing entropy (typical)
    \item Transfer energy in the ``wrong'' direction while still increasing entropy (fluctuation)
    \item Have zero net heat transfer while increasing entropy (isothermal irreversible process)
\end{itemize}

Heat direction is contingent; entropy increase is necessary.
\end{proof}

\subsection{Implications for the Demon}

\begin{corollary}[Demon's Irrelevance to Heat Direction]
The demon's door operation produces entropy increase regardless of heat flow direction. Even if a particular collision transfers heat from cold to hot, total entropy still increases.
\end{corollary}

\begin{proof}
The demon operates at the single-molecule level, where heat flow is fluctuating. Some operations will transfer heat hot$\to$cold, some cold$\to$hot.

But every operation creates phase-lock correlations, hence entropy increase.

The demon cannot exploit individual fluctuations because entropy is not fluctuating---it monotonically increases through categorical completion.
\end{proof}

\begin{theorem}[Complete Demon Defeat]
\label{thm:complete_defeat}
The demon is defeated at every level:
\begin{enumerate}
    \item \textbf{Individual collisions}: Entropy increases regardless of energy direction
    \item \textbf{Statistical average}: Net heat flows hot$\to$cold (Second Law on average)
    \item \textbf{Categorical structure}: Every operation advances categorical completion
\end{enumerate}
The demon cannot violate the Second Law because:
\begin{itemize}
    \item It cannot control individual collision outcomes (quantum/thermal uncertainty)
    \item Even ``favourable'' outcomes (heat cold$\to$hot) increase entropy
    \item The quantity it tries to manipulate (heat) is not the conserved quantity (entropy)
\end{itemize}
\end{theorem}

\subsection{The Insight Formalised}

\begin{equation}
\boxed{
\begin{aligned}
\text{Heat} &: \text{energy accounting (can flow either way)} \\
\text{Entropy} &: \text{categorical completion (always increases)}
\end{aligned}
}
\end{equation}

Maxwell asked: ``Can the demon sort molecules to transfer heat from cold to hot?''

The answer is: \textit{It doesn't matter.} Even if a particular collision transfers heat cold$\to$hot, entropy still increases. The demon is defeated not by heat flow constraints but by the inexorable advance of categorical completion.

\begin{remark}[Historical Irony]
The Second Law was historically formulated in terms of heat (``heat cannot spontaneously flow from cold to hot''). This formulation, while correct macroscopically, obscures the microscopic reality: the Second Law constrains \textit{entropy}, not \textit{heat}. The heat formulation is a consequence, not the foundation.

Maxwell's Demon exploits the historical conflation. By framing the paradox in terms of heat, Maxwell created a puzzle that seemed to require information-theoretic resolution. Framed in terms of entropy, the paradox dissolves: entropy increases regardless of heat direction.
\end{remark}

\subsection{Summary}

The analysis of door collisions reveals the fundamental decoupling of heat and entropy:

\begin{enumerate}
    \item Heat can flow in either direction in individual collisions
    \item Entropy \textit{always} increases through categorical completion
    \item The Second Law constrains entropy, not heat direction
    \item Heat obeys thermodynamic constraints only statistically
    \item The demon measures and manipulates heat---the wrong quantity
    \item Entropy, the actually conserved quantity, is immune to the demon's strategy
\end{enumerate}

This decoupling provides the most fundamental resolution of Maxwell's Demon: the demon attacks a statistical emergent property (heat) while the Second Law protects a categorical fundamental property (entropy). The demon's entire strategy is misdirected.

